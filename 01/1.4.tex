\documentclass{article}

\title{Dummit \& Foote Ch. 1.4: Matrix Groups}
\author{Scott Donaldson}
\date{Mar. 2023}
\usepackage{amsmath, amsthm, amsfonts, enumitem}

% 1.4.9 has a matrix multiplication that overflows but looks OK, so override the warning here
\hfuzz=14pt

\begin{document}

\maketitle

\section*{1. (3/16/23)}

Prove that $|GL_2(\mathbb{F}_2)| = 6$.

\begin{proof}
    Matrices in $GL_2(\mathbb{F}_2)$ have the form 
    $\begin{pmatrix}a & b\\c & d\end{pmatrix}$, 
    where $a, b, c, d \in \{0, 1\}$. There are 16 possible matrices of this form (2 options for each entry over 4 entries, $2^4 = 16$).
    
    From the definition of $GL_2$, we discount matrices with determinant 0. A $2 \times 2$ matrix has determinant 0 when $ad - bc = 0$, that is, $ad = bc$. This happens only when $ad = bc = 1$ or $ad = bc = 0$. There is only one matrix where $ad = bc = 1, \begin{pmatrix}1 & 1\\1 & 1\end{pmatrix}$. Matrices with determinant 0 have one of $a, d$ and $b, c$ equal to 0. They are the matrices with all zero entries (1), with three zero entries (4), and with two zero entries ($a$ and $b$, or $a$ and $c$, or $b$ and $d$, or $c$ and $d$) (4).

    This leaves us with $16 - 1 - 1 - 4 - 4 = 6$ matrices with nonzero determinants, so the order of $GL_2(\mathbb{F}_2) = 6$.
\end{proof}

\section*{2. (3/16/23)}

Write out all the elements of $GL_2(\mathbb{F}_2)$ and compute the order of each element.

\begin{itemize}
    \item $\begin{pmatrix}1 & 0\\0 & 1\end{pmatrix}$: 1 (identity)
    \item $\begin{pmatrix}1 & 1\\0 & 1\end{pmatrix}$: 2
    \item $\begin{pmatrix}1 & 0\\1 & 1\end{pmatrix}$: 2
    \item $\begin{pmatrix}0 & 1\\1 & 1\end{pmatrix}$: 3
    \item $\begin{pmatrix}1 & 1\\1 & 0\end{pmatrix}$: 3
    \item $\begin{pmatrix}0 & 1\\1 & 0\end{pmatrix}$: 2
\end{itemize}

\section*{3. (3/16/23)}

Show that $GL_2(\mathbb{F}_2)$ is non-abelian.

\begin{proof}
    To prove that $GL_2(\mathbb{F}_2)$ is non-abelian, we need only show that it contains two non-commuting elements. We have:
    \begin{align*}
        \begin{pmatrix}0 & 1\\1 & 1\end{pmatrix} \begin{pmatrix}0 & 1\\1 & 0\end{pmatrix} &= \begin{pmatrix}1 & 0\\1 & 1\end{pmatrix}, \text{ while } \\
        \begin{pmatrix}0 & 1\\1 & 0\end{pmatrix} \begin{pmatrix}0 & 1\\1 & 1\end{pmatrix} &= \begin{pmatrix}1 & 1\\0 & 1\end{pmatrix}
    \end{align*}
    Since these products are not equal, $GL_2(\mathbb{F}_2)$ is non-abelian.
\end{proof}

\section*{4. (3/18/23)}

Show that if $n$ is not prime then $\mathbb{Z}/n\mathbb{Z}$ is not a field.

\begin{proof}
    Let $n$ be a composite positive integer and let $a$ divide $n$ with $a > 1$. We will show that $a$ does not have a multiplicative inverse in $\mathbb{Z}/n\mathbb{Z}$, and therefore $\mathbb{Z}/n\mathbb{Z}$ is not a field.

    We will show that there is no integer $c$ such that $ac = 1$ mod $n$. Since $a$ divides $n$, let $ab = n = 0$ mod $n$. So $a(b + 1) = ab + a = n + a = a$ mod $n$. That is, for the pair of consecutive integers $b$ and $b + 1$, we have $ab = 0 < 1$ and $a(b + 1) = a > 1$. Then there is no integer $c$ strictly between $b$ and $b + 1$ such that $ac = 1$ mod $n$. For any larger integers, we note that $abk = nk = 0$ mod $n$, and $a(bk + 1) = abk + a = nk + a = a$ mod $n$, and therefore there is no integer $c$ among all of $\mathbb{Z}^+$ with $ac = 1$. Therefore, since $a$ has no multiplicative inverse, $\mathbb{Z}/n\mathbb{Z}$ is not a field.
\end{proof}

\section*{5. (3/18/23)}

Show that $GL_n(F)$ is a finite group if and only if $F$ has a finite number of elements.

\begin{proof}
    Let $F$ be a field with $m < \infty$ elements and, for some $n > 1$, let $GL_n(F)$ be the general linear group of degree $n$ on $F$. The total possible number of $n \times n$ matrices with entries from $F$ is $m^{n^2}$. Since the number of elements in $GL_n(F)$ is at most this value, it is a finite group (in 6. we will show that it is strictly less than).

    To prove the converse, we will show that, if $F$ is an infinite field, then $GL_n(F)$ must not be a finite group. Let $F$ be an infinite field. For every $x \in F$ (excluding $x = 0$), we can construct an $n \times n$ matrix whose diagonal entries are $x$ and all other entries are 0. By definition, the determinant of such a matrix is the product of the diagonal entries, $x^n \neq 0$. Therefore such a matrix belongs to $GL_n(F)$. This is a bijection between $F$ and $GL_n(F)$, and so they have the same cardinality, that is, $GL_n(F)$ must not be a finite group.

    Thus, $GL_n(F)$ is a finite group if and only if $F$ has a finite number of elements.
\end{proof}

\section*{6. (3/19/23)}

If $|F| = q$ is finite prove that $|GL_n(F)| < q^{n^2}$.

\begin{proof}
    An element of $GL_n(F)$ is an invertible $n \times n$ matrix whose entries come from $F$. For each entry, there are $q$ possibilities, and there are $n^2$ total entries, so there are $q^{n^2}$ possible such matrices (before discounting those with determinant = 0). It is guaranteed that some number of $n \times n$ matrices have determinant 0; for example, the matrix whose entries are all 0 obviously has determinant 0. So the number of elements of $GL_n(F)$ is always strictly less than $q^{n^2}$.
\end{proof}

\section*{7. (3/19/23)}

Let $p$ be a prime. Prove that the order of $GL_2(\mathbb{F}_p)$ is $p^4 - p^3 - p^2 + p$.

\begin{proof}
    From 5. and 6., there are $p^{2^2} = p^4$ possible $2 \times 2$ matrices, and the order of $GL_2(\mathbb{F}_p)$ is strictly less than this number. Let us count the ways in which an element of $GL_2(\mathbb{F}_p)$ might have a determinant equal to 0.

    A $2 \times 2$ matrix in $GL_2(\mathbb{F}_p)$ has the form $\begin{pmatrix}a & b\\c & d\end{pmatrix}$, with $a, b, c, d \in F_p$. The determinant of a $2 \times 2$ matrix is $ad - bc$. First, consider the cases in which $a, b, c, d \neq 0$. Setting the determinant equal to 0, we can see that $d$ must equal $bc / a$. So there are $p - 1$ choices for $a, b, c$, and $d$ is fixed based on the other entries. Then there are $(p - 1)^3$ matrices with 4 nonzero entries with determinant equal to 0.

    Next, consider $2 \times 2$ matrices with one entry equal to 0, for example, $\begin{pmatrix}a & b\\c & 0\end{pmatrix}$. The determinant of this matrix is $a \cdot 0 - bc = bc$. In order for this to equal 0, at least one of either $b$ or $c$ must equal zero. Then there are no matrices with exactly 1 zero entry with determinant equal to 0.

    Now consider $2 \times 2$ matrices with two entries equal to 0. Such matrices have the form $\begin{pmatrix}a & b\\0 & 0\end{pmatrix}, \begin{pmatrix}a & 0\\c & 0\end{pmatrix}, \begin{pmatrix}0 & 0\\c & d\end{pmatrix},$ or $\begin{pmatrix}0 & b\\0 & d\end{pmatrix}$. There are $p - 1$ possible choices for both of the nonzero entries, so there are $4(p - 1)^2$ matrices with exactly 2 nonzero entries with determinant equal to 0.

    Matrices with three entries equal to 0 have the form $\begin{pmatrix}a & 0\\0 & 0\end{pmatrix}, \begin{pmatrix}0 & b\\0 & 0\end{pmatrix},\newline \begin{pmatrix}0 & 0\\c & 0\end{pmatrix},$ or $\begin{pmatrix}0 & 0\\0 & d\end{pmatrix}$. There are $4(p - 1)$ such matrices.

    Finally, there is the single matrix with all 0 entries, $\begin{pmatrix}0 & 0\\0 & 0\end{pmatrix}$.

    So, the total number of elements of $GL_2(\mathbb{F}_p)$ is:
    \begin{multline*}
        p^4 - (p - 1)^3 - 4(p - 1)^2 - 4(p - 1) - 1 = \\
        p^4 - (p^3 - 3p^2 + 3p - 1) - (4p^2 - 8p + 4) - (4p - 4) - 1 = \\
        p^4 - p^3 + 3p^2 - 3p + 1 - 4p^2 + 8p - 4 - 4p + 4 - 1 = \\
        p^4 - p^3 + (3 - 4)p^2 + (-3 + 8 - 4)p + (1 - 4 + 4 - 1) = \\
        p^4 - p^3 - p^2 + p
    \end{multline*}
    as desired.
\end{proof}

\section*{8. (3/21/23)}

Show that $GL_n(F)$ is non-abelian for any $n \geq 2$ and $F$.

\begin{proof}
    To show that $GL_n(F)$ is non-abelian, we need to show that it contains two elements that are noncommutative. By definition of general linear groups, $GL_n(F)$ consists of invertible $n \times n$ matrices whose entries come from the field $F$. Further, by definition of fields, $F$ contains an additive identity 0 and a multiplicative identity 1. Therefore, if we consider only matrices in $GL_n(F)$ whose entries are 0 or 1 and whose product's entries are 0 or 1 (in $\mathbb{Z}$), these are elements of every $GL_n(F)$ regardless of which $F$ we choose.

    Let $A$ be the transpose of the identity matrix and let $B$ be equal to the identity matrix with the final two columns swapped:
    
    $A = \begin{pmatrix}
        0 & 0 & \cdots & 0 & 1\\
        0 & 0 & \cdots & 1 & 0\\
        \vdots & \vdots & \ddots & \vdots & \vdots \\
        0 & 1 & \cdots & 0 & 0\\
        1 & 0 & \cdots & 0 & 0
    \end{pmatrix}, 
    B = \begin{pmatrix}
        1 & 0 & \cdots & 0 & 0\\
        0 & 1 & \cdots & 0 & 0\\
        \vdots & \vdots & \ddots & \vdots & \vdots \\
        0 & 0 & \cdots & 0 & 1\\
        0 & 0 & \cdots & 1 & 0
    \end{pmatrix}$.

    The upper-right entry of $AB$ is the dot product of the first row of $A$ with the last column of $B$: $0 \cdot 0 + 0 \cdot 0 + \ldots + 0 \cdot 1 + 1 \cdot 0 = 0$.

    The upper-right entry of $BA$ is the dot product of the first row of $B$ with the last column of $A$: $1 \cdot 1 + 0 \cdot 0 + \ldots + 0 \cdot 0 + 0 \cdot 0 = 1$.

    Because $AB$ and $BA$ do not contain exactly the same entries, they are not equal matrices. Therefore, $A$ and $B$ do not commute. Further, because for every $n \geq 2$ and every field $F$, $GL_n(F)$ contains the elements $A$ and $B$, $GL_n(F)$ is non-abelian.
\end{proof}

\section*{9. (3/21/23)}

Prove that the binary operation of multiplication of $2 \times 2$ matrices is associative.

\begin{proof}
    Let $A = \begin{pmatrix}a_1 & a_2\\a_3 & a_4\end{pmatrix}, B = \begin{pmatrix}b_1 & b_2\\b_3 & b_4\end{pmatrix}, C = \begin{pmatrix}c_1 & c_2\\c_3 & c_4\end{pmatrix}$.

    \begin{align*}
        A(BC) &= \begin{pmatrix}a_1 & a_2\\a_3 & a_4\end{pmatrix}\Bigl(\begin{pmatrix}b_1 & b_2\\b_3 & b_4\end{pmatrix}\begin{pmatrix}c_1 & c_2\\c_3 & c_4\end{pmatrix}\Bigr) \\ 
        &= \begin{pmatrix}a_1 & a_2\\a_3 & a_4\end{pmatrix}\begin{pmatrix}b_1 c_1 + b_2 c_3 & b_1 c_2 + b_2 c_4\\b_3 c_1 + b_4 c_3 & b_3 c_2 + b_4 c_4\end{pmatrix} \\
        &= \begin{pmatrix}
            a_1 (b_1 c_1 + b_2 c_3) + a_2 (b_3 c_1 + b_4 c_3) & a_1 (b_1 c_2 + b_2 c_4) + a_2 (b_3 c_2 + b_4 c_4)\\
            a_3 (b_1 c_1 + b_2 c_3) + a_4 (b_3 c_1 + b_4 c_3) & a_3 (b_1 c_2 + b_2 c_4) + a_4 (b_3 c_2 + b_4 c_4)
        \end{pmatrix} \\
        &= \begin{pmatrix}
            (a_1 b_1 + a_2 b_3) c_1 + (a_1 b_2 + a_2 b_4) c_3 & (a_1 b_1 + a_2 b_3) c_2 + (a_1 b_2 + a_2 b_4) c_4\\
            (a_3 b_1 + a_4 b_3) c_1 + (a_3 b_2 + a_4 b_4) c_3 & (a_3 b_1 + a_4 b_3) c_2 + (a_3 b_2 + a_4 b_4) c_4
        \end{pmatrix} \\
        &= \begin{pmatrix}a_1 b_1 + a_2 b_3 & a_1 b_2 + a_2 b_4\\a_3 b_1 + a_4 b_3 & a_3 b_2 + a_4 b_4\end{pmatrix}\begin{pmatrix}c_1 & c_2 \\ c_3 & c_4\end{pmatrix} \\
        &= \Bigl(\begin{pmatrix}a_1 & a_2\\a_3 & a_4\end{pmatrix}\begin{pmatrix}b_1 & b_2\\b_3 & b_4\end{pmatrix}\Bigr)\begin{pmatrix}c_1 & c_2\\c_3 & c_4\end{pmatrix} \\
        &= (AB)C.
    \end{align*}
\end{proof}

\section*{10. (3/22/23)}

Let $G = \{ \begin{pmatrix}a & b \\ 0 & c\end{pmatrix} \mid a, b, c \in \mathbb{R}, a \neq 0, c \neq 0 \}$.

\begin{enumerate}[label=(\alph*)]
    \item Compute the product of $\begin{pmatrix}a_1 & b_1 \\ 0 & c_1\end{pmatrix}$ and $\begin{pmatrix}a_2 & b_2 \\ 0 & c_2\end{pmatrix}$ to show that $G$ is closed under matrix multiplication.
    
          $\begin{pmatrix}a_1 & b_1 \\ 0 & c_1\end{pmatrix}\begin{pmatrix}a_2 & b_2 \\ 0 & c_2\end{pmatrix} = \begin{pmatrix}a_1 a_2 & a_1 b_2 + b_1 c_2 \\ 0 & c_1 c_2\end{pmatrix}$. $\mathbb{R}$ is closed under addition and multiplication, so the entries of the matrix product are all in $\mathbb{R}$, and so the product is an element of $G$.

    \item Find the matrix inverse of $\begin{pmatrix}a & b \\ 0 & c\end{pmatrix}$ and deduce that $G$ is closed under inverses.

          The inverse of $\begin{pmatrix}a & b \\ 0 & c\end{pmatrix}$ is the $2 \times 2$ matrix $\begin{pmatrix}d & e \\ f & g\end{pmatrix}$ such that \newline $\begin{pmatrix}a & b \\ 0 & c\end{pmatrix}\begin{pmatrix}d & e \\ f & g\end{pmatrix} = \begin{pmatrix}1 & 0 \\ 0 & 1\end{pmatrix}$.

          Looking at lower-left entry first, we have $0 \cdot d + cf = 0$. We know that $c$ is nonzero, so $f = 0$.

          Next, looking at the upper-left entry, we have $ad + b \cdot 0 = 1 \Rightarrow ad = 1$. So $d = 1/a$. Similarly for the lower-right entry, $cg = 1 \Rightarrow g = 1/c$.

          Finally, looking at the upper-right entry, we have $ae + bg = ae + b/c = 0$. So $e = -b/ac$. Therefore the inverse matrix is $\begin{pmatrix}1/a & -b/ac \\ 0 & 1/c\end{pmatrix}$.

    \item Deduce that $G$ is a subgroup of $GL_2(\mathbb{R})$.

          From 9., matrix multiplication is associative for $2 \times 2$ matrices. As shown in a), $G$ is closed under the operation of matrix multiplication, and in b), inverses of elements in $G$ are also in $G$. Thus $G$ is a subgroup of $GL_2(\mathbb{R})$.

    \item Prove that the set of elements of $G$ whose diagonal entries are equal (i.e. $a = c$) is also a subgroup of $GL_2(\mathbb{R})$.

          Now let $H$ be the set of elements of $G$ whose diagonal entries are equal; that is, matrices of the form $\begin{pmatrix}a & b \\ 0 & a\end{pmatrix}, a \neq 0$.

          $H$ is closed under matrix multiplication: \newline $\begin{pmatrix}a & b \\ 0 & a\end{pmatrix}\begin{pmatrix}c & d \\ 0 & c\end{pmatrix} = \begin{pmatrix}ac & ad + bc \\ 0 & ac\end{pmatrix}$.

          From b), the inverse of $\begin{pmatrix}a & b \\ 0 & a\end{pmatrix}$ is $\begin{pmatrix}1/a & -b/a^2 \\ 0 & 1/a\end{pmatrix}$, which is also in $H$.

          Thus this set is also a subgroup of $GL_2(\mathbb{R})$.
\end{enumerate}

\section*{11. (3/23/23)}

Let $H(F) = \Bigl\{ \begin{pmatrix}1 & a & b \\ 0 & 1 & c \\ 0 & 0 & 1\end{pmatrix} \mid a, b, c \in F \Bigr\}$ — called the \emph{Heisenberg group} over $F$. Let $X = \begin{pmatrix}1 & a & b \\ 0 & 1 & c \\ 0 & 0 & 1\end{pmatrix}$ and $Y = \begin{pmatrix}1 & d & e \\ 0 & 1 & f \\ 0 & 0 & 1\end{pmatrix}$ be elements of $H(F)$.

\begin{enumerate}[label=(\alph*)]
    \item Compute the matrix product $XY$ and deduce that $H(F)$ is closed under matrix multiplication. Exhibit explicit matrices such that $XY \neq YX$ (so that $H(F)$ is always non-abelian).

          $XY = \begin{pmatrix}1 & a & b \\ 0 & 1 & c \\ 0 & 0 & 1\end{pmatrix}\begin{pmatrix}1 & d & e \\ 0 & 1 & f \\ 0 & 0 & 1\end{pmatrix} = \begin{pmatrix}1 & a + d & b + e + af \\ 0 & 1 & c + f \\ 0 & 0 & 1\end{pmatrix}$.

          Given this product, the only entry that has a different value when the order of multiplication is reversed is the upper-right, which is $b + e + af$ for $XY$ and $b + e + cd$ for $YX$. So we need to find $af \neq cd$. Let $a = f = 1$ and $c = d = 0$.

          Then:\newline
          $XY = \begin{pmatrix}1 & 1 & b \\ 0 & 1 & 0 \\ 0 & 0 & 1\end{pmatrix}\begin{pmatrix}1 & 0 & e \\ 0 & 1 & 1 \\ 0 & 0 & 1\end{pmatrix} = \begin{pmatrix}1 & 1 & b + e + 1 \\ 0 & 1 & 1 \\ 0 & 0 & 1\end{pmatrix}$ and \newline
          $YX = \begin{pmatrix}1 & 0 & e \\ 0 & 1 & 1 \\ 0 & 0 & 1\end{pmatrix}\begin{pmatrix}1 & 1 & b \\ 0 & 1 & 0 \\ 0 & 0 & 1\end{pmatrix} = \begin{pmatrix}1 & 1 & b + e \\ 0 & 1 & 1 \\ 0 & 0 & 1\end{pmatrix}$.

          For no $b, e \in F$ is it possible for $b + e$ to equal $b + e + 1$. Thus $H(F)$ is non-abelian.

    \item Find an explicit formula for the matrix inverse $X^{-1}$ and deduce that $H(F)$ is closed under inverses.

          Suppose $D = \begin{pmatrix}d_{11} & d_{12} & d_{13} \\ d_{21} & d_{22} & d_{23} \\ d_{31} & d_{32} & d_{33}\end{pmatrix}$ such that $XD = I_3$. Carrying out the multiplication, we obtain the following set of equations:
          
          \begin{itemize}
            \item $d_{11} + ad_{21} + bd_{31} = 1$
            \item $d_{12} + ad_{22} + bd_{32} = 0$
            \item $d_{13} + ad_{23} + bd_{33} = 0$
            \item $d_{21} + cd_{31} = 0$
            \item $d_{22} + cd_{32} = 1$
            \item $d_{23} + cd_{33} = 1$
            \item $d_{31} = 0$
            \item $d_{32} = 0$
            \item $d_{33} = 1$
          \end{itemize}

          Substituting $d_{31} = d_{32} = 0, d_{33} = 1$ into the first equations, we obtain the following:

          \begin{itemize}
            \item $d_{11} + ad_{21} = 1$
            \item $d_{12} + ad_{22} = 0$
            \item $d_{13} + ad_{23} + b = 0$
            \item $d_{21} = 0$
            \item $d_{22} = 1$
            \item $d_{23} + c = 1$
          \end{itemize}

          So we have $d_{23} = -c$, and substituting that, as well as $d_{21} = 0, d_{22} = 1$ into the above, we obtain:

          \begin{itemize}
            \item $d_{11} = 1$
            \item $d_{12} + a = 0$
            \item $d_{13} - ac + b = 0$
          \end{itemize}

          Then $D = \begin{pmatrix}1 & -a & ac - b \\ 0 & 1 & -c \\ 0 & 0 & 1\end{pmatrix}$. Since $XD = I_3, D = X^{-1}$, and we see that $H(F)$ is closed under inverses.
    
    \item Prove the associative law for $H(F)$ and deduce that $H(F)$ is a group of order $|F|^3$.

          Since $H(F)$ is closed under matrix multiplication, we can ignore all but the upper-triangular entries (upper-middle, upper-right, and middle-right), since they will always be either 0 or 1.
          
          From a), the upper-middle and middle-right entries of $XY$ are $a + d$ and $c + f$, respectively. If we multiply their product by a third matrix, these entries will be the sum of the respective entries of the three entries, which is associative.

          The upper-right entry of $XY$ is $b + e + af$. Let $Z$ be a matrix with upper-triangular entries $g, h, i$. Then $(XY)Z$ has the upper-right entry $(b + e + af) + h + (a + d)i = b + e + h + af + ai + di$. The upper-right entry of $X(YZ)$ is $(e + h + di) + b + a(f + i) = b + e + h + af + ai + di$. Thus, $(XY)Z = X(YZ)$, and so $H(F)$ is associative.

          Since $H(F)$ is closed under matrix multiplication, inverses, and is associative, it is a group. In choosing elements of $H(F)$, we can freely choose from three elements $a, b, c \in F$. Therefore $|H(F)| = |F|^3$.

    \item Find the order of each element of the finite group $H(\mathbb{Z}/2\mathbb{Z})$.

          \begin{itemize}
            \item $\begin{pmatrix}1 & 0 & 0 \\ 0 & 1 & 0 \\ 0 & 0 & 1\end{pmatrix}$: 1
            \item $\begin{pmatrix}1 & 1 & 0 \\ 0 & 1 & 0 \\ 0 & 0 & 1\end{pmatrix}$: 2
            \item $\begin{pmatrix}1 & 0 & 1 \\ 0 & 1 & 0 \\ 0 & 0 & 1\end{pmatrix}$: 2
            \item $\begin{pmatrix}1 & 0 & 0 \\ 0 & 1 & 1 \\ 0 & 0 & 1\end{pmatrix}$: 2
            \item $\begin{pmatrix}1 & 1 & 1 \\ 0 & 1 & 0 \\ 0 & 0 & 1\end{pmatrix}$: 2
            \item $\begin{pmatrix}1 & 1 & 0 \\ 0 & 1 & 1 \\ 0 & 0 & 1\end{pmatrix}$: 3
            \item $\begin{pmatrix}1 & 0 & 1 \\ 0 & 1 & 1 \\ 0 & 0 & 1\end{pmatrix}$: 2
            \item $\begin{pmatrix}1 & 1 & 1 \\ 0 & 1 & 1 \\ 0 & 0 & 1\end{pmatrix}$: 3
          \end{itemize}

    \item Prove that every nonidentity element of the group $H(\mathbb{R})$ has infinite order.

          \begin{proof}
            Let $X = \begin{pmatrix}1 & a & b \\ 0 & 1 & c \\ 0 & 0 & 1\end{pmatrix} \in H(\mathbb{R})$. We will show by induction that the upper-middle, upper-right, and middle-right entries of $X^n$ are $na, \newline (n(n - 1)/2)ac + nb$, and $nc$, respectively.

            For the base case, $X^1 = X$ has the upper-middle, upper-right, and middle-right entries $a = 1 \cdot a, b = 0 \cdot ac + 1 \cdot b, c = 1 \cdot c$.

            For the induction step, suppose for $X^n$, the relevant entries are $na, (n(n - 1)/2)ac + nb$, and $nc$. Then
            
            $X^{n + 1} = X^n X = \begin{pmatrix}1 & na & (n(n - 1)/2)ac + nb \\ 0 & 1 & nc \\ 0 & 0 & 1\end{pmatrix}\begin{pmatrix}1 & a & b \\ 0 & 1 & c \\ 0 & 0 & 1\end{pmatrix} = \\ \begin{pmatrix}1 & (n + 1)a & b + nac + (n(n - 1)/2)ac + nb \\ 0 & 1 & (n + 1)c \\ 0 & 0 & 1\end{pmatrix}$. The upper-middle and middle-right entries satisfy the induction hypothesis, and the upper-right entry is:
            \begin{multline*}
                b + nac + (n(n - 1)/2)ac + nb = (n + n(n - 1)/2)ac + (n + 1)b = \\
                ((2n + n^2 - n)/2)ac + (n + 1)b = ((n^2 + n)/2)ac + (n + 1)b = \\
                (n(n + 1)/2)ac + (n + 1)b.
            \end{multline*}
            Thus, $X^{n + 1}$ satisfies the induction hypothesis.

            Now if $|X| < \infty$, then for some $n$, $X^n = I_3$. So we need:
            
            $na = (n(n - 1)/2)ac + nb = nc = 0$. For $na$ and $nc$, since $n > 0$, we need $a = c = 0$. Then $nb = 0$, so $b = 0$. Therefore the identity matrix is the only element of $H(\mathbb{R})$ with finite order.
          \end{proof}
\end{enumerate}

\end{document}