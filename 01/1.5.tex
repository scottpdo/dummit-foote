\documentclass{article}

\title{Dummit \& Foote Ch. 1.5: The Quaternion Group}
\author{Scott Donaldson}
\date{Mar. 2023}
\usepackage{amsmath, amsthm, amsfonts, enumitem}

\begin{document}

\maketitle

\section*{1. (3/24/23)}

Compute the order of each of the elements of $Q_8$.

\begin{center}
    \begin{tabular}{ |c|c|c|c|c|c|c|c|c| } 
        \hline
        Element & 1 & $-1$ & $i$ & $-i$ & $j$ & $-j$ & $k$ & $-k$ \\
        \hline
        Order & 1 & 2 & 4 & 4 & 4 & 4 & 4 & 4 \\
        \hline
    \end{tabular}
\end{center}

\section*{2. (3/24/23)}

Write out the group tables for $S_3, D_8$ and $Q_8$.

\begin{center}
    \begin{tabular}{ |c|c|c|c|c|c|c| } 
        \hline
        1 & $(1,2)$ & $(1,3)$ & $(2,3)$ & $(1,2,3)$ & $(1,3,2)$ \\
        \hline
        $(1,2)$ & 1 & $(1,3,2)$ & $(1,2,3)$ & $(2,3)$ & $(1,3)$ \\
        \hline
        $(1,3)$ & $(1,2,3)$ & 1 & $(1,3,2)$ & $(1,2)$ & $(2,3)$ \\
        \hline
        $(2,3)$ & $(1,3,2)$ & $(1,2,3)$ & 1 & $(1,3)$ & $(1,2)$ \\
        \hline
        $(1,2,3)$ & $(1,3)$ & $(2,3)$ & $(1,2)$ & $(1,3,2)$ & 1 \\
        \hline
        $(1,3,2)$ & $(2,3)$ & $(1,2)$ & $(1,3)$ & 1 & $(1,2,3)$ \\
        \hline
    \end{tabular}

    \vspace{1em}

    \begin{tabular}{ |c|c|c|c|c|c|c|c|c| } 
        \hline
        1 & $r$ & $r^2$ & $r^3$ & $s$ & $sr$ & $sr^2$ & $sr^3$ \\
        \hline
        $r$ & $r^2$ & $r^3$ & 1 & $sr^3$ & $s$ & $sr$ & $sr^2$ \\
        \hline
        $r^2$ & $r^3$ & 1 & $r$ & $sr^2$ & $sr^3$ & $s$ & $sr$ \\
        \hline
        $r^3$ & 1 & $r$ & $r^2$ & $sr$ & $sr^2$ & $sr^3$ & $s$ \\
        \hline
        $s$ & $sr$ & $sr^2$ & $sr^3$ & 1 & $r$ & $r^2$ & $r^3$ \\
        \hline
        $sr$ & $sr^2$ & $sr^3$ & $s$ & $r^3$ & 1 & $r$ & $r^2$ \\
        \hline
        $sr^2$ & $sr^3$ & $s$ & $sr$ & $r^2$ & $r^3$ & 1 & $r$ \\
        \hline
        $sr^3$ & $s$ & $sr$ & $sr^2$ & $r$ & $r^2$ & $r^3$ & 1 \\
        \hline
    \end{tabular}    \vspace{1em}

    \begin{tabular}{ |c|c|c|c|c|c|c|c|c| } 
        \hline
        1 & $-1$ & $i$ & $-i$ & $j$ & $-j$ & $k$ & $-k$ \\
        \hline
        $-1$ & 1 & $-i$ & $i$ & $-j$ & $j$ & $-k$ & $k$ \\
        \hline
        $i$ & $-i$ & $-1$ & 1 & $k$ & $-k$ & $-j$ & $j$ \\
        \hline
        $-i$ & $i$ & 1 & $-1$ & $-k$ & $k$ & $j$ & $-j$ \\
        \hline
        $j$ & $-j$ & $-k$ & $k$ & $-1$ & 1 & $i$ & $-i$ \\
        \hline
        $-j$ & $j$ & $k$ & $-k$ & 1 & $-1$ & $-i$ & $i$ \\
        \hline
        $k$ & $-k$ & $j$ & $-j$ & $-i$ & $i$ & $-1$ & 1 \\
        \hline
        $-k$ & $k$ & $-j$ & $j$ & $i$ & $-i$ & 1 & $-1$ \\
        \hline
    \end{tabular}
\end{center}

\section*{3. (3/24/23)}

Find a set of generators and relations for $Q_8$.

\begin{proof}
    $Q_8 = \{ 1, -1, i, -i, j, -j, k, -k \}$.

    Consider the presentation $\langle a, b, c \mid b^2 = c^2 = a, a^2 = 1, cb = abc \rangle$. If we replace the given generators with $a = -1, b = i, c = j$ (which satisfy the given relations), then we can show that every element of $Q_8$ is indeed generated: $1 = a^2, -1 = a, i = b, -i = ab, j = c, -j = ac, k = bc, -k = abc$. Thus, the presentation generates $Q_8$.

    However, it remains to be shown that these generators and relations are not the presentation for some larger group that contains $Q_8$ as a subgroup. Consider an arbitrary product of $a, b, c$. $a$ commutes with $b$:
    \begin{equation*}
        b^2 = a \Rightarrow ab^2 = b^2 a = 1 \Rightarrow (ab)b = b(ba) = 1 \Rightarrow b = (ab)^{-1} = (ba)^{-1} \Rightarrow ab = ba
    \end{equation*}
    and the same logic shows that $a$ and $c$ also commute. Therefore we can rewrite an arbitrary product of $a, b, c$ so that all the $a$ factors are at the start (ex. $b^{n_1} a^{n_2} c^{n_3} a^{n_4} = a^{n_2 + n_4} b^{n_1} c^{n_3}$). Further, because $a^2 = 1$, this initial $a^n$ can be reduced to either $a$ (if $n$ odd) or removed (if $n$ even).

    So any unique element generated must have the form $a^{p}b^{n_1}c^{n_2}...b^{n_k}c^{n_{k + 1}}$, $p \in \{0, 1\}$. Next, we can rewrite any arbitrary product of factors $b$ and $c$ by replacing any $cb$ with $abc$ and reducing so that the element has the form $a^{p}b^{q}c^{r}$. Further, while we already have $p \in \{0, 1\}$, we must also have $q, r \in \{0, 1\}$, because $b^2 = c^2 = a$. So the only elements that can be generated are $1, a, b, c, ab, ac, bc, abc$. By replacing $a$ with $-1$, $b$ with $i$, and $c$ with $j$, we have $1, -1, i, -i, j, -j, ij, -ij$, and if we let $k = ij$, then we have precisely $Q_8$.
\end{proof}

\end{document}