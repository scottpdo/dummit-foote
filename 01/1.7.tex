\documentclass{article}

\title{Dummit \& Foote Ch. 1.7: Group Actions}
\author{Scott Donaldson}
\date{Apr. - May 2023}
\usepackage{amsmath, amsthm, amsfonts, enumitem}

\begin{document}

\maketitle

\section*{1. (4/27/23)}

Let $F$ be a field. Show that the multiplicative group of nonzero elements of $F$ (denoted by $F^\times$) acts on the set $F$ by $g \cdot a = ga$, where $g \in F^\times, a \in F$ and $ga$ is the usual product in $F$ of the two field elements.

\begin{proof}
    To show that $F^\times$ acts on $F$, we must show that $g_1 \cdot (g_2 \cdot a) = (g_1 g_2) \cdot a$ for all $g_1, g_2 \in F^\times, a \in F$, and $1 \cdot a = a$ for all $a \in F$.

    First, let $g_1, g_2 \in F^\times$ and $a \in F$. By the definition of the action, $g_1 \cdot (g_2 \cdot a) = g_1 \cdot (g_2 a) = g_1 g_2 a$. By the associativity of multiplication, $g_1 g_2 a = (g_1 g_2) a$. Again by the action definition, this equals $(g_1 g_2) \cdot a$.
    
    It follows directly from the field axiom of multiplicative identity that $1 \cdot a = a$ for all $a \in A$. Thus $F^\times$ acts on $F$ by $g \cdot a = ga$.
\end{proof}

\section*{2. (4/27/23)}

Show that the additive group $\mathbb{Z}$ acts on itself by $z \cdot a = z + a$ for all $z, a \in \mathbb{Z}$.

\begin{proof}
    First, $z_1 \cdot (z_2 \cdot a) = z_1 \cdot (z_2 + a) = z_1 + z_2 + a = (z_1 + z_2) + a = (z_1 + z_2) \cdot a$.

    Also, $0 \cdot a = 0 + a = a$ for all $a \in \mathbb{Z}$. Thus $\mathbb{Z}$ acts on itself by $z \cdot a = z + a$.
\end{proof}

\section*{3. (4/27/23)}

Show that the additive group $\mathbb{R}$ acts on the $x, y$ plane $\mathbb{R} \times \mathbb{R}$ by $r \cdot (x, y) = (x + ry, y)$.

\begin{proof}
    First, $r_1 \cdot (r_2 \cdot (x, y)) = r_1 \cdot (x + r_2 y, y) = (x + r_2 y + r_1 y, y) = \\ (x + (r_1 + r_2)y, y) = (r_1 + r_2) \cdot (x, y)$.

    Also, $0 \cdot (x, y) = (x + 0y, y) = (x, y)$ for all $(x, y) \in \mathbb{R} \times \mathbb{R}$. Thus $\mathbb{R}$ acts on $\mathbb{R} \times \mathbb{R}$ by $r \cdot (x, y) = (x + ry, y)$.
\end{proof}

\section*{4. (4/27/23)}

Let $G$ be a group acting on a set $A$ and fix some $a \in A$. Show that the following sets are subgroups of $G$:

\begin{enumerate}[label=(\alph*)]
    \item the kernel of the action,
          \begin{proof}
            The kernel of $G$ is the set $\{ g \in G \mid g \cdot a = a \text{ for all } a \in A \}$. It is closed under the binary operation of $G$: If $g_1, g_2$ are in the kernel, then $g_1 \cdot (g_2 \cdot a) = g_1 \cdot a = a$ for all $a \in A$. And, by definition of a group action, $g_1 \cdot (g_2 \cdot a) = (g_1 g_2) \cdot a$, which implies that $(g_1 g_2) \cdot a = a$, so $g_1 g_2$ is in the kernel of $G$.

            The kernel is also closed under inverses: Let $g$ be in the kernel of $G$. Then $1 \cdot a = (g^{-1} g) \cdot a = g^{-1} \cdot (g \cdot a) = g^{-1} \cdot a$. By definition, $1 \cdot a = a$, so $g^{-1} \cdot a = a$ for all $a$, so $g^{-1}$ is in the kernel. Thus the kernel of the action is a subgroup of $G$.
          \end{proof}
    \item $\{ g \in G \mid ga = a \}$ — this subgroup is called the \emph{stabilizer} of $G$.
          \begin{proof}
            The proof that this set of elements if a subgroup is identical to the one immediately above, but for a fixed $a$ as opposed to all $a \in A$.
          \end{proof}
\end{enumerate}

\section*{5. (4/28/23)}

Prove that the kernel of an action of the group $G$ on the set $A$ is the same as the kernel of the corresponding permutation representation $G \rightarrow S_A$.

\begin{proof}
    Let $\varphi$ be the permutation representation $G \rightarrow S_A$ corresponding to $G$ acting on $A$. Let $g$ be in the kernel of the action of $G$ (to show that $\varphi(g)$ is in the kernel of $\varphi$). Then $g \cdot a = a$ for all $a \in A$. If $\sigma_g$ is the permutation of $S_A$ corresponding to $g$, then $\sigma_g$ is the identity permutation, because $\sigma_g(a) = a$ for all $a \in A$. Thus $\sigma_g = \varphi(g)$ is in the kernel of $\varphi$.

    Next, let $\varphi(g)$ be in the kernel of $\varphi$ (to show that $g$ is in the kernel of $G$). Then $\varphi(g)$ is the identity permutation, so $\varphi(g) \cdot a = \sigma_g(a) = a$ for all $a \in A$. Also, by definition, $\sigma_g(a) = g \cdot a$, so $g \cdot a = a$ for all $a \in A$. Thus $g$ is in the kernel of the action of $G$.

    Having shown that membership in one implies membership in the other, this proves that the kernel of $G$ acting on $A$ is thus equal to the kernel of the permutation representation $\varphi: G \rightarrow S_A$.
\end{proof}

\section*{6. (4/28/23)}

Prove that a group $G$ acts faithfully on a set $A$ if and only if the kernel of the action is the set consisting of only the identity.

\begin{proof}
    First, let $G$ act on $A$. Suppose that $G$ acts on $A$ faithfully (to show that the kernel of the action of $G$ is the set consisting of only the identity). Consider the permutation representation $\varphi: G \rightarrow S_A$. Since $G$ acts on $A$ faithfully, $\varphi$ is injective (that is, $g_1, g_2 \in G$ induce different permutations $\varphi(g_1), \varphi(g_2)$). Thus the identity permutation $\varphi(1)$ is the only permutation that assigns $a$ to $a$ for all $a \in A$. From 5., the kernel of the action of $G$ is the same as the kernel of $\varphi$, so the identity of $G$ is the only element in the kernel of the action of $G$.

    Next, suppose that the kernel of the action of $G = \{ 1 \}$ (to show that $G$ acts on $A$ faithfully). Suppose for some $g_1, g_2 \in G$, we have $\varphi(g_1) = \varphi(g_2)$, that is, $\sigma_{g_1}(a) = \sigma_{g_2}(a)$ for all $a \in A$. Consider the permutation obtained by composing $\varphi(g_1)^{-1} \circ \varphi(g_2)$. Applying the resulting permutation to some $a \in A$ (and saying that $\sigma_{g_1}(a) = \sigma_{g_2}(a) = b$), we obtain $(\varphi(g_1)^{-1} \circ \varphi(g_2))(a) = \sigma_{g_1}^{-1}(\sigma_{g_2}(a)) = \sigma_{g_1}^{-1}(b) = a$. This implies that $\varphi(g_1)^{-1} \circ \varphi(g_2)$ is the identity permutation. Since $\varphi$ is a homomorphism, $\varphi(g_1)^{-1} \circ \varphi(g_2) = \varphi(g_1^{-1}) \circ \varphi(g_2) = \varphi(g_1^{-1} g_2)$. However, because the kernel of the action of $G$ is $\{ 1 \}$, and from 5., the kernel of $\varphi$ is also $\{ 1 \}$, this implies that $g_1^{-1} g_2 = 1 \Rightarrow g_1 = g_2$.
\end{proof}

\section*{7. (4/29/23)}

Prove that the action of the multiplicative group $\mathbb{R}^\times$ on $\mathbb{R}^n$ defined by \\ $\alpha \cdot (r_1, r_2, ..., r_n) = (\alpha r_1, \alpha r_2, ..., \alpha r_n)$ is faithful.

\begin{proof}
    From 6., a group acts faithfully on a set if and only if the kernel of the action consists only of the group's identity. Therefore, to show that the given action of $\mathbb{R}^\times$ on $\mathbb{R}^n$ is faithful, it suffices to show that the kernel of the action is $\{ 1 \}$.

    By definition, the kernel of the action is the set of all $\alpha \in \mathbb{R}$ such that $\alpha \cdot (r_1, r_2, ..., r_n) = (r_1, r_2, ..., r_n)$ for all such elements of $\mathbb{R}^n$. By definition of the group action, then, for an element $\alpha$ of $\mathbb{R}^\times$ to be in the kernel of the action, we must have $\alpha r_1 = r_1, \alpha r_2 = r_2, ..., \alpha r_n = r_n$. The only element for which this holds is $1$. Thus the kernel of the action is $\{ 1 \}$, and so $\mathbb{R}^\times$ acts faithfully on $\mathbb{R}^n$.
\end{proof}

\section*{8. (4/30/23)}

Let $A$ be a nonempty set and let $k$ be a positive integer with $k \leq |A|$. The symmetric group $S_A$ acts on $B$ consisting of all subsets of $A$ of cardinality $k$ by $\sigma \cdot \{ a_1, ..., a_k \} = \{ \sigma(a_1), ..., \sigma(a_k) \}$.

\begin{enumerate}[label=(\alph*)]
    \item Prove that this is a group action.
          \begin{proof}
            The identity permutation acts on an arbitrary element of $B$ by $(1) \cdot \{ a_1, ..., a_k \} = \{ a_1, ..., a_k \}$, as desired.

            Further, $\sigma_1 \cdot (\sigma_2 \cdot \{ a_1, ..., a_k \}) = \sigma_1 \cdot \{ \sigma_2(a_1), ..., \sigma_2(a_k) \} = \{ \sigma_1(\sigma_2(a_1)), ..., \\ \sigma_1(\sigma_2(a_k)) \} = \{ (\sigma_1 \circ \sigma_2)(a_1), ..., (\sigma_1 \circ \sigma_2)(a_k) \} = (\sigma_1 \circ \sigma_2) \cdot \{ a_1, ..., a_k \}$. Together these two equations prove that this action of $S_A$ on $B$ is a group action.
          \end{proof}
    \item Describe exactly how the permutations $(1, 2)$ and $(1, 2, 3)$ act on the six 2-element subsets of $\{1, 2, 3, 4\}$.
          \begin{itemize}
            \item $(1, 2) \cdot \{1, 2\} = \{2, 1\} = \{1, 2\}$
            \item $(1, 2) \cdot \{1, 3\} = \{2, 3\}$
            \item $(1, 2) \cdot \{1, 4\} = \{2, 4\}$
            \item $(1, 2) \cdot \{2, 3\} = \{1, 3\}$
            \item $(1, 2) \cdot \{2, 4\} = \{1, 4\}$
            \item $(1, 2) \cdot \{3, 4\} = \{3, 4\}$
            \item $(1, 2, 3) \cdot \{1, 2\} = \{2, 3\}$
            \item $(1, 2, 3) \cdot \{1, 3\} = \{2, 1\} = \{1, 2\}$
            \item $(1, 2, 3) \cdot \{1, 4\} = \{2, 4\}$
            \item $(1, 2, 3) \cdot \{2, 3\} = \{3, 1\} = \{1, 3\}$
            \item $(1, 2, 3) \cdot \{2, 4\} = \{3, 4\}$
            \item $(1, 2, 3) \cdot \{3, 4\} = \{1, 4\}$
          \end{itemize}
\end{enumerate}

\section*{9. (4/30/23)}

Do both parts of the preceding exercise with "ordered $k$-tuples" in place of "$k$-element subsets," where the action on $k$-tuples is defined as above but with set braces replaced by parentheses (note that, for example, the 2-tuples $(1, 2)$ and $(2, 1)$ are different even though the sets $\{1, 2\}$ and $\{2, 1\}$ are the same).

\begin{enumerate}[label=(\alph*)]
    \item The proof is identical to that in 8., but with set braces replaced by parentheses. For the identity permutation, $(1) \cdot ( a_1, ..., a_k ) = ( a_1, ..., a_k )$. Similarly for arbitrary $\sigma_1, \sigma_2$ and $( a_1, ..., a_k )$, the logic holds.
    \item Describe exactly how the permutations $(1, 2)$ and $(1, 2, 3)$ act on the twelve 2-element tuples of $(1, 2, 3, 4)$.
          \begin{itemize}
            \item $(1, 2) \cdot (1, 2) = (2, 1); (1, 2) \cdot (2, 1) = (1, 2)$
            \item $(1, 2) \cdot (1, 3) = (2, 3); (1, 2) \cdot (3, 1) = (3, 2)$
            \item $(1, 2) \cdot (1, 4) = (2, 4); (1, 2) \cdot (4, 1) = (4, 2)$
            \item $(1, 2) \cdot (2, 3) = (1, 3); (1, 2) \cdot (3, 2) = (3, 1)$
            \item $(1, 2) \cdot (2, 4) = (1, 4); (1, 2) \cdot (4, 2) = (4, 1)$
            \item $(1, 2) \cdot (3, 4) = (3, 4); (1, 2) \cdot (4, 3) = (4, 3)$
            \item $(1, 2, 3) \cdot (1, 2) = (2, 3); (1, 2, 3) \cdot (2, 1) = (3, 2)$
            \item $(1, 2, 3) \cdot (1, 3) = (2, 1); (1, 2, 3) \cdot (3, 1) = (1, 2)$
            \item $(1, 2, 3) \cdot (1, 4) = (2, 4); (1, 2, 3) \cdot (4, 1) = (4, 2)$
            \item $(1, 2, 3) \cdot (2, 3) = (3, 1); (1, 2, 3) \cdot (3, 2) = (1, 3)$
            \item $(1, 2, 3) \cdot (2, 4) = (3, 4); (1, 2, 3) \cdot (4, 2) = (4, 3)$
            \item $(1, 2, 3) \cdot (3, 4) = (1, 4); (1, 2, 3) \cdot (4, 3) = (4, 1)$
          \end{itemize}
\end{enumerate}

\section*{10. (5/4/23)}

With reference to the two preceding exercises determine:

\begin{enumerate}[label=(\alph*)]
  \item for which values of $k$ the action of $S_n$ on $k$-element subsets is faithful, and
  \item for which values of $k$ the action of $S_n$ on ordered $k$-tuples is faithful.
\end{enumerate}

For the action of $S_n$ on $k$-element subsets, the action is faithful if $n > 1$ and $k < n$.
\begin{proof}
  In the case where $n = 1$, then the action is trivially faithful (because the symmetric group $S_n$ consists only of the identity).
  
  So suppose that $n > 1$ and let $k < n$, with $B$ the set of all $k$-element subsets of $A = \{ 1, 2, ..., n \}$. Let $\sigma \in S_n$ be a non-identity permutation. Then $\sigma$ assigns at least one element of $A$ to a different element of $A$. Suppose that $\sigma(a_1) = a_2$ for some $a_1, a_2 \in A$. Because $k < n$, there exists a subset $b \in B$ such that $a_1 \in b$ and $a_2 \notin b$. Then $\sigma \cdot b = \{ \sigma(a_1), ... \} = \{ a_2, ... \} \neq b$, and so $\sigma$ is not in the kernel of the action. Therefore the kernel of the action consists only of the identity permutation, and so the action is faithful.

  Now, let $n > 1$ and let $k = n$. Then $B$, the set of all $k$-element subsets of $A = \{ 1, 2, ..., n \}$, consists only of $A$ itself. Now let $\sigma \in S_n$ and let $a_1, a_2 \in A$ with $\sigma(a_1) = a_2$. For all $b \in B$ (because $b = A$), $a_1, a_2 \in b \Rightarrow \sigma(a_2) \in b$. Therefore $\sigma \cdot b = b$ for all $b \in B$. Thus every permutation of $S_n$ is in the kernel of the action, and so the action is not faithful.

  This proves that the action of $S_n$ on $k$-element subsets is faithful if and only if $n > 1$ and $k < n$.
\end{proof}

For the action of $S_n$ on ordered $k$-tuples, the action is faithful for all values of $k$ (if $n > 1$).

\begin{proof}
  As above, the action is trivially faithful if $n = 1$, so suppose that $n > 1$, let $\sigma$ be a non-identity permutation in $S_n$, and let $1 \leq k \leq n$, such that $B$ is the set of all $k$-element tuples of $A = \{ 1, 2, ..., n \}$ (ex. $(1,2)$ and $(2,1) \in B$). Let $a_1 \in A$ and let $a_2 = \sigma(a_1)$. Let $b$ be the $k$-tuple consisting only of $a_1$, that is, $\underbrace{(a_1, ..., a_1)}_{\text{$k$ times}}$. Then $\sigma \cdot b = \sigma \cdot (a_1, ..., a_1) = (\sigma(a_1), ..., \sigma(a_1)) = (a_2, ..., a_2)$. Then for all non-identity $\sigma \in S_n$, there exists a $b \in B$ such that $\sigma \cdot b \neq b$. Therefore the only permutation in the kernel of the action is the identity permutation, and so the action is faithful for all values of $k$.
\end{proof}

\section*{11. (5/4/23)}

Write out the cycle decomposition of the eight permutations in $S_4$ corresponding to the elements of $D_8$ given by the action of $D_8$ on the vertices of a square.

\begin{itemize}
  \item $1: (1)$
  \item $r: (1, 2, 3, 4)$
  \item $r^2: (1, 3)(2, 4)$
  \item $r^3: (1, 4, 3, 2)$
  \item $s: (2, 4)$
  \item $sr: (1, 4)(2, 3)$
  \item $sr^2: (1, 3)$
  \item $sr^3: (1, 2)(3, 4)$
\end{itemize}

\section*{12. (5/5/23)}

Assume $n$ is an even positive integer and show that $D_{2n}$ acts on the set consisting of pairs of opposite vertices of a regular $n$-gon. Find the kernel of this action.

\begin{proof}
  Let $A$ be the set of pairs of opposite vertices of a regular $n$-gon:
  \begin{equation*}
    \bigl\{ \{1,\frac{n}{2} + 1\}, \{2, \frac{n}{2} + 2\}, ..., \{\frac{n}{2} - 1, n - 1\} \bigr\}.
  \end{equation*}
  We will show that the following is an action of $D_{2n}$ on the element $\{k, \frac{n}{2} + k\} \in A, 1 \leq k < \frac{n}{2}$ defined on the generators of $D_{2n}$: 
  \begin{itemize}
    \item $s \cdot \{ k, \frac{n}{2} + k \} = \{ n - k + 1, \frac{n}{2} - k + 1 \}$, and
    \item $r \cdot \{ k, \frac{n}{2} + k \} = \{ k + 1, \frac{n}{2} + k + 1 \}$, where all values are taken mod $n$.
  \end{itemize}

  In order to prove that this is a group action, we will show that the relations of $D_{2n}$ hold when acting on elements of $A$, that is, for all $a \in A$, we have $a = 1 \cdot a = (s^2) \cdot a = s \cdot s \cdot a$, that $a = 1 \cdot a = (r^n) \cdot a = \underbrace{r \cdot ... \cdot r}_\text{$n$ times} \cdot a$, and finally, that $s \cdot r \cdot a = r^{-1} \cdot s \cdot a$.

  First, $s \cdot s \cdot \{ k, \frac{n}{2} + k \} = s \cdot \{ n - k + 1, \frac{n}{2} - k + 1 \}$ by definition. In turn, this equals $\{ n - (n - k + 1) + 1, \frac{n}{2} - (n - k + 1) + 1 \} = \{ k, -\frac{n}{2} + k \}$. Since all values are taken mod $n$, $-\frac{n}{2} + k = \frac{n}{2} + k$, and so $s \cdot s \cdot \{ k, \frac{n}{2} + k \} = \{ k, \frac{n}{2} + k \}$. Therefore $s \cdot s \cdot a = a$ for all $a \in A$.

  Next, to show that $\underbrace{r \cdot ... \cdot r}_\text{$n$ times} \cdot a = a$, we will first prove by induction that $\underbrace{r \cdot ... \cdot r}_\text{$m$ times} \cdot \{ k, \frac{n}{2} + k \} = \{ k + m, \frac{n}{2} + m \}, m \geq 0$. The base case $r \cdot \{ k, \frac{n}{2} + k \} = \{ k + 1, \frac{n}{2} + k + 1 \}$ holds by definition. So suppose for some $m$, $\underbrace{r \cdot ... \cdot r}_\text{$m$ times} \cdot \{ k, \frac{n}{2} + k \} = \{ k + m, \frac{n}{2} + m \}$ (mod $n$). Then: 
  \begin{multline*}
    \underbrace{r \cdot ... \cdot r}_\text{$m + 1$ times} \cdot \{ k, \frac{n}{2} + k \} = r \cdot \underbrace{r \cdot ... \cdot r}_\text{$m$ times} \cdot \{ k, \frac{n}{2} + k \} = r \cdot \{ k + m, \frac{n}{2} + k + m \} = \\
    \{ k + (m + 1), \frac{n}{2} + k + (m + 1) \}.
  \end{multline*}
  Thus the induction case holds, and so:
  \begin{multline*}
    \underbrace{r \cdot ... \cdot r}_\text{$n$ times} \cdot a = \underbrace{r \cdot ... \cdot r}_\text{$n$ times} \cdot\{ k, \frac{n}{2} + k \} = \\ \{ k + n, \frac{n}{2} + k + n \} = \{ k, \frac{n}{2} + k \} = a \text{ for all } a \in A.
  \end{multline*}

  Finally, to show that $s \cdot r \cdot a = r^{-1} \cdot s \cdot a$, we first note that $r^{-1} \cdot a = \{ k - 1, \\ \frac{n}{2} + k - 1 \}$. Now:
  \begin{align*}
    s \cdot r \cdot \{ k, \frac{n}{2} + k \} &= & r^{-1} \cdot s \cdot \{ k, \frac{n}{2} + k \} &= \\
    s \cdot \{ k + 1, \frac{n}{2} + k + 1 \} &= & r^{-1} \cdot \{ n - k + 1, \frac{n}{2} - k + 1 \} &= \\
    \{ n - (k + 1) + 1, \frac{n}{2} - (k + 1) + 1 \} &= & \{ n - k, \frac{n}{2} - k \}. \\
    \{ n - k, \frac{n}{2} - k \}, \text{ and }
  \end{align*}
  Therefore $s \cdot r \cdot a = r^{-1} \cdot s \cdot a$ for all $a \in A$. Together, these relations show that the above is a group action.

  We will now consider the kernel of this action. This consists of elements $s^{\{0, 1\}}r^m$ of $D_{2n}$ such that $s^{\{0, 1\}}r^m \cdot a = a$ for all $a \in A$. We will consider the two cases $r^m$ and $sr^m$ separately.
  \begin{itemize}
    \item $r^m$: From above, $r^m \cdot \{ k, \frac{n}{2} + k \} = \{ k + m, \frac{n}{2} + k + m \}$ for all $m \geq 0$. Clearly $m = 0 \Rightarrow r = 1$ satisfies this equality. Since values are taken mod $n$ and these are sets, not tuples, also note that $k = \frac{n}{2} + k + m \Rightarrow 0 = \frac{n}{2} + m \Rightarrow m = \frac{n}{2}$. So among elements of the form $r^m$, only 1 and $r^{n/2}$ are in the kernel of the action.
    \item $sr^m$: From above, $sr^m \cdot \{ k, \frac{n}{2} + k \} = s \cdot \{ k + m, \frac{n}{2} + k + m \} = \\ \{ n - (k + m) + 1, \frac{n}{2} - (k + m) + 1 \}$. Considering the first elements of each set together, we have $k = n - (k + m) + 1 \Rightarrow 2k = n - m + 1 \Rightarrow m = 1 - 2k$. Since $k$ is variable, we cannot fix $m$, and so there is no value of $m$ for which $sr^m \cdot a = a$ for all $a \in A$.
  \end{itemize}
  Thus the kernel of this action is $\{ 1, r^{n/2} \}$.
\end{proof}

\section*{13. (5/5/23)}

Find the kernel of the left regular action.

\begin{proof}
  The left regular action of $G$ on itself is defined by $g \cdot a = ga$ for $g, a \in G$. The kernel of this action consists of all $g \in G$ such that $ga = a$ for all $a \in G$. Let $a \in G$ and suppose that $ga = a$ for some $g \in G$. By definition of the group identity, $1 \cdot a = a$, so $ga = 1 \cdot a$. We right-multiply both sides by $a^{-1}$ to obtain $g = 1$. Then the kernel of the left regular action is $\{ 1 \}$, and so the action is faithful.
\end{proof}

\section*{14. (5/5/23)}

Let $G$ be a group and let $A = G$. Show that if $G$ is non-abelian then the maps defined by $g \cdot a = ag$ for all $g, a \in G$ do \emph{not} satisfy the axioms of a (left) group action of $G$ on itself.

\begin{proof}
  Since $G$ is non-abelian, there exist $g_1, g_2 \in G$ such that $g_1 g_2 \neq g_2 g_1$. Then for all $a \in G$:
  \begin{equation*}
    g_1 \cdot g_2 \cdot a = g_1 \cdot (a g_2) = a g_2 g_1 \neq a g_1 g_2 = (g_1 g_2) \cdot a.
  \end{equation*}
  Thus this map is not a group action for non-abelian groups.
\end{proof}

\section*{15. (5/12/23)}

Let $G$ be any group and let $A = G$. Show that the maps defined by $g \cdot a \mapsto ag^{-1}$ for all $g, a \in G$ \emph{do} satisfy the axioms of a (left) group action of $G$ on itself.

\begin{proof}
  For the identity, $1 \cdot a = a (1^{-1}) = a$ for all $a \in G$.
  
  Then, for all $g_1, g_2 \in G$, $g_1 \cdot g_2 \cdot a = g_1 \cdot (a g_2^{-1}) = a g_2^{-1} g_1^{-1} = a (g_1 g_2)^{-1} = (g_1 g_2)^{-1} \cdot a$.

  Therefore this satisfies the axioms of a group action of $G$ on itself.
\end{proof}

\section*{16. (5/12/23)}

Let $G$ be any group and let $A = G$. Show that the maps defined by $g \cdot a \mapsto gag^{-1}$ for all $g, a \in G$ \emph{do} satisfy the axioms of a (left) group action of $G$ (this action of $G$ on itself is called \emph{conjugation}).

\begin{proof}
  As in 15., the requirement for identity is trivial, since $1 = 1^{-1}$.

  Then, for all $g_1, g_2 \in G$, we have:

  \begin{multline*}
    g_1 \cdot g_2 \cdot a = g_1 \cdot (g_2 a g_2^{-1}) = g_1 g_2 a g_2^{-1} g_1^{-1} = (g_1 g_2) a (g_1 g_2)^{-1} = (g_1 g_2) \cdot a,
  \end{multline*}

  as desired. Therefore conjugation satisfies the axioms of a group action of $G$ on itself.
\end{proof}

\section*{17. (5/12/23)}

Let $G$ be a group and let $G$ act on itself by left conjugation, so each $g \in G$ maps $G$ to $G$ by $x \mapsto gxg^{-1}$. For fixed $g \in G$, prove that conjugation by $g$ is an isomorphism from $G$ onto itself. Deduce that $x$ and $gxg^{-1}$ have the same order for all $x \in G$ and that for any subset $A$ of $G$, $|A| = |gAg^{-1}|$ (here $gAg^{-1} = \{ gag^{-1} \mid a \in A \}$).

\begin{proof}
  Let $g \in G$ and let $\varphi_g: G \rightarrow G$ be defined by $\varphi_g(x) = gxg^{-1}$. We will show that $\varphi_g$ is a homomorphism, is injective, and is surjective; it follows then that $\varphi_g$ is an automorphism of $G$.

  Let $x, y \in G$. Then $\varphi_g(x) \varphi_g(y) = gxg^{-1}gyg^{-1} = gxyg^{-1} = \varphi_g(xy)$. Thus $\varphi_g$ is a homomorphism.

  Next, to show that $\varphi_g$ is one-to-one, let $\varphi_g(x) = \varphi_g(y)$. Then $gxg^{-1} = gyg^{-1}$. Right-multiplying by $g^{-1}$ and then left-multiplying by $g$, we obtain $x = y$, so $\varphi_g$ is injective.

  Lastly, to show that $\varphi_g$ is onto, let $z \in G$. Let $x = g^{-1}zg$. Then $\varphi_g(x) = gxg^{-1} = gg^{-1}zgg^{-1} = z$, so $\varphi_g$ is surjective. Since it is a bijective homomorphism, $\varphi_g$ is thus an automorphism of $G$.

  Since $\varphi$ is an automorphism that uniquely maps $x$ to $gxg^{-1}$, we deduce that $|x| = |gx^{-1}|$ (from Ch. 1.6, Exercise 2.). It follows that, for any $A \subseteq G$, $gAg^{-1} \subseteq G$ contains as many elements (of equal order) as does $A$.
\end{proof}

\section*{18. (5/12/23)}

Let $H$ be a group acting on a set $A$. Prove that the relation $\sim$ on $A$ defined by
\begin{equation*}
  a \sim b \text{ if and only if } a = hb \text{ for some } h \in H
\end{equation*}
is an equivalence relation. (For each $x \in A$ the equivalence class of $x$ under $\sim$ is called the \emph{orbit} of $x$ under the action of $H$. The orbits under the action of $H$ partition the set $A$.)

\begin{proof}
  In order to show that $\sim$ is an equivalence relation on $A$, we must show that, for all $a, b, c \in A$, i) $a \sim a$; ii) $a \sim b \Rightarrow b \sim a$, and iii) $a \sim b, b \sim c \Rightarrow \\ a \sim c$.

  First, for all $a \in A$, $1_H \cdot a = a$ (by definition of a group action), so $a \sim a$.

  Next, suppose that $a \sim b$. Then $a = hb$ for some $h \in H$. This implies that $b = h^{-1} a$, and since $h^{-1}$, it follows that $b \sim a$ as well.

  Finally, suppose that $a \sim b$ and $b \sim c$. Then $a = h_1 b, b = h_2 c$ for some $h_1, h_2 \in H$. Then $a = h_1 h_2 c$, and since $H$ is closed under its group operation, the product $h_1 h_2 \in H$, so $a \sim c$.

  Thus $\sim$ is an equivalence relation on $A$.
\end{proof}

\section*{19. (5/15/23)}

Let $H$ be a subgroup of the finite group $G$ and let $H$ act on $G$ by left multiplication. Let $x \in G$ and let $\mathcal{O}$ be the orbit of $x$ under the action of $H$. Prove that the map $H \rightarrow \mathcal{O}$ defined by $h \mapsto hx$ is a bijection (hence that all orbits have cardinality $|H|$). From this and the preceding exercise deduce \emph{Lagrange's Theorem}:
\begin{center}
  \emph{if $G$ is a finite group and $H$ is a subgroup of $G$ then $|H|$ divides $|G|$.}
\end{center}

\begin{proof}
  To show that the map from the subgroup $H$ to the orbit $\mathcal{O}$ of $x \in G$ defined by left multiplication is a bijection, let $\varphi: H \rightarrow \mathcal{O}$ be defined by $\varphi(h) = hx$. We will show that $\varphi$ is injective and surjective.

  First, to show that $\varphi$ is one-to-one, let $\varphi(y_1) = \varphi(y_2)$ for $y_1, y_2 \in H$. Then $y_1 x = y_2 x$, which implies that $y_1 = y_2$, so $\varphi$ is injective.

  Next, to show that $\varphi$ is onto, let $z \in \mathcal{O}$ (to show that $z = \varphi(y)$ for some $y \in H$). Since $z$ is in the orbit of the given element $x$, $x = hz$ for some $h \in H$. Then $z = h^{-1} x$, which implies that $z = \varphi(h^{-1})$, and since $H$ is a subgroup of $G$ and is therefore closed under inverses, $\varphi$ is surjective.

  Since $\varphi$ is a bijection from the subgroup $H$ to $\mathcal{O},$ the orbit of $x$, the two have the same cardinality. From 18., the orbits of the elements of $G$ partition $G$; that is, the orbits of $x$ and $y$ under the action of $H$ are either disjoint or equal. Let $|H| = |\mathcal{O}| = n$. There are $|G| - n$ elements in $G$ not in the orbit of $x$ under the action of $H$. For another element $y \in G$, either the orbit of $y$ is equal to the orbit of $x$, or it contains $|H| = n$ elements, and in the case of the latter, there are $|G| - 2n$ elements in $G$ not in the orbits of $x$ and $y$. Continuing this process for each element of $G$, the orbit of each element must contain $n$ elements and we must eventually arrive at 0 remaining elements, so $|G| = kn$, that is, the order of the subgroup $H$ divides the order of $G$.
\end{proof}

\section*{20. (5/18/23)}

Show that the group of rigid motions of a tetrahedron is isomorphic to a subgroup of $S_4$.

\begin{proof}
  Let $G$ be the group of rigid motions of a tetrahedron. From Ch. 1.2, Ex. 9., this group has order 12. Order the vertices 1, 2, 3, and 4 and consider fixing vertex 1. There are two rotations (excluding the identity) that permute the remaining vertices 2, 3, and 4. The first sends 2 to 3, 3 to 4, and 4 to 2, and the second (the inverse of the first) sends 2 to 4, 4 to 3, and 3 to 2. For each of the four vertices, there are two such non-identity rotations, that is, eight such rotations. They can be written as cycle decompositions of permutations of $S_4$ (in fact, they are all the 3-cycles of $S_4$): $(1, 2, 3), (1, 3, 2), (1, 2, 4), (1, 4, 2), (1, 3, 4), (1, 4, 3), \\ (2, 3, 4), (2, 4, 3)$.

  There are also the rotations to consider that swap the positions of pairs of vertices. For example, one such rotation permutes 1 and 2 (and in doing so, also swaps 3 and 4). Similarly, these can be represented by the cycle decompositions made up of two disjoint 2-cycles: $(1, 2)(3, 4), (1, 3)(2, 4), \text{ and } (1, 4)(2, 3)$. Finally, there is the identity, bringing the total number of rigid motions to 12, as shown in Ch. 1.2, Ex. 9.

  Intuitively, this collection should be a subgroup, because each rigid motion has an inverse achieved by rotating it in the opposite direction, and there is no way to obtain a permutation of the vertices of a tetrahedron that is not already included in the given motions. However, to show that this collection as denoted by permutations in $S_4$ is a subgroup of $S_4$, we will show that it is closed under inverses and under the binary operation of $S_4$. Each 3-cycle has an inverse in the other 3-cycle that contains the same three numbers; that is, the inverse of $(1, 2, 3)$ is $(1, 3, 2)$. Each permutation whose cycle decomposition is a pair of disjoint 2-cycles is its own inverse, since the product of a 2-cycle with itself is identity. Thus the collection is closed under inverses.

  Next, to show that it is closed under the binary operation, first note that the collection consists of the identity, all eight 3-cycles, and all three pairs of disjoint 2-cycles. So it suffices to show that the product of any two of these elements is a 3-cycle or is a pair of disjoint 2-cycles. Further, by exhaustively showing that this holds for one representative 3-cycle (against the remaining elements) and one representative pair of disjoint 2-cycles (against the remaining elements), we will have shown that it holds for all 3-cycles and all pairs of disjoint 2-cycles. Then:

  \begin{itemize}
    \item $(1, 2, 3)(1, 2, 3) = (1, 3, 2)$
    \item $(1, 2, 3)(1, 3, 2) = (1)$
    \item $(1, 2, 3)(1, 2, 4) = (1, 3)(2, 4)$
    \item $(1, 2, 3)(1, 4, 2) = (1, 4, 3)$
    \item $(1, 2, 3)(1, 3, 4) = (2, 3, 4)$
    \item $(1, 2, 3)(1, 4, 3) = (1, 4)(2, 3)$
    \item $(1, 2, 3)(2, 3, 4) = (1, 2)(3, 4)$
    \item $(1, 2, 3)(2, 4, 3) = (1, 2, 4)$
    \item $(1, 2, 3)(1, 2)(3, 4) = (1, 3, 4)$
    \item $(1, 2, 3)(1, 3)(2, 4) = (2, 4, 3)$
    \item $(1, 2, 3)(1, 4)(2, 3) = (1, 4, 2)$
  \end{itemize}

  Since the collection is the products of $(1, 2, 3)$ and every other element are in the collection, and we can replace 1, 2, 3 with any other 3-cycle, it is closed for all 3-cycles. We will show this holds similarly for $(1, 2)(3, 4)$:

  \begin{itemize}
    \item $(1, 2)(3, 4)(1, 2, 3) = (2, 4, 3)$
    \item $(1, 2)(3, 4)(1, 3, 2) = (1, 4, 3)$
    \item $(1, 2)(3, 4)(1, 2, 4) = (2, 3, 4)$
    \item $(1, 2)(3, 4)(1, 4, 2) = (1, 3, 4)$
    \item $(1, 2)(3, 4)(1, 3, 4) = (1, 4, 2)$
    \item $(1, 2)(3, 4)(1, 4, 3) = (1, 3, 2)$
    \item $(1, 2)(3, 4)(2, 3, 4) = (1, 2, 4)$
    \item $(1, 2)(3, 4)(2, 4, 3) = (1, 2, 3)$
    \item $(1, 2)(3, 4)(1, 2)(3, 4) = (1)$
    \item $(1, 2)(3, 4)(1, 3)(2, 4) = (1, 4)(2, 3)$
    \item $(1, 2)(3, 4)(1, 4)(2, 3) = (1, 3)(2, 4)$
  \end{itemize}

  Thus, since it is closed under the binary operation of $S_4$ and is closed under inverses, this collection is a subgroup of $S_4$.
\end{proof}

\end{document}