\documentclass{article}

\title{Dummit \& Foote Ch. 1.2: Dihedral Groups}
\author{Scott Donaldson}
\date{Jan. - Feb. 2023}
\usepackage{amsmath, amsthm, amsfonts, enumitem}

\begin{document}

\maketitle

\section*{1. (1/23/23)}

Compute the order of each of the elements in the following groups:

\begin{enumerate}[label=(\alph*)]
    \item $D_6$
        \begin{itemize}
            \item $r, r^2$: 3
            \item $s, sr, sr^2$: 2
        \end{itemize}
    
    \item $D_8$
        \begin{itemize}
            \item $r$: 4
            \item $r^2$: 2
            \item $r^3$: 4
            \item $s, sr, sr^2, sr^3$: 2
        \end{itemize}

    \item $D_{10}$
        \begin{itemize}
            \item $r, r^2, r^3, r^4$: 5
            \item $s, sr, sr^2, sr^3, sr^4$: 2
        \end{itemize}
    
\end{enumerate}

\section*{2. (1/23/23)}

Use the generators and relations of $D_{2n} = \langle r, s\hspace{0.5ex}|\hspace{0.5ex} r^n = s^2 = 1, rs = sr^{-1} \rangle$ to show that if $x$ is any element of $D_{2n}$ which is not a power of $r$, then $rx = xr^{-1}$. 

\begin{proof}
    Let $x \in D_{2n}$ such that $x \neq r^k$ for all $k \in \mathbb{Z}$. Then, since all elements of $D_{2n}$ can be written as a product of generators $s$ and $r$, we must have $x = sr^k$ for some $k \in \{ 1, 2, ..., n - 1 \}$. Therefore:
    \begin{equation*}
        rx = rsr^k = sr^{-1} r^k = sr^{k - 1} = sr^k r^{-1} = x r^{-1},
    \end{equation*}
    as desired.
\end{proof}

\section*{3. (1/25/23)}

Use the generators and relations above to show that every element of $D_{2n}$ which is not a power of $r$ has order 2. Deduce that $D_{2n}$ is generated by the two elements $s$ and $sr$, both of which have order 2.

\begin{proof}
    Let $sr^k \in D_{2n}$. $(sr^k)(sr^k) = s(r^k s) r^k = s (s r^{-k}) r^{k} = ss r^{-k} r^k = 1 \cdot 1 = 1$. Thus the order of elements of the form $sr^k$, that is, every element which is not a power of $r$, has order 2.

    To show that $D_{2n}$ is generated by $s$ and $sr$, let $r^k, sr^k \in D_{2n}$. Now $s \cdot sr = r$, so $(s \cdot sr)^k = r^k$. To obtain $sr^k$, we simply left-multiply the previous by $s$: $s(s \cdot sr)^k = sr^k$. Thus every element of $D_{2n}$ can be written as a product of $s$ and $sr$, and so $\langle s, sr \rangle$ is a generator for $D_{2n}$.
\end{proof}

\section*{4. (1/25/23)}

If $n = 2k$ is even and $n \geq 4$, show that $z = r^k$ is an element of order 2 which commutes with all elements of $D_{2n}$. Show also that $z$ is the only nonidentity element of $D_{2n}$ which commutes with all elements of $D_{2n}$.

\begin{proof}
    Let $n = 2k, n \geq 4$, and let $z = r^k \in D_{2n}$. $z \cdot z = r^k r^k = r^{2k} = r^n = 1$, so $z$ has order 2.

    Since $r^k r^k = 1$, it follows that $r^k = r^{-k}$ (equivalently, $z = z^{-1}$). Elements of the form $r^m$ obviously commute with each other, so we only need to show that $z = r^k$ commutes with elements of the form $sr^m$. Now:

    \begin{multline*}
        r^k sr^m = r^k r^{-m} s = r^{-k} r^{-m} s = r^{-k - m} s = (r^{k + m})^{-1} s = \\
        s r^{k + m} = s r^{m + k} = s r^m r^k,
    \end{multline*}

    which shows that $z = r^k$ commutes with elements of the form $sr^m$.

    Finally, to show that $z$ is the only nonidentity element which commutes with all elements, we will consider the possible separate cases of the forms of arbitrary elements of $D_{2n}$. Let $a, b \in D_{2n}$.

    \begin{itemize}
        \item Let $a = r^m$. From above, $a$ commutes with all elements of the form $r^p$. Does $a$ commute with elements of the form $s r^p$? $r^m sr^p = r^m r^{-p} s = r^{m - p} s$. On the other hand, we have $sr^p r^m = sr^{p + m} = r^{-p - m} s$. These two are equal when $m - p = -p - m$, that is, when $m = -m$ (in $\mathbb{Z}/n\mathbb{Z}$). This only occurs when $m = n / 2 = k$, and so $z = r^k$ is the only element of the form $r^m$ which commutes with all elements of $D_{2n}$.
        \item Let $a = sr^m$. As a counterexample, it suffices to show that there is at least one element of $D_{2n}$ which $a$ does not commute with: $r$. $sr^m r = sr^{m + 1}$, while $r sr^m = r r^{-m} s = r^{1 - m} s = sr^{m - 1}$. Because $n \geq 4$, there are no values of $m \in \mathbb{Z}/n\mathbb{Z}$ for which $m + 1 = m - 1$. Thus elements of the form $sr^m$ do not commute in $D_{2n}$.
    \end{itemize}
    This completes the proof that $z = r^k$ is the only nonidentity element of $D_{2n}$ which commutes with all other elements.

\end{proof}

\section*{5. (1/26/23)}

If $n$ is odd and $n \geq 3$, show that the identity is the only element of $D_{2n}$ which commutes with all elements of $D_{2n}$.

\begin{proof}
    This proof is nearly identical to that of Exercise 4. above, only with $n$ odd instead of even. The proof that elements of the form $sr^m$ is the same as above. To show that elements of the form $r^m$ do not commute, we again consider $r^m sr^p$ and $sr^p r^m$ and see that we must have $m = -m$ (in $\mathbb{Z}/n\mathbb{Z}$). Adding $m$ to both sides, we must have $2m = 0 \Rightarrow 2m = n$. However, because $n$ is odd, this does not occur, and so there are no nonidentity elements of $D_{2n}$ which commute with all elements of $D_{2n}$.
\end{proof}

\section*{6. (1/26/23)}

Let $x, y$ be elements of order 2 in any group $G$. Prove that if $t = xy$ then $tx = xt^{-1}$ (so that if $n = |xy| < \infty$ then $x, t$ satisfy the same relations in $G$ as $s, r$ do in $D_{2n}$).

\begin{proof}
    Let $x, y \in G, |x| = |y| = 2$ and let $t = xy$. From $x^2 = y^2 = 1$, we have $x = x^{-1}$ and $y = y^{-1}$. Then:

    \begin{equation*}
        t = xy \Rightarrow tx = xyx = x(y^{-1} x^{-1}) = x(xy)^{-1} = xt^{-1},
    \end{equation*}
    as desired.

    If $|xy| = |t| = n < \infty$, then we have $t^n = x^2 = 1, tx = xt^{-1}$. These are the same relations in $G$ for $x, t$ as $s, r$ do in $D_{2n}$.
\end{proof}

\section*{7. (1/26/23)}

Show that $\langle a, b\hspace{0.5ex}|\hspace{0.5ex} a^2 = b^2 = (ab)^n = 1 \rangle$ gives a presentation for $D_{2n}$ in terms of the two generators $a = s$ and $b = sr$ of order 2 computed in Exercise 3 above.

\begin{proof}
    First, we will show that the relations for $r, s$ follow from the relations for $a, b$. Let $a = s$, so $s^2 = 1$. Let $r = ab, so r^n = (ab)^n = 1$. The orders of $r$ and $s$ are correct, but it remains to be shown that $sr = r^{-1}s$. Now $r = ab = sb$, so left-multiplying both sides by $s$, we obtain $sr = b$. Also, $r^{-1}s = (ab)^{-1} a = b^{-1} a^{-1} a = b^{-1} = b$. Thus $sr = r^{-1}s$, and so the relations for $r, s$ can be derived from those for $a, b$.
    
    Next, we will prove the converse, that the relations for $a, b$ follow from those for $r, s$. Let $a = s$, so $a^2 = 1$. Let $b = sr$, so (from Exercise 3.) $b^2 = (sr)^2 = 1$. It remains to be shown that $(ab)^n = 1$. Now $ab = s(sr) = r$, and $r^n = 1$, so $(ab)^n = 1$. Thus the relations for $a, b$ can be derived from those for $s, r$.

    Since each set of relations implies the other, they are identical, and thus present the same group, that is, $D_{2n}$.
\end{proof}

\section*{8. (1/26/23)}

Find the order of the cyclic subgroup of $D_{2n}$ generated by $r$.

\begin{proof}
    Let $R$ be the cyclic subgroup of $D_{2n}$ generated by $r$, consisting of the elements $\{1, r, r^2, ..., r^{n - 1}\}$. Intuitively it contains $n$ elements (half the order of $D_{2n}$). If less, then some $r^k, k \in \{0, 1, 2, ..., n - 1\}$ is excluded from the subgroup, contra the definition of $R$. If more, then for some element $r^k$ we must have $k > n$ (or else it would not be a unique element). However, since $r^n = 1$, we would then have $r^k = r^{k - n} r^n = r^{k - n}$. If $k - n$ is still greater than $n$, we would continue this process until we arrive at a $k - mn \in \{0, 1, 2, ..., n - 1\}$. In either case, $r^k$ is not unique. Therefore the order of $R$ is exactly $n$.
\end{proof}

\section*{9. (5/15/23)}

Let $G$ be the group of rigid motions in $\mathbb{R}^3$ of a tetrahedron. Show that $|G| = 12$.

\begin{proof}
    Label the vertices of the tetrahedron 1, 2, 3, 4. It has six edges, each labeled by its vertices: $1\textrm{-}2, 1\textrm{-}3, 1\textrm{-}4, 2\textrm{-}3, 2\textrm{-}4, 3\textrm{-}4$. A rigid motion maps one edge to another. Rotations in $\mathbb{R}^3$ could map the edge $1\textrm{-}2$ to $2\textrm{-}3$, or could rotate the edge about itself ($1\textrm{-}2$ to $2\textrm{-}1$). The identity maps $1\textrm{-}2$ to itself.

    If we consider that a motion might send one edge to six possible edges, each with two possible orientations (reflected or not), then there must be 12 unique rigid motions in $\mathbb{R}^3$ of a tetrahedron.
\end{proof}

\section*{10. (2/2/23)}

Let $G$ be the group of rigid motions in $\mathbb{R}^3$ of a cube. Show that $|G| = 24$.

\begin{proof}
    Following the pattern of the proof in Exercise 9., there are twelve edges on a cube (labeled by pairs of eight vertices). So a motion might send one edge to twelve possible edges, each with two possible orientations. Thus there are 24 unique rigid motions in $\mathbb{R}^3$ of a cube.
\end{proof}

\section*{11. (2/2/23)}

Let $G$ be the group of rigid motions in $\mathbb{R}^3$ of an octahedron. Show that $|G| = 24$.

\begin{proof}
    Like the cube, the octahedron has twelve edges, and therefore $12 \cdot 2 = 24$ unique rigid motions.
\end{proof}

\section*{12. (2/3/23)}

Let $G$ be the group of rigid motions in $\mathbb{R}^3$ of a dodecahedron. Show that $|G| = 60$.

\begin{proof}
    The dodecahedron has 30 edges. As with the above proofs, it therefore has 60 rigid motions.
\end{proof}

\section*{13. (2/3/23)}

Let $G$ be the group of rigid motions in $\mathbb{R}^3$ of an icosahedron. Show that $|G| = 60$.

\begin{proof}
    The icosahedron has 30 edges. As with the above proofs, it therefore has 60 rigid motions.
\end{proof}

\section*{14. (2/3/23)}

Find a set of generators for $\mathbb{Z}$.

\begin{proof}
    $\mathbb{Z}$ is generated by $\langle 1, -1 \rangle$. Every element $n \in \mathbb{Z}$ can be written as $\underbrace{1 + ... + 1}_{\text{$n$ times}}$ (if $n > 0$), $\underbrace{(-1) + ... + (-1)}_{\text{$n$ times}}$ (if $n < 0$), or $1 + (-1)$ (for $n = 0$).
\end{proof}

\section*{15. (2/11/23)}

Find a set of generators and relations for $\mathbb{Z}/n\mathbb{Z}$.

\begin{proof}
    $\mathbb{Z}/n\mathbb{Z}$ is generated by $\langle 1\hspace{0.5ex}|\hspace{0.5ex} k = \underbrace{1 + ... + 1}_{\text{$k$ times}}$ if $k > 0$, and $0 = \underbrace{1 + ... + 1}_{\text{$n$ times}} \rangle$.
\end{proof}

\section*{16. (2/11/23)}

Show that the group $\langle x_1, y_1 \hspace{0.5ex}|\hspace{0.5ex} x_1^2 = y_1^2 = (x_1 y_1)^2 = 1 \rangle$ is the dihedral group $D_4$.

\begin{proof}
    Let $x_1 = r$ and $y_1 = s$. Then the given group can be rewritten with the presentation $\langle r, s \hspace{0.5ex}|\hspace{0.5ex} r^2 = s^2 = (rs)^2 = 1 \rangle$. $(rs)^2 = 1 \Rightarrow rsrs = 1 \Rightarrow rsr = s$ (right-multiplying by $s$), which implies that $rs = sr^{-1}$ (right multiplying by $r^{-1}$). The latter relation is that of the dihedral group, specifically $D_4$ since $r^2 = 1$.
\end{proof}

\section*{17. (2/11/23)}

Let $X_{2n} = \langle x, y \hspace{0.5ex}|\hspace{0.5ex} x^n = y^2 = 1, xy = yx^2 \rangle$.

\begin{enumerate}[label=(\alph*)]
    \item Show that if $n = 3k, k > 0$, then $X_{2n}$ has order 6, and it has the same generators and relations as $D_6$.
          \begin{proof}
            Assume that $x$ and $y$ are unique and distinct from $1$. From $xy = yx^2$, right-multiply by $y$ and cancel to obtain:
            \begin{equation*}
                x = yx^2y = yx(xy) = yxyx^2 = y(xy)x^2 = yyx^2x^2 = x^4.
            \end{equation*}
            Now $x = x^4$ implies that $x^3 = 1$. So we have $1, x, x^2$ as unique elements of $X_{2n}$, as well as the left and right products of $y$ with each: $\{1, x, x^2, y, xy, x^2y, yx, yx^2\}$. However, we also have $xy = yx^2$, and note that $yx = yx x^3 = yx^4 = yx^2 x^2 = xy x^2 = xxy = x^2 y$, so both right products can be removed as non-unique elements, leaving us with:\newline $\{1, x, x^2, y, yx, yx^2\}$. If we let $x = r$, $y = s$, this is the same presentation as $D_6$.
          \end{proof}
    \item Show that if $(3, n) = 1$, then $x$ satisfies the additional relation: $x = 1$.
          \begin{proof}
            Let $(3, n) = 1$, so $n = 3k + 1$ or $n = 3k + 2$. From part a), $x^3 = 1$. From the relation $x^n = 1$, we then have (in the case where $n = 3k + 1$):
            \begin{equation*}
                x^{3k + 1} = 1 \Rightarrow x^{3k} x = 1 \Rightarrow (x^3)^k x = 1 \Rightarrow x = 1.
            \end{equation*}

            And, if $n = 3k + 2$:
            \begin{equation*}
                x^{3k + 2} = 1 \Rightarrow x^{3k} x^2 = 1 \Rightarrow (x^3)^k x^2 = 1 \Rightarrow x^2 = 1.
            \end{equation*}
            Since we also have $x^3 = 1$, this implies that $x^2 = x^3 \Rightarrow x = 1$.

            Assuming that $y$ is distinct from $1$, the group reduces to $\{1, y\}$.
          \end{proof}
\end{enumerate}

\section*{18. (2/12/23)}

Let $Y = \langle u, v \hspace{0.5ex}|\hspace{0.5ex} u^4 = v^3 = 1, uv = v^2 u^2 \rangle$.

\begin{enumerate}[label=(\alph*)]
    \item Show that $v^2 = v^{-1}$.
          $v^3 = 1$. Multiplying both sides by $v^{-1}$, we obtain $v^2 = v^{-1}$.
    \item Show that $v$ commutes with $u^3$.
          \begin{equation*}
            v^2 u^3 v = (v^2 u^2) uv = (v^2 u^2)(v^2 u^2) = uv(v^2 u^2) = uv^3 u^2 = u^3.
          \end{equation*}
          From part a), $v^2$ = $v^{-1}$, so $v^2 u^3 v = v^{-1} u^3 v$. And, from above, this equals $u^3$, so we left-multiply by $v$ to obtain $u^3 v = v u^3$.
    \item Show that $v$ commutes with $u$.
          From the given relation $u^4 = 1$, we obtain $u^8 = 1$, and $u^9 = u$. Now $uv = u^9 v = u^3 u^3 u^3 v$. Since $v$ commutes with $u^3$, we can rewrite this as $v u^3 u^3 u^3 = v u^9 = vu$. Since $uv = vu$, they are commuting elements.
    \item Show that $uv = 1$.
          From the relation $uv = v^2 u^2$ and the fact that $u$ and $v$ commute, we see that $vu = v^2 u^2$. Left-multiplying by $v^{-1}$, $u = v u^2 = u^2 v$. Now left-multiply by $u^{-1}$ to obtain $1 = uv$.
    \item Show that $u = 1$ and $v = 1$.
          Finally, from $u^4 = 1$ and $v^3 = 1$, $u^4 v^3 = 1$. Since $u$ and $v$ commute, we can rewrite this as $u(uv)^3 = 1$. From part d), $uv = 1$, so $u = 1$. And because the identity is its own inverse, we also have $v = 1$.
\end{enumerate}

Thus the group $Y$ degenerates to the trivial group of order 1.

\end{document}