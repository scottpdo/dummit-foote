\documentclass{article}

\title{Dummit \& Foote Ch. 1: Groups}
\author{Scott Donaldson}
\date{Nov. 2022 - Jan. 2023}
\usepackage{amsmath, amsthm, amsfonts, enumitem}

\begin{document}

\maketitle

\section*{1. (11/14/22)}

Let $G$ be a group. Determine which of the following binary operations are associative:

\begin{enumerate}[label=(\alph*)]
    \item The operation $\star$ on $\mathbb{Z}$ defined by $a \star b = a - b:$
    
          Not associative. $3 \star (2 \star 1) = 3 - 1 = 2$ but $(3 \star 2) \star 1 = 3 - 2 = 1$.
    \item The operation $\star$ on $\mathbb{R}$ defined by $a \star b = a + b + ab:$
    
          Associative. 
          \begin{equation*}
            a \star (b \star c) = a \star (b + c + bc) = a + b + c + bc + ab + ac + abc =
            (a + b + ab) \star c = (a \star b) \star c
          \end{equation*}
    \item The operation $\star$ on $\mathbb{Q}$ defined by $a \star b = \frac{a + b}{5}$:
    
          Not associative. $0 \star (1 \star 1) = 0 + 2/5 = 2/5$ but $(0 \star 1) \star 1 = 1/5 \star 1 = 6/5 * 1/5 = 6/25$.
    \item The operation $\star$ on $\mathbb{Z} \times \mathbb{Z}$ defined by $(a, b) \star (c, d) = (ad + bc, bd):$
    
          Associative.
          \begin{multline*}
            ((a,b)\star(c,d))\star(e,f) = (ad + bc, bd)\star(e,f) = \\
            (adf + bcf + bde, bdf) = (a,b)\star(cf + de, df) = (a,b)\star((c,d)\star(e,f)).
          \end{multline*}
    \item The operation $\star$ on $\mathbb{Q} - \{0\}$ defined by $a \star b = a / b:$
    
          Not associative. $(1 \star 2) \star 3 = 1/6$ but $1 \star (2 \star 3) = 3/2$.
\end{enumerate}

\section*{2. (11/14/22)}

Decide which of the binary operations in the preceding exercise are commutative.

\begin{enumerate}[label=(\alph*)]
    \item Not commutative. $1 - 2 = -1$ but $2 - 1 = 1$.
    \item Commutative. $a \star b = a + b + ab = b + a + ba = b \star a$.
    \item Commutative. $a \star b = \frac{a + b}{5} = \frac{b + a}{5} = b \star a$.
    \item Commutative. $(a,b)\star(c,d) = (ad + bc, bd) = (cb + da, db) = (c,d)\star(a,b)$.
    \item Not commutative. $1 / 2 = 1/2$ but $2 / 1 = 2$.
\end{enumerate}

\section*{3. (11/16/22)}

Prove that addition of residue classes in $\mathbb{Z}/n\mathbb{Z}$ is associative.

\begin{proof}

      First, we will show that subtraction in $\mathbb{Z}/n\mathbb{Z}$ is well-defined. Given a representative element $\bar{a}$, $1 \leq \bar{a} \leq n - 1$, the element $n - \bar{a}$ is $\bar{a}$'s inverse. $1 \leq n - \bar{a} \leq n - 1$, so $n - \bar{a}$ is also a representative element. Also, $\bar{a} + (n - \bar{a}) = n \sim 0$. Thus, subtracting an element $\bar{a}$ from $\bar{b}$ is the same as adding $n - \bar{a}$ to $\bar{b}$, and so subtraction is well-defined.

      Now, to show that addition is associative, let $\bar{a}, \bar{b}, \bar{c} \in \mathbb{Z}/n\mathbb{Z}$. Suppose that $(\bar{a} + \bar{b}) + \bar{c} = \bar{d}$ and $\bar{a} + (\bar{b} + \bar{c}) = \bar{e}$. Then:

      \begin{equation*}
            \bar{d} - \bar{c} = \bar{a} + \bar{b} \Rightarrow \bar{a} = (\bar{d} - \bar{c}) - \bar{b}
      \end{equation*}

      And:

      \begin{equation*}
            \bar{e} - \bar{a} = \bar{b} + \bar{c} \Rightarrow \bar{e} = ((\bar{d} - \bar{c}) - \bar{b}) + \bar{b} + \bar{c} = \bar{d} - \bar{c} + \bar{c} = \bar{d} 
      \end{equation*}

      Therefore $\bar{d} = \bar{e}$, so $(\bar{a} + \bar{b}) + \bar{c} = \bar{a} + (\bar{b} + \bar{c})$.
\end{proof}

\section*{4. (11/16/22)}

Prove that multiplication of residue classes in $\mathbb{Z}/n\mathbb{Z}$ is associative.

\begin{proof}
      Let $\bar{a}, \bar{b}, \bar{c} \in \mathbb{Z}/n\mathbb{Z}$. Then:

      \begin{equation*}
            \overline{a}(\overline{b}\overline{c}) = \overline{a}(\overline{bc}) = \overline{a(bc)}
      \end{equation*}

      Since the latter expression involves arbitrary integers $a, b, c$ whose representative elements in $\mathbb{Z}/n\mathbb{Z}$ are $\overline{a}, \overline{b}, \overline{c}$, we can use the associative property of standard multiplication:

      \begin{equation*}
            \overline{a(bc)} = \overline{(ab)c} = (\overline{ab})\overline{c} = (\overline{a}\overline{b})\overline{c} 
      \end{equation*}

      Therefore multiplication of residue classes is associative.
\end{proof}

\section*{5. (11/16/22)}

Prove for all $n > 1$ that $\mathbb{Z}/n\mathbb{Z}$ is not a group under multiplication of residue classes.

\begin{proof}
      Let $\mathbb{Z}/n\mathbb{Z}$ with $n > 1$. The element $1$ is the identity element, since (by multiplication of standard integers), $1 \cdot \bar{a} = \bar{a}$ for all $\bar{a} \in \mathbb{Z}/n\mathbb{Z}$. However, the element $0$ has no inverse, since (again by standard multiplication), there is no element $\bar{a}$ such that $0 \cdot \bar{a} = 1$. Thus, $\mathbb{Z}/n\mathbb{Z}$ is not a group under multiplication.
\end{proof}

\section*{6. (11/18/22)}

Determine which of the following are sets are groups under addition:

\begin{enumerate}[label=(\alph*)]
      \item the set of rational numbers (including $0 = 0/1$) in lowest terms whose denominators are odd:
      
            This is a group. The identity element is $0$ and addition is associative by definition. Each element $a$ has an inverse in $-a = -1 \cdot a$. It remains to be shown that the set is closed under addition. Let $\frac{a}{b}$ and $\frac{c}{d}$ be two elements of the set. Then $\frac{a}{b} + \frac{c}{d} = \frac{ad + bc}{bd}$. The product of two odd numbers is odd, so $bd$ is odd. Further, if $\frac{ad + bc}{bd}$ is not in lowest terms, then the denominator must remain negative, since an odd number has no even divisors. Thus the set is closed under addition.
      \item the set of rational numbers (including $0 = 0/1$) in lowest terms whose denominators are even:

            Not a group. $1/2 + 1/2 = 1/1$, a rational number whose denominator is odd.
      \item the set of rational numbers of absolute value $< 1$.

            Not a group. $3/4 + 3/4 = 3/2$, a rational number whose absolute value is $\geq 1$.
      \item the set of rational numbers of absolute value $\geq 1$ together with $0$.

            Not a group. $3/2 + (-3/4) = 1/4$, a rational number whose absolute value is $< 1$. 
      \item the set of rational numbers with denominators equal to $1$ or $2$.

            This is a group. Identity, associativity, and inverses are trivial. Let $a, b$ be members of the set. If both have denominator $1$ or $2$, then their sum has denominator $1$. Otherwise, if one has denominator $1$ and the other denominator $2$, their sum has denominator $2$. Therefore the set is closed under addition.
      \item the set of rational numbers with denominators equal to $1$, $2$, or $3$.

            Not a group. $1/2 + 1/3 = 5/6$.
\end{enumerate}

\section*{7. (11/18/22)}

Let $G = \{x \in \mathbb{R} \mid 0 \leq x < 1\}$ and for $x, y \in G$ let $x \star y$ be the fractional part of $x + y$. Prove that $\star$ is a well-defined binary operation on $G$ and that $G$ is an abelian group under $\star$ (called the \emph{real numbers mod 1}).

\begin{proof}
      $\star$ is a well-defined binary operation on $G$. Let $x, y \in G$. Then $x, y \in [0, 1)$. Suppose that $x + y = z \in \mathbb{R}$. By definition, $x \star y$ is the fractional part of $z$, which is unique. Therefore $\star$ is well-defined, and commutative, since $+$ is commutative.

      The identity element of $G$ is $0$, since for all $x \in [0, 1)$, $0 + x = x$.

      For all $x \in G$, $x$ has an inverse $1-x \in G$, since $x + (1-x) = 1$, and so $x \star (1-x) = 0$.

      $G$ is closed under $\star$. For any $z = x + y$, the fractional part of $z$ is (by definition) greater than or equal to $0$ and strictly less than $1$. Therefore $x \star y$ is in $G$.

      Finally, $\star$ is associative. Let $a, b, c \in G$. $(a \star b) \star c$ is equal to the fractional part of $(a \star b) + c$. And, $a \star b$ is equal to the fractional part of $a + b$. Now, taking the fractional part of a number is an idempotent operation; that is, performing it more than once yields the same value. So the fractional part of $(a \star b) + c$, that is, the fractional part of the fractional part of $(a + b) + c$ is just the fractional part of $(a + b) + c = a + b + c$. Similarly, $a \star (b \star c)$ is equal to the fractional part of $a + b + c$, and so $\star$ is associative.

      Thus $G$ is an abelian group under $\star$.
\end{proof}

\section*{8. (11/18/22)}

Let $G = \{z \in \mathbb{C} \mid z^n = 1$ for some $n \in \mathbb{Z}^+\}$. Prove that $G$ is a group under multiplication (called the \emph{roots of unity}) but not under addition.

\begin{proof}
      $1$ is the identity element of $G$. $1^1 = 1$, so $1 \in G$, and by definition $1 \cdot z = z$ for all $z \in \mathbb{C}$. Multiplication is by definition associative, so it remains to be shown that elements in $G$ have inverses and that $G$ is closed under multiplication.

      Let $z \in G$ (to show elements have inverses). Then $z^n = 1$ for some $n \in \mathbb{Z}^+$. Since $1/1 = 1$, we also have $1/(z^n) = 1$. It follows that $(1/z)^n = 1$, and so $1/z \in G$. $z \cdot 1/z = 1$, and therefore $z$ has an inverse $1/z$.

      Let $a, b \in G$ (to show that $G$ is closed under multiplication). It follows that $a^n = 1$ and $b^m = 1$ for some $n, m \in \mathbb{Z}^+$. Then $1 = a^n b^m = (ab)^{nm}$. The product of $ab$ raised to the $nm$ power is $1$, so it is an element of $G$, and thus $G$ is closed under addition.

      $G$ is not a group under addition. Both $1$ and the imaginary number $i$ are elements of $G$, but their sum $1 + i$ is not. Consider the modulus of a complex number $z = x + iy$, $\sqrt{x^2 + y^2}$. The modulus of $1 + i$ is $\sqrt{2}$. The modulus of the product of two complex numbers is equal to the product of the modulus of each number (proof omitted). The modulus of $(1 + i)^2$ is $\sqrt{2} \cdot \sqrt{2} = 2$. The modulus of $(1 + i)^3$ is then $2\sqrt{2}$. For each successive $n$, then, the modulus of $(1 + i)^n$ is strictly increasing. However, the modulus of $1 \in \mathbb{C}$ is $1$, so $(1 + i)^n$ is never $1$, and therefore $1 + i$ is not in $G$.
\end{proof}

\section*{9. (11/19/22)}

Let $G = \{a + b\sqrt{2} \in \mathbb{R} \mid a, b \in \mathbb{Q} \}$. Prove that $G$ is a group under addition and that the nonzero elements of $G$ are a group under multiplication.

\begin{proof}
      For addition, let $0 = 0 + 0\sqrt{2}$ be the identity element and note that addition is by definition is associative. The inverse of $a + b\sqrt{2}$ is simply $-a - b\sqrt{2}$. To show that $G$ is closed, let $a + b\sqrt{2}$ and $c + d\sqrt{2}$ be elements of $G$. Then $a + b\sqrt{2} + c + d\sqrt{2} = (a + c) + (b + d)\sqrt{2}.$ Since the rational numbers are closed under addition, $a + c, b + d \in \mathbb{Q}$ and so $G$ is closed under addition. Thus $G$ is a group under addition.

      Next consider the set $G - \{0\}$ under multiplication. $1 = 1 + 0\sqrt{2}$ is the identity element and multiplication is by definition associative. The inverse of $a + b\sqrt{2}$ is: 
      \begin{equation*}
            \frac{1}{a + b\sqrt{2}} = \frac{a - b\sqrt{2}}{a^2 - 2b^2} = \bigl(\frac{a}{a^2 - 2b^2}\bigr) - \bigl(\frac{b}{a^2 - 2b^2}\bigr)\sqrt{2}
      \end{equation*}

      The expressions inside the parentheticals are rational numbers, so elements in $G - \{0\}$ have inverses that are in $G$ (note that the denominator $a^2 - 2b^2$ is only $0$ when $a = b\sqrt{2}$; however, this is impossible, as $a \notin \mathbb{Q}$).
      
      To show that $G - \{0\}$ is closed, let $a + b\sqrt{2}$ and $c + d\sqrt{2}$ be elements of $G - \{0\}$. Then 
      \begin{equation*}
            (a + b\sqrt{2}) \cdot (c + d\sqrt{2}) = ac + ad\sqrt{2} + bc\sqrt{2} + 2bd = (ac + 2bd) + (ad + bc)\sqrt{2}
      \end{equation*}

      Therefore $G - \{0\}$ is closed under multiplication, and is thus a group under multiplication.

\end{proof}

\section*{10. (11/20/22)}

Prove that a finite group is abelian if and only if its group table is a symmetric matrix.

\begin{proof}
      Let $G$ be a finite group with elements $\{g_1, g_2, ... , g_n\}, g_1 = 1$ and let $A$ be its group table, a matrix with the $i,j$-th entry equal to $g_i g_j$.

      First, suppose that $G$ is an abelian group. So for all $g_i, g_j \in G$, $g_i g_j = g_j g_i$. Then the $i,j$-th entry, $g_i g_j$, is equal to the $j,i$-th entry, $g_j g_i$. Thus $A$ is symmetric.

      Next, suppose that $A$ is a symmetric matrix. Then the $i,j$-th entry is equal to the $j,i$-th entry, that is, $g_i g_j = g_j g_i$. Since all possible combinations of elements of $G$ commute with each other, $G$ is thus an abelian group.
\end{proof}

\section*{11. (11/20/22)}

Find the orders of each element of the additive group $\mathbb{Z}/12\mathbb{Z}$.

\begin{enumerate}[label=$\ast$]
      \item $\bar{0}$: $1$.
      \item $\bar{1}$: $12$.
      \item $\bar{2}$: $6$.
      \item $\bar{3}$: $4$.
      \item $\bar{4}$: $3$.
      \item $\bar{5}$: $12$.
      \item $\bar{6}$: $2$.
      \item For each subsequent element $\bar{a}$, the order is the same as that of its inverse (listed above), $12 - \bar{a}$.
\end{enumerate}

\section*{12. (11/20/22)}

Find the orders of the following elements of the multiplicative group $\bigl(\mathbb{Z}/12\mathbb{Z}\bigr)^\times$.

\begin{enumerate}[label=$\ast$]
      \item $\overline{1}$: $1$.
      \item $\overline{-1}$: $-1 \times -1 = 1$. Order $2$.
      \item $\overline{5}$: $5 \times 5 = 25 \sim 1$. Order $2$.
      \item $\overline{7}$: $7 \times 7 = 49 \sim 1$. Order $2$.
      \item $\overline{-7}$: $-7 ~ 5$. Order $2$.
      \item $\overline{13}$: $13 \sim 1$. Order $1$.
\end{enumerate}

\section*{13. (11/20/22)}

Find the orders of the following elements of the additive group $\mathbb{Z}/36\mathbb{Z}$.

\begin{enumerate}[label=$\ast$]
      \item $\overline{1}$: $36$.
      \item $\overline{2}$: $18$.
      \item $\overline{6}$: $6$.
      \item $\overline{9}$: $4$.
      \item $\overline{10}$: $18$.
      \item $\overline{12}$: $3$.
      \item $\overline{-1}$: $36$.
      \item $\overline{-10}$: $18$.
      \item $\overline{-18}$: $2$.
\end{enumerate}

\section*{14. (11/30/22)}

Find the orders of the following elements of the multiplicative group $\bigl(\mathbb{Z}/36\mathbb{Z}\bigr)^\times$.

\begin{enumerate}[label=$\ast$]
      \item $\overline{1}$: $1$.
      \item $\overline{-1}$: $2$.
      \item $\overline{5}$: $6$.
      \item $\overline{13}$: $3$.
      \item $\overline{-13}$: $6$.
      \item $\overline{17}$: $2$.
\end{enumerate}

\section*{15. (11/30/22)}

Prove that $(a_1 a_2 ... a_n)^{-1} = a_n^{-1} a_{n-1}^{-1} ... a_1^{-1}$ for all $a_1, a_2, ... , a_n \in G$.

\begin{proof}
      Let $a_1 a_2 ... a_n = b$. Then $a_1 a_2 ... a_{n - 1} = b a_n^{-1}$. We can continue multiplying by the inverse of each right-most element until $1 = b a_n^{-1} a_{n - 1}^{-1} ... a_2^{-1} a_1^{-1}$. Then $b^{-1} = a_n^{-1} a_{n - 1}^{-1} ... a_2^{-1} a_1^{-1}$, and so $(a_1 a_2 ... a_n)^{-1} = a_n^{-1} a_{n-1}^{-1} ... a_1^{-1}$.
\end{proof}

\section*{16. (12/20/22)}

Let $x$ be an element of $G$. Prove that $x^2 = 1$ if and only if $|x|$ is either $1$ or $2$.

\begin{proof}
      First, suppose that $|x|$ is $1$. Then $x = 1$, so $x^2 = 1 \cdot 1 = 1$. If $|x|$ is $2$, then by definition $x^2 = 1$. So if $|x|$ is either $1$ or $2$, then $x^2 = 1$.

      Next, suppose that $x^2 = 1$. By definition, the order of $x$ cannot be greater than $2$, so it must be either $1$ or $2$.
\end{proof}

\section*{17. (12/19/22)}

Let $x \in G$ with $|x| = n$, $n \in \mathbb{Z}^+$. Prove that $x^{-1} = x^{n - 1}$.

\begin{proof}
      Let $x \in G$ with $|x| = n$. So $x^n = 1$.

      Multiply both sides by $x^{-1}$ to obtain $x^n x^{-1} = x^{-1}$. Thus $x^{n - 1} = x^{-1}$.
\end{proof}

\section*{18. (12/20/22)}

Prove that $xy = yx$ if and only if $y^{-1} xy = x$ if and only if $x^{-1} y^{-1} xy = 1$.

\begin{proof}
      First, to prove that $xy = yx$ implies that $y^{-1} xy = x$, let $xy = yx$ and left-multiply both sides by $y^{-1}$. Then $y^{-1} xy = y^{-1} yx = x$.

      Next, to prove that $y^{-1} xy = x$ implies that $x^{-1} y^{-1} xy = 1$, let $y^{-1} xy = x$ and left-multiply both sides by $x^{-1}$. Then $x^{-1} y^{-1} xy = x^{-1} x = 1$.

      Finally, to prove that $x^{-1} y^{-1} xy = 1$ implies that $xy = yx$, let $x^{-1} y^{-1} xy = 1$ and left-multiply both sides by $x$, then $y$. Then $xy = yx$.
\end{proof}

\section*{19. (12/29/22)}

Let $x \in G$ and let $a, b \in \mathbb{Z}^+$. 

\begin{enumerate}[label=(\alph*)]
      \item Prove that $x^a x^b = x^{a + b}$ and $(x^a)^b = x^{ab}$.\newline
            $x^a x^b = \underbrace{x \cdot ... \cdot x}_{\text{$a$ times}} \cdot \underbrace{x \cdot ... \cdot x}_{\text{$b$ times}} = \underbrace{x \cdot ... \cdot x}_{\text{$a + b$ times}} = x^{a + b}$.\newline
            Similarly, $(x^a)^b = \underbrace{x^a \cdot ... \cdot x^a}_{\text{$b$ times}} = \underbrace{\underbrace{x \cdot ... \cdot x}_{\text{$a$ times}} \cdot ... \cdot \underbrace{x \cdot ... \cdot x}_{\text{$a$ times}}}_{\text{$b$ times}} = \underbrace{x \cdot ... \cdot x}_{\text{$ab$ times}} = x^{ab}$.
      \item Prove that $(x^a)^{-1} = x^{-a}$.\newline
            Let $x^a = b$. Right-multiply this equation by $x^{-1}$ to obtain $x^a x^{-1} = x^{a - 1} = b x^{-1}$. Continue doing this until we obtain $1 = b \underbrace{x^{-1} \cdot ... \cdot x^{-1}}_{\text{$a$ times}}$, that is, $1 = b x^{-a}$. Then, left-multiply by $b^{-1}$ to obtain $b^{-1} = x^{-a}$. Since $b = x^a$, $(x^{a})^{-1} = x^{-a}$.
      \item Establish part a) for arbitrary integers $a$ and $b$.\newline
            In the case where either $a$ or $b$ is $0$, the equalities hold because for any $x \in G$, by definition $x^0 = 1$, and so $x^a x^0 = x^a \cdot 1 = x^a = x^{a + 0}$ and $(x^a)^0 = 1 = x^0 = x^{a \cdot 0}$ (also, $(x^0)^a = 1 = x^0 = x^{0 \cdot a}$).
            
      Next, consider $x^a x^b$ with both exponents negative, written differently, $x^{-a} x^{-b}$. From part b), this is equal to $(x^a)^{-1} (x^b)^{-1} = (x^b x^a)^{-1} = (x^{a + b})^{-1} = x^{-a - b}$, as desired. If $a$ and $b$ have different signs, that is, $x^a x^{-b}$, we have $x^a (x^{-1})^b = \underbrace{x \cdot ... \cdot x}_{\text{$a$ times}} \cdot \underbrace{x^{-1} \cdot ... \cdot x^{-1}}_{\text{$b$ times}}$. Each pair of $x \cdot x^{-1}$ reduces to the identity, leaving us with (in the case where $a > -b$) $x^{a - b}$, or (if $a < -b$), $(x^{-1})^{b - a} = x^{a - b}$, as desired.

      Finally, consider $(x^a)^{-b}$. From part b), this is equal to $((x^a)^b)^{-1} = (x^{ab})^{-1} = x^{-ab}$. Similarly, $(x^{-a})^b = ((x^{-1})^a)^b = (x^{-1})^{ab} = x^{-ab}$. And, if both $a$ and $b$ are negative, then:
      
      \begin{equation*}
            (x^{-a})^{-b} = (((x^a)^{-1})^b)^{-1} = ((x^a)^{-b})^{-1} = (x^{-ab})^{-1} = x^{ab}.
      \end{equation*}
\end{enumerate}

\section*{20. (12/29/22)}

For an element $x \in G$, show that $x$ and $x^{-1}$ have the same order.

\begin{proof}
      Let $x \in G$. Suppose that $|x| = n$. Then $x^n = 1$. Multiply both sides of this equation by $x^{-n}$ to obtain $x^n x^{-n} = x^{n - n} = x^0 = 1$ on the left, and $x^{-n} = (x^{-1})^n$ on the right. Thus the order of $x^{-1}$ is at most $n$. However, if its order were any natural number $m$ less than $n$, then we would have\newline
      $(x^{-1})^m = 1 \Rightarrow 1 = x^m$, contradicting $|x| = n$. The same logic shows that if $x$ has infinite order, $x^{-1}$ cannot have finite order and vice-versa. Thus $x$ and $x^{-1}$ must have the same order.
\end{proof}

\section*{21. (12/30/22)}

Let $G$ be a finite group and let $x$ be an element of $G$ of order $n$. Prove that if $n$ is odd, then $x = (x^2)^k$ for some $k$.

\begin{proof}
      Let $x \in G$ with $|x| = 2k - 1$ for some $k \in \mathbb{N}$. Then $x^{2k - 1} = 1$, which implies that $x^{2k} x^{-1} = 1$. Right-multiplying both sides of the equation by $x$, we have $x^{2k} = x$, so $x = (x^2)^k$, as desired.
\end{proof}

\section*{22. (12/31/22)}

If $x$ and $g$ are elements of the group $G$, prove that $|x| = |g^{-1}xg|$. Deduce that $|ab| = |ba|$ for all $a, b \in G$.

\begin{proof}
      First, we will prove a useful lemma, that $(g^{-1}xg)^n = g^{-1}x^ng$.
      \begin{multline*}
            (g^{-1}xg)^n = \underbrace{(g^{-1}xg)...(g^{-1}xg)}_\text{$n$ times} = g^{-1}\underbrace{(xgg^{-1})...(xgg^{-1})}_\text{$n - 1$ times}xg = g^{-1}x^{n - 1}xg = \\g^{-1} x^n g.
      \end{multline*}

      Now if $|x|$ is infinite, then there is no $n \in \mathbb{Z}^+$ such that $x^n = 1$. Suppose toward contradiction that $|g^{-1}xg| = n$. Then we have $(g^{-1}xg)^n = 1 \Rightarrow g^{-1}x^n g = 1$. We can left-multiply by $g$ and then right-multiply by $g^{-1}$ to obtain $x^n = gg^{-1} = 1$, contradicting $x$ having infinite order. Therefore $|g^{-1}xg|$ is also infinite.

      Suppose then that $|x| = n$, $n \in \mathbb{Z}^+$. So $x^n = 1$. Left-multiply by $g^{-1}$ and then right-multiply by $g$ to obtain $g^{-1}x^n g = g^{-1}g = 1$. From the above lemma, then, we have $(g^{-1}xg)^n = 1$. So the order of $g^{-1}xg$ must be less than or equal to $n$. Suppose that the order is $m$, $m < n$. Then $(g^{-1}xg)^m = 1 \Rightarrow g^{-1}x^m g = 1 \Rightarrow x^m = 1$, contradicting the order of $x$ being $n$. Thus the order of $g^{-1}xg$ is the same as the order of $x$.

      Suppose for some $a, b \in G$ that $|ab| = n$. Then $(ab)^n = 1 \Rightarrow \underbrace{ab \cdot ... \cdot ab}_\text{$n$ times} \Rightarrow a(ba)^{n - 1}b = 1$. Now we can left-multiply both sides of this equation by $b$ and then right-multiply by $b^{-1}$ to obtain $ba(ba)^{n - 1}bb^{-1} = bb^{-1} \Rightarrow (ba)^n = 1$. By similar logic to above, the order of $ba$ must be at most $n$, and can in fact be no less than it, and is thus equal to the order of $ab$. 
\end{proof}

\section*{23. (12/31/22)}

Suppose $x \in G$ and $|x| = n < \infty$. If $n = st$ for some positive integers $s$ and $t$, prove that $|x^s| = t$.

\begin{proof}
      The order of $x$ is $n$, so $x^n = 1$. Then $x^{st} = 1$. From 19., $(x^s)^t = 1$. So the order of $x^s$ is at most $t$.

      Suppose that the order of $x^s$ is $r < t$. Then $(x^s)^r = 1 \Rightarrow x^{sr} = 1$, and so the order of $x$ is at most $sr < st = n$, a contradiction. Therefore the order of $x^s$ is exactly $t$. 
\end{proof}

\section*{24. (1/5/23)}

If $a$ and $b$ are commuting elements of $G$, prove that $(ab)^n = a^n b^n$ for all $n \in \mathbb{Z}$.

\begin{proof}
      We will prove this statement first for non-negative integers only, using induction. First, note that (trivially) $(ab)^0 = 1$ and $a^0 b^0 = 1 \cdot 1 = 1$, so $(ab)^0 = a^0 b^0$.

      Next, suppose that $(ab)^n = a^n b^n$ for some positive integer $n$ (in order to show that the statement holds for $n + 1$). By our inductive hypothesis, $(ab)^{n + 1} = (ab)^n ab = a^n b^n ab$. Since $a$ and $b$ commute, so do any non-negative powers of $a$ and $b$, specifically, $ab^n = b^n a$. Thus $a^n b^n ab = a^n a b^n b = a^{n + 1} b^{n + 1}$, as desired.

      Having established this for positive integers, we can now do the same for negative integers. For the base case of $-1$, let $ab = ba = x$. Then $(ab)^{-1} = x^{-1}$, which implies that $b^{-1} a^{-1} = x^{-1}$. Also, we have $(ba)^{-1} = a^{-1} b^{-1} = x^{-1}$, so $(ab)^{-1} = x^{-1} = a^{-1} b^{-1}$.

      Now suppose that $(ab)^{-n} = a^{-n} b^{-n}$ for some positive integer $n$. Following the logic for non-negative integers, we see that $(ab)^{-n - 1} = (ab)^{-n} (ab)^{-1} = a^{-n} b^{-n} a^{-1} b^{-1}$. Having established that negative powers of $a$ and $b$ commute just as do non-negative powers, we have $a^{-n} b^{-n} a^{-1} b^{-1} = a^{-n - 1} b^{-n - 1}$, as desired.
\end{proof}

\section*{25. (1/12/23)}

Prove that if $x^2 = 1$ for all $x \in G$ then $G$ is abelian.

\begin{proof}
      Suppose that $G$ is a group such that, for all $x \in G$, $x^2 = 1$. Left-multiplying by $x^{-1}$, this implies that $x = x^{-1}$; that is, each element of $G$ is its own inverse.

      Let $a, b \in G$. Then $ab = (ab)^{-1} = b^{-1} a^{-1}$, and since each element is its own inverse, this equals $ba$. Thus all elements of $G$ commute, so $G$ is an abelian group.
\end{proof}

\section*{26. (1/12/23)}

Assume $H$ is a nonempty subset of $(G, \star)$ which is closed under the binary operation on $G$ and is closed under inverses, i.e., for all $h, k \in H$, $hk$ and $h^{-1} \in H$. Prove that $H$ is a group under the operation $\star$ restricted to $H$ (a \emph{subgroup} of $G$).

\begin{proof}
      For $H$ to be a group under the operation $\star$, it must fulfill associativity, existence of identity, and existence of inverses.

      Associativity is given by the fact that the operation $\star$ is associative on $G$, since $G$ is a group. Inverses are also given. It remains to be proven that $H$ contains the identity element.

      Let $h \in H$. Since $H$ is closed under inverses, $h^{-1} \in H$. $H$ is closed under $\star$, so $h h^{-1} \in H$. By definition, $h h^{-1} = 1$, so $1 \in H$.
\end{proof}

\section*{27. (1/12/23)}

Prove that if $x$ is an element of $G$ then $\{x^n\hspace{0.5ex}|\hspace{0.5ex}n \in \mathbb{Z}\}$ is a subgroup of $G$ (called the \emph{cyclic subgroup} of $G$ generated by $x$).

\begin{proof}
      Let $x \in G$ and let $X = \{x^n\hspace{0.5ex}|\hspace{0.5ex}n \in \mathbb{Z}\}$. We must prove that $X$ is associative and contains the identity element and inverses for each element.

      Let $x^n, x^m, x^k \in X$. 
      \begin{equation*}
            (x^n x^m) x^k = (x^{n + m}) x^k = x^{n + m + k} = x^n (x^{m + k}) = x^n (x^m x^k),
      \end{equation*}
      so $X$ is associative.

      $0 \in \mathbb{Z}$, so $x^0 = 1 \in X$, and so $X$ contains the identity element.

      Finally, let $x^n \in X$. $-n \in \mathbb{Z} \Rightarrow x^{-n} \in X$. Since $x^n x^{-n} = x^0 = 1$, there is an inverse for each element of $X$ in $X$. Thus $X$ is a subgroup of $G$.
\end{proof}

\section*{28. (1/14/23)}

Let $(A, \star)$ and $(B, \diamond)$ be groups and let $A \times B$ be their direct product. Verify all the group axioms for $A \times B$: associativity, identity, and inverses.

\begin{proof}
      To prove that $A \times B$ is associative, let $a_1, a_2, a_3 \in A$ and $b_1, b_2, b_3 \in B$. Consider $(a_1, b_1)[(a_2, b_2)(a_3, b_3)]$. This equals $(a_1, b_1)(a_2 \star a_3, b_2 \diamond b_3)$, which equals $\bigl(a_1 \star (a_2 \star a_3), b_1 \diamond (b_2 \diamond b_3)\bigr)$. Now since $A$ and $B$ are themselves associative, we can rewrite this as $\bigl((a_1 \star a_2) \star a_3, (b_1 \diamond b_2) \diamond b_3\bigr)$, which is equal to $(a_1 \star a_2, b_1 \star b_2)(a_3, b_3)$, which in turn equals $[(a_1, b_1)(a_2, b_2)](a_3, b_3)$. Thus $A \times B$ is associative.

      Next, since $A$ and $B$ are groups, they each contain an identity element, $1_A, 1_B$, respectively. By definition, $A \times B$ contains $(1_A, 1_B)$. For any $(a, b) \in A \times B$, $(a, b)(1_A, 1_B) = (a \star 1_A)(b \diamond 1_B) = (a, b)$. Thus $A \times B$ contains the identity element $(1_A, 1_B)$.

      Finally, let $(a, b) \in A \times B$. $A$ and $B$ contain inverses for each element, so $(a^{-1}, b^{-1}) \in A \times B$. Now $(a, b)(a^{-1}, b^{-1}) = (a \star a^{-1}, b \diamond b^{-1}) = (1_A, 1_B)$, the identity element of $A \times B$. Thus $A \times B$ also contains an inverse for each element.

      $A \times B$ satisfies the three group axioms of associativity, identity, and inverses, and is thus a group itself.
\end{proof}

\section*{29. (1/17/23)}

Prove that $A \times B$ is an abelian group if and only if $A$ and $B$ are both abelian.

\begin{proof}
      First, we will show that if $A$ and $B$ are abelian groups under their respective operations $\star$ and $\diamond$, then $A \times B$ is as well. We see that $(a_1, b_1)(a_2, b_2) = (a_1 \star a_2, b_1 \diamond b_2)$. Since elements of $A$ and $B$ commute, this is equal to $(a_2 \star a_1, b_2 \diamond b_1)$, which, by definition of $A \times B$, is equal to $(a_2, b_2)(a_1, b_1)$. Thus $A \times B$ is an abelian group.

      Next, let $A \times B$ be an abelian group. So we have $(a_2 \star a_1, b_2 \diamond b_1) = (a_1 \star a_2, b_1 \diamond b_2)$. Therefore we have $a_2 \star a_1 = a_1 \star a_2$ and $b_2 \diamond b_1 = b_1 \diamond b_2$, so $A$ and $B$ must both be abelian groups.
\end{proof}

\section*{30. (1/17/23)}

Prove that the elements $(a, 1)$ and $(1, b)$ of $A \times B$ commute and deduce that the order of $(a, b)$ is the least common multiple of $|a|$ and $|b|$.

\begin{proof}
      To show that $(a, 1)$ and $(1, b)$ commute, we note that:
      \begin{equation*}
            (a, 1)(1, b) = (a \star 1, 1 \diamond b) = (a, b) = (1 \star a, b \diamond 1) = (1, b)(a, 1).
      \end{equation*}

      Suppose that $|a| = n, |b| = m$, with $n$ and $m$ both positive integers (if one is infinite then the element $(a, b)$ of $A \times b$ obviously has infinite order). If we let $k$ be the least common multiple of $n$ and $m$, then $(a, b)^k = (a^k, b^k) = (1, 1)$, the identity element of $A \times B$ (since $k$ is a multiple of both $n$ and $m$). Further, $(a, b)^j = (a^j, b^j) \neq (1,1)$ for any $j < k$: If $a^j = 1$, then $b^j \neq 1$ (and vice- versa), or else $j$ would be the least common multiple of $n$ and $m$.
\end{proof}

\section*{31. (1/17/23)}

Prove that any finite group $G$ of even order contains an element of order $2$.

\begin{proof}
      Let $t(G) = \{g \in G\hspace{0.5ex}|\hspace{0.5ex}g \neq g^{-1}\}$. For all $x \in t(G)$, $x \neq x^{-1} \Rightarrow x^2 \neq 1$, that is, $t(G)$ is a subset of $G$ consisting of elements of order not equal to $2$. Also, $1 \notin t(G)$ (since $1 = 1^{-1}$), so $|G| > |t(G)|$.
      
      Let $x \in t(G)$. Then $x \neq x^{-1}$. Since $x = (x^{-1})^{-1}$, we also have $x^{-1} \neq (x^{-1})^{-1}$, and so $x^{-1} \in t(G)$. For every element in $t(G)$, its inverse must also be in $t(G)$. Because (from above), $t(G)$ cannot contain the identity element, its order must be even. The order of $G$ is even, and since the difference of two even numbers is also even, the order of $G - t(G)$ is even as well.

      Now since the order of $t(G)$ is both even and strictly less than that of $G$, we know that $G$ contains (at least) $2$ elements not in $t(G)$, namely, the identity and some other element whose order is $2$. Thus any finite group $G$ of even order contains an element of order $2$.
\end{proof}

\section*{32. (1/22/23)}

If $x$ is an element of finite order $n$ in $G$, prove that the elements $1, x, x^2, ..., x^{n - 1}$ are all distinct. Deduce that $|x| \leq |G|$.

\begin{proof}
      Let $x \in G$ with $x^n = 1$. Suppose for some $k < m < n$, we have $x^m = x^k$. Then $x^m = x^k \Rightarrow x^m x^{-k} = 1 \Rightarrow x^{m - k} = 1$. Since $m - k < n$, this contradicts $x$ having order $n$. Therefore for no two elements $x^m$ and $x^k$, with $m$ and $k$ less than $n$, are those elements equal.

      If $|G|$ is infinite, then the order of $x$ is by definition less than that of $G$. Suppose $|G| = p$, and that $|x| = n > p$. Then the cyclic subgroup generated by $x$, $\{x^k\hspace{0.5ex}|\hspace{0.5ex}0 \leq k < n\}$, which has $n$ distinct elements and is a subset of $G$, contains more elements than $G$'s $p$ elements, a contradiction. Therefore the order of $x$ must be no greater than $|G|$.
\end{proof}

\section*{33. (1/22/23)}

Let $x$ be an element of finite order $n$ in $G$.

\begin{enumerate}[label=(\alph*)]
      \item Prove that if $n$ is odd then $x^i \neq x^{-i}$ for all $i = 1, 2, ..., n - 1$.
            \begin{proof}
                  Consider the smallest even $k$ such that $x^k = 1$. The order of $x$ is $n$, so $k > n$. And since $x^{2n} = x^n x^n = 1 \cdot 1 = 1$, $k$ is at most $2n$. Suppose $n < k < 2n$. Then we have $x^{2n} = x^k x^{2n - k}$. We know that $x^{2n}$ and $x^k$ are both the identity, so it follows that $1 = x^{2n - k}$. However, since $k > n$, $2n - k < 2n - n = n$, which contradicts $|x| = n$. Therefore $k$ cannot be less than $2n$, and so $k = 2n$ is the smallest even power of $x$ equaling identity.

                  Note that if $x^i = x^{-i}$, then $x^{2i} = 1$. However, from above, the smallest possible value of $i$ for this to occur is $n$. That is, for no $1 \leq i < n$ do we have $x^{2i} = 1$, and therefore $x^i \neq x^{-i}$ for all such values of $i$.
            \end{proof}
      \item Prove that if $n = 2k$ and $1 \leq i < n$ then $x^i = x^{-i}$ if and only if $i = k$.
            \begin{proof}
                  Let $|x| = n = 2k$ and let $1 \leq i < n$. First, in order to show that $i = k$, let $x^i = x^{-i} \Rightarrow x^{2i} = 1$. Suppose that $i \neq k$. Because $|x| = 2k$, we cannot have $i < k$, or else $x^{2i} = 1$ would be a contradiction. So we must have $k < i < n$. Additionally, we have $2k = n \Rightarrow 2i > n$. $x^n = 1$ implies that $x^{-n} = 1$, so we see that $x^{2i} x^{-n} = x^{2i - n} = 1$. By assumption,
                  \begin{equation*}
                        k = \frac{n}{2} < i < n \Rightarrow n < 2i < 2n \Rightarrow 0 < 2i - n < n.
                  \end{equation*}
                  Thus $2i - n$ is a positive integer less than $n$ such that $x^{2i - n} = 1$, a contradiction. Therefore $i = k$.

                  Next, in order to show that $x^i = x^{-i}$, let $i = k$. Then we have $x^n = x^{2k} = x^{2i} = 1$. Multiplying both sides by $x^{-i}$, it follows that $x^i = x^{-i}$.
            \end{proof}
\end{enumerate}

\section*{34. (1/22/23)}

If $x$ is an element of infinite order in $G$, prove that the elements $x^n, n \in \mathbb{Z}$ are all distinct.

\begin{proof}
      Toward contradiction, suppose that for some $m > n \in \mathbb{Z}$, $x^m = x^n$. Then $x^m x^{-n} = 1 \Rightarrow x^{m - n} = 1$. Since $m \neq n$, $m - n$ is a positive integer such that $x^{m - n} = 1$, and so $|x|$ is an integer greater than or equal to $m - n$, contradicting $x$ having infinite order. Therefore the elements $x^n, n \in \mathbb{Z}$ are all distinct.
\end{proof}

\section*{35. (1/22/23)}

If $x$ is an element of finite order $n$ in $G$, use the Division Algorithm to show that \emph{any} integral power of $x$ equals one of the elements in the set $\{1, x, x^2, ..., x^{n - 1} \}$.

\begin{proof}
      Let $k > n$. From the Division Algorithm, there are unique $q, r \in \mathbb{Z}$ such that $k = qn + r$ and $0 \leq r < n$. Now:
      \begin{equation*}
            x^k = x^{qn + r} = x^{qn} x^r = (x^n)^q x^r = 1^q x^r = x^r,
      \end{equation*}
      and since $0 \leq r < n$, $x^r = x^k$ is an element of the cyclic subgroup of $G$ generated by $x$. Therefore \emph{any} integral power of $x$ is contained in its cyclic subgroup.
\end{proof}

\section*{36. (1/22/23)}

Assume $G = \{1, a, b, c\}$ is a group of order 4 with identity 1. Assume also that $G$ has no elements of order 4 (so by Exercise 32, every element has order $\leq 3$). Use the cancellation laws to show that there is a unique group table for $G$. Deduce that $G$ is abelian.

\begin{proof}
      Suppose, toward contradiction (and without loss of generality), that $ab \neq ba$. We know that $a$ and $b$ are both distinct from $1$. If $ab$ equals either $a$ or $b$, this is a contradiction, since it implies that either $b$ or $a$ is 1, respectively (the same holds for $ba$). Therefore we must have either $ab = c$ or $ab = 1$. Suppose that $ab = c$. Then, since $ab \neq ba$, it follows that $ba = 1$. But then $b = a^{-1}$, and so $ab = c \Rightarrow a a^{-1}  = c \Rightarrow 1 = c$, a contradiction. Therefore we must have $ab = ba$. The same logic holds for any pair among $a, b$, and $c$, and so $G$ is an abelian group.
\end{proof}

\end{document}