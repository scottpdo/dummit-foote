\documentclass{article}

\title{Dummit \& Foote Ch. 1.3: Symmetric Groups}
\author{Scott Donaldson}
\date{Feb. - Mar. 2023}
\usepackage{amsmath, amsthm, amsfonts, enumitem}

\begin{document}

\maketitle

\section*{1. (2/16/23)}

\begin{itemize}[itemsep=0em]
    \item $\sigma: (1, 3, 5)(2, 4)$
    \item $\tau: (1, 5)(2, 3)$
    \item $\sigma^2: (1, 5, 3)$
    \item $\sigma \tau: (2, 5, 3, 4)$
    \item $\tau \sigma: (1, 2, 4, 3)$
    \item $\tau^2 \sigma: (1, 3, 5)(2, 4)$ (because $\tau^2 = 1$, so $\tau^2 \sigma = \sigma$)
\end{itemize}

\section*{2. (2/16/23)}

\begin{itemize}[itemsep=0em]
    \item $\sigma: (1,13,5,10)(3,15,8)(4,14,11,7,12,9)$
    \item $\tau: (1, 14)(2, 9, 15, 13, 4)(3, 10)(5, 12, 7)(8, 11)$
    \item $\sigma^2: (1, 5)(3, 8, 15)(4, 11, 12)(7, 9, 4)(10, 13)$
    \item $\sigma \tau: (1, 11, 3)(2, 4)(5, 9, 8, 7, 10, 15)(13, 14)$
    \item $\tau \sigma: (1, 4)(2, 9)(3, 13, 12, 15, 11, 5)(8, 10, 14)$
    \item $\tau^2 \sigma: (1, 2, 15, 8, 3, 4, 14, 11, 12, 13, 7, 5, 10)$
\end{itemize}

\section*{3. (2/16/23)}

Compute the order of each of the permutations whose cycle decompositions were computed above.

\begin{enumerate}[itemsep=0em]
    \item $|\sigma| = 6$; $|\tau| = 2$; $|\sigma^2| = 3$; $|\sigma \tau| = 4$; $|\tau \sigma| = 4$; $|\tau^2 \sigma| = 6$
    \item $|\sigma| = 12$; $|\tau| = 30$; $|\sigma^2| = 6$; $|\sigma \tau| = 6$; $|\tau \sigma| = 6$; $|\tau^2 \sigma| = 13$
\end{enumerate}

\section*{4. (2/16/23)}

Compute the order of each of the elements in the following groups:

\begin{enumerate}[label=(\alph*)]
    \item $S_3$
        \begin{itemize}[itemsep=0em]
            \item $(1)$: 1
            \item $(1, 2); (1, 3); (2, 3)$: 2
            \item $(1, 2, 3); (1, 3, 2)$: 3
        \end{itemize}
    \item $S_4$
        \begin{itemize}[itemsep=0em]
            \item $(1)$: 1
            \item $(1, 2); (1, 3); (1, 4); (2, 3); (2, 4); (3, 4); (1, 2)(3, 4);\\(1, 3)(2, 4); (1, 4)(2, 3)$: 2
            \item $(1, 2, 3); (1, 3, 2); (1, 2, 4); (1, 4, 2); (1, 3, 4); (1, 4, 3);\\(2, 3, 4); (2, 4, 3)$: 3
            \item $(1, 2, 3, 4); (1, 4, 2, 3); (1, 3, 2, 4); (1, 3, 4, 2); (1, 4, 2, 3); (1, 4, 3, 2)$: 4
        \end{itemize}
\end{enumerate}

\section*{5. (2/16/23)}

Find the order of $(1, 12, 8, 10, 4)(2, 13)(5, 11, 7)(6, 9)$.

\begin{proof}
    The order of a permutation in a symmetric group is the least common multiple of its cycles. However, since we have not yet proven this, we will calculate the first few multiples of the permutation manually, and extrapolate from there. Let $\sigma = (1, 12, 8, 10, 4)(2, 13)(5, 11, 7)(6, 9)$.

    $\sigma^2 = (1, 8, 4, 12, 10)(5, 7, 11)$.

    $\sigma^3 = (1, 10, 12, 4, 8)(2, 13)(6, 9)$.

    $\sigma^4 = (1, 4, 10, 8, 12)(5, 11, 7)$.
    
    $\sigma^5 = (2, 13)(5, 7, 11)(6, 9)$.

    From this pattern, we see that each constituent cycle vanishes from the cycle decomposition when the exponent is a multiple of the cycle's length. Thus, the order of $\sigma$ is the least common multiple of the lengths of its cycles, which is $2 \cdot 3 \cdot 5 = 30$.
\end{proof}

\section*{6. (2/17/23)}

Write out the cycle decomposition of each element of order 4 in $S_4$.

\begin{itemize}[itemsep=0em]
    \item $(1, 2, 3, 4)$
    \item $(1, 2, 4, 3)$
    \item $(1, 3, 2, 4)$
    \item $(1, 3, 4, 2)$
    \item $(1, 4, 2, 3)$
    \item $(1, 4, 3, 2)$
\end{itemize}

\section*{7. (2/20/23)}

Write out the cycle decomposition of each element of order 2 in $S_4$.

\begin{itemize}[itemsep=0em]
    \item $(1, 2)$
    \item $(1, 3)$
    \item $(1, 4)$
    \item $(2, 3)$
    \item $(2, 4)$
    \item $(3, 4)$
    \item $(1, 2)(3, 4)$
    \item $(1, 3)(2, 4)$
    \item $(1, 4)(2, 3)$
\end{itemize}

\section*{8. (2/22/23)}

Prove that if $\Omega = \{1, 2, 3... \}$ then $S_\Omega$ is an infinite group.

\begin{proof}

    Let $\Omega = \{1, 2, 3... \}$. Consider the subset $T$ consisting of all elements whose cycle decomposition is a single 2-cycle permuting $1 \in \Omega$, for example $(1, 2), (1, 10)$ but not $(2, 3)$.

    There is a bijection $f: \mathbb{Z}^+ \rightarrow T$ defined by $f(n) = (1, n + 1)$. Because there is a bijection between these two sets, they have the same cardinality; that is, like $\mathbb{Z}^+$, $T$ is infinite.

    Because $\Omega$ contains a proper subset of infinite size, $\Omega$ has infinite elements and is therefore an infinite group.

\end{proof}

\section*{9. (2/22/23)}

\begin{enumerate}[label=(\alph*)]
    \item Let $\sigma$ be the 12-cycle $(1, 2, 3, 4, 5, 6, 7, 8, 9, 10, 11, 12)$. For which positive integers $i$ is $\sigma^i$ also a 12-cycle?
        \begin{itemize}[itemsep=0em]
            \item $\sigma^5 = (1, 6, 11, 4, 9, 2, 7, 12, 5, 10, 3, 8)$
            \item $\sigma^7 = (1, 8, 3, 10, 5, 12, 7, 2, 9, 4, 11, 6)$
            \item $\sigma^11 = (1, 12, 11, 10, ..., 2)$
        \end{itemize}
    \item Let $\tau$ be the 8-cycle $(1, 2, 3, 4, 5, 6, 7, 8)$. For which positive integers $i$ is $\tau^i$ also a 12-cycle?
        \begin{itemize}[itemsep=0em]
            \item $\tau^3 = (1, 4, 7, 2, 5, 8, 3, 6)$
            \item $\tau^5 = (1, 6, 3, 8, 5, 2, 7, 4)$
            \item $\tau^7 = (1, 8, 7, 6, 5, 4, 3, 2)$
        \end{itemize}
    \item Let $\omega$ be the 14-cycle $(1, 2, 3, 4, 5, 6, 7, 8, 9, 10, 11, 12, 13, 14)$. For which positive integers $i$ is $\omega^i$ also a 12-cycle?
        \begin{itemize}[itemsep=0em]
            \item $\omega^3 = (1, 4, 7, 10, 13, 2, 5, 8, 11, 14, 3, 6, 9, 12)$
            \item $\omega^5 = (1, 6, 11, 2, 7, 12, 3, 8, 13, 4, 9, 14, 5, 10)$
            \item $\omega^9 = (1, 10, 5, 14, 9, 4, 13, 8, 3, 12, 7, 2, 11, 6)$
            \item $\omega^{11} = (1, 12, 9, 6, 3, 14, 11, 8, 5, 2, 13, 10, 7, 4)$
            \item $\omega^{13} = (1, 14, 13, 12, ..., 2)$
        \end{itemize}
\end{enumerate}

\section*{10. (2/23/23)}

Prove that if $\sigma$ is the $m$-cycle $(a_1, a_2, ..., a_m)$, then for all $i \in \{1, 2, 3, ..., m \},\\
\sigma^i(a_k) = a_{k + i}$, where $k + i$ is replaced by its least residue mod $m$ when $k + i > m$. Deduce that $|\sigma| = m$.

\begin{proof}
    We will prove this by induction on $i$. For the base case, $i = 1$, we have, by definition, $\sigma^1 = \sigma$ and $\sigma(a_k) = a_{k + 1}$ for $k < m$. For $m$, $\sigma(a_m) = a_1$, and since $1 = (m + 1)$ mod $m$, this holds for all $i \in \{1, 2, 3, ..., m \}$.

    For the induction case, suppose that for some $n \in \{1, 2, 3, ..., m \}$, $\sigma^n(a_k) = a_{k + n}$ (where $k + n$ is assumed mod $m$) for all valid $k$. Consider $\sigma^{n + 1} = \sigma^n \sigma^1 = \sigma^n \sigma$. For an arbitrary element $a_k$, then, $\sigma^{n + 1}(a_k) = \sigma^n(\sigma(a_k)) = \sigma^n(a_{k + 1})$ (by the base case), which equals $a_{k + n + 1}$ (by the induction hypothesis). Therefore, $\sigma^{n + 1}(a_k) = a_{k + (n + 1)}$.

    Thus, by induction, $\sigma^i(a_k) = a_{k + i}$ for all $i \in \{1, 2, 3, ..., m \}$. It follows that $\sigma^m(a_k) = a_{k + m} = a_{k}$ for all $k$, so $\sigma^m = 1$. Therefore, $|\sigma| = m$.
\end{proof}

\section*{11. (3/12/23)}

Let $\sigma$ be the $m$-cycle $(1, 2, ..., m)$. Show that $\sigma^i$ is also an $m$-cycle if and only if $i$ is relatively prime to $m$.

\begin{proof}
    First, we will show that if $i$ and $m$ are relatively prime, then $\sigma^i$ is also an $m$-cycle. Let $\sigma$ be the $m$-cycle $(1, 2, ..., m)$ and let $i$ be relatively prime to $m$.

    From 10., we know that $\sigma^i(k) = k + i$ (mod $m$). So $\sigma^i$ is the permutation which sends 1 to $1 + i$, $1 + i$ to $1 + 2i$, and so on. We can represent it with the cycle decomposition $(1, 1 + i, 1 + 2i, ...)$ (all mod $m$). It may be the case that this includes multiple disjoint cycles -- we have not yet proven that $\sigma^i$ can be represented by a single cycle. However, every element of the first cycle in its cycle decomposition can be represented as $1 + ki$, with $0 \leq k < m$. The final value in the first cycle is $1 + ki$ such that $1 + (k + 1)i = 1$. For this to occur, because $i$ and $m$ are relatively prime, $m$ must divide $k + 1$, and so $m \leq k + 1$. In fact, we cannot have $m < k + 1$, because when the sequence arrives at $1 + mi$, this is equal to 1. So $m = k + 1$; that is, $\sigma^i$ is also an $m$-cycle.

    Next, we will prove the contrapositive: Namely, that if $i$ and $m$ are not relatively prime, then $\sigma^i$ is not an $m$-cycle. Without loss of generality, let $\sigma$ be the $m$-cycle $(1, 2, ..., m)$ and let $j = $ gcd$\{i, m\} > 1$. Again from 10., $\sigma^i(k) = k + i$ (mod $m$). The first (and possibly only) cycle of the cycle decomposition of $\sigma^i$ is $(1, 1 + i, 1 + 2i, ...)$ (all mod $m$). The last element of this cycle decomposition is $1 + (\frac{m}{j} - 1)n$. Its successor, $1 + \frac{mn}{j}$, is equal to 1 mod $m$, because $m$ divides $\frac{mn}{j}$. In fact, $\frac{mn}{j}$ is the least common multiple of $m$ and $n$ (because $j$ is the greatest common divisor of $m$ and $n$). It follows that $\frac{m}{j} - 1$ is the smallest candidate integer for the coefficient of $n$ in the final element of the first cycle. Therefore, $\sigma^i$ contains a cycle of length $\frac{m}{j} < m$, and so it is not an $m$-cycle. Further, $\sigma^i$ contains exactly $j$ disjoint cycles of length $\frac{m}{j}$.
\end{proof}

\section*{12. (3/12/23)}

\begin{enumerate}[label=(\alph*)]
    \item If $\tau = (1, 2)(3, 4)(5, 6)(7, 8)(9, 10)$ determine whether there is an $n$-cycle $\sigma$ ($n \geq 10$) with $\tau = \sigma^k$ for some integer $k$.

          Let $\sigma$ be the 10-cycle $(1, 3, 5, 7, 9, 2, 4, 6, 8, 10)$.

          $\sigma^2 = (1, 5, 9, 4, 8)(2, 6, 10, 3, 7)$.

          $\sigma^3 = (1, 7, 4, 10, 5, 2, 8, 3, 9, 6)$.

          $\sigma^4 = (1, 9, 8, 5, 4)(2, 10, 7, 6, 3)$, and

          $\sigma^5 = (1, 2)(3, 4)(5, 6)(7, 8)(9, 10)$.
    \item If $\tau = (1, 2)(3, 4, 5)$ determine whether there is an $n$-cycle $\sigma$ ($n \geq 5$) with $\tau = \sigma^k$ for some integer $k$.

          From 11., we know that if $\sigma$ is a 5-cycle, then $\sigma^k$ must be another 5-cycle (because 5 is prime). And, if $\sigma$ is an $n$-cycle with $n > 5$, then its cycle decomposition contains disjoint cycles of equal length. Because $\tau = (1, 2)(3, 4, 5)$ is the product of a 2-cycle and a 3-cycle, there are no $n$-cycles $\sigma$ with $n > 5$ and $\tau = \sigma^k$.
\end{enumerate}

\section*{13. (3/12/23)}

Show that an element has order 2 in $S_n$ if and only if its cycle decomposition is a product of commuting 2-cycles.

\begin{proof}
    Let $\sigma \in S_n$. First, we will prove that if $\sigma$'s cycle decomposition is a product of commuting 2-cycles, then $\sigma$ has order 2. Suppose $\sigma$'s cycle decomposition can be written $(a_1, a_2)(a_3, a_4)...(a_k, a_{k + 1})$ (with $k$ even). 
    
    Consider $\sigma^2(a_j)$ for $a_j \in \{a_1, a_2, ..., a_k, a_{k + 1}\}$. Generally, $\sigma(a_j) = a_{j + 1}$ if $j$ is odd, and $a_{j - 1}$ if $j$ is even. Then, if $j$ is odd, $\sigma^2(a_j) = \sigma(\sigma(a_j)) = \sigma(a_{j + 1}) = a_j$ (because $j + 1$ is even). And, if $j$ is even, $\sigma^2(a_j) = \sigma(\sigma(a_j)) = \sigma(a_{j - 1}) = a_j$ (because $j - 1$ is odd). $\sigma^2$ assigns every $a_j \in \{a_1, a_2, ..., a_k, a_{k + 1}\}$ to itself, and every element outside to itself, and is therefore the identity permutation. Thus $|\sigma| = 2$.

    Next, let $|\sigma| = 2$, so $\sigma^2$ is the identity permutation. Let $(a_1, a_2, ..., a_k)$ be a cycle in $\sigma$'s cycle decomposition. If $k > 2$, then $\sigma^2(a_1) = \sigma(\sigma(a_1)) = \sigma(a_2) = a_3$. However, this implies that $\sigma^2$ is not the identity, a contradiction. Therefore we must have $k \leq 2$. But if $k = 1$, then $\sigma$ is itself the identity. Thus $\sigma$'s cycle decomposition is a product of commuting 2-cycles.
\end{proof}

\section*{14. (3/12/23)}

Let $p$ be prime. Show that an element has order $p$ in $S_n$ if and only if its cycle decomposition is a product of commuting $p$-cycles. Show by an explicit example that this need not be the case if $p$ is not prime.

\begin{proof}
    First, to show that if an element in $S_n$ has order $p$, it is the product of commuting $p$-cycles, let $\sigma \in S_n$ with $|sigma| = p$. Let $(a_1, a_2, ..., a_k)$ be a cycle in $\sigma$'s cycle decomposition. From 10., $\sigma^p(a_i) = a_{i + p}$ (with $i + p$ mod $k$). Since $\sigma^p = 1$, we must have $i + p$ mod $k$ = $i$, so $p$ mod $k = 0$. That is, $k$ divides $p$. Now since $p$ is prime, $k$ must be either 1 or $p$ itself. If $k = 1$, then $\sigma$ is the identity, which has order $1$. Therefore we must have $k = p$. Since an arbitrary cycle of $\sigma$ is a $p$-cycle, $\sigma$ must be a product of commuting $p$-cycles.

    Next, let $\sigma \in S_n$ have a cycle decomposition which is a product of commuting $p$-cycles. Consider an arbitrary cycle $(a_1, a_2, ..., a_p)$. From 10., this cycle has order $p$. So every such cycle in $\sigma$ has order $p$. Thus $\sigma^p$ is the product of 1-cycles, and so is the identity permutation. Therefore $|\sigma| = p$.
\end{proof}

\section*{15. (3/14/23, revised 10/6/23)}

Prove that the order of an element in $S_n$ equals the least common multiple of the lengths of the cycles in its cycle decomposition.

\begin{proof}
    Let $\sigma \in S_n$. We write the cycle decomposition of $\sigma = \sigma_1 \sigma_2 ... \sigma_m$, where each $\sigma_i$ is a single cycle. We will show by induction that the order of $\sigma$ is equal to the least common multiple of the lengths of the cycles in its cycle decomposition.

    If $\sigma \in S_n$ is itself a single cycle, then from 10., its order is equal to its length.

    For the induction step, suppose now that every permutation with a cycle decomposition containing $m$ disjoint cycles has order equal to the least common multiple of the lengths of its cycles. Let $\sigma' \in S_n$ be a permutation whose cycle decomposition contains $m + 1$ cycles. We can write $\sigma' = \sigma \sigma_{m + 1}$, where $\sigma$ is a permutation whose cycle decomposition contains $m$ cycles. Let us suppose $|\sigma| = k$ and $|\sigma_{m + 1}| = j$ (equivalently, $\sigma_{m + 1}$ has length $j$), and let $r$ be the least common multiple of $k$ and $j$. Then:
    \begin{equation*}
        \sigma^r = (\sigma \sigma_{m + 1})^r = \sigma^r \sigma_{m + 1}^r = \underbrace{\sigma^{q_1k} \sigma_{m + 1}^{q_2k}}_{\text{for some $q_1, q_2 \in \mathbb{Z}^+$}} = (1),
    \end{equation*}
    that is, the order of $\sigma$ divides $r$. For no positive integer $i$ smaller than $r$ do we have $\sigma^i = 1$ \emph{and} $\sigma_{m + 1}^i = 1$, or else $i$ would be a multiple of both $k$ and $j$, contradicting $r$ being their least common multiple. This shows that, if all permutations whose cycle decompositions consist of $m$ cycles have order equal to the least common multiple of the lengths of their cycles, then all permutations of $m + 1$ cycles do as well.

    By induction, this completes the proof that all permutations in $S_n$ have order equal to the least common multiple of the lengths of the cycles in their cycle decompositions.
\end{proof}

\section*{16. (3/15/23)}

Show that if $n \geq m$ then the number of $m$-cycles in $S_n$ is given by

\begin{equation*}
    \frac{n(n - 1)(n - 2)...(n - m + 1)}{m}.
\end{equation*}
    
\begin{proof}
    Let $\Omega = \{1, 2, 3, ..., n\}$, let $S_n$ be the symmetric group on $\Omega$, and let $\sigma \in S_n$ be a permutation on elements of $\Omega$ with a cycle decomposition that is a single $m$-cycle with $m \leq n$. To count the ways of forming $\sigma$, first consider all $m$-cycles. There are $n$ choices for the first element, $n - 1$ for the second, $n - 2$ for the third, and so on, so that there are $n(n - 1)(n - 2)...(n - m + 1)$ ways to form an $m$-cycle in general.

    However, this ignore the fact that many $m$-cycles represent the same $\sigma$, for example $(1, 2, 3) = (2, 3, 1)$. Let $\sigma = (a_1, a_2, ..., a_m)$ be an $m$-cycle. The cycles $(a_2, a_3, ..., a_m, a_1), (a_3, a_4, ..., a_m, a_1, a_2), ..., (a_m, a_1, ..., a_{m - 1})$ all represent the same permutation as $\sigma$. That is, there are $m$ possible ways of representing an arbitrary $m$-cycle, and therefore the number of unique $m$-cycles in $S_n$ is $n(n - 1)(n - 2)...(n - m + 1)/m$.
\end{proof}

\section*{17. (3/15/23)}

Show that if $n \geq 4$ then the number of representations in $S_n$ which are the product of two disjoint 2-cycles is $n(n - 1)(n - 2)(n - 3)/8$.

\begin{proof}
    Let $\Omega = \{1, 2, 3, ..., n\}$, let $S_n$ be the symmetric group on $\Omega$, and let $\sigma \in S_n$ be a permutation on elements of $\Omega$ with a cycle decomposition that is a product of two disjoint 2-cycles. Consider forming a single 2-cycle. From 16., there are $n(n - 1)/2$ ways of forming an arbitrary 2-cycle. For the next 2-cycle, there are $n - 2$ elements remaining, so there are $(n - 2)(n - 3)/2$ ways of forming the second 2-cycle, and so there are $n(n - 1) / 2 \cdot (n - 2)(n - 3) / 2 = n(n - 1)(n - 2)(n - 3) / 4$ ways of forming a pair of 2-cycles.

    However, this ignores duplicates such as $(1, 2)(3, 4) = (3, 4)(1, 2)$. From 16., duplicates such as $(1, 2)(3, 4) = (2, 1)(4, 3)$ have already been discounted, but for each pair of 2-cycles, we need to discount the cycle decomposition with a representation that swaps the order of the 2-cycles. That is, $n(n - 1)(n - 2)(n - 3) / 4$ is double the number of unique permutations, so the number of representations is $n(n - 1)(n - 2)(n - 3) / 8$.
\end{proof}

\section*{18. (3/15/23)}

Find all numbers $n$ such that $S_5$ contains an element of order $n$.

\begin{itemize}[itemsep=0em]
    \item $1: 1$
    \item $2: (1, 2)$
    \item $3: (1, 2, 3)$
    \item $4: (1, 2, 3, 4)$
    \item $5: (1, 2, 3, 4, 5)$
    \item $6: (1, 2)(3, 4, 5)$
\end{itemize}

There are no elements with an order greater than 6, because the least common multiple of the lengths of any number of disjoint cycles out of only 5 elements does not exceed 6.

\section*{19. (3/15/23)}

Find all numbers $n$ such that $S_7$ contains an element of order $n$.

\begin{itemize}[itemsep=0em]
    \item $1-7:$ Cycles of the form $(1, ... n)$ for $n \in \{1, 2, ..., 7\}$
    \item $8: (1, 2, 3, 4)(5, 6)$
    \item $10: (1, 2, 3, 4, 5)(6, 7)$
    \item $12: (1, 2, 3, 4)(5, 6, 7)$
\end{itemize}

There are no elements with an order greater than 12, because the least common multiple of the lengths of any number of disjoint cycles out of only 7 elements does not exceed 12.

\section*{20. (3/15/23)}

Find a set of generators and relations for $S_3$.

\begin{proof}
    $S_3$ = $\{1, (1, 2), (1, 3), (2, 3), (1, 2, 3), (1, 3, 2)\}$. A set of generators and relations for $S_3$ are given by $\langle a, b \hspace{0.5ex}|\hspace{0.5ex} a^2 = b^3 = 1 \rangle$, where $a = (1, 2)$ and $b = (1, 2, 3)$.

    Before substituting, we see that the group generated by $a, b$, and the relations $a^2 = b^3 = 1$ consists of the elements $1, a, b, ab, ba, b^2$.

    Now $(2, 3) = (1, 2)(1, 2, 3) = ab$. $(1, 3) = (1, 2, 3)(1, 2) = ba$. And finally, $(1, 3, 2) = (1, 2, 3)(1, 2, 3) = b^2$. So the group $S_3$ is indeed generated by $(1, 2), (1, 2, 3)$ and the relations $(1, 2)^2 = (1, 2, 3)^3 = 1$.

    Further, this is the same group structure as the dihedral group $D_6$.
\end{proof}

\end{document}