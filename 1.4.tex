\documentclass{article}

\title{Dummit \& Foote Ch. 1.4: Matrix Groups}
\author{Scott Donaldson}
\date{Mar. 2023}
\usepackage{amsmath, amsthm, amsfonts, enumitem}

\begin{document}

\maketitle

\section*{1. (3/16/23)}

Prove that $|GL_2(\mathbb{F}_2)| = 6$.

\begin{proof}
    Matrices in $GL_2(\mathbb{F}_2)$ have the form 
    $\begin{pmatrix}a & b\\c & d\end{pmatrix}$, 
    where $a, b, c, d \in \{0, 1\}$. There are 16 possible matrices of this form (2 options for each entry over 4 entries, $2^4 = 16$).
    
    From the definition of $GL_2$, we discount matrices with determinant 0. A $2 \times 2$ matrix has determinant 0 when $ad - bc = 0$, that is, $ad = bc$. This happens only when $ad = bc = 1$ or $ad = bc = 0$. There is only one matrix where $ad = bc = 1, \begin{pmatrix}1 & 1\\1 & 1\end{pmatrix}$. Matrices with determinant 0 have one of $a, d$ and $b, c$ equal to 0. They are the matrices with all zero entries (1), with three zero entries (4), and with two zero entries ($a$ and $b$, or $a$ and $c$, or $b$ and $d$, or $c$ and $d$) (4).

    This leaves us with $16 - 1 - 1 - 4 - 4 = 6$ matrices with nonzero determinants, so the order of $GL_2(\mathbb{F}_2) = 6$.
\end{proof}

\section*{2. (3/16/23)}

Write out all the elements of $GL_2(\mathbb{F}_2)$ and compute the order of each element.

\begin{itemize}
    \item $\begin{pmatrix}1 & 0\\0 & 1\end{pmatrix}$: 1 (identity)
    \item $\begin{pmatrix}1 & 1\\0 & 1\end{pmatrix}$: 2
    \item $\begin{pmatrix}1 & 0\\1 & 1\end{pmatrix}$: 2
    \item $\begin{pmatrix}0 & 1\\1 & 1\end{pmatrix}$: 3
    \item $\begin{pmatrix}1 & 1\\1 & 0\end{pmatrix}$: 3
    \item $\begin{pmatrix}0 & 1\\1 & 0\end{pmatrix}$: 2
\end{itemize}

\section*{3. (3/16/23)}

Show that $GL_2(\mathbb{F}_2)$ is non-abelian.

\begin{proof}
    To prove that $GL_2(\mathbb{F}_2)$ is non-abelian, we need only show that it contains two non-commuting elements.

    $\begin{pmatrix}0 & 1\\1 & 1\end{pmatrix} \times \begin{pmatrix}0 & 1\\1 & 0\end{pmatrix} = \begin{pmatrix}1 & 0\\1 & 1\end{pmatrix}$.

    However, $\begin{pmatrix}0 & 1\\1 & 0\end{pmatrix} \times \begin{pmatrix}0 & 1\\1 & 1\end{pmatrix} = \begin{pmatrix}1 & 1\\0 & 1\end{pmatrix}$. These products are not equal, so $GL_2(\mathbb{F}_2)$ is non-abelian.
\end{proof}

\section*{4. (3/18/23)}

Show that if $n$ is not prime then $\mathbb{Z}/n\mathbb{Z}$ is not a field.

\begin{proof}
    Let $n$ be a composite positive integer and let $a$ divide $n$ with $a > 1$. We will show that $a$ does not have a multiplicative inverse in $\mathbb{Z}/n\mathbb{Z}$, and therefore $\mathbb{Z}/n\mathbb{Z}$ is not a field.

    We will show that there is no integer $c$ such that $ac = 1$ mod $n$. Since $a$ divides $n$, let $ab = n = 0$ mod $n$. So $a(b + 1) = ab + a = n + a = a$ mod $n$. That is, for the pair of consecutive integers $b$ and $b + 1$, we have $ab = 0 < 1$ and $a(b + 1) = a > 1$. Then there is no integer $c$ strictly between $b$ and $b + 1$ such that $ac = 1$ mod $n$. For any larger integers, we note that $abk = nk = 0$ mod $n$, and $a(bk + 1) = abk + a = nk + a = a$ mod $n$, and therefore there is no integer $c$ among all of $\mathbb{Z}^+$ with $ac = 1$. Therefore, since $a$ has no multiplicative inverse, $\mathbb{Z}/n\mathbb{Z}$ is not a field.
\end{proof}

\section*{5. (3/18/23)}

Show that $GL_n(F)$ is a finite group if and only if $F$ has a finite number of elements.

\begin{proof}
    Let $F$ be a field with $m < \infty$ elements and, for some $n > 1$, let $GL_n(F)$ be the general linear group of degree $n$ on $F$. The total possible number of $n \times n$ matrices with entries from $F$ is $m^{n^2}$. Since the number of elements in $GL_n(F)$ is at most this value, it is a finite group (in 6. we will show that it is strictly less than).

    To prove the converse, we will show that, if $F$ is an infinite field, then $GL_n(F)$ must not be a finite group. Let $F$ be an infinite field. For every $x \in F$ (excluding $x = 0$), we can construct an $n \times n$ matrix whose diagonal entries are $x$ and all other entries are 0. By definition, the determinant of such a matrix is the product of the diagonal entries, $x^n \neq 0$. Therefore such a matrix belongs to $GL_n(F)$. This is a bijection between $F$ and $GL_n(F)$, and so they have the same cardinality, that is, $GL_n(F)$ must not be a finite group.

    Thus, $GL_n(F)$ is a finite group if and only if $F$ has a finite number of elements.
\end{proof}

\section*{6. (3/19/23)}

If $|F| = q$ is finite prove that $|GL_n(F)| < q^{n^2}$.

\begin{proof}
    An element of $GL_n(F)$ is an invertible $n \times n$ matrix whose entries come from $F$. For each entry, there are $q$ possibilities, and there are $n^2$ total entries, so there are $q^{n^2}$ possible such matrices (before discounting those with determinant = 0). It is guaranteed that some number of $n \times n$ matrices have determinant 0; for example, the matrix whose entries are all 0 obviously has determinant 0. So the number of elements of $GL_n(F)$ is always strictly less than $q^{n^2}$.
\end{proof}

\section*{7. (3/19/23)}

Let $p$ be a prime. Prove that the order of $GL_2(\mathbb{F}_p)$ is $p^4 - p^3 - p^2 + p$.

\begin{proof}
    From 5. and 6., there are $p^{2^2} = p^4$ possible $2 \times 2$ matrices, and the order of $GL_2(\mathbb{F}_p)$ is strictly less than this number. Let us count the ways in which an element of $GL_2(\mathbb{F}_p)$ might have a determinant equal to 0.

    A $2 \times 2$ matrix in $GL_2(\mathbb{F}_p)$ has the form $\begin{pmatrix}a & b\\c & d\end{pmatrix}$, with $a, b, c, d \in F_p$. The determinant of a $2 \times 2$ matrix is $ad - bc$. First, consider the cases in which $a, b, c, d \neq 0$. Setting the determinant equal to 0, we can see that $d$ must equal $bc / a$. So there are $p - 1$ choices for $a, b, c$, and $d$ is fixed based on the other entries. Then there are $(p - 1)^3$ matrices with 4 nonzero entries with determinant equal to 0.

    Next, consider $2 \times 2$ matrices with one entry equal to 0, for example, $\begin{pmatrix}a & b\\c & 0\end{pmatrix}$. The determinant of this matrix is $a \cdot 0 - bc = bc$. In order for this to equal 0, at least one of either $b$ or $c$ must equal zero. Then there are no matrices with exactly 1 zero entry with determinant equal to 0.

    Now consider $2 \times 2$ matrices with two entries equal to 0. Such matrices have the form $\begin{pmatrix}a & b\\0 & 0\end{pmatrix}, \begin{pmatrix}a & 0\\c & 0\end{pmatrix}, \begin{pmatrix}0 & 0\\c & d\end{pmatrix},$ or $\begin{pmatrix}0 & b\\0 & d\end{pmatrix}$. There are $p - 1$ possible choices for both of the nonzero entries, so there are $4(p - 1)^2$ matrices with exactly 2 nonzero entries with determinant equal to 0.

    Matrices with three entries equal to 0 have the form $\begin{pmatrix}a & 0\\0 & 0\end{pmatrix}, \begin{pmatrix}0 & b\\0 & 0\end{pmatrix},\newline \begin{pmatrix}0 & 0\\c & 0\end{pmatrix},$ or $\begin{pmatrix}0 & 0\\0 & d\end{pmatrix}$. There are $4(p - 1)$ such matrices.

    Finally, there is the single matrix with all 0 entries, $\begin{pmatrix}0 & 0\\0 & 0\end{pmatrix}$.

    So, the total number of elements of $GL_2(\mathbb{F}_p)$ is:
    \begin{multline*}
        p^4 - (p - 1)^3 - 4(p - 1)^2 - 4(p - 1) - 1 = \\
        p^4 - (p^3 - 3p^2 + 3p - 1) - (4p^2 - 8p + 4) - (4p - 4) - 1 = \\
        p^4 - p^3 + 3p^2 - 3p + 1 - 4p^2 + 8p - 4 - 4p + 4 - 1 = \\
        p^4 - p^3 + (3 - 4)p^2 + (-3 + 8 - 4)p + (1 - 4 + 4 - 1) = \\
        p^4 - p^3 - p^2 + p
    \end{multline*}
    as desired.
\end{proof}

\section*{8. (3/21/23)}

Show that $GL_n(F)$ is non-abelian for any $n \geq 2$ and $F$.

\begin{proof}
    To show that $GL_n(F)$ is non-abelian, we need to show that it contains two elements that are noncommutative. By definition of general linear groups, $GL_n(F)$ consists of invertible $n \times n$ matrices whose entries come from the field $F$. Further, by definition of fields, $F$ contains an additive identity 0 and a multiplicative identity 1. Therefore, if we consider only matrices in $GL_n(F)$ whose entries are 0 or 1 and whose product's entries are 0 or 1 (in $\mathbb{Z}$), these are elements of every $GL_n(F)$ regardless of which $F$ we choose.

    Let $A$ be the transpose of the identity matrix and let $B$ be equal to the identity matrix with the final two columns swapped:
    
    $A = \begin{pmatrix}
        0 & 0 & \cdots & 0 & 1\\
        0 & 0 & \cdots & 1 & 0\\
        \vdots & \vdots & \ddots & \vdots & \vdots \\
        0 & 1 & \cdots & 0 & 0\\
        1 & 0 & \cdots & 0 & 0
    \end{pmatrix}, 
    B = \begin{pmatrix}
        1 & 0 & \cdots & 0 & 0\\
        0 & 1 & \cdots & 0 & 0\\
        \vdots & \vdots & \ddots & \vdots & \vdots \\
        0 & 0 & \cdots & 0 & 1\\
        0 & 0 & \cdots & 1 & 0
    \end{pmatrix}$.

    The upper-left entry of $AB$ is the dot product of the first row of $A$ with the first column of $B$, that is, $0 \cdot 0 + 0 \cdot 0 + \ldots + 1 \cdot 1 = 1$.

    The upper-left entry of $BA$ is the dot product of the first row of $B$ with the first column of $A$, that is, $1 \cdot 0 + 0 \cdot 0 + \ldots + 0 \cdot 1 = 0$.

    Because $AB$ and $BA$ do not contain exactly the same entries, they are not equal matrices. Therefore, $A$ and $B$ do not commute. Further, because for every $n \geq 2$ and every field $F$, $GL_n(F)$ contains the elements $A$ and $B$, $GL_n(F)$ is non-abelian.
\end{proof}

\end{document}