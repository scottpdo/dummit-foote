\documentclass{article}

\title{Dummit \& Foote Ch. 2.2: Centralizers and Normalizers, Stabilizers and Kernels}
\author{Scott Donaldson}
\date{Jun. 2023}
\usepackage{amsmath, amsthm, amsfonts, enumitem}

\begin{document}

\maketitle

\section*{1. (6/5/23)}

Prove that $C_G(A) = \{ g \in G \mid g^{-1}ag = a \text{ for all } a \in A \}$.

\begin{proof}
    By definition, $C_G(A) = \{ g \in G \mid gag^{-1} = a \text{ for all } a \in A \}$ (that is, it is the set of elements of $G$ that commute with all elements of $A$).

    Let $g \in C_G(A), a \in A$. Then $gag^{-1} = a$, which implies that $ga = ag$, and so left-multiplying by $g^{-1}$ we obtain $a = g^{-1}ag$. Therefore, equivalently, $C_G(A)$ is the set of elements $g \in G$ such that $g^{-1}ag = a$ for all $a \in A$.
\end{proof}

\section*{2. (6/5/23)}

Prove that $C_G(Z(G)) = G$ and deduce that $N_G(Z(G)) = G$.

\begin{proof}
    Recall that $Z(G) = \{g \in G \mid gx = xg \text{ for all } x \in G \}$. Let $z \in Z(G)$, so $z$ commutes with every element of $G$.
    
    Also recall that $C_G(A) = \{ g \in G \mid gag^{-1} = a \text{ for all } a \in A \}$. When $A = Z(G)$, then every element of $g$ commutes with every element of $A$. Therefore for all $g \in G$, $g \in C_G(Z(G))$. Thus $C_G(Z(G)) = G$.

    Note that, since $C_G(A) \leq N_G(A)$ for all subsets $A$, we must have $G = C_G(Z(G)) \leq N_G(Z(G))$. Since there is no greater set of elements, we also have $N_G(Z(G)) = G$.
\end{proof}

\section*{3. (6/8/23)}

Prove that if $A$ and $B$ are subsets of $G$ with $A \subseteq B$ then $C_G(B)$ is a subgroup of $C_G(A)$.

\begin{proof}
    Let $a \in A \text{ and } g \in C_G(B)$. Then $g$ commutes with every element of $b$, that is, $gb = bg \Rightarrow gbg^{-1} = b$ for all $b \in B$. Since $A \subseteq B$, we also have $gag^{-1} = a$ for all $a \in A$. Therefore $g \in C_G(A)$, which implies that $C_G(B) \subseteq C_G(A)$.

    From the introduction to this chapter, centralizers are subgroups, so both $C_G(B) \leq G$ and $C_G(A) \leq G$. Since $C_G(B)$ is contained within $C_G(A)$ and both are subgroups of $G$, $C_G(B)$ must be closed within $C_G(A)$ and closed under inverses within $C_G(A)$, so it is also a subgroup of $C_G(A)$.
\end{proof}

\section*{4. (6/8/23)}

For each of $S_3$, $D_8$, and $Q_8$ compute the centralizers of each element and find the center of each group.

\subsection*{$S_3$}

\begin{itemize}
    \item $C_{S_3}((1)) = S_3$
    \item $C_{S_3}((1, 2)) = \{ (1), (1, 2) \}$
    \item $C_{S_3}((1, 3)) = \{ (1), (1, 3) \}$
    \item $C_{S_3}((2, 3)) = \{ (1), (2, 3) \}$
    \item $C_{S_3}((1, 2, 3)) = C_{S_3}((1, 3, 2)) = \{ (1), (1, 2, 3), (1, 3, 2) \}$
\end{itemize}

The center $Z(S_3)$ consists only of the identity permutation.

\subsection*{$D_8$}

\begin{itemize}
    \item $C_{D_8}(1) = D_8$
    \item $C_{D_8}(r) = C_{D_8}(r^2) = C_{D_8}(r^3) = \{ 1, r, r^2, r^3 \}$
    \item $C_{D_8}(s) = C_{D_8}(sr^2) = \{ 1, r^2, s, sr^2 \}$
    \item $C_{D_8}(sr) = C_{D_8}(sr^3) = \{ 1, r^2, sr, sr^3 \}$
\end{itemize}

The center $Z(D_8)$ is $\{ 1, r^2 \}$.

\subsection*{$Q_8$}

\begin{itemize}
    \item $C_{D_8}(1) = C_{D_8}(-1) = Q_8$
    \item $C_{D_8}(i) = C_{D_8}(-i) = \{ 1, -1, i, -i \}$
    \item $C_{D_8}(j) = C_{D_8}(-j) = \{ 1, -1, j, -j \}$
    \item $C_{D_8}(k) = C_{D_8}(-k) = \{ 1, -1, k, -k \}$
\end{itemize}

The center $Z(Q_8)$ is $\{ 1, -1 \}$.

\section*{5. (6/8/23)}

In each of parts (a) through (c) show that for the specified group $G$ and subgroup $A$ of $G$, $C_G(A) = A$ and $N_G(A) = G$.

\begin{enumerate}[label=(\alph*)]
    \item $G = S_3$ and $A = \{ (1), (1, 2, 3), (1, 3, 2) \}$.
          \begin{proof}
            From Exercise 4, we have $C_G((1, 2, 3)) = C_G((1, 3, 2)) = A$. No other non-identity permutation is in any of the centralizers of any element of $A$, therefore $C_G(A) = A$.

            Next, consider $\sigma^{-1}(1, 2, 3)\sigma$ for some other permutation in $S_3$, for example $(1, 2)(1, 2, 3)(1, 2)$. This is equal to $(1, 3, 2)$, which is an element of $A$, so $(1, 2)$ is in the normalizer of $A$. Since $C_G(A) \leq N_G(A)$ for all $A$, $A \subseteq N_G(A)$, and it follows that $N_G(A)$ consists of at least $A$ and the element $(1, 2)$. Then, because $N_G(A)$ is a subgroup, it is closed under permutation composition, and therefore must contain all elements of $S_3$.
          \end{proof}
    \item $G = D_8$ and $A = \{ 1, s, r^2, sr^2 \}$.
          \begin{proof}
            We know that $C_G(A)$ is a subgroup of $G$, and from Exercise 4, we have $A \leq C_G(A)$ (since $A$ is commutative). Then $|C_G(A)| \geq 4$. By Lagrange's Theorem, the order of $C_G(A)$ divides the order of $G$, 8. Then we must have either $C_G(A) = A$ or $C_G(A) = G$. However, $r$ is not in the centralizer of $A$, because $rsr^{-1} = rsr^3 = sr^{-1}r^3 = sr^2 \neq s$. Therefore $C_G(A) = A$.

            When we consider the normalizer of $A$, note that $rsr^{-1} = sr^2 \in A$. Thus $N_G(A)$ is a subgroup of $G$ that contains both $A$ and the element $r$. By closing the subgroup, we obtain $N_G(A) = G$.
          \end{proof}
    \item $G = D_{10}$ and $A = \{ 1, r, r^2, r^3, r^4 \}$.
          \begin{proof}
            Since $A$ consists only of powers of $r$, $A$ is commutative, and so (as above) $A \leq C_G(A)$. The centralizer of $A$ does not contain the element $s$, because $s^{-1}rs = srs = ssr^4 = r^4 \neq r$. Then we must have $|A| = 5 \leq |C_G(A)| \leq 9 = |G - \{ s \}|$. Again by Lagrange's Theorem, the order of $C_G(A)$ must divide 10, and since it at least 5 and at most 9, it must be 5. Therefore $C_G(A) = A$.

            When we consider the normalizer of $A$, note that $s^{-1}r^4s = r \in A$. Thus $N_G(A)$ is a subgroup of $G$ that contains both $A$ and the element $s$. By closing the subgroup, we obtain $N_G(A) = G$.
          \end{proof}
\end{enumerate}

\end{document}