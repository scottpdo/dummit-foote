\documentclass{article}

\title{Dummit \& Foote Ch. 2.4: Subgroups Generated by Subsets of a Group}
\author{Scott Donaldson}
\date{Jul. 2023}
\usepackage{amsmath, amsthm, amsfonts, enumitem}

\begin{document}

\maketitle

\section*{1. (7/13/23)}

Prove that if $H$ is a subgroup of $G$ then $\langle H \rangle = H$.

\begin{proof}
    Let $H \leq G$. To show that $\langle H \rangle = H$, we must show that each is contained in the other. By definition, $H \subseteq \langle H \rangle$, so it remains to be proven that $\langle H \rangle \subseteq H$.

    Let $h \in \langle H \rangle$. Recall that:
    \begin{equation*}
        \langle H \rangle = \bigcap_{\substack{H \subseteq K \\ K \leq G}} K,
    \end{equation*}
    that is, for all subset $K \leq G$ with $H \subseteq K$, we have $h \in K$. In particular, since $H$ is a subgroup of $G$, we have $h \in H$, since $H \leq G$ and $H \subseteq H$. Therefore $\langle H \rangle \subseteq H$, and it follows that $\langle H \rangle = H$.
\end{proof}

\section*{2. (7/17/23)}

Prove that if $A$ is a subset of $B$ then $\langle A \rangle \leq \langle B \rangle$. Give an example where $A \subseteq B$ with $A \neq B$ but $\langle A \rangle = \langle B \rangle$.

\begin{proof}
    Let $G$ be a group and let $A \subseteq B \subseteq G$. Recall that one definition of $\langle A \rangle$ is the set of all finite words of elements and inverses of elements of $A$, that is, every element of $\langle A \rangle$ can be written $a_1^{\varepsilon_1} a_2^{\varepsilon_2} ... a_n^{\varepsilon_n}$, where $n \in \mathbb{Z}, n \geq 0$ and $a_i \in A, \varepsilon_i = \pm 1$ for each $i$. Since $A$ is a subset of $B$, $a_i \in A \Rightarrow a_i \in B$, and so each element $a_1^{\varepsilon_1} a_2^{\varepsilon_2} ... a_n^{\varepsilon_n} \in \langle A \rangle$ is also in $\langle B \rangle$. Therefore $\langle A \rangle \leq \langle B \rangle$.

    Now let $G = \mathbb{Z}/3\mathbb{Z}$, $A = \{ 1 \}$, and $B = \{ 0, 1 \}$. Then we have $A \subseteq B$ with $A \neq B$ but $\langle A \rangle = \langle B \rangle = G$.
\end{proof}

\section*{3. (7/17/23)}

Prove that if $H$ is an abelian subgroup of $G$ then $\langle H, Z(G) \rangle$ is abelian. Give an explicit example of an abelian subgroup $H$ of a group $G$ such that $\langle H, C_G(H) \rangle$ is not abelian.

\begin{proof}
    Let $G$ be a group and let $H$ be an abelian subgroup of $G$. Recall that $Z(G) = \{ g \in G \mid xg = gx \text{ for all } x \in G \}$, that is, the set of elements of $G$ that commute with every element of $G$. We will show that $\langle H, Z(G) \rangle$ is an abelian subgroup of $G$.

    First, we will show that the product of any two elements commutes with both elements. Let $a, b \in G$ be commuting elements. Then:
    \begin{equation*}
        (ab)a = aba = aab = a(ab), \text{ and } (ab)b = abb = bab = b(ab),
    \end{equation*}
    as desired.

    Now the generated subgroup $\langle H, Z(G) \rangle$ is constructed from finite words of elements and inverses of elements from $H$ and $Z(G)$. Since $H$ is an abelian subgroup and elements of $Z(G)$ (and therefore their inverses) commute with every element of $G$ (and therefore $H$), it follows that every element in $\langle H, Z(G) \rangle$ is a product of commuting elements. Every such element therefore commutes with every other element in $H$ and $Z(G)$, as well as any other product of elements of $H$ and $Z(G)$. Thus $\langle H, Z(G) \rangle$ is an abelian subgroup of $G$.

    However, it does not follow that $\langle H, C_G(H) \rangle$ is an abelian subgroup of $G$. Let $G = D_8$ and $H = \{ 1, r^2 \}$. The centralizer of $H$ in $G$ is all of $G$, since every element of $H$ commutes with every other element of $G$ (that is, $H = Z(G)$). Then the generated subgroup $\langle H, C_G(H) \rangle = \langle H, G \rangle = G$, which is non-abelian.
\end{proof}

\section*{4. (7/17/23)}

Prove that if $H$ is a subgroup of $G$ then $H$ is generated by the set $H - \{ 1 \}$.

\begin{proof}
    Let $H \leq G$ and consider $\langle H - \{ 1 \} \rangle$. If $H = \{ 1 \}$, then $H - \{ 1 \} = \emptyset$, and so by definition $\langle H - \{ 1 \} \rangle = \{ 1 \} = H$.

    Suppose $H \neq \{ 1 \}$. Then there exists some $h \in H$ with $h \neq 1$. Since $H$ is a subgroup, it is closed under inverses, so $h^{-1} \in H$. We generate $\langle H - \{ 1 \} \rangle$ by taking finite products of elements of $H$, and so $hh^{-1} = 1 \in \langle H - \{ 1 \} \rangle$. Further, we cannot construct any element outside of $H$ by taking products of elements of $H$, so we must therefore have $\langle H - \{ 1 \} \rangle = (H - \{ 1 \}) \cup \{ 1 \} = H$.
\end{proof}

\section*{5. (7/20/23)}

Prove that the subgroup generated by any two distinct elements of order 2 in $S_3$ is all of $S_3$.

\begin{proof}
    The elements of order 2 in $S_3$ are $(1, 2), (1, 3)$, and $(2, 3)$. Since any two of these elements permute one of $\{ 1, 2, 3 \}$ to the other two, without loss of generality we can consider the subgroup generated by a single pair of them. We will consider the subgroup generated by $(1, 2)$ and $(1, 3)$.

    The subgroup contains the identity element, since $(1, 2)(1, 2) = (1)$. It also contains both elements of order 3, since $(1, 2)(1, 3) = (1, 3, 2)$ and $(1, 3)(1, 2) = (1, 2, 3)$. Finally, the subgroup contains the third element of order 2, since $(1, 2)(1, 2, 3) = (2, 3)$. Together these are all the elements of $S_3$.

    Therefore the subgroup generated by any two elements of $S_3$ is all of $S_3$.
\end{proof}

\section*{6. (7/20/23)}

Prove that the subgroup of $S_4$ generated by $(1, 2)$ and $(1, 2)(3, 4)$ is a noncyclic group of order 4.

\begin{proof}
    Let us construct the subgroup of $S_4$ generated by $(1, 2)$ and $(1, 2)(3, 4)$. Both elements have order 2, so we will not consider any higher powers of each. Their product is $(3, 4)$, which also has order 2. At this point the subgroup consists of $\{ (1), (1, 2), (1, 2)(3, 4), (3, 4) \}$. Taking the product of $(3, 4)$ with either of $(1, 2)$ or $(1, 2)(3, 4)$ results in the other element, respectively. Therefore there is no way to obtain new elements not already in this subgroup.

    Thus the subgroup of $S_4$ generated by $(1, 2)$ and $(1, 2)(3, 4)$ has order 4. Further, it is noncyclic, since it contains no elements of order 4 (in fact, it is isomorphic to the Klein 4-group $V_4$).
\end{proof}

\end{document}