\documentclass{article}

\title{Dummit \& Foote Ch. 2.4: Subgroups Generated by Subsets of a Group}
\author{Scott Donaldson}
\date{Jul. 2023}
\usepackage{amsmath, amsthm, amsfonts, enumitem}

\begin{document}

\maketitle

\section*{1. (7/13/23)}

Prove that if $H$ is a subgroup of $G$ then $\langle H \rangle = H$.

\begin{proof}
    Let $H \leq G$. To show that $\langle H \rangle = H$, we must show that each is contained in the other. By definition, $H \subseteq \langle H \rangle$, so it remains to be proven that $\langle H \rangle \subseteq H$.

    Let $h \in \langle H \rangle$. Recall that:
    \begin{equation*}
        \langle H \rangle = \bigcap_{\substack{H \subseteq K \\ K \leq G}} K,
    \end{equation*}
    that is, for all subset $K \leq G$ with $H \subseteq K$, we have $h \in K$. In particular, since $H$ is a subgroup of $G$, we have $h \in H$, since $H \leq G$ and $H \subseteq H$. Therefore $\langle H \rangle \subseteq H$, and it follows that $\langle H \rangle = H$.
\end{proof}

\end{document}