\documentclass{article}

\title{Dummit \& Foote Ch. 2.3: Cyclic Groups and Cyclic Subgroups}
\author{Scott Donaldson}
\date{Jun. 2023}
\usepackage{amsmath, amsthm, amsfonts, enumitem}

\begin{document}

\maketitle

\section*{1. (6/18/23)}

Find all subgroups of $Z_{45} = \langle x \rangle$, giving a generator for each. Describe the containments between these subgroups.

\begin{proof}
    The subgroups of $Z_{45} = \langle x \rangle$ are those cyclic groups generated by $x^n$, where $n$ divides 45. These are:
    \begin{itemize}[itemsep=0em]
        \item $\langle 1 \rangle = \{ 1 \}$, the trivial subgroup
        \item $\langle x^{15} \rangle = \{ 1, x^{15}, x^{30} \} \cong \mathbb{Z}/3\mathbb{Z}$
        \item $\langle x^9 \rangle = \{ 1, x^9, x^{18}, x^{27}, x^{36} \} \cong \mathbb{Z}/5\mathbb{Z}$
        \item $\langle x^5 \rangle = \{ 1, x^5, x^{10}, x^{15}, x^{20}, x^{25}, x^{30}, x^{35}, x^{40} \} \cong \mathbb{Z}/9\mathbb{Z}$
        \item $\langle x^3 \rangle = \{ 1, x^3, x^6, ..., x^{39}, x^{42} \} \cong \mathbb{Z}/15\mathbb{Z}$
        \item $\langle x \rangle = Z_{45}$ itself
    \end{itemize}
    Among these subgroups, we have $\langle 1 \rangle$ contained within every other subgroup, as well as $\langle x^{15} \rangle \leq \langle x^5 \rangle$, $\langle x^{15} \rangle \leq \langle x^3 \rangle$, and $\langle x^9 \rangle \leq \langle x^3 \rangle$.
\end{proof}

\section*{2. (6/19/23)}

If $x$ is an element of the finite group $G$ and $|x| = |G|$, prove that $G = \langle x \rangle$. Give an explicit example to show that this result need not be true if $G$ is an infinite group.

\begin{proof}
    Let $|x| = |G| = n < \infty$. By definition, $G$ is closed, so it contains all powers of $x: 1, x, x^2, ..., x^{n - 1}$. These are exactly $n$ elements, so $G$ contains no other elements. It is therefore generated by $x$, that is, $G = \langle x \rangle$.

    However, if $G$ is an infinite group and $x \in G$ with $|x| = \infty$, then this is not necessarily the case. For example, if $G = \mathbb{Z}$ and $x = 2$, then $x$ generates all even integers in $\mathbb{Z}$, but does not generate the element 5.
\end{proof}

\section*{3. (6/19/23)}

Find all generators for $\mathbb{Z}/48\mathbb{Z}$.

\begin{proof}
    From Proposition 6., the generators for $\mathbb{Z}/48\mathbb{Z}$ are those positive integers $n < 48$ for which $n$ is relatively prime to 48. These are: 1, 5, 7, 11, 13, 17, 19, 23, 25, 29, 31, 35, 37, 41, 43, and 47.
\end{proof}

\section*{4. (6/19/23)}

Find all generators for $\mathbb{Z}/202\mathbb{Z}$.

\begin{proof}
    As above, the generators for $\mathbb{Z}/202\mathbb{Z}$ are those positive integers $n < 202$ for which $n$ is relatively prime to 202. The integer 202 only has two divisors greater than 1, namely 2 and 101. Therefore the generators of $\mathbb{Z}/202\mathbb{Z}$ are every odd positive integer less than 202 except for 101.
\end{proof}

\section*{5. (6/19/23)}

Find the number of generators for $\mathbb{Z}/49000\mathbb{Z}$.

\begin{proof}
    We are concerned with the number of integers $n$ between 0 and 48999 for which $n$ is relatively prime to 49000. It will be helpful to write 49000 uniquely as the product of primes: $2^3 \cdot 5^3 \cdot 7^2$.

    Let us first consider the generators for $\mathbb{Z}/49000\mathbb{Z}$ between 0 and 69, that is, all the numbers that are relatively prime to 49000 between 0 and 69: 1, 3, 9, 11, 13, 17, 19, 23, 27, 29, 31, 33, 37, 39, 41, 43, 47, 51, 53, 57, 59, 61, 67, and 69. There are 24 such generators.

    Next, we show that, for any $n \in \{0, ..., 48999\}$, the greatest common divisor of $n$ and 49000 is equal to the greatest common divisor of $n \text{ mod } 70$ and 49000. This is because 70 is equal to the product of the bases of the prime factors of 49000: $70 = 2 \cdot 5 \cdot 7$. So for any $n$, we have $n = m + 70k = m + (2 \cdot 5 \cdot 7)k$, where $m \in \{0, ..., 69\}$ and $k \geq 0$. Suppose that $m$ is \emph{not} in the list of the above generators (that is, that the greatest common divisor of $m$ and 49000 is greater than 1). Then either 2, 5, or 7 divides $m$ (otherwise $m$ would be relatively prime to 49000). Without loss of generality, suppose that 2 divides $m$, and write $m = 2p$. We can then rewrite $n$ as:
    \begin{equation*}
        n = m + (2 \cdot 5 \cdot 7)k = 2p + (2 \cdot 5 \cdot 7)k = 2 \bigl( p + (5 \cdot 7)k \bigr),
    \end{equation*}
    that is, 2 divides $n$, so it is not relatively prime to 49000 (similarly, if 5 or 7 divide $m$, then 5 or 7 also divide $n$, respectively). It follows that the generators for $\mathbb{Z}/49000\mathbb{Z}$ between 0 and 69 repeat (mod 70) over the rest of 49000. Since $49000/70 = 700$, there are thus $700 \cdot 24 = 16800$ generators for $\mathbb{Z}/49000\mathbb{Z}$.
\end{proof}

\section*{6. (6/20/23)}

In $\mathbb{Z}/48\mathbb{Z}$ write out all elements of $\langle \overline{a} \rangle$ for every $\overline{a}$. Find all inclusions between subgroups in $\mathbb{Z}/48\mathbb{Z}$.

\begin{itemize}[itemsep=0em]
    \item Subgroup of order 48: $\langle \overline{1} \rangle = \langle \overline{5} \rangle = \langle \overline{7} \rangle = \langle \overline{11} \rangle = \langle \overline{13} \rangle = \langle \overline{17} \rangle = \langle \overline{19} \rangle = \langle \overline{23} \rangle = \langle \overline{25} \rangle = \langle \overline{29} \rangle = \langle \overline{31} \rangle = \langle \overline{35} \rangle = \langle \overline{37} \rangle = \langle \overline{41} \rangle = \langle \overline{43} \rangle = \langle \overline{47} \rangle$.

    \item Subgroup of order 24: $\langle \overline{2} \rangle = \langle \overline{10} \rangle = \langle \overline{14} \rangle = \langle \overline{22} \rangle = \langle \overline{26} \rangle = \langle \overline{34} \rangle = \langle \overline{38} \rangle = \langle \overline{46} \rangle$.

    \item Subgroup of order 16: $\langle \overline{3} \rangle = \langle \overline{9} \rangle = \langle \overline{15} \rangle = \langle \overline{21} \rangle = \langle \overline{27} \rangle = \langle \overline{33} \rangle = \langle \overline{39} \rangle = \langle \overline{45} \rangle$.

    \item Subgroup of order 12: $\langle \overline{4} \rangle = \langle \overline{20} \rangle = \langle \overline{28} \rangle = \langle \overline{44} \rangle$.

    \item Subgroup of order 8: $\langle \overline{6} \rangle = \langle \overline{18} \rangle = \langle \overline{30} \rangle = \langle \overline{42} \rangle$.

    \item Subgroup of order 6: $\langle \overline{8} \rangle = \langle \overline{40} \rangle$.

    \item Subgroup of order 4: $\langle \overline{12} \rangle = \langle \overline{36} \rangle$.

    \item Subgroup of order 3: $\langle \overline{16} \rangle = \langle \overline{32} \rangle$.

    \item Subgroup of order 2: $\langle \overline{24} \rangle$.

    \item Subgroup of order 1, the trivial subgroup: $\{ 0 \}$.
\end{itemize}

Among these subgroups, all contain the trivial subgroup. The subgroups of order 2 and 3 are distinct, but both are contained in the subgroup of order 6. The subgroup of order 2 is also contained in the subgroup of order 4. The subgroups of order 4 and 6 are both contained in the subgroup of order 12. The subgroup of order 4 is also contained in the subgroup of order 8. The subgroups of order 8 and 12 are both contained in the subgroup of order 24. The subgroup of order 8 is also contained in the subgroup of order 16.

\section*{7. (6/22/23)}

Let $Z_{48} = \langle x \rangle$ and use the isomorphism $\mathbb{Z}/48\mathbb{Z} \cong Z_{48}$ given by $\overline{1} \mapsto x$ to list all subgroups of $Z_{48}$ as computed in the preceding exercise.

\begin{itemize}[itemsep=0em]
    \item Subgroup of order 48: $\{ 1, x, x^2, ..., x^{47} \}$.
    \item Subgroup of order 24: $\{ 1, x^2, x^4, ..., x^{46} \}$.
    \item Subgroup of order 16: $\{ 1, x^3, x^6, ..., x^{45} \}$.
    \item Subgroup of order 12: $\{ 1, x^4, x^8, ..., x^{44} \}$.
    \item Subgroup of order 8: $\{ 1, x^6, x^{12}, x^{18}, x^{24}, x^{30}, x^{36}, x^{42} \}$.
    \item Subgroup of order 6: $\{ 1, x^8, x^{16}, x^{24}, x^{32}, x^{40} \}$.
    \item Subgroup of order 4: $\{ 1, x^{12}, x^{24}, x^{36} \}$.
    \item Subgroup of order 3: $\{ 1, x^{16}, x^{32} \}$.
    \item Subgroup of order 2: $\{ 1, x^{24} \}$.
    \item Subgroup of order 1, the trivial subgroup: $\{ 1 \}$.
\end{itemize}

\section*{8. (6/23/23)}

Let $Z_{48} = \langle x \rangle$. For which integers $a$ does the map $\varphi_a$ defined by $\varphi_a: \overline{1} \mapsto x^a$ extend to an \emph{isomorphism} from $\mathbb{Z}/48\mathbb{Z}$ onto $Z_{48}$?

\begin{proof}
    We will show that $\varphi_a$ is an isomorphism from $\mathbb{Z}/48\mathbb{Z}$ onto $Z_{48}$ if and only if $a \in \mathbb{Z}$ is relatively prime to 48.

    First, let $m, n \in \mathbb{Z}/48\mathbb{Z}$. Then $\varphi_a(m)\varphi_a(n) = (x^a)^m (x^a)^n = (x^a)^{m + n} = \varphi_a(m + n)$. So $\varphi_a$ is a homomorphism.

    Next, $\varphi_a$ is one-to-one. Let $\varphi_a(n) = \varphi_a(m)$ for $m, n \in \mathbb{Z}/48\mathbb{Z}$. Then $(x^a)^m = (x^a)^n \Rightarrow x^{am} = x^{an}$, and so $am = an$ (mod 48). Since $a$ is relatively prime to 48, we must therefore have $m = n$, and it follows that $\varphi_a$ is injective. (Note, however, that if $k > 1$ divides both $a$ and 48, then $am = an$ does not imply that $m = n$, and $\varphi_a$ is therefore not injective. For example, if $a = 14$, then $\varphi_a(7) = (x^14)^7 = x^{98} = x^2$ and $\varphi_a(31) = (x^14)^31 = x^{434} = x^2$).

    Finally, $\varphi_a$ is onto. Let $x^b \in Z_{48}$. Suppose there exists some $n \in \mathbb{Z}/48\mathbb{Z}$ such that $\varphi_a(n) = x^b$, that is, $(x^a)^n = x^b$. Then we must have $an = b$ (mod 48). Since $a$ is relatively prime to 48, any integer between 0 and 47 can be written as $an$ for some $n \in \mathbb{Z}/48\mathbb{Z}$, and so $\varphi_a$ is onto.

    Thus for $a$ relatively prime to 48, $\varphi_a: \overline{1} \mapsto x^a$ is an isomorphism from $\mathbb{Z}/48\mathbb{Z}$ onto $Z_{48}$.
\end{proof}

\section*{9. (7/2/23)}

Let $Z_{36} = \langle x \rangle$. For which integers $a$ does the map $\varphi_a$ defined by $\varphi_a: \overline{1} \mapsto x^a$ extend to a \emph{well defined homomorphism} from $\mathbb{Z}/48\mathbb{Z}$ onto $Z_{36}$? Can $\varphi_a$ ever be a surjective homomorphism?

\begin{proof}
    We will show that $\varphi_a: \mathbb{Z}/48\mathbb{Z} \rightarrow Z_{36}$ is a well defined homomorphism if and only if $a$ is a multiple of 3.

    For $\varphi_a$ to be a homomorphism, we must have $\varphi_a(b)\varphi_a(c) = \varphi_a(b + c)$ for all $b, c \in \mathbb{Z}/48\mathbb{Z}$. Now $\varphi_a(b)\varphi_a(c) = (x^a)^b (x^a)^c = (x^a)^{b + c} = x^{a(b + c)}$ and $\varphi_a(b + c) = (x^a)^{b + c} = x^{a(b + c)}$. Superficially these appear identical already. However, note that in $\varphi_a(b)\varphi_a(c)$ we compute $ab + ac$ mod 36, while in $\varphi_a(b + c)$ we first take $b + c$ mod 48 before then computing $a(b + c)$. That is, $a$ must satisfy
    \begin{equation*}
        a(b + c \text{ mod } 48) \text{ mod } 36 = a(b + c) \text{ mod } 36
    \end{equation*}
    for all $b, c \in \mathbb{Z}/48\mathbb{Z}$. If $b + c < 48$, then the two are equal for all $a \in \mathbb{Z}$. So suppose that $b + c \geq 48$. Then $b + c \text{ mod } 48 = b + c - 48$, so we must have
    \begin{flalign*}
        a(b + c - 48) \text{ mod } 36 &= a(b + c) \text{ mod } 36 \\
        ab + ac - 48a \text{ mod } 36 &= ab + ac \text{ mod } 36 \\
        -48a \text{ mod } 36 &= 0 \text{ mod } 36 \\
        -48a &\cong 36 \Rightarrow 48a \cong 36,
    \end{flalign*}
    that is, $a$ is some integer which, when multiplied by 48, results in a multiple of 36. Writing 48 as the product of its prime factors gives $2^4 \cdot 3$, while $36 = 2^2 \cdot 3^2$. Note that 36 has one more factor of 3, and so when $a$ is a multiple of 3, $48a$ will be a multiple of 36. Only these values satisfy the exponents in the equation above, and thus $\varphi_a$ is a homomorphism if and only if $a$ is a multiple of 3.

    It is not possible for $\varphi_a$ to be a surjective homomorphism. Because $a$ must be a multiple of 3, we have $\varphi_a(1) = x^a = x^{3n} = (x^3)^n$ for some $n \in \mathbb{Z}$. In turn, $\varphi_a$ generates only the values $\varphi_a(2) = (x^6)^n, \varphi_a(3) = (x^9)^n$, ..., that is, it only generates powers of $x^3$ in $Z_{36}$. By counterexample, there is no value in $\mathbb{Z}/48\mathbb{Z}$ whose image under $\varphi_a$ is $x$, and so $\varphi_a$ cannot be surjective.
\end{proof}

\section*{10. (7/2/23)}

What is the order of $\overline{30}$ in $\mathbb{Z}/54\mathbb{Z}$? Write out all the elements and their orders in $\langle \overline{30} \rangle$.

\begin{proof}
    First, the group $\langle \overline{30} \rangle$ (ordered by multiples of $\overline{30}$ consists of the elements $\{ 0, 30, 6, 36, 12, 42, 18, 48, 24 \}$. This implies that the order of $\overline{30} = |\langle \overline{30} \rangle| = 9$.

    The orders of each of the elements of $\langle \overline{30} \rangle$ are:
    \begin{itemize}[itemsep=0em]
        \item 0: 1
        \item 6: 9
        \item 12: 9
        \item 18: 3
        \item 24: 9
        \item 30: 9
        \item 36: 3
        \item 42: 9
        \item 48: 9
    \end{itemize}
\end{proof}

\section*{11. (7/2/23)}

Find all cyclic subgroups of $D_8$ Find a proper subgroup of $D_8$ which is not cyclic.

\begin{proof}
    Recall that $D_8 = \{ 1, r, r^2, r^3, s, sr, sr^2, sr^3 \}$. A cyclic subgroup of $D_8$ must be generated by one element, so it cannot contain both $s$ and a multiple of $r$. Therefore the cyclic subgroups of $D_8$ are:
    \begin{itemize}[itemsep=0em]
        \item $\langle 1 \rangle = \{ 1 \}$
        \item $\langle r \rangle = \langle r^3 \rangle = \{ 1, r, r^2, r^3 \}$
        \item $\langle r^2 \rangle = \{ 1, r^2 \}$
        \item $\langle s \rangle = \{ 1, s \}$
    \end{itemize}
    The group $D_8$ also contains as a subgroup $\{ 1, r^2, s, sr^2 \}$, which is generated by the two elements $r^2$ and $s$, and is therefore not cyclic.
\end{proof}

\section*{12. (7/2/23)}

Prove that the following groups are \emph{not} cyclic:

\begin{enumerate}[label=(\alph*), itemsep=0em]
    \item $Z_2 \times Z_2$
          \begin{proof}
            This group consists of the elements $\{ (0, 0), (0, 1), (1, 0), (1, 1) \}$. So each non-identity element has order 2, and there is no element of order 4 (the size of the group). Therefore it is not generated by any single element, and so it is not a cyclic group.
          \end{proof}
    \item $Z_2 \times \mathbb{Z}$
          \begin{proof}
            Now $Z_2 \times \mathbb{Z} = \{ (a, b) \mid a = 0 \text{ or } 1, b \in \mathbb{Z} \}$. So a generating element must be of the form $(0, b)$ or $(1, b)$. Elements of the form $(0, b)$ can only generate $(0, 2b), (0, 3b), ...$ but never $(1, nb)$, so a generating element must be of the form $(1, b)$. Multiples of $(1, b)$ include $(0, 2b), (1, 3b), (0, 4b), ...$, that is, $(0, nb)$ and $(1, mb)$ for even $n$ and odd $m$, respectively. However, then this element cannot generate $(1, nb)$, and so it is not a generating element. Since both candidates fail to generate the group, it is not cyclic.
          \end{proof}
    \item $\mathbb{Z} \times \mathbb{Z}$
          \begin{proof}
            Similar to $Z_2 \times \mathbb{Z}$, consider a generating element of $\mathbb{Z} \times \mathbb{Z}$, $(a, b)$. Multiples of this element include $(2a, 2b), (3a, 3b), ...$, that is, $(na, nb)$ for $n \in \mathbb{Z}$. However, this element cannot generate $(a, nb)$ (where $n \neq 1$), and so it is not a generating element. Since all elements of $\mathbb{Z} \times \mathbb{Z}$ are of this form, there is no generating element, and so the group is not cyclic.
          \end{proof}
\end{enumerate}

\section*{13. (7/5/23)}

Prove that the following groups are \emph{not} isomorphic:

\begin{enumerate}[label=(\alph*), itemsep=0em]
    \item $\mathbb{Z} \times Z_2$ and $\mathbb{Z}$
          \begin{proof}
            The group of the integers under addition contains no elements of finite order other than the identity, 0. However, the group $\mathbb{Z} \times Z_2$ contains the element $(0, 1)$, which has order 2. Since there is no corresponding element of order 2 in $\mathbb{Z}$, the groups are not isomorphic.
          \end{proof}
    \item $\mathbb{Q} \times Z_2$ and $\mathbb{Q}$
          \begin{proof}
            The proof that $\mathbb{Q} \times Z_2$ and $\mathbb{Q}$ are not isomorphic is identical to the proof that $\mathbb{Z} \times Z_2$ and $\mathbb{Z}$ are not isomorphic.
          \end{proof}
\end{enumerate}

\section*{14. (7/5/23)}

Let $\sigma = (1, 2, 3, 4, 5, 6, 7, 8, 9, 10, 11, 12)$. For each of the following integers $a$ compute $\sigma^a$:

\begin{itemize}[itemsep=0em]
    \item $a = 13$: $\sigma^{13} = \sigma$
    \item $a = 65$: $\sigma^{65} = \sigma^5 = (1, 6, 11, 4, 9, 2, 7, 12, 5, 10, 3, 8)$
    \item $a = 626$: $\sigma^{626} = \sigma^2 = (1, 3, 5, 7, 9, 11)(2, 4, 6, 8, 10, 12)$
    \item $a = 1195$: $\sigma^{1195} = \sigma^7 = (1, 8, 3, 10, 5, 12, 7, 2, 9, 4, 11, 6)$
    \item $a = -6$: $\sigma^{-6} = \sigma^6 = (1, 7)(2, 8)(3, 9)(4, 10)(5, 11)(6, 12)$
    \item $a = -81$: $\sigma^{-81} = \sigma^3 = (1, 4, 7, 10)(2, 5, 8, 11)(3, 6, 9, 12)$
    \item $a = -570$: $\sigma^{-570} = \sigma^6$
    \item $a = -1211$: $\sigma^{-1211} = \sigma^{-11} = \sigma$
\end{itemize}

\section*{15. (7/5/23)}

Prove that $\mathbb{Q} \times \mathbb{Q}$ is not cyclic.

\begin{proof}
    If $\mathbb{Q} \times \mathbb{Q}$ were cyclic, then it could be generated from a single element. Suppose toward contradiction that some element $(x, y)$ generates $\mathbb{Q} \times \mathbb{Q}$. Under addition in $\mathbb{Q}$ for each element of the ordered pair, we can generate elements of the form $(0, 0), (\pm x, \pm y), (\pm 2x, \pm 2y), (\pm 3x, \pm 3y), ...$. However, we cannot generate the element $(x/2, y/2)$, which is an element of $\mathbb{Q} \times \mathbb{Q}$. Therefore an arbitrary element $(x, y)$ cannot generate $\mathbb{Q} \times \mathbb{Q}$, and so there is no generator. Thus $\mathbb{Q} \times \mathbb{Q}$ is not a cyclic group.
\end{proof}

\section*{16. (7/8/23)}

Assume $|x| = n$ and $|y| = m$. Suppose that $x$ and $y$ \emph{commute}: $xy = yx$. Prove that $|xy|$ divides the least common multiple of $m$ and $n$. Need this be true if $x$ and $y$ do \emph{not} commute? Give an example of commuting elements $x, y$ such that the order of $xy$ is not equal to the least common multiple of $|x|$ and $|y|$.

\begin{proof}
    Given $|x| = n, |y| = m$, note that $x^n = y^m = 1$ implies that $x^{mn} y^{mn} = (xy)^{mn} = 1$. So $xy$ has finite order. Suppose that $|xy| = k < \infty$. Then, from Ch. 1, Ex. 24., $(xy)^k = x^k y^k = 1$.
    
    First, consider that if $x^k = a \neq 1$, then $y^k = a^{-1}$. It follows that $x^k = (y^k)^{-1}$, and so $x = y^{-1}$. Then $|xy| = |1| = 1$, which trivially divides the least common multiple of $m$ and $n$.

    In the other case, we must have $x^k = y^k = 1$. Since the orders of $x$ and $y$ are $n$ and $m$, respectively, the orders of both elements divide $k$, that is, $k$ is a multiple of both $n$ and $m$. It follows that $k$ must be the least common multiple of $m$ and $n$.
\end{proof}

\end{document}