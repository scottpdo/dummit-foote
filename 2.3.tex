\documentclass{article}

\title{Dummit \& Foote Ch. 2.3: Cyclic Groups and Cyclic Subgroups}
\author{Scott Donaldson}
\date{Jun. 2023}
\usepackage{amsmath, amsthm, amsfonts, enumitem}

\begin{document}

\maketitle

\section*{1. (6/18/23)}

Find all subgroups of $Z_{45} = \langle x \rangle$, giving a generator for each. Describe the containments between these subgroups.

\begin{proof}
    The subgroups of $Z_{45} = \langle x \rangle$ are those cyclic groups generated by $x^n$, where $n$ divides 45. These are:
    \begin{itemize}
        \item $\langle 1 \rangle = \{ 1 \}$, the trivial subgroup
        \item $\langle x^{15} \rangle = \{ 1, x^{15}, x^{30} \} \cong \mathbb{Z}/3\mathbb{Z}$
        \item $\langle x^9 \rangle = \{ 1, x^9, x^{18}, x^{27}, x^{36} \} \cong \mathbb{Z}/5\mathbb{Z}$
        \item $\langle x^5 \rangle = \{ 1, x^5, x^{10}, x^{15}, x^{20}, x^{25}, x^{30}, x^{35}, x^{40} \} \cong \mathbb{Z}/9\mathbb{Z}$
        \item $\langle x^3 \rangle = \{ 1, x^3, x^6, ..., x^{39}, x^{42} \} \cong \mathbb{Z}/15\mathbb{Z}$
        \item $\langle x \rangle = Z_{45}$ itself
    \end{itemize}
    Among these subgroups, we have $\langle 1 \rangle$ contained within every other subgroup, as well as $\langle x^{15} \rangle \leq \langle x^5 \rangle$, $\langle x^{15} \rangle \leq \langle x^3 \rangle$, and $\langle x^9 \rangle \leq \langle x^3 \rangle$.
\end{proof}

\section*{2. (6/19/23)}

If $x$ is an element of the finite group $G$ and $|x| = |G|$, prove that $G = \langle x \rangle$. Give an explicit example to show that this result need not be true if $G$ is an infinite group.

\begin{proof}
    Let $|x| = |G| = n < \infty$. By definition, $G$ is closed, so it contains all powers of $x: 1, x, x^2, ..., x^{n - 1}$. These are exactly $n$ elements, so $G$ contains no other elements. It is therefore generated by $x$, that is, $G = \langle x \rangle$.

    However, if $G$ is an infinite group and $x \in G$ with $|x| = \infty$, then this is not necessarily the case. For example, if $G = \mathbb{Z}$ and $x = 2$, then $x$ generates all even integers in $\mathbb{Z}$, but does not generate the element 5.
\end{proof}

\section*{3. (6/19/23)}

Find all generators for $\mathbb{Z}/48\mathbb{Z}$.

\begin{proof}
    From Proposition 6., the generators for $\mathbb{Z}/48\mathbb{Z}$ are those positive integers $n < 48$ for which $n$ is relatively prime to 48. These are: 1, 5, 7, 11, 13, 17, 19, 23, 25, 29, 31, 35, 37, 41, 43, and 47.
\end{proof}

\section*{4. (6/19/23)}

Find all generators for $\mathbb{Z}/202\mathbb{Z}$.

\begin{proof}
    As above, the generators for $\mathbb{Z}/202\mathbb{Z}$ are those positive integers $n < 202$ for which $n$ is relatively prime to 202. The integer 202 only has two divisors greater than 1, namely 2 and 101. Therefore the generators of $\mathbb{Z}/202\mathbb{Z}$ are every odd positive integer less than 202 except for 101.
\end{proof}

\section*{5. (6/19/23)}

Find the number of generators for $\mathbb{Z}/49000\mathbb{Z}$.

\begin{proof}
    We are concerned with the number of integers $n$ between 0 and 48999 for which $n$ is relatively prime to 49000. It will be helpful to write 49000 uniquely as the product of primes: $2^3 \cdot 5^3 \cdot 7^2$.

    Let us first consider the generators for $\mathbb{Z}/49000\mathbb{Z}$ between 0 and 69, that is, all the numbers that are relatively prime to 49000 between 0 and 69: 1, 3, 9, 11, 13, 17, 19, 23, 27, 29, 31, 33, 37, 39, 41, 43, 47, 51, 53, 57, 59, 61, 67, and 69. There are 24 such generators.

    Next, we show that, for any $n \in \{0, ..., 48999\}$, the greatest common divisor of $n$ and 49000 is equal to the greatest common divisor of $n \text{ mod } 70$ and 49000. This is because 70 is equal to the product of the bases of the prime factors of 49000: $70 = 2 \cdot 5 \cdot 7$. So for any $n$, we have $n = m + 70k = m + (2 \cdot 5 \cdot 7)k$, where $m \in \{0, ..., 69\}$ and $k \geq 0$. Suppose that $m$ is \emph{not} in the list of the above generators (that is, that the greatest common divisor of $m$ and 49000 is greater than 1). Then either 2, 5, or 7 divides $m$ (otherwise $m$ would be relatively prime to 49000). Without loss of generality, suppose that 2 divides $m$, and write $m = 2p$. We can then rewrite $n$ as:
    \begin{equation*}
        n = m + (2 \cdot 5 \cdot 7)k = 2p + (2 \cdot 5 \cdot 7)k = 2 \bigl( p + (5 \cdot 7)k \bigr),
    \end{equation*}
    that is, 2 divides $n$, so it is not relatively prime to 49000 (similarly, if 5 or 7 divide $m$, then 5 or 7 also divide $n$, respectively). It follows that the generators for $\mathbb{Z}/49000\mathbb{Z}$ between 0 and 69 repeat (mod 70) over the rest of 49000. Since $49000/70 = 700$, there are thus $700 \cdot 24 = 16800$ generators for $\mathbb{Z}/49000\mathbb{Z}$.
\end{proof}

\section*{6. (6/20/23)}

In $\mathbb{Z}/48\mathbb{Z}$ write out all elements of $\langle \overline{a} \rangle$ for every $\overline{a}$. Find all inclusions between subgroups in $\mathbb{Z}/48\mathbb{Z}$.

\begin{proof}
    Subgroup of order 48: $\langle \overline{1} \rangle = \langle \overline{5} \rangle = \langle \overline{7} \rangle = \langle \overline{11} \rangle = \langle \overline{13} \rangle = \langle \overline{17} \rangle = \langle \overline{19} \rangle = \langle \overline{23} \rangle = \langle \overline{25} \rangle = \langle \overline{29} \rangle = \langle \overline{31} \rangle = \langle \overline{35} \rangle = \langle \overline{37} \rangle = \langle \overline{41} \rangle = \langle \overline{43} \rangle = \langle \overline{47} \rangle$.

    Subgroup of order 24: $\langle \overline{2} \rangle = \langle \overline{10} \rangle = \langle \overline{14} \rangle = \langle \overline{22} \rangle = \langle \overline{26} \rangle = \langle \overline{34} \rangle = \langle \overline{38} \rangle = \langle \overline{46} \rangle$.

    Subgroup of order 16: $\langle \overline{3} \rangle = \langle \overline{9} \rangle = \langle \overline{15} \rangle = \langle \overline{21} \rangle = \langle \overline{27} \rangle = \langle \overline{33} \rangle = \langle \overline{39} \rangle = \langle \overline{45} \rangle$.

    Subgroup of order 12: $\langle \overline{4} \rangle = \langle \overline{20} \rangle = \langle \overline{28} \rangle = \langle \overline{44} \rangle$.

    Subgroup of order 8: $\langle \overline{6} \rangle = \langle \overline{18} \rangle = \langle \overline{30} \rangle = \langle \overline{42} \rangle$.

    Subgroup of order 6: $\langle \overline{8} \rangle = \langle \overline{40} \rangle$.

    Subgroup of order 4: $\langle \overline{12} \rangle = \langle \overline{36} \rangle$.

    Subgroup of order 3: $\langle \overline{16} \rangle = \langle \overline{32} \rangle$.

    Subgroup of order 2: $\langle \overline{24} \rangle$.

    Subgroup of order 1, the trivial subgroup: $\{ 0 \}$.

    Among these subgroups, all contain the trivial subgroup. The subgroups of order 2 and 3 are distinct, but both are contained in the subgroup of order 6. The subgroup of order 2 is also contained in the subgroup of order 4. The subgroups of order 4 and 6 are both contained in the subgroup of order 12. The subgroup of order 4 is also contained in the subgroup of order 8. The subgroups of order 8 and 12 are both contained in the subgroup of order 24. The subgroup of order 8 is also contained in the subgroup of order 16.
\end{proof}

\end{document}