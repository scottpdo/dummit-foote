\documentclass{article}

\title{Dummit \& Foote Ch. 4.2: Groups Acting on Themselves by Left Multiplication — Cayley's Theorem}
\author{Scott Donaldson}
\date{Feb. 2024}
\usepackage{amsmath, amsthm, amsfonts, amssymb, enumitem, tabu, tikz}

\begin{document}

\maketitle

Let $G$ be a group and let $H$ be a subgroup of $G$.

\section*{1. (2/12/24)}

Let $G = \{ 1, a, b, c \}$ be the Klein 4-group whose group table is written out in Section 2.5.
\begin{enumerate}[label=(\alph*), itemsep=0em]
    \item Label $1, a, b, c$ with the integers $1, 2, 4, 3$, respectively, and prove that under the left regular representation of $G$ into $S_4$ the nonidentity elements are mapped as follows:
        \begin{align*}
            a \mapsto (1\, 2)(3\, 4) & & b \mapsto(1\, 4)(2\, 3) & & c \mapsto (1\, 3)(2\, 4).
        \end{align*}
        \begin{proof}
            The left regular representation of $G$ into $S_4$ is the homomorphism $\varphi: G \to S_4$ defined by $\varphi(g) = \sigma_g$, where $\sigma_g: G \to G$ is the permutation of $G$ defined by $\sigma_g(x) = gx$ for all $x \in G$.

            Each non-identity element maps the elements as follows:
            \begin{align*}
                \sigma_a(1) = a1 = a & & \sigma_a(a) = a^2 = 1 & & \sigma_a(b) = ab = c & & \sigma_a(c) = ac = b \\
                \sigma_b(1) = b1 = b & & \sigma_b(a) = ba = c & & \sigma_b(b) = b^2 = 1 & & \sigma_b(c) = bc = a \\
                \sigma_c(1) = c1 = c & & \sigma_c(a) = ca = b & & \sigma_c(b) = cb = a & & \sigma_c(c) = c^2 = 1. 
            \end{align*}
            By the given labeling, this assigns the elements $a, b$, and $c$ to the pairs of 2-cycles shown above.
        \end{proof}
    \item Relabel $1, a, b, c$ as $1, 4, 2, 3$, respectively, and compute the image of each element of $G$ under the left regular representation of $G$ into $S_4$. Show that the image of $G$ in $S_4$ under this labeling is the same \emph{subgroup} as the image of $G$ in part (a) (even though the nonidentity elements individually map to different permutations under the two different labelings).
    \begin{proof}
        Under this labeling, the elements $a, b$, and $c$ are mapped to the permutations $(1\, 4)(2\, 3), (1\, 2)(3\, 4)$, and $(1\, 3)(2\, 4)$, respectively. Although each element maps to a different permutation from part (a), the subgroup of $S_4$ is the same in both cases.
    \end{proof}
\end{enumerate}

\end{document}