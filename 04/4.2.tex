\documentclass{article}

\title{Dummit \& Foote Ch. 4.2: Groups Acting on Themselves by Left Multiplication — Cayley's Theorem}
\author{Scott Donaldson}
\date{Feb. 2024}
\usepackage{amsmath, amsthm, amsfonts, amssymb, enumitem, tabu, tikz}

\begin{document}

\maketitle

Let $G$ be a group and let $H$ be a subgroup of $G$.

\section*{1. (2/12/24)}

Let $G = \{ 1, a, b, c \}$ be the Klein 4-group whose group table is written out in Section 2.5.
\begin{enumerate}[label=(\alph*), itemsep=0em]
    \item Label $1, a, b, c$ with the integers $1, 2, 4, 3$, respectively, and prove that under the left regular representation of $G$ into $S_4$ the nonidentity elements are mapped as follows:
        \begin{align*}
            a \mapsto (1\, 2)(3\, 4) & & b \mapsto(1\, 4)(2\, 3) & & c \mapsto (1\, 3)(2\, 4).
        \end{align*}
        \begin{proof}
            The left regular representation of $G$ into $S_4$ is the homomorphism $\varphi: G \to S_4$ defined by $\varphi(g) = \sigma_g$, where $\sigma_g: G \to G$ is the permutation of $G$ defined by $\sigma_g(x) = gx$ for all $x \in G$.

            Each non-identity element maps the elements as follows:
            \begin{align*}
                \sigma_a(1) = a1 = a & & \sigma_a(a) = a^2 = 1 & & \sigma_a(b) = ab = c & & \sigma_a(c) = ac = b \\
                \sigma_b(1) = b1 = b & & \sigma_b(a) = ba = c & & \sigma_b(b) = b^2 = 1 & & \sigma_b(c) = bc = a \\
                \sigma_c(1) = c1 = c & & \sigma_c(a) = ca = b & & \sigma_c(b) = cb = a & & \sigma_c(c) = c^2 = 1. 
            \end{align*}
            By the given labeling, this assigns the elements $a, b$, and $c$ to the pairs of 2-cycles shown above.
        \end{proof}
    \item Relabel $1, a, b, c$ as $1, 4, 2, 3$, respectively, and compute the image of each element of $G$ under the left regular representation of $G$ into $S_4$. Show that the image of $G$ in $S_4$ under this labeling is the same \emph{subgroup} as the image of $G$ in part (a) (even though the nonidentity elements individually map to different permutations under the two different labelings).
    \begin{proof}
        Under this labeling, the elements $a, b$, and $c$ are mapped to the permutations $(1\, 4)(2\, 3), (1\, 2)(3\, 4)$, and $(1\, 3)(2\, 4)$, respectively. Although each element maps to a different permutation from part (a), the subgroup of $S_4$ is the same in both cases.
    \end{proof}
\end{enumerate}

\section*{2. (2/12/24)}

List the elements of $S_3$ as $1, (1\, 2), (2\, 3), (1\, 3), (1\, 2\, 3), (1\, 3\, 2)$ and label these with the integers $1, 2, 3, 4, 5, 6$, respectively. Exhibit the image of each element of $S_3$ under the left regular representation of $S_3$ into $S_6$.

\begin{proof}[Solution]
    First, consider the element $(1\, 2)$. We see that:
    \begin{align*}
        (1\,2)1 &= (1\,2) \mapsto 2 &  (1\,2)(1\,2) &= 1 \mapsto 1 \\
        (1\,2)(2\,3) &= (1\,2\,3) \mapsto 5 & (1\,2)(1\,3) &= (1\,3\,2) \mapsto 6 \\
        (1\,2)(1\,2\,3) &= (2\,3) \mapsto 3 & (1\,2)(1\,3\,2) &= (1\,3) \mapsto 4.
    \end{align*}
    So the left regular representation of $(1\,2)$ under the given labeling in $S_6$ is $(1\,2)(3\,4)(5\,6)$.

    The left regular representations of the remaining elements are:
    \begin{align*}
        (2\,3) &\mapsto (1\,3)(2\,6)(4\,5) \\
        (1\,3) &\mapsto (1\,4)(2\,5)(3\,6) \\
        (1\,2\,3) &\mapsto (1\,5\,6)(2\,4\,3) \\
        (1\,3\,2) &\mapsto (1\,6\,5)(2\,3\,4).
    \end{align*}
\end{proof}

\section*{3. (2/12/24)}

Let $r$ and $s$ be the usual generators for the dihedral group of order 8.
\begin{enumerate}[label=(\alph*), itemsep=0em]
    \item List the elements of $D_8$ as $1, r, r^2, r^3, s, sr, sr^2, sr^3$ and label these with the integers $1, 2, ..., 8$, respectively. Exhibit the image of each element of $D_8$ under the left regular representation of $D_8$ into $S_8$.
        \begin{align*}
            1 &\mapsto 1 \\
            r &\mapsto (1\,2\,3\,4)(5\,8\,7\,6) \\
            r^2 &\mapsto (1\,3)(2\,4)(5\,7)(6\,8) \\
            r^3 &\mapsto (1\,4\,3\,2)(5\,6\,7\,8) \\
            s &\mapsto (1\,5)(2\,6)(3\,7)(4\,8) \\
            sr &\mapsto (1\,6)(2\,7)(3\,8)(4\,5) \\
            sr^2 &\mapsto (1\,7)(2\,8)(3\,5)(4\,6) \\
            sr^3 &\mapsto (1\,8)(2\,5)(3\,6)(4\,7)
        \end{align*}
    \item Relabel this same list of elements of $D_8$ with the integers $1, 3, 5, 7, 2, 4, 6, 8$ respectively and recompute the image of each element of $D_8$ under the left regular representation with respect to this new labeling. Show that the two subgroups of $S_8$ obtained in parts (a) and (b) are different.
        \begin{align*}
            1 &\mapsto 1 \\
            r &\mapsto (1\,3\,5\,7)(2\,8\,6\,4) \\
            r^2 &\mapsto (1\,5)(2\,6)(3\,7)(4\,8) \\
            r^3 &\mapsto (1\,7\,5\,3)(2\,4\,6\,8) \\
            s &\mapsto (1\,2)(3\,4)(5\,6)(7\,8) \\
            sr &\mapsto (1\,4)(2\,7)(3\,6)(5\,8) \\
            sr^2 &\mapsto (1\,6)(2\,5)(3\,8)(4\,7) \\
            sr^3 &\mapsto (1\,8)(2\,3)(4\,5)(6\,7).
        \end{align*}
        We see that the generators of the subgroups of $S_8$ in parts (a) and (b) are different, and so these are different subgroups of $S_8$.
\end{enumerate}

\section*{4. (2/12/24)}

Use the left regular representation of $Q_8$ to produce two elements of $S_8$ which generate a subgroup of $S_8$ isomorphic to the quaternion group $Q_8$.

\begin{proof}
    We know that the elements $i$ and $j$ generate the quaternion group $Q_8$. Labeling the elements $1, -1, i, -i, j, -j, k, -k$ with $1, 2, ..., 8$ respectively, the elements $i$ and $j$ map to the following permutations in $S_8$:
    \begin{align*}
        i &\mapsto (1\,3\,2\,4)(5\,7\,6\,8) \\
        j &\mapsto (1\,5\,2\,6)(3\,8\,4\,7).
    \end{align*}
    Since the left regular representation of $Q_8$ in $S_8$ is a homomorphism, these two permutations generate a subgroup of $S_8$ isomorphic to $Q_8$.
\end{proof}

\section*{5. (2/12/24)}

Let $r$ and $s$ be the usual generators for the dihedral group of order 8 and let $H = \langle s \rangle$. List the left cosets of $H$ in $D_8$ as $1H, rH, r^2H, r^3H$.
\begin{enumerate}[label=(\alph*), itemsep=0em]
    \item Label these cosets with the integers $1, 2, 3, 4$, respectively. Exhibit the image of each element of $D_8$ under the representation $\pi_H$ of $D_8$ into $S_4$ obtained from the action of $D_8$ by left multiplication on the set of 4 left cosets of $H$ in $D_8$. Deduce that this representation is faithful (i.e., the elements of $S_4$ obtained form a subgroup isomorphic to $D_8$).
        \begin{align*}
            1 &\mapsto 1 & s &\mapsto (2\,4) \\
            r &\mapsto (1\,2\,3\,4) & sr &\mapsto (1\,4)(2\,3) \\
            r^2 &\mapsto (1\,3)(2\,4) & sr^2 &\mapsto (1\,3) \\
            r^3 &\mapsto (1\,4\,3\,2) & sr^3 &\mapsto (1\,2)(3\,4).
        \end{align*}
        Since each element of $D_8$ induces a unique permutation in $S_4$, the resulting image under the left regular representation is isomorphic to $D_8$, and so this representation is faithful.
    \item Repeat part (a) with the list of cosets relabeled by the integers $1, 3, 2, 4$, respectively. Show that the permutations obtained from this labeling form a subgroup of $S_4$ that is different from the subgroup obtained in part (a).
        \begin{align*}
            1 &\mapsto 1 & s &\mapsto (3\,4) \\
            r &\mapsto (1\,3\,2\,4) & sr &\mapsto (1\,4)(2\,3) \\
            r^2 &\mapsto (1\,2)(3\,4) & sr^2 &\mapsto (1\,2) \\
            r^3 &\mapsto (1\,4\,2\,3) & sr^3 &\mapsto (1\,3)(2\,4).
        \end{align*}
        Since the generators (the images of $r$ and $s$) of this subgroup of $S_4$ are different from those in part (a), this is a different subgroup from part (a).
    \item Let $K = \langle sr \rangle$, list the cosets of $K$ in $D_8$ as $1K, rK, r^2K, r^3K$, and label these with the integers $1, 2, 3, 4$. Prove that, with respect to this labeling, the image of $D_8$ under the representation $\pi_K$ obtained from left multiplication on the cosets of $K$ is the same \emph{subgroup} of $S_4$ as in part (a) (even though the subgroups $H$ and $K$ are different and some of the elements of $D_8$ map to different permutations under the two homomorphisms).
        \begin{proof}
            Consider the images of the generators $r$ and $s$ under $\pi_K$:
            \begin{align*}
                r \cdot 1K &= rK & s \cdot 1K &= rK \\
                r \cdot rK &= r^2K & s \cdot rK &= 1K \\
                r \cdot r^2K &= r^3K & s \cdot r^2K &= r^3K \\
                r \cdot r^3K &= 1K & s \cdot r^3K &= r^2K.
            \end{align*}
            So $r$ and $s$ map to $(1\,2\,3\,4)$ and $(1\,2)(3\,4) \in S_4$, respectively. These elements are both in the subgroup in part (a) above, and so they are the same subgroup, but the image of $s$ is different.
        \end{proof}
\end{enumerate}

\section*{6. (2/15/24)}

Let $r$ and $s$ be the usual generators for the dihedral group of order 8 and let $N = \langle r^2 \rangle$. List the left cosets of $N$ in $D_8$ as $1N, rN, sN$, and $srN$. Label these cosets with the integers $1, 2, 3, 4$ respectively. Exhibit the image of each element of $D_8$ under the representation $\pi_N$ of $D_8$ into $S_4$ obtained from the action of $D_8$ by left multiplication on the set of 4 left cosets of $N$ in $D_8$. Deduce that this representation is not faithful and prove that $\pi_N(D_8)$ is isomorphic to the Klein 4-group.

\begin{proof}[Solution]
    \begin{align*}
        1 &\mapsto 1 & s &\mapsto (1\,3)(2\,4) \\
        r &\mapsto (1\,2)(3\,4) & sr &\mapsto (1\,4)(2\,3) \\
        r^2 &\mapsto 1 & sr^2 &\mapsto (1\,3)(2\,4) \\
        r^3 &\mapsto (1\,2)(3\,4) & sr^3 &\mapsto (1\,4)(2\,3).
    \end{align*}
    The left regular representation assigns 1 and $r^2$ to the identity permutation, so this action is not faithful.

    The image of $D_8$ under $\pi_N$ consists of the four permutations $1, (1\,2)(3\,4), \newline (1\,3)(2\,4)$, and $(1\,4)(2\,3)$. From Ch. 2.5, Exercise 10, this is isomorphic to the Klein 4-group $V_4$.
\end{proof}

\section*{7. (2/15/24)}

Let $Q_8$ be the quaternion group of order 8.

\begin{enumerate}[label=(\alph*), itemsep=0em]
    \item Prove that $Q_8$ is isomorphic to a subgroup of $S_8$.
        \begin{proof}
            From Exercise 4, $Q_8$ is isomorphic to
            \begin{equation*}
                \langle (1\,3\,2\,4)(5\,7\,6\,8), (1\,5\,2\,6)(3\,8\,4\,7) \rangle \in S_8.
            \end{equation*}
        \end{proof}
    \item Prove that $Q_8$ is not isomorphic to a subgroup of $S_n$ for any $n \leq 7$.
        \begin{proof}
            Let $A$ be a set with $|A| = n \leq 7$, let $a \in A$, and let $\cdot$ be the action of $Q_8$ on $A$. We attempt to find a subgroup of $S_n$ that is isomorphic to $Q_8$ by considering the permutation representations of the elements of $Q_8$.
            
            Now if $i \cdot a = j \cdot a$, then the permutation representations $\sigma_i$ and $\sigma_j$ are equal to each other, and so $Q_8$ is not isomorphic to the resulting subgroup of $S_n$. Further (without loss of generality), if $i \cdot a = -i \cdot a$, then:
            \begin{equation*}
                i \cdot a = -i \cdot a \Rightarrow -i \cdot i \cdot a = -i \cdot -i \cdot a \Rightarrow a = -1 \cdot a,
            \end{equation*}
            and so the permutation representation of $-1$ is equal to the identity permutation, which implies that $Q_8$ is not isomorphic to the subgroup of $S_n$. Therefore the elements $\pm i, \pm j, \pm k$ must all assign $a$ to different elements. However, these 6 unique elements together with $a$ are at least all of $A$, and so we must have $-1 \cdot a = a$. Thus $Q_8$ is not isomorphic to a subgroup of $S_n$.
        \end{proof}
\end{enumerate}

\section*{9. (2/16/24)}

Prove that if $p$ is a prime and $G$ is a group of order $p^a$ for some $a \in \mathbb{Z}^+$, then every subgroup of index $p$ is normal in $G$. Deduce that every group of order $p^2$ has a normal subgroup of order $p$.

\begin{proof}
    Let $H$ be a subgroup of $G$ with $[G:H] = p$. Let $gH$ be the left coset of $H$ by some element $g \in G$.
    
    Suppose that, for some $n < p$, $g^n \in H$ and let $h = g^n$. Since $p$ is a prime, there exists a positive integer $k$ such that $kn = 1$ (mod $p$). Then $h^k = g^{kn} = g$, which implies that $g \in H$. We conclude that, if we restrict to $g \notin H$, then for all $n < p$, $g^n \notin H$. This implies that $\{ H, gH, g^2 H, ..., g^{p - 1} H \}$ is a set of $p$ distinct cosets of $H$. Because the index of $H$ in $G$ is $p$, this must be all the cosets of $H$ in $G$, and so $g^p \in H$.

    Now by the operation defined on left cosets of $H$ by $aH \cdot bH = (ab)H$, we see that this is isomorphic to the cyclic group $Z_p$. We conclude by Theorem 6(d) of Chapter 3.1 that $H$ is normal in $G$.

    Further, if $|G| = p^2$, then by Cauchy's Theorem it contains an element of order $p$ which generates a subgroup of order $p$. This subgroup has index $p^2/p = p$, and so from above, is a normal subgroup of order $p$.
\end{proof}

\section*{10. (2/20/24)}

Prove that every non-abelian group of order 6 has a nonnormal subgroup of order 2. Use this to classify groups of order 6.

\begin{proof}
    Let $G$ be a group of order 6. From Cauchy's Theorem, let $g, x \in G$ such that $|g| = 2$ and $|x| = 3$. If $gx = xg$, then $G \cong Z_2 \times Z_3 \cong Z_6$, and so $G$ is abelian; therefore we must have $gx \neq xg$.
    
    So $G$ contains the elements $1, g, x, x^2, gx$, and $xg$. Consider the element $gx^2$. By the cancellation laws we can see that it is not equal to $1, g, x, x^2$, or $gx$. Therefore it must be equal to $xg$, so we can write $xg = gx^2$. We now see that $G \cong D_6 = \langle s, r \mid s^2 = r^3 = 1, rs = sr^2 \rangle$. We note that $\langle s \rangle$ is a nonnormal subgroup of $D_6$, because $rsr^{-1} = rsr^2 = sr^2 r^2 = sr \notin \langle s \rangle$.
    
    Finally, we conclude that every group of order 6 is isomorphic to either the cyclic group or the dihedral group of order 6.
\end{proof}

\section*{11. (2/20/24)}

Let $G$ be a finite group and let $\pi: G \rightarrow S_G$ be the left regular representation. Prove that if $x$ is an element of $G$ of order $n$ and $|G| = mn$, then $\pi(x)$ is a product of $m$ $n$-cycles.

Deduce that $\pi(x)$ is an odd permutation if and only if $|x|$ is even and $\dfrac{|G|}{|x|}$ is odd.

\begin{proof}
    Let $x \in G$, $|x| = n$, $|G| = mn$, and consider the \emph{right} cosets of the cyclic subgroup $\langle x \rangle$. There are $m$ such cosets of $\langle x \rangle$; let $1, y_2, ..., y_m$ be representatives of the right cosets, so that $\{ \langle x \rangle 1, \langle x \rangle y_2, ..., \langle x \rangle y_m \}$ forms a partition of $G$.

    Now consider the cycle decomposition of $\pi(x)$. Certainly it contains at least one $n$-cycle, namely the cycle that contains the elements of the cyclic subgroup $\langle x \rangle$: $(1\, x\, x^2\, ...\, x^{n - 1})$. Every successive representative of the above right cosets of $\langle x \rangle$ also induces a (disjoint) $n$-cycle with the same order that we would list the elements of the coset, for example, $(y_2\, xy_2\, x^2y_2\, ...\, x^{n - 1}y_2)$. Since there are $m$ representatives each with a unique $n$-cycle, we conclude that the cycle decomposition of $\pi(x)$ consists of $m$ $n$-cycles.

    From Chapter 3.5, Proposition 25, a permutation is odd if and only if the number of cycles of even length in its cycle decomposition is odd. Therefore we conclude that $\pi(x)$ is odd if and only if $m = \dfrac{|G|}{|x|}$ (the number of cycles) is odd, and $n = |x|$ (the cycle length) is even.
\end{proof}

\section*{12. (2/20/24)}

Let $G$ and $\pi$ be as in the preceding exercise. Prove that if $\pi(G)$ contains an odd permutation then $G$ has a subgroup of index 2.

\begin{proof}
    Consider the homomorphism $\epsilon: \pi(G) \rightarrow \{ \pm{1} \}$ described in Chapter 3.5 which assigns a permutation to 1 if it is an even permutation and $-1$ if it is an odd permutation. Since $\pi(G)$ contains an odd permutation, $\epsilon$ is surjective, and so $\ker \pi(G) \neq \pi(G)$. Since $\epsilon$ is a homomorphism, there must be the same number of elements assigned to both 1 and $-1$. Therefore $|\pi(G) : \ker \pi(G)| = 2$, and so $G$ contains a subgroup of index 2.
\end{proof}

\section*{13. (2/20/24)}

Prove that if $|G| = 2k$ where $k$ is odd then $G$ has a subgroup of index 2.

\begin{proof}
    From Cauchy's Theorem, let $x \in G$ such that $|x| = 2$. From Exercise 11, since $|x| = 2$, even, and $\dfrac{|G|}{|x|} = \dfrac{2k}{2} = k$, odd, $\pi(x)$ is an odd permutation. From Exercise 12, since $\pi(G)$ contains an odd permutation, $G$ contains a subgroup of index 2.
\end{proof}

\section*{14. (2/20/24)}

Let $G$ be a finite group of composite order $n$ with the property that $G$ has a subgroup of order $k$ for each positive integer $k$ dividing $n$. Prove that $G$ is not simple.

\begin{proof}
    By definition, a group $G$ is simple if it contains no proper normal subgroups other than 1 and $G$ itself. Therefore, it suffices to show that $G$ contains at least one proper normal subgroup.

    Let $p$ be the smallest prime dividing $n$ and let $n = pk$. Then $G$ contains a subgroup of order $k$ that has index $n/k = p$. By Corollary 5, this subgroup is normal and, since it is a proper subgroup, $G$ is therefore not simple.
\end{proof}

\end{document}