\documentclass{article}

\title{Dummit \& Foote Ch. 4.3: Groups Acting on Themselves by Conjugation — The Class Equation}
\author{Scott Donaldson}
\date{Feb. 2024}
\usepackage{amsmath, amsthm, amsfonts, amssymb, enumitem, tabu, tikz}

\hfuzz=3pt

\begin{document}

\maketitle

Let $G$ be a group.

\section*{1. (2/22/24)}

Suppose $G$ has a left action on a set $A$, denoted by $g \cdot a$ for all $g \in G$ and $a \in A$. Denote the corresponding right action on $A$ by $a \cdot g$. Prove that the (equivalence) relations $\sim$ and $\sim'$ defined by
\begin{equation*}
    a \sim b \text{ if and only if } a = g \cdot b \text{ for some } g \in G
\end{equation*}
and
\begin{equation*}
    a \sim' b \text{ if and only if } a = b \cdot g \text{ for some } g \in G
\end{equation*}
are the same relation (i.e., $a \sim b$ if and only $a \sim' b$).

\begin{proof}
    To show that $a \sim b$ implies $a \sim' b$, we must show that, given a $g \in G$ with $a = g \cdot b$, there exists an $h \in G$ such that $a = b \cdot h$. By definition, the corresponding right action of a left action is specified to be $g \cdot x = x \cdot g^{-1}$ for all $g \in G$, $x \in A$. Letting $h = g^{-1}$, we have found an element where $a = g \cdot b = b \cdot h$, and so $a \sim' b$.

    The proof for $a \sim' b$ implies $a \sim b$ is identical, letting $h = g^{-1}$ but with $h$ acting on the left.
\end{proof}

\section*{2. (2/22/24)}

Find all conjugacy classes and their sizes in the following groups:
\begin{enumerate}[label=(\alph*), itemsep=0em]
    \item $D_8$:
        \begin{align*}
            \{ 1 \}_1 & & \{ r^2 \}_1 & & \{ r, r^3 \}_2 & & \{ s, sr^2 \}_2 & & \{ sr, sr^3 \}_2
        \end{align*}
    \item $Q_8$:
        \begin{align*}
            \{ 1 \}_1 & & \{ -1 \}_1 & & \{ \pm{i} \}_2 & & \{ \pm{j} \}_2 & & \{ \pm{k} \}_2
        \end{align*}
    \item $A_4$:
        \begin{gather*}
            \{ 1 \}_1 \hspace{1em} \{ (1\, 2\, 3), (1\, 3\, 4), (1\, 4\, 2), (2\, 4\, 3) \}_4 \hspace{1em} \{ (1\, 3\, 2), (1\, 2\, 4), (1\, 4\, 3), (2\, 3\, 4) \}_4 \\ \{ (1\, 2)(3\, 4), (1\, 3)(2\, 4), (1\, 4)(2\, 3) \}_3
        \end{gather*}
\end{enumerate}

\section*{3. (2/22/24)}

Find all the conjugacy classes and their sizes in the following groups:
\begin{enumerate}[label=(\alph*), itemsep=0em]
    \item $Z_2 \times S_3$:
        \begin{gather*}
            \{ (0, 1) \}_1 \hspace{1em} \{ (1, 1) \}_1 \hspace{1em} \{ (0, (1\, 2)), (0, (1\, 3)), (0, (2\, 3)) \}_3 \\ \{ (1, (1\, 2)), (1, (1\, 3)), (1, (2\, 3)) \}_3 \hspace{1em} \{ (0, (1\, 2\, 3)), (0, (1\, 3\, 2)) \}_2 \\ \{ (1, (1\, 2\, 3)), (1, (1\, 3\, 2)) \}_2
        \end{gather*}
    \item $S_3 \times S_3$:
        \begin{gather*}
            \{ (1, 1) \}_1 \hspace{1em} \{ (1, \text{2-cycle}) \}_3 \hspace{1em} \{ (\text{2-cycle}, 1) \}_3 \hspace{1em} \{ (1, \text{3-cycle}) \}_2 \hspace{1em} \{ (\text{3-cycle}, 1) \}_2 \\
            \{ (\text{2-cycle}, \text{2-cycle}) \}_9 \hspace{1em} \{ (\text{2-cycle}, \text{3-cycle}) \}_6 \hspace{1em} \{ (\text{3-cycle}, \text{2-cycle}) \}_6 \\* \{ (\text{3-cycle}, \text{3-cycle}) \}_4
        \end{gather*}
    \item $Z_3 \times A_4$ (using representatives from the conjugacy classes of $A_4$ above):
        \begin{gather*}
            \{ (0, 1) \}_1 \hspace{1em} \{ (1, 1) \}_1 \hspace{1em} \{ (2, 1) \}_1 \\
            \{ (0, \overline{(1\, 2\, 3)}) \}_4 \hspace{1em} \{ (1, \overline{(1\, 2\, 3)}) \}_4 \hspace{1em} \{ (2, \overline{(1\, 2\, 3)}) \}_4 \\
            \{ (0, \overline{(1\, 3\, 2)}) \}_4 \hspace{1em} \{ (1, \overline{(1\, 3\, 2)}) \}_4 \hspace{1em} \{ (2, \overline{(1\, 3\, 2)}) \}_4 \\
            \{ (0, \overline{(1\, 2)(3\, 4)}) \}_3 \hspace{1em} \{ (1, \overline{(1\, 2)(3\, 4)}) \}_3 \hspace{1em} \{ (2, \overline{(1\, 2)(3\, 4)}) \}_3
        \end{gather*}
\end{enumerate}

\end{document}