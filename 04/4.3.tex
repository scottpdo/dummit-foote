\documentclass{article}

\title{Dummit \& Foote Ch. 4.3: Groups Acting on Themselves by Conjugation — The Class Equation}
\author{Scott Donaldson}
\date{Feb. 2024}
\usepackage{amsmath, amsthm, amsfonts, amssymb, enumitem, tabu, tikz}

\hfuzz=3pt

\begin{document}

\maketitle

Let $G$ be a group.

\section*{1. (2/22/24)}

Suppose $G$ has a left action on a set $A$, denoted by $g \cdot a$ for all $g \in G$ and $a \in A$. Denote the corresponding right action on $A$ by $a \cdot g$. Prove that the (equivalence) relations $\sim$ and $\sim'$ defined by
\begin{equation*}
    a \sim b \text{ if and only if } a = g \cdot b \text{ for some } g \in G
\end{equation*}
and
\begin{equation*}
    a \sim' b \text{ if and only if } a = b \cdot g \text{ for some } g \in G
\end{equation*}
are the same relation (i.e., $a \sim b$ if and only $a \sim' b$).

\begin{proof}
    To show that $a \sim b$ implies $a \sim' b$, we must show that, given a $g \in G$ with $a = g \cdot b$, there exists an $h \in G$ such that $a = b \cdot h$. By definition, the corresponding right action of a left action is specified to be $g \cdot x = x \cdot g^{-1}$ for all $g \in G$, $x \in A$. Letting $h = g^{-1}$, we have found an element where $a = g \cdot b = b \cdot h$, and so $a \sim' b$.

    The proof for $a \sim' b$ implies $a \sim b$ is identical, letting $h = g^{-1}$ but with $h$ acting on the left.
\end{proof}

\section*{2. (2/22/24)}

Find all conjugacy classes and their sizes in the following groups:
\begin{enumerate}[label=(\alph*), itemsep=0em]
    \item $D_8$:
        \begin{align*}
            \{ 1 \}_1 & & \{ r^2 \}_1 & & \{ r, r^3 \}_2 & & \{ s, sr^2 \}_2 & & \{ sr, sr^3 \}_2
        \end{align*}
    \item $Q_8$:
        \begin{align*}
            \{ 1 \}_1 & & \{ -1 \}_1 & & \{ \pm{i} \}_2 & & \{ \pm{j} \}_2 & & \{ \pm{k} \}_2
        \end{align*}
    \item $A_4$:
        \begin{gather*}
            \{ 1 \}_1 \hspace{1em} \{ (1\, 2\, 3), (1\, 3\, 4), (1\, 4\, 2), (2\, 4\, 3) \}_4 \hspace{1em} \{ (1\, 3\, 2), (1\, 2\, 4), (1\, 4\, 3), (2\, 3\, 4) \}_4 \\ \{ (1\, 2)(3\, 4), (1\, 3)(2\, 4), (1\, 4)(2\, 3) \}_3
        \end{gather*}
\end{enumerate}

\section*{3. (2/22/24)}

Find all the conjugacy classes and their sizes in the following groups:
\begin{enumerate}[label=(\alph*), itemsep=0em]
    \item $Z_2 \times S_3$:
        \begin{gather*}
            \{ (0, 1) \}_1 \hspace{1em} \{ (1, 1) \}_1 \hspace{1em} \{ (0, (1\, 2)), (0, (1\, 3)), (0, (2\, 3)) \}_3 \\ \{ (1, (1\, 2)), (1, (1\, 3)), (1, (2\, 3)) \}_3 \hspace{1em} \{ (0, (1\, 2\, 3)), (0, (1\, 3\, 2)) \}_2 \\ \{ (1, (1\, 2\, 3)), (1, (1\, 3\, 2)) \}_2
        \end{gather*}
    \item $S_3 \times S_3$:
        \begin{gather*}
            \{ (1, 1) \}_1 \hspace{1em} \{ (1, \text{2-cycle}) \}_3 \hspace{1em} \{ (\text{2-cycle}, 1) \}_3 \hspace{1em} \{ (1, \text{3-cycle}) \}_2 \hspace{1em} \{ (\text{3-cycle}, 1) \}_2 \\
            \{ (\text{2-cycle}, \text{2-cycle}) \}_9 \hspace{1em} \{ (\text{2-cycle}, \text{3-cycle}) \}_6 \hspace{1em} \{ (\text{3-cycle}, \text{2-cycle}) \}_6 \\* \{ (\text{3-cycle}, \text{3-cycle}) \}_4
        \end{gather*}
    \item $Z_3 \times A_4$ (using representatives from the conjugacy classes of $A_4$ above):
        \begin{gather*}
            \{ (0, 1) \}_1 \hspace{1em} \{ (1, 1) \}_1 \hspace{1em} \{ (2, 1) \}_1 \\
            \{ (0, \overline{(1\, 2\, 3)}) \}_4 \hspace{1em} \{ (1, \overline{(1\, 2\, 3)}) \}_4 \hspace{1em} \{ (2, \overline{(1\, 2\, 3)}) \}_4 \\
            \{ (0, \overline{(1\, 3\, 2)}) \}_4 \hspace{1em} \{ (1, \overline{(1\, 3\, 2)}) \}_4 \hspace{1em} \{ (2, \overline{(1\, 3\, 2)}) \}_4 \\
            \{ (0, \overline{(1\, 2)(3\, 4)}) \}_3 \hspace{1em} \{ (1, \overline{(1\, 2)(3\, 4)}) \}_3 \hspace{1em} \{ (2, \overline{(1\, 2)(3\, 4)}) \}_3
        \end{gather*}
\end{enumerate}

\section*{4. (2/22/24)}

Prove that if $S \subseteq G$ and $g \in G$ then $gN_g(S)g^{-1} = N_G(gSg^{-1})$ and $gC_g(S)g^{-1} = C_G(gSg^{-1})$.

\begin{proof}
    Let $x \in N_G(S)$. So $xsx^{-1} \in S$ for all $s \in S$. Then
        \begin{align*}
            gxsx^{-1}g^{-1} &\in gSg^{-1} \\
            gxg^{-1}gsg^{-1}gx^{-1}g^{-1} &\in gSg^{-1} \\
            (gxg^{-1})gsg^{-1}(gx^{-1}g^{-1}) &\in gSg^{-1} \\
            (gxg^{-1})gsg^{-1}(gxg^{-1})^{-1} &\in gSg^{-1},
        \end{align*}
    which implies that $gxg^{-1} \in N_G(gSg^{-1})$, and so $gN_G(S)g^{-1} \subseteq N_G(gSg^{-1})$.

    Conversely, let $x \in N_G(gSg^{-1})$. So $xgsg^{-1}x^{-1} \in gSg^{-1}$ for all $s \in S$. Then
        \begin{align*}
            xgsg^{-1}x^{-1} &\in gSg^{-1} \\
            g^{-1}xgsg^{-1}x^{-1} &\in Sg^{-1} \\
            g^{-1}xgsg^{-1}x^{-1}g &\in S \\
            (g^{-1}xg)s(g^{-1}xg)^{-1} &\in S \\
            g^{-1}xg &\in N_G(S) \\
            x &\in gN_G(S)g^{-1},
        \end{align*}
    which shows that $N_G(gSg^{-1}) \subseteq gN_G(S)g^{-1}$. This proves that $N_G(gSg^{-1}) = gN_G(S)g^{-1}$.

    Next, let $x \in C_G(S)$. So $xs = sx$ for all $s \in S$. Then
        \begin{align*}
            xs &= sx \\
            gsxg^{-1} &= gsxg^{-1} \\
            gsg^{-1}gxg^{-1} &= gsg^{-1}gxg^{-1} \\
            (gsg^{-1})(gxg^{-1}) &= (gsg^{-1})(gxg^{-1}),
        \end{align*}
        and so $gxg^{-1} \in C_G(gSg^{-1})$, which implies that $gC_G(S)g^{-1} \subseteq C_G(gSg^{-1})$.
    Finally, let $x \in C_G(gSg^{-1})$. So $x(gsg^{-1}) = (gsg^{-1})x$ for all $x \in S$. Then
        \begin{align*}
            xgsg^{-1} &= gsg^{-1}x \\
            g^{-1}xgsg^{-1} &= sg^{-1}x \\
            g^{-1}xgs &= sg^{-1}xg \\
            (g^{-1}xg)s &= s(g^{-1}xg),
        \end{align*}
        which implies that $g^{-1}xg \in C_G(S)$, so $x \in gC_G(S)g^{-1}$. It follows that \\ $C_G(gSg^{-1}) \subseteq gC_G(S)g^{-1}$, and therefore $gC_g(S)g^{-1} = C_G(gSg^{-1})$.
\end{proof}

% \section*{5. (2/26/24)}

% If the center of $G$ is of index $n$, prove that every conjugacy class has at most $n$ elements.

% \begin{proof}
% \end{proof}

\section*{9. (3/7/24)}

Show that $|C_{S_n}((1\,2)(3\,4))| = 8 \cdot (n - 4)!$ for all $n \geq 4$. Determine the elements in this centralizer explicitly.

\begin{proof}
    In $S_4$, the permutations that commute with $(1\,2)(3\,4)$ are the four elements of the cyclic subgroup generated by it, as well as the transpositions $(1\,2)$ and $(3\,4)$, and the 4-cycles $(1\,3\,2\,4)$ and $(1\,4\,2\,3)$.

    Now let $n > 4$. Consider the product of one of the elements of $C_{S_4}((1\,2)(3\,4))$ with an element of $S_{n}$. If the permutation only acts on ${1, 2, 3, 4}$, then it is already in $C_{S_4}((1\,2)(3\,4))$. If the permutation only acts on $\{ 5, ..., n \}$ then it is disjoint with (thus commutes with) the permutations in $C_{S_4}((1\,2)(3\,4))$. Now $S_{\{5, ..., n\}} \cong S_{n - 4}$, therefore there are $(n - 4)!$ such permutations. Since the product of any of these permutations with an element of $C_{S_4}((1\,2)(3\,4))$ must commute with $(1\,2)(3\,4)$, there are thus $8 \cdot (n - 4)!$ elements in $C_{S_n}((1\,2)(3\,4))$.
\end{proof}

\section*{10. (2/28/24)}

Let $\sigma$ be the 5-cycle $(1\, 2\, 3\, 4\, 5)$ in $S_5$. In each of (a) to (c) find an explicit element $\tau \in S_5$ which accomplishes the specified conjugation:

\begin{enumerate}[label=(\alph*), itemsep=0em]
    \item $\tau \sigma \tau^{-1} = \sigma^2 = (1\, 3\, 5\, 2\, 4)$. Let $\tau = (2\, 3\, 5\, 4)$. Then $\tau \sigma \tau^{-1} = \\ (\tau(1)\, \tau(2)\, \tau(3)\, \tau(4)\, \tau(5)) = (1\, 3\, 5\, 2\, 4) = \sigma^2$.
    \item $\tau \sigma \tau^{-1} = \sigma^{-1} = (1\, 5\, 4\, 3\, 2)$. Let $\tau = (2\, 5)(3\, 4)$. Then $\tau \sigma \tau^{-1} = \sigma^{-1}$.
    \item $\tau \sigma \tau^{-1} = \sigma^{-2} = (1\, 4\, 2\, 5\, 3)$. Let $\tau = (2\, 4\, 5\, 3)$. Then $\tau \sigma \tau^{-1} = \sigma^{-2}$.
\end{enumerate}

\section*{11. (2/28/24)}

In each of (a) - (d) determine whether $\sigma_1$ and $\sigma_2$ are conjugate. If they are, give an explicit permutation $\tau$ such that $\tau \sigma_1 \tau^{-1} = \sigma_2$.

\begin{enumerate}[label=(\alph*), itemsep=0em]
    \item $\sigma_1 = (1\, 2)(3\, 4\, 5)$ and $\sigma_2 = (1\, 2\, 3)(4\, 5)$. Both have cycle type $1,1,3$ and so they are conjugate. Let $\tau = (1\, 4\, 2\, 5\, 3)$. Then $\tau \sigma_1 \tau^{-1} = \sigma_2$.
    \item $\sigma_1 = (1\, 5)(3\, 7\, 2)(10\, 6\, 8\, 11)$ and $\sigma_2 = (3\, 7\, 5\, 10)(4\, 9)(13\, 11\, 2)$. In $S_13$, both have cycle type $1,1,1,1,2,3,4$ and so they are conjugate. Let $\tau = \\ (1\, 4)(2\, 11\, 10\, 3)(5\, 9\, 6\, 7\, 13\, 8)$. Then $\tau \sigma_1 \tau^{-1} = \sigma_2$.
    \item $\sigma_1 = (1\, 5)(3\, 7\, 2)(10\, 6\, 8\, 11)$ and $\sigma_2 = \sigma_1^3 = (1\, 5)(10\, 11\, 8\, 6)$. They do not have the same cycle type ($\sigma_1$ contains a 3-cycle that $\sigma_2$ does not), and so they are not conjugate.
    \item $\sigma_1 = (1\, 3)(2\, 4\, 6)$ and $\sigma_2 = (3\, 5)(2\, 4)(5\, 6) = (2\, 4)(3\, 5\, 6)$. Let $\tau = (1\, 2\, 3\, 4\, 5)$. Then $\tau \sigma_1 \tau^{-1} = \sigma_2$.
\end{enumerate}

\section*{13. (2/28/24)}

Find all finite groups which have exactly two conjugacy classes.

\begin{proof}
    Let $G$ be a non-trivial finite group. Since the conjugacy class of $1$ is $\{ 1 \}$, if $G$ has exactly two conjugacy classes, then every other element in $G$ must have the same conjugacy class, namely $G - \{ 1 \}$.

    From Proposition 6, for any $g \in G$, the number of conjugates of $g$ (i.e. the cardinality of the conjugacy class of $g$) is the index of the centralizer of $g$, $|G:C_G(g)|$. Therefore the size of the conjugacy class of $g$ must divide the order of $G$.

    Let $|G| = n$. Then the size of the conjugacy class of $g$ is $|G - \{ 1 \}| = n - 1$. This is only possible when $|G| = 2$, and so $G$ must be the unique group of order two.
\end{proof}

\section*{17. (3/25/24)}

Let $A$ be a nonempty set and let $X$ be any subset of $A$. Let
\begin{equation*}
    F(X) = \{ a \in A \mid \sigma(a) = a \text{ for all } \sigma \in X \} \text{ — the \emph{fixed set} of } X.
\end{equation*}
Let $M(X) = A - F(X)$ be the elements which are \emph{moved} by some element of $X$. Let $D = \{ \sigma \in S_A \mid |M(\sigma)| < \infty \}$. Prove that $D$ is a normal subgroup of $S_A$.

\begin{proof}
    Let $\sigma \in D$. Then $M(\sigma)$ is a finite subset of $A$. Let $\tau \in S_A$ and consider $M(\tau \sigma \tau^{-1})$. If $|M(\tau \sigma \tau^{-1})|$ is also finite, then $\tau \in D$, and so $D \unlhd S_A$. Now $|M(\sigma)| = |A - F(\sigma)| = |A| - |F(\sigma)|$, and $|A|$ is constant. Therefore, by proving that $|F(\sigma)| = |F(\tau \sigma \tau^{-1})|$, it follows that $|M(\sigma)| = |M(\tau \sigma \tau^{-1})|$, which proves that $D \unlhd S_A$. We will now construct a bijection between $|F(\sigma)|$ and $|F(\tau \sigma \tau^{-1})|$.
    
    Let $\varphi: F(\sigma) \rightarrow F(\tau \sigma \tau^{-1})$ be defined by $\varphi(a) = \tau(a)$. The map $\varphi$ is well-defined, because:
    \begin{equation*}
        a \in F(\sigma) \Rightarrow a = \sigma(a) \Rightarrow \tau(a) = \tau(\sigma(a)) = (\tau \sigma \tau^{-1}) (\tau(a)),
    \end{equation*}
    which implies that $\varphi(a) = \tau(a) \in F(\tau \sigma \tau^{-1})$.

    It is injective: Let $a, b \in F(\sigma)$ and suppose that $\varphi(a) = \varphi(b)$. Then $\tau(a) = \tau(b)$, and since $\tau$ is a permutation, by definition we have $a = b$.
    
    Finally, $\varphi$ is surjective. Let $b \in F(\tau \sigma \tau^{-1})$ (to show that there exists an $a \in F(\sigma)$ such that $\varphi(a) = b$). Let $a = \tau^{-1}(b)$. Then $b = \tau(a)$, so $\varphi(a) = b$. We show that $\sigma(a) = a$:
    \begin{align*}
        \tau \sigma \tau^{-1}(b) &= b \\
        (\tau \sigma \tau^{-1})(\tau(a)) &= \tau(a) \\
        \tau (\sigma(a)) &= \tau(a) \\
        \sigma(a) &= a,
    \end{align*}
    which implies that $a \in F(\sigma)$, so $\varphi$ is surjective. Therefore $\varphi$ is a bijection between $F(\sigma)$ and $F(\tau \sigma \tau^{-1})$, which (as noted above), proves that $D$ is a normal subgroup of $S_A$.
\end{proof}

\end{document}