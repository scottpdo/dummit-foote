\documentclass{article}

\title{Dummit \& Foote Ch. 4.1: Group Actions and Permutation Representations}
\author{Scott Donaldson}
\date{Dec. 2023 - Jan. 2024}
\usepackage{amsmath, amsthm, amsfonts, amssymb, enumitem, tabu, tikz}

\begin{document}

\maketitle

Let $G$ be a group and $A$ be a nonempty set.

\section*{1. (12/24/23)}

Let $G$ act on the set $A$. Prove that if $a, b \in A$ and $b = g \cdot a$ for some $g \in G$, then $G_b = gG_ag^{-1}$ ($G_a$ is the stabilizer of $a$). Deduce that if $G$ acts transitively on $A$ then the kernel of the action is $\cap_{g \in G}\hspace{0.25em}gG_ag^{-1}$.

\begin{proof}
    We will show first that $G_b$, the stabilizer of $b$, is contained in $gG_ag^{-1}$, and then show the converse, which proves that they are equal.

    Let $x \in G_b$, so $x \cdot b = b$. Then:
    \begin{align*}
        x \cdot g \cdot a &= g \cdot a \text{ ($b = g \cdot a$)} \\
        (gg^{-1}) \cdot (xg) \cdot a &= g \cdot a \text{ ($gg^{-1} = 1, 1 \cdot a = a$)} \\
        g \cdot (g^{-1}xg) \cdot a &= g \cdot a \\
        (g^{-1}xg) \cdot a &= a,
    \end{align*}
    which implies that $g^{-1}xg \in G_a$, and therefore $x \in gG_ag^{-1}$, so $G_b \subseteq gG_ag^{-1}$.

    The converse, that $gG_ag^{-1} \subseteq G_b$, can be shown by following the above proof in reverse (that is, let $x \in gG_ag^{-1}$, so $g^{-1}xg \in G_a$, which implies that $(g^{-1}xg) \cdot a = a$, and each assertion holds from bottom to top). Since each is contained in the other, we have $G_b = gG_ag^{-1}$.

    Now we already know that the kernel of the group action of $G$ on $A$ is the intersection of the stabilizers of all the elements of $A$, that is, $\cap_{b \in A}\hspace{0.25em}G_b$. If $G$ acts transitively on $A$, fixing $a \in A$, then for all $b \in A$, we can write $b = g \cdot a$ for some $g \in G$, which from above implies that $G_b = gG_ag^{-1}$. We deduce that the kernel can be expressed in terms of a fixed element $a$, namely:
    \begin{equation*}
        \cap_{b \in A}\hspace{0.25em}G_b = \cap_{b \in A}\hspace{0.25em}\underbrace{gG_ag^{-1}}_{b = g \cdot a} = \cap_{g \in G}\hspace{0.25em}gG_ag^{-1}.
    \end{equation*}
    We know that $\cap_{g \in G}\hspace{0.25em}gG_ag^{-1}$ intersects all of the same conjugates as does $\cap_{b \in A}$, since $G$ acts transitively on $A$. And, since $b = g \cdot a \Rightarrow G_b = gG_ag^{-1}$, it intersects no conjugates not represented by $G_b$ for all $b \in A$.
\end{proof}

\section*{2. (1/2/24)}

Let $G$ be a \emph{permutation group} on the set $A$ (i.e., $G \leq S_A$), let $\sigma \in G$ and let $a \in A$. Prove that $\sigma G_a \sigma^{-1} = G_{\sigma(a)}$. Deduce that if $G$ acts transitively on $A$ then
\begin{equation*}
    \bigcap_{\sigma \in G} \sigma G_a \sigma^{-1} = 1.
\end{equation*}

\begin{proof}
    We first show that $\sigma G_a \sigma^{-1} \subseteq G_{\sigma(a)}$, and then show the converse. To begin, let $\tau \in G_a$ and consider $\sigma \tau \sigma^{-1} \in \sigma G_a \sigma^{-1}$. We note that:
    \begin{equation*}
        (\sigma \tau \sigma^{-1})(\sigma(a)) = (\sigma \tau \sigma^{-1} \sigma)(a) = (\sigma \tau)(a) = \underbrace{\sigma(\tau(a)) = \sigma(a)}_{\tau \in G_a \Rightarrow \tau(a) = a},
    \end{equation*}
    and so $\sigma \tau \sigma^{-1}$ stabilizes $\sigma(a)$, which implies that $\sigma G_a \sigma^{-1} \subseteq G_{\sigma(a)}$.

    For the converse, let $\tau \in G$ and suppose that $\sigma \tau \sigma^{-1} \in G_{\sigma(a)}$. Then:
    \begin{align*}
        (\sigma \tau \sigma^{-1})(\sigma(a)) &= \sigma(a) \\
        (\sigma \tau \sigma^{-1} \sigma)(a) &= \sigma(a) \\
        (\sigma \tau)(a) &= \sigma(a) \\
        \sigma(\tau(a)) &= \sigma(a) \\
        \tau(a) &= a,
    \end{align*}
    so $\tau$ is in the stabilizer of $a$, which implies that $\sigma \tau \sigma^{-1} \in \sigma G_a \sigma^{-1}$, and so $G_{\sigma(a)} \subseteq \sigma G_a \sigma^{-1}$.
    
    This concludes the proof that $\sigma G_a \sigma^{-1} = G_{\sigma(a)}$.

    Now if $G$ acts transitively on $A$, then there is only one orbit; that is, given some $a \in A$, for all $b \in A$, there is a $\sigma \in G$ such that $b = \sigma(a)$.

    From above, we conclude:
    \begin{equation*}
        \bigcap_{\sigma \in G} \sigma G_a \sigma^{-1} = \bigcap_{\sigma \in G} G_{\sigma(a)} =
        \bigcap_{a \in A} G_a \text{ (because $G$ acts transitively on $A$),}
    \end{equation*}
    and since the only permutation that fixes every element of $A$ is the identity, this intersection consists therefore only the identity permutation.
\end{proof}

\section*{3. (1/2/24)}

Assume that $G$ is an abelian, transitive subgroup of $S_A$. Show that $\sigma(a) \neq a$ for all $\sigma \in G - \{ 1 \}$ and all $a \in A$. Deduce that $|G| = |A|$. [Use the preceding exercise.]

\begin{proof}
    Suppose that $\sigma_1$ fixes $a$, so $\sigma_1(a) = a$, and let $\sigma_2(a) = b$. Then:
    \begin{align*}
        (\sigma_1 \circ \sigma_2)(a) = \sigma_1(\sigma_2(a)) &= \sigma_1(b), \text{ and } \\
        (\sigma_2 \circ \sigma_1)(a) = \sigma_2(\sigma_1(a)) &= \sigma_2(a).
    \end{align*}
    Since $G$ is abelian, these must be equal, and so $\sigma_1(b) = \sigma_2(a) = b$. Then $\sigma_1$ also fixes $b$.
    
    Since $G$ is transitive, for every $b \in A$, there exists a $\sigma \in G$ such that $\sigma(b) = a$, which implies that $\sigma_1$ fixes every element of $A$ and is therefore the identity. Thus the identity is the only element of $G$ for which $\sigma(a) = a$; equivalently, $\sigma(a) \neq a$ for all $\sigma \in G - \{ 1 \}$ and all $a \in A$.

    Now let $A = \{ 1, ..., n \}$. Since $G$ is transitive, it must contain at least $n$ permutations. For all $i \in A$, define $\sigma_i$ such that $\sigma_i(1) = i$ (with $\sigma_1$ the identity permutation). Suppose that $\tau$ is another permutation in $G$. Since $A$ only contains $n$ elements, we must have $\tau(1) = i$ for some $i \in A$, so $\tau(1) = \sigma_i(1)$. Then:
    \begin{align*}
        (\tau \circ \sigma_i)(1) = \tau(\sigma_i(1)) &= \tau(i), \text{ and } \\
        (\sigma_i \circ \tau)(1) = \sigma_i(\tau(1)) &= \sigma_i(i).
    \end{align*}
    Since $G$ is abelian, these are equal, so $\tau(i) = \sigma_i(i)$. It follows that, if $j = \tau(i) = \sigma_i(i)$, then $\tau(j) = \sigma_i(j)$, and so on for every element which $\sigma_i$ permutes. Therefore $\tau = \sigma_i$, so $G$ contains exactly $n$ permutations. We conclude that $|G| = |A|$.
\end{proof}

\section*{4. (1/3/24)}

Let $S_3$ act on the set $\Omega$ of ordered pairs: $\{ (i, j) \mid 1 \leq i, j \leq 3 \}$ by $\sigma((i, j)) = (\sigma(i), \sigma(j))$. Find the orbits of $S_3$ on $\Omega$. For each $\sigma \in S_3$ find the cycle decomposition of $\sigma$ under this action (i.e., find its cycle decomposition when $\sigma$ is considered as an element of $S_9$ — first fix a labelling of these nine ordered pairs). For each orbit $\mathcal{O}$ of $S_3$ acting on these nine points pick some $a \in \mathcal{O}$ and find the stabilizer of $a$ in $S_3$.

\begin{proof}[Solution]
    The elements $(1, 1), (2, 2),$ and $(3, 3)$ all belong to the same orbit. We see that $(1\,2) \cdot (1, 1) = (2, 2)$ and $(2\,3) \cdot (2, 2) = (3, 3)$. Further, since for any $(i, i) \in \Omega$, the action of $\sigma \in S_3$ on it results in an element with the same coordinates, there is no $(i, j) \in \Omega$ with $i \neq j$ such that $\sigma((i, i)) = (i, j)$. Therefore these three elements constitute one orbit.

    The other orbit consists of the remaining six elements. Beginning with $(1, 2)$, we have:
    \begin{align*}
        (1\,2\,3)(1, 2) &= (2, 3), \text{ then} \\
        (1\,2\,3)(2, 3) &= (3, 1), \\
        (1\,2)(3, 1) &= (3, 2), \\
        (1\,2\,3)(3, 2) &= (1, 3), \text{ and finally} \\
        (1\,2\,3)(1, 3) &= (2, 1).
    \end{align*}
    Conversely to the first orbit, for no $(i, j) \in \Omega$ with $i \neq j$ do we have $\sigma((i, j)) = (i, i)$. Thus these are the two disjoint orbits of $S_3$ on $\Omega$.

    Next, let us label the elements of $\Omega$:
    \begin{align*}
        (1, 1) \rightarrow 1 & & (1, 2) \rightarrow 2 & & (1, 3) \rightarrow 3 \\
        (2, 1) \rightarrow 4 & & (2, 2) \rightarrow 5 & & (2, 3) \rightarrow 6 \\
        (3, 1) \rightarrow 7 & & (3, 2) \rightarrow 8 & & (3, 3) \rightarrow 9
    \end{align*}
    Then we can describe the cycle decomposition of each permutation of $S_3$ by how it acts on these elements:
    \begin{align*}
        1 &\rightarrow 1 \\
        (1\,2) &\rightarrow (1\,5)(2\,4)(3\,6)(7\,8) \\
        (1\,3) &\rightarrow (1\,9)(2\,8)(3\,7)(4\,6) \\
        (2\,3) &\rightarrow (2\,3)(4\,7)(5\,9)(6\,8) \\
        (1\,2\,3) &\rightarrow (1\,5\,9)(2\,6\,7)(3\,4\,8) \\
        (1\,3\,2) &\rightarrow (1\,9\,5)(2\,7\,6)(3\,8\,4)
    \end{align*}
    For the first orbit, let us choose the point $(1, 1)$ and find its stabilizer. The permutations in $S_3$ that fix this element must fix 1. Obviously this includes the identity. Neither of the 3-cycles fix 1. Only one of the 2-cycles, $(2\,3)$, fixes 1. Therefore the stabilizer of $(1, 1)$ is the subgroup $\{ 1, (2\,3) \}$.

    For the second orbit, we choose $(1, 2)$. Since all non-identity permutations of $S_3$ reassign either 1 or 2 (or both), the stabilizer consists of only the identity.
\end{proof}

\section*{5. (1/3/24)}

For each of parts (a) and (b) repeat the preceding exercise but with $S_3$ acting on the specified set:
\begin{enumerate}[itemsep=0em,label=(\alph*)]
    \item the set of 27 triples $\{ (i, j, k) \mid 1 \leq i, j, k \leq 3 \}$
        \begin{itemize}[itemsep=0em]
            \item The orbit of any point $(i, j, k)$ consists of those other points whose coordinates are equal if $i, j$, or $k$ are equal or not if they are not. For example, the orbit of $(1, 1, 2)$ contains all those points whose first and second coordinates are the same and whose third is different, that is:
            \begin{equation*}
                (1, 1, 3), (2, 2, 1), (2, 2, 3), (3, 3, 1), (3, 3, 2)
            \end{equation*}
            (It is simple to find a 2-cycle that acts on $(1, 1, 2)$ to send it to any other point in this orbit.)

            For each point where at least one of the coordinates differs from the others (of the type $(i, i, j), (i, j, i), (j, i, i)$, or $(i, j, k)$), its orbit contains the 6 points of the same type. The three points $(1, 1, 1), (2, 2, 2)$, and $(3, 3, 3)$ are all in the same remaining orbit.
            \item Next, we label the 27 triples:
            \begin{align*}
                (1, 1, 1) \rightarrow 1 & & (1, 1, 2) \rightarrow 2 & & (1, 1, 3) \rightarrow 3 & & (1, 2, 1) \rightarrow 4 \\
                (1, 2, 2) \rightarrow 5 & & (1, 2, 3) \rightarrow 6 & & (1, 3, 1) \rightarrow 7 & & (1, 3, 2) \rightarrow 8 \\
                (1, 3, 3) \rightarrow 9 & & (2, 1, 1) \rightarrow 10 & & (2, 1, 2) \rightarrow 11 & & (2, 1, 3) \rightarrow 12 \\
                (2, 2, 1) \rightarrow 13 & & (2, 2, 2) \rightarrow 14 & & (2, 2, 3) \rightarrow 15 & & (2, 3, 1) \rightarrow 16 \\
                (2, 3, 2) \rightarrow 17 & & (2, 3, 3) \rightarrow 18 & & (3, 1, 1) \rightarrow 19 & & (3, 1, 2) \rightarrow 20 \\
                (3, 1, 3) \rightarrow 21 & & (3, 2, 1) \rightarrow 22 & & (3, 2, 2) \rightarrow 23 & & (3, 2, 3) \rightarrow 24 \\
                (3, 3, 1) \rightarrow 25 & & (3, 3, 2) \rightarrow 26 & & (3, 3, 3) \rightarrow 27
            \end{align*}
            So we can describe each permutation in $S_3$ by how it acts on these elements:
            \begin{align*}
                1 &\rightarrow 1 \\
                (1\,2) &\rightarrow (1\,14)(2\,13)(3\,15)(4\,11)(5\,10)(6\,12)(7\,17)\\ &\hspace{1.5em}(8\,16)(9\,18)(19\,23)(20\,22)(21\,24)(25\,26) \\
                (1\,3) &\rightarrow (1\,27)(2\,26)(3\,25)(4\,24)(5\,23)(6\,22)(7\,21)\\ &\hspace{1.5em}(8\,20)(9\,19)(10\,18)(11\,17)(12\,16)(13\,15) \\
                (2\,3) &\rightarrow (2\,3)(4\,7)(5\,9)(6\,8)(10\,19)(11\,21)(12\,20)\\ &\hspace{1.5em}(13\,25)(14\,27)(15\,26)(16\,22)(17\,24)(18\,23) \\
                (1\,2\,3) &\rightarrow (1\,14\,27)(2\,15\,25)(3\,13\,26)(4\,17\,21)(5\,18\,19)\\ &\hspace{1.5em}(6\,16\,20)(7\,11\,24)(8\,12\,22)(9\,10\,23) \\
                (1\,3\,2) &\rightarrow (1\,27\,14)(2\,25\,15)(3\,26\,13)(4\,21\,17)(5\,19\,18)\\ &\hspace{1.5em}(6\,20\,16)(7\,24\,11)(8\,22\,12)(9\,23\,10)
            \end{align*}
            \item Finally, for each orbit, we choose an element from it and find its stabilizer:
            \begin{align*}
                (1, 1, 1) \text{ has stabilizer } \{ 1, (2, 3) \}.
            \end{align*}
            For each remaining orbit, since an element from has at least two coordinates that are different, each non-identity element of $S_3$ reassigns it, so the stabilizer of all elements from the other orbits is simply the identity.
        \end{itemize}
    \item the set $\mathcal{P}(\{ 1, 2, 3 \}) - \{ \emptyset \}$ of all 7 nonempty subsets of $\{ 1, 2, 3 \}$.
        \begin{itemize}[itemsep=0em]
            \item There are three orbits, which partition the power set by how many elements each child set contains (ex. one orbit consists of the three singleton sets, namely $\{ \{1\}, \{2\}, \{3\} \}$).
            \item We label each element:
            \begin{align*}
                \{ 1 \} \rightarrow 1 & & \{ 2 \} \rightarrow 2 & & \{ 3 \} \rightarrow 3 \\
                \{ 1, 2 \} \rightarrow 4 & & \{ 1, 3 \} \rightarrow 5 & & \{ 2, 3 \} \rightarrow 6 \\
                \{ 1, 2, 3 \} \rightarrow 7
            \end{align*}
            So we can describe each permutation in $S_3$ by how it acts on these elements:
            \begin{align*}
                1 &\rightarrow 1 \\
                (1\,2) &\rightarrow (1\,2)(5\,6) \\
                (1\,3) &\rightarrow (1\,3)(4\,6) \\
                (2\,3) &\rightarrow (2\,3)(4\,5) \\
                (1\,2\,3) &\rightarrow (1\,2\,3)(4\,6\,5) \\
                (1\,3\,2) &\rightarrow (1\,3\,2)(4\,5\,6)
            \end{align*}
            \item Finally, for each orbit, we choose an element from it and find its stabilizer. In the orbit of singletons, the stabilizer of $\{ 1 \}$ is $\{ 1, (2\,3) \}$. In the orbit of doubles, the stabilizer of $\{ 1, 2 \}$ is $\{ 1, (1\,2) \}$. In the orbit that consists only of the triple $\{ 1, 2, 3 \}$, the stabilizer is all of $S_3$.
        \end{itemize}
\end{enumerate}

\section*{6. (1/18/24)}

As in Exercise 12 of Section 2.2 let $R$ be the set of all polynomials with integer coefficients in the independent variables $x_1, x_2, x_3, x_4$ and let $S_4$ act on $R$ by permuting the indices of the four variables:
\begin{equation*}
    \sigma \cdot p(x_1, x_2, x_3, x_4) = p(x_{\sigma(1)}, x_{\sigma(2)}, x_{\sigma(3)}, x_{\sigma(4)})
\end{equation*}
for all $\sigma \in S_4$.

\begin{enumerate}[itemsep=0em, label=(\alph*)]
    \item Find the polynomials in the orbit of $S_4$ on $R$ containing $x_1 + x_2$ (recall from Exercise 12 in Section 2.2 that the stabilizer of this polynomial has order 4).
        \begin{align*}
            x_1 + x_2 & & x_1 + x_3 & & x_1 + x_4 \\
            x_2 + x_3 & & x_2 + x_4 & & x_3 + x_4
        \end{align*}
    \item Find the polynomials in the orbit of $S_4$ on $R$ containing $x_1 x_2 + x_3 x_4$ (recall from Exercise 12 in Section 2.2 that hte stabilizer of this polynomial has order 8).
        \begin{align*}
            x_1 x_2 + x_3 x_4 & & x_1 x_3 + x_2 x_4 & & x_1 x_4 + x_2 x_3
        \end{align*}
    \item Find the polynomials in the orbit of $S_4$ on $R$ containing $(x_1 + x_2)(x_3 + x_4)$.
        \begin{align*}
            (x_1 + x_2)(x_3 + x_4) & & (x_1 + x_3)(x_2 + x_4) & & (x_1 + x_4)(x_2 + x_3)
        \end{align*}
\end{enumerate}

\section*{8. (1/19/24)}

A transitive permutation group $G$ on a set $A$ is called \emph{doubly transitive} if for any (hence all) $a \in A$ the subgroup $G_a$ is transitive on the set $A - \{ a \}$.

\begin{enumerate}[itemsep=0em, label=(\alph*)]
    \item Prove that $S_n$ is doubly transitive on $\{ 1, 2, ..., n \}$ for all $n \geq 2$.
        \begin{proof}
            From Chapter 3.2, Exercise 15, the stabilizer of any element of $\{ 1, 2, ..., n \}$ in $S_n$ is isomorphic to $S_{n - 1}$ and consists of all the permutations (in particular, transpositions) of the remaining $n - 1$ elements. Therefore the stabilizer acts transitively on the remaining elements, and $S_n$ is thus doubly transitive.
        \end{proof}
    \item Prove that a doubly transitive group is primitive. Deduce that $D_8$ is not doubly transitive in its action on the 4 vertices of a square.
        \begin{proof}
            Let $G$ be doubly transitive on $A$. Suppose that $B \subseteq A$ is a block and $a, b \in B$.

            Now suppose that $c \in A$ is not in the block $B$. Since $G$ is doubly transitive, $G_a$ acts transitively on $A - \{ a \}$. Therefore there exists $g \in G_a$ such that $c = g \cdot b$. However, since $g \in G_a$, we know that $g \cdot a = a \in B$, which implies that $g \cdot b$ also lies in $B$, contradicting $c = g \cdot b \notin B$. Then $c$ must \emph{also} lie in the block $B$, and so $B$ is all of $G$.

            Since any block containing at least two elements must be the entire group, this proves that a doubly transitive group is primitive.

            Next, suppose that $D_8$ acts on the 4 vertices of a square $1, 2, 3, 4$:
            \begin{align*}
                r \cdot 1 = 2 & & r \cdot 2 = 3 & & r \cdot 3 = 4 & & r \cdot 4 = 1 \\
                s \cdot 1 = 1 & & s \cdot 2 = 4 & & s \cdot 3 = 3 & & s \cdot 4 = 2
            \end{align*}
            Then the stabilizer of 1 in $D_8$ — the subgroup $\langle s \rangle$ — is not transitive on $\{ 2, 3, 4 \}$, because there is no element $x \in \langle s \rangle$ such that $2 = x \cdot 3$ (in fact, 3 is a fixed point). Therefore $D_8$ is not doubly transitive in its action on the 4 vertices of a square.
        \end{proof}
\end{enumerate} 

\end{document}