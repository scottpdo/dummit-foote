\documentclass{article}

\title{Dummit \& Foote Ch. 2.2: Centralizers and Normalizers, Stabilizers and Kernels}
\author{Scott Donaldson}
\date{Jun. 2023}
\usepackage{amsmath, amsthm, amsfonts, enumitem}

\begin{document}

\maketitle

\section*{1. (6/5/23)}

Prove that $C_G(A) = \{ g \in G \mid g^{-1}ag = a \text{ for all } a \in A \}$.

\begin{proof}
    By definition, $C_G(A) = \{ g \in G \mid gag^{-1} = a \text{ for all } a \in A \}$ (that is, it is the set of elements of $G$ that commute with all elements of $A$).

    Let $g \in C_G(A), a \in A$. Then $gag^{-1} = a$, which implies that $ga = ag$, and so left-multiplying by $g^{-1}$ we obtain $a = g^{-1}ag$. Therefore, equivalently, $C_G(A)$ is the set of elements $g \in G$ such that $g^{-1}ag = a$ for all $a \in A$.
\end{proof}

\section*{2. (6/5/23)}

Prove that $C_G(Z(G)) = G$ and deduce that $N_G(Z(G)) = G$.

\begin{proof}
    Recall that $Z(G) = \{g \in G \mid gx = xg \text{ for all } x \in G \}$. Let $z \in Z(G)$, so $z$ commutes with every element of $G$.
    
    Also recall that $C_G(A) = \{ g \in G \mid gag^{-1} = a \text{ for all } a \in A \}$. When $A = Z(G)$, then every element of $g$ commutes with every element of $A$. Therefore for all $g \in G$, $g \in C_G(Z(G))$. Thus $C_G(Z(G)) = G$.

    Note that, since $C_G(A) \leq N_G(A)$ for all subsets $A$, we must have $G = C_G(Z(G)) \leq N_G(Z(G))$. Since there is no greater set of elements, we also have $N_G(Z(G)) = G$.
\end{proof}

\section*{3. (6/8/23)}

Prove that if $A$ and $B$ are subsets of $G$ with $A \subseteq B$ then $C_G(B)$ is a subgroup of $C_G(A)$.

\begin{proof}
    Let $a \in A \text{ and } g \in C_G(B)$. Then $g$ commutes with every element of $b$, that is, $gb = bg \Rightarrow gbg^{-1} = b$ for all $b \in B$. Since $A \subseteq B$, we also have $gag^{-1} = a$ for all $a \in A$. Therefore $g \in C_G(A)$, which implies that $C_G(B) \subseteq C_G(A)$.

    From the introduction to this chapter, centralizers are subgroups, so both $C_G(B) \leq G$ and $C_G(A) \leq G$. Since $C_G(B)$ is contained within $C_G(A)$ and both are subgroups of $G$, $C_G(B)$ must be closed within $C_G(A)$ and closed under inverses within $C_G(A)$, so it is also a subgroup of $C_G(A)$.
\end{proof}

\section*{4. (6/8/23)}

For each of $S_3$, $D_8$, and $Q_8$ compute the centralizers of each element and find the center of each group.

\subsection*{$S_3$}

\begin{itemize}
    \item $C_{S_3}((1)) = S_3$
    \item $C_{S_3}((1, 2)) = \{ (1), (1, 2) \}$
    \item $C_{S_3}((1, 3)) = \{ (1), (1, 3) \}$
    \item $C_{S_3}((2, 3)) = \{ (1), (2, 3) \}$
    \item $C_{S_3}((1, 2, 3)) = C_{S_3}((1, 3, 2)) = \{ (1), (1, 2, 3), (1, 3, 2) \}$
\end{itemize}

The center $Z(S_3)$ consists only of the identity permutation.

\subsection*{$D_8$}

\begin{itemize}
    \item $C_{D_8}(1) = D_8$
    \item $C_{D_8}(r) = C_{D_8}(r^2) = C_{D_8}(r^3) = \{ 1, r, r^2, r^3 \}$
    \item $C_{D_8}(s) = C_{D_8}(sr^2) = \{ 1, r^2, s, sr^2 \}$
    \item $C_{D_8}(sr) = C_{D_8}(sr^3) = \{ 1, r^2, sr, sr^3 \}$
\end{itemize}

The center $Z(D_8)$ is $\{ 1, r^2 \}$.

\subsection*{$Q_8$}

\begin{itemize}
    \item $C_{D_8}(1) = C_{D_8}(-1) = Q_8$
    \item $C_{D_8}(i) = C_{D_8}(-i) = \{ 1, -1, i, -i \}$
    \item $C_{D_8}(j) = C_{D_8}(-j) = \{ 1, -1, j, -j \}$
    \item $C_{D_8}(k) = C_{D_8}(-k) = \{ 1, -1, k, -k \}$
\end{itemize}

The center $Z(Q_8)$ is $\{ 1, -1 \}$.

\section*{5. (6/8/23)}

In each of parts (a) through (c) show that for the specified group $G$ and subgroup $A$ of $G$, $C_G(A) = A$ and $N_G(A) = G$.

\begin{enumerate}[label=(\alph*)]
    \item $G = S_3$ and $A = \{ (1), (1, 2, 3), (1, 3, 2) \}$.
          \begin{proof}
            From Exercise 4, we have $C_G((1, 2, 3)) = C_G((1, 3, 2)) = A$. No other non-identity permutation is in any of the centralizers of any element of $A$, therefore $C_G(A) = A$.

            Next, consider $\sigma^{-1}(1, 2, 3)\sigma$ for some other permutation in $S_3$, for example $(1, 2)(1, 2, 3)(1, 2)$. This is equal to $(1, 3, 2)$, which is an element of $A$, so $(1, 2)$ is in the normalizer of $A$. Since $C_G(A) \leq N_G(A)$ for all $A$, $A \subseteq N_G(A)$, and it follows that $N_G(A)$ consists of at least $A$ and the element $(1, 2)$. Then, because $N_G(A)$ is a subgroup, it is closed under permutation composition, and therefore must contain all elements of $S_3$.
          \end{proof}
    \item $G = D_8$ and $A = \{ 1, s, r^2, sr^2 \}$.
          \begin{proof}
            We know that $C_G(A)$ is a subgroup of $G$, and from Exercise 4, we have $A \leq C_G(A)$ (since $A$ is commutative). Then $|C_G(A)| \geq 4$. By Lagrange's Theorem, the order of $C_G(A)$ divides the order of $G$, 8. Then we must have either $C_G(A) = A$ or $C_G(A) = G$. However, $r$ is not in the centralizer of $A$, because $rsr^{-1} = rsr^3 = sr^{-1}r^3 = sr^2 \neq s$. Therefore $C_G(A) = A$.

            When we consider the normalizer of $A$, note that $rsr^{-1} = sr^2 \in A$. Thus $N_G(A)$ is a subgroup of $G$ that contains both $A$ and the element $r$. By closing the subgroup, we obtain $N_G(A) = G$.
          \end{proof}
    \item $G = D_{10}$ and $A = \{ 1, r, r^2, r^3, r^4 \}$.
          \begin{proof}
            Since $A$ consists only of powers of $r$, $A$ is commutative, and so (as above) $A \leq C_G(A)$. The centralizer of $A$ does not contain the element $s$, because $s^{-1}rs = srs = ssr^4 = r^4 \neq r$. Then we must have $|A| = 5 \leq |C_G(A)| \leq 9 = |G - \{ s \}|$. Again by Lagrange's Theorem, the order of $C_G(A)$ must divide 10, and since it at least 5 and at most 9, it must be 5. Therefore $C_G(A) = A$.

            When we consider the normalizer of $A$, note that $s^{-1}r^4s = r \in A$. Thus $N_G(A)$ is a subgroup of $G$ that contains both $A$ and the element $s$. By closing the subgroup, we obtain $N_G(A) = G$.
          \end{proof}
\end{enumerate}

\section*{6. (6/9/23)}

Let $H$ be a subgroup of the group $G$.

\begin{enumerate}[label=(\alph*)]
    \item Show that $H \leq N_G(H)$. Give an example to show that this is not necessarily true if $H$ is not a subgroup.
          \begin{proof}
            Let $h_1, h_2 \in H$ (to show that $h_1 \in N_G(H)$). Because $H$ is a subgroup of $G$, it is closed and closed under inverses, so $h_1 h_2 h_1^{-1} \in H$. So the conjugate of every element with every other element of $H$ is in $H$, which implies that $H \leq N_G(H)$.

            However, this does not follow if $H$ is merely a subset of $G$. For example, let $G = D_6$ and $H = \{ s, r \}$. Then $rsr^{-1} = sr^2r^2 = sr \notin H$, which implies that $r \notin H$. Therefore $H$ is not contained within its normalizer.
          \end{proof}
    \item Show that $H \leq C_G(H)$ if and only if $H$ is abelian.
          \begin{proof}
            First, let $H$ be abelian and let $h_1, h_2 \in H$. Because $H$ is abelian, we have $h_1 h_2 = h_2 h_1 \Rightarrow h_2 = h_1 h_2 h_1^{-1}$, so the conjugate of $h_2$ by $h_1$ is $h_2$. Thus the arbitrary element $h_1$ is in the centralizer of $H$, and so $H \leq C_G(H)$.

            Next, let $H \leq C_G(H)$. Then for all $h_1, h_2 \in H$, $h_2 = h_1 h_2 h_1^{-1} \Rightarrow h_2 h_1 = h_1 h_2$, and so $H$ is an abelian subgroup of $G$.
          \end{proof}
\end{enumerate}

\section*{7. (6/13/23)}

Let $n \in \mathbb{Z}$ with $n \geq 3$. Prove the following:

\begin{enumerate}[label=(\alph*)]
    \item $Z(D_{2n}) = \{ 1 \}$ if $n$ is odd
          \begin{proof}
            Recall that $Z(D_{2n}) = \{ x \in D_{2n} \mid xy = yx \text{ for all } y \in D_{2n} \}$. Let $x \in Z(D_{2n}), y \in D_{2n}$. We will consider separately the cases where $x = r^k$ and $x = sr^k$.

            Suppose $x = r^k$ for some $0 < k < n$ (clearly if $x = r^0 = 1$, then it is in the center of $D_{2n}$). If $y = s$, then $xy = r^ks = sr^{-k}$ and $yx = sr^k$. These are only equal when $k = -k$ (mod $n$); since $n$ is odd there are no values of $k$ that satisfy this equality, and so $x = r^k$ does not commute with every element of $D_{2n}$ and is not in $Z(D_{2n})$.

            Next, suppose $x = sr^k$. Then if $y = r$, we have $xy = sr^kr = sr^{k + 1}$ and $yx = rsr^k = sr^{-1}r^k = sr^{k - 1}$. No values of $k$ satisfy this equality and so no $x$ of the form $sr^k$ is in $Z(D_{2n})$. Thus the center of $D_{2n}$ consists of only the identity when $n$ is odd.
          \end{proof}
    \item $Z(D_{2n}) = \{ 1, r^k \} $ if $n = 2k$
          \begin{proof}
            The case where $x = sr^k$ is identical to the above proof; if $y = r$ then they do not commute and so no $x$ of the form $sr^k$ is in $Z(D_{2n})$.

            Consider $x = r^k$ for some $0 < k < n$. If $y = r^p, 0 \leq p < n$, then they commute because both elements are powers of $r$. So let $y = sr^p$. Then $xy = r^ksr^p = sr^{-k}r^p = sr^{p - k}$ and $yx = sr^pr^k = sr^{p + k}$. These are equal to each other when $p - k = p + k$, that is, when $-k = k$ (mod $n$), which implies that $2k = n$. Since $n$ is even, there is a value of $k$ for which this occurs, $n / 2$.

            Thus the center of $D_{2n}$ when $n = 2k$ is $\{ 1, r^k \}$.
          \end{proof}
\end{enumerate}

\section*{8. (6/13/23)}

Let $G = S_n$, fix an $i \in \{ 1, 2, ..., n \}$ and let $G_i = \{ \sigma \in G \mid \sigma(i) = i \}$ (the stabilizer of $i$ in $G$). Use group actions to prove that $G_i$ is a subgroup of $G$. Find $|G_i|$.

\begin{proof}
    There is a group action of $G$ on $\{ 1, ..., n \}$ defined by $\sigma \cdot k = \sigma(k)$. The identity permutation applied to any $k$ is always $k$, and closure is easily demonstrated by composition of permutations.

    Now let $\sigma_1, \sigma_2 \in G_i$ (to show that $\sigma_1 \circ \sigma_2 \in G_i$). Then $\sigma_1(i) = i$ and $\sigma_2(i) = i$. It follows that $\sigma_1(\sigma_2(i)) = \sigma_1(i) = i$, and since this is equal to $(\sigma_1 \circ \sigma_2)(i)$, $\sigma_1 \circ \sigma_2$ is in $G_i$, so it is closed.

    Next, note that $\sigma(i) = i$ for some $\sigma \in G_i$ implies that $i = \sigma^{-1}(i)$, so $\sigma^{-1}$ is also in $G_i$ and it is therefore closed under inverses. Thus $G_i$ is a subgroup of $G$.

    To find the order of $G_i$, recall from Ch. 1.3 that the order of $S_n$ is $n!$. Further, $G_i$ consists of those permutations of $S_n$ whose cycle decompositions do not include $i$. We will show that $G_i$ has the same cardinality as $S_{n - 1}$ and that its order is therefore $(n - 1)!$.

    Let $\varphi: G_i \rightarrow S_{n - 1}$ be defined on elements of $\{ 1, ..., n \}$ by $\varphi(\sigma(m)) = \sigma(m)$  if $m < i$ and $= \sigma(m) - 1$ if $m > i$. For example, if $i = 10$, $\varphi$ maps the permutation with cycle decomposition $(1, 5, 9, 13, 17)$ to $(1, 5, 9, 12, 16)$.
    
    $\varphi$ is one-to-one: If $\varphi(\sigma_1(m)) = \varphi(\sigma_2(m))$, then they are by definition equal if $\sigma_1(m)$ and $\sigma_2(m)$ are either both less than or both greater than $i$. Without loss of generality, suppose that $\sigma_1(m) < i$ and $\sigma_2(m) > i$. Then $\varphi(\sigma_1(m)) < i$ and $\varphi(\sigma_2(i)) \geq i$, so they cannot be equal.

    $\varphi$ is onto: Let $\sigma \in S_{n - 1}$. There is a unique permutation $G_i$ that maps to $\sigma$ whose cycle decomposition contains the same values in the same positions as $\sigma$ when those values are less than $i$, and the successor of those values in the same positions as $\sigma$ when those values are greater than $i$. Formally, the inverse $\varphi^{-1}: S_{n - 1} \rightarrow G_i$ is well-defined by $\varphi(\sigma(m)) = \sigma(m)$ if $m < i$ and $= \sigma(m) + 1$ if $m > i$.

    This proves that $\varphi$ is a bijection (note that the additional requirement that it is an isomorphism is unnecessary because we are only concerned with the size of these groups). Therefore $|G_i| = |S_{n - 1}| = (n - 1)!$.
\end{proof}

\section*{9. (6/13/23)}

For any subgroup $H$ of $G$ and any nonempty subset $A$ of $G$ define $N_H(A)$ to be the set $\{ h \in H \mid hAh^{-1} = A \}$. Show that $N_H(A) = N_G(A) \cap H$ and deduce that $N_H(A)$ is a subgroup of $H$ (note that $A$ need not be a subset of $H$).

\begin{proof}
    To show that $N_H(A) = N_G(A) \cap H$, we will show that membership in one implies membership in the other, and vice-versa.

    First, let $h \in N_H(A)$ (to show that $h \in N_G(A) \cap H$). Then $hAh^{-1} = A$. Also, by definition, $N_G(A) = \{ g \in G \mid gAg^{-1} = A \}$, so $h \in N_G(A)$. Further, since $N_H(A)$ consists of only those $h \in N_G(A)$ that are also in $H$, it follows that $h \in N_G(A) \cap H$.

    Next, let $h \in N_G(A) \cap H$, that is, $h \in N_G(A)$ and $h \in H$. Since $h \in N_G(A)$, $hAh^{-1} = A$. It follows immediately that $h \in N_H(A)$. Therefore $N_H(A) = N_G(A) \cap H$.

    Now from Ch. 2.1, exercise 10., the intersection of two subgroups (of $G$) is again a subgroup (of $G$). Since $N_H(A)$ is also restricted to $H$ and containment of subgroups is transitive, we deduce that $N_H(A)$ is a subgroup of $H$.
\end{proof}

\section*{10. (6/13/23)}

Let $H$ be a subgroup of order 2 in $G$. Show that $N_G(H) = C_G(H)$. Deduce that if $N_G(H) = G$ then $H \leq Z(G)$.

\begin{proof}
    Let $H = \{ 1, h \} \leq G$. In order to prove that $N_G(H) = C_G(H)$, we will show that membership in one implies membership in the other, and vice-versa.

    For some $g \in G$, let $g \in N_G(H)$. Then $gHg^{-1} = H$. Since $g \cdot 1 \cdot g^{-1} = 1$, we must have $ghg^{-1} = h$, which implies that $gh = hg$. Then $g$ commutes with both 1 and $h$, that is, with every element of $H$, and so $g \in C_G(H)$. Since we know that $C_G(H) \leq N_G(H)$, this proves that $N_G(H) = C_G(H)$.

    Next suppose that $N_G(H) = G$. Then for every $g \in G$, $gh = hg$. So an arbitrary element $g$ commutes with every element of $H$. Put differently, every element of $H$ commutes with every element of $G$. It follows that $H$ is contained in the center of $G$, that is, $H \leq Z(G)$.
\end{proof}

\section*{11. (6/14/23)}

Prove that $Z(G) \leq N_G(A)$ for any subset $A$ of $G$.

\begin{proof}
    Let $A$ be a subset of $G$ and let $g \in Z(G)$. Then $g$ commutes with every other element of $G$, so in particular $ga = ag$ for all $a \in A$. It follows that $gag^{-1} = a$ for all $a \in A$, and therefore that $gAg^{-1} = A$. Thus $g \in N_G(A)$, and so $Z(G)$ is contained in $N_G(A)$. By the transitivity of subgroups, we must also have $Z(G) \leq N_G(A)$.
\end{proof}

\section*{12. (6/17/23)}

Let $R$ be the set of all polynomials with integer coefficients in the independent variables $x_1, x_2, x_3, x_4$ i.e., the members of $R$ are finite sums of the form $ax_1^{r_1}x_2^{r_2}x_3^{r_3}x_4^{r_4}$, where $a$ is any integer and $r_1, ..., r_4$ are nonnegative integers. For example,
\begin{equation*}
    12x_1^5 x_2^7 x_4 - 18x_2^3 x_3 + 11 x_1^6 x_2 x_3^3 x_4^{23}
\end{equation*}
is a typical element of $R$. Each $\sigma \in S_4$ gives a permutation of $\{ x_1, ..., x_4 \}$ by defining $\sigma \cdot x_i = x_{\sigma(i)}$. This may be extended to a map from $R$ to $R$ by defining
\begin{equation*}
    \sigma \cdot p(x_1, x_2, x_3, x_4) = p(x_{\sigma(1)}, x_{\sigma(2)}, x_{\sigma(3)}, x_{\sigma(4)})
\end{equation*}
for all $p(x_1, x_2, x_3, x_4) \in R$ (i.e., $\sigma$ simply permutes the indices of the variables).

\begin{enumerate}[label=(\alph*)]
    \item Let $p = p(x_1, x_2, x_3, x_4)$ be the polynomial above, let $\sigma = (1, 2, 3, 4)$ and $\tau = (1, 2, 3)$. Compute:
        \begin{itemize}
            \item $\sigma \cdot p = 12x_1 x_2^5 x_3^7 - 18x_3^3 x_4 + 11 x_1^{23} x_2^6 x_3 x_4^3$
            \item $\begin{aligned}[t]
                \tau \cdot (\sigma \cdot p) = \tau \cdot 12x_1 x_2^5 x_3^7 - 18x_3^3 x_4 + 11 x_1^{23} x_2^6 x_3 x_4^3 = \\ 12x_1^7 x_2 x_3^5 - 18x_1^3 x_4 + 11x_1 x_2^{23} x_3^6 x_4^3\end{aligned}$
            \item $(\tau \circ \sigma) \cdot p = (1, 3, 4, 2) \cdot p = 12x_1^7 x_2 x_3^5 - 18x_1^3 x_4 + 11x_1 x_2^{23} x_3^6 x_4^3$
            \item $(\sigma \circ \tau) \cdot p = (1, 3, 2, 4) \cdot p = 12x_1 x_3^5 x_4^7 - 18x_2 x_4^3 + 11x_1^{23} x_2^3 x_3^6 x_4$
        \end{itemize}
    \item Prove that these definitions give a (left) group action of $S_4$ on $R$.
          \begin{proof}
            To show that these definitions give a group action, we have to show that $(1) \cdot p = p$, and $\sigma_1 \cdot (\sigma_2 \cdot p) = (\sigma_1 \circ \sigma_2) \cdot p$ for all $p \in R, \sigma_1, \sigma_2 \in S_4$.

            First, $(1) \cdot p(x_1, x_2, x_3, x_4) = p(x_1, x_2, x_3, x_4)$ satisfies the identity condition.

            Next, let $\sigma_1, \sigma_2 \in S_4$. Then:
            \begin{flalign*}
                \sigma_1 \cdot (\sigma_2 \cdot p(x_1, x_2, x_3, x_4)) = \sigma_1 \cdot p(x_{\sigma_2(1)}, x_{\sigma_2(2)}, x_{\sigma_2(3)}, x_{\sigma_2(4)}) = \\ p(x_{\sigma_1(\sigma_2(1))}, x_{\sigma_1(\sigma_2(2))}, x_{\sigma_1(\sigma_2(3))}, x_{\sigma_1(\sigma_2(4))}) = \\ (\sigma_1 \circ \sigma_2) \cdot p(x_1, x_2, x_3, x_4),
            \end{flalign*}
            
            as desired. Thus the definitions give a group action of $S_4$ on $R$.
          \end{proof}
    \item Exhibit all permutations in $S_4$ that stabilize $x_4$ and prove that they form a subgroup isomorphic to $S_3$.
          \begin{proof}
            Given the above group action of $S_4$ on $R$, a permutation stabilizes $x_4$ if its cycle decomposition does not include 4. For example, $(1, 3)$ stabilizes $x_4$ because it maps $x_4$ to $x_4$, but $(1, 4)$ does not stabilize $x_4$ because it maps $x_4$ to $x_1$. The permutations in $S_4$ whose cycle decompositions do not include 4 are: $(1), (1, 2), (1, 3), (2, 3), (1, 2, 3), \text{ and } (1, 3, 2)$. In fact these are exactly those permutations that make up the group $S_3$ (which is closed and closed under inverses). Thus the permutations in $S_4$ that stabilize $x_4$ form a subgroup isomorphic to $S_3$.
          \end{proof}
    \item Exhibit all permutations in $S_4$ that stabilize the element $x_1 + x_2$ and prove that they form an abelian subgroup of order 4.
          \begin{proof}
            A permutation $\sigma$ stabilizes $x_1 + x_2$ if it stabilizes $x_1$ and $x_2$, or if it assigns $x_1$ to $x_2$ and vice-versa (since $x_1 + x_2 = x_2 + x_1$). The permutations in $S_4$ of this form comprise the set $\{ (1), (1, 2), (3, 4), (1, 2)(3, 4) \}$. In fact, this is a commutative subgroup of $S_4$ where each non-identity permutation has order 2 (thus isomorphic to the Klein 4-group $V_4$).
          \end{proof}
    \item Exhibit all permutations in $S_4$ that stabilize the element $x_1 x_2 + x_3 x_4$ and prove that they form a subgroup isomorphic to the dihedral group of order 8.
          \begin{proof}
            Consider all the presentations of $x_1 x_2 + x_3 x_4$ that might be formed by permuting the subscripts but leaving the value unchanged. Including the above presentation, these are on the left, with the corresponding permutation in $S_4$ on the right in the table below:

            \begin{center}
              \begin{tabular}{ |c|c| } 
                \hline
                $x_1 x_2 + x_3 x_4$ & $(1)$ \\
                \hline
                $x_1 x_2 + x_4 x_3$ & $(3, 4)$ \\
                \hline
                $x_2 x_1 + x_3 x_4$ & $(1, 2)$ \\
                \hline
                $x_2 x_1 + x_4 x_3$ & $(1, 2)(3, 4)$ \\
                \hline
                $x_3 x_4 + x_1 x_2$ & $(1, 3)(2, 4)$ \\
                \hline
                $x_3 x_4 + x_2 x_1$ & $(1, 3, 2, 4)$ \\
                \hline
                $x_4 x_3 + x_1 x_2$ & $(1, 4, 2, 3)$ \\
                \hline
                $x_4 x_3 + x_2 x_1$ & $(1, 4)(2, 3)$ \\
                \hline
              \end{tabular}
            \end{center}

            Now let $\varphi$ be a map from $D_8$ to the set of permutations above defined on generators by $\varphi(s) = (1, 2)$ and $\varphi(r) = (1, 3, 2, 4)$. We will prove that $\varphi$ is an isomorphism. The order of $(1, 2)$ is 2 and the order of $(1, 3, 2, 4)$ is 4, so this satisfies the requirements that $\varphi(s^2) = \varphi(s)^2 = (1)$ and $\varphi(r^4) = \varphi(r)^4 = (1)$.

            Consider the additional relation that $sr = r^{-1}s$. To show that this holds under $\varphi$, we must show that $\varphi(s)\varphi(r) = \varphi(r)^{-1}\varphi(s)$ remains true. On the left we have $(1, 2)(1, 3, 2, 4) = (1, 3)(2, 4)$ and on the right we have $(1, 3, 2, 4)^{-1}(1, 2) = (1, 4, 2, 3)(1, 2) = (1, 3)(2, 4)$. Since the generators and relations of $D_8$ hold under $\varphi$ in the stabilizer shown above, $\varphi$ is a homomorphism. Finally it can be shown exhaustively that $\varphi$ is a bijection, and is thus an isomorphism; thus the stabilizer is isomorphic to the dihedral group $D_8$.
          \end{proof}
    \item Show that the permutations in $S_4$ that stabilize the element $(x_1 + x_2)(x_3 + x_4)$ are exactly the same as those found in part (e).
          \begin{proof}
            By similar method to the above table, we expand $(x_1 + x_2)(x_3 + x_4) = x_1 x_3 + x_1 x_4 + x_2 x_3 + x_2 x_4$ and create a table of the possible alternate presentations with their corresponding permutations:

            \begin{center}
              \begin{tabular}{ |c|c| } 
                \hline
                $x_1 x_3 + x_1 x_4 + x_2 x_3 + x_2 x_4$ & $(1)$ \\
                \hline
                $x_1 x_4 + x_1 x_3 + x_2 x_4 + x_2 x_3$ & $(3, 4)$ \\
                \hline
                $x_2 x_3 + x_2 x_4 + x_1 x_3 + x_1 x_4$ & $(1, 2)$ \\
                \hline
                $x_2 x_4 + x_2 x_3 + x_1 x_4 + x_1 x_3$ & $(1, 2)(3, 4)$ \\
                \hline
                $x_3 x_1 + x_3 x_2 + x_4 x_1 + x_4 x_2$ & $(1, 3)(2, 4)$ \\
                \hline
                $x_3 x_2 + x_3 x_1 + x_4 x_2 + x_4 x_1$ & $(1, 3, 2, 4)$ \\
                \hline
                $x_4 x_1 + x_4 x_2 + x_3 x_1 + x_3 x_2$ & $(1, 4, 2, 3)$ \\
                \hline
                $x_4 x_2 + x_4 x_1 + x_3 x_2 + x_3 x_1$ & $(1, 4)(2, 3)$ \\
                \hline
              \end{tabular}
            \end{center}

            Thus the above permutations (isomorphic to the dihedral group $D_8$) are the same as those that stabilize $x_1 x_2 + x_3 x_4$.
          \end{proof}
\end{enumerate}

\section*{13. (6/17/23)}

Let $n$ be a positive integer and let $R$ be the set of all polynomials with integer coefficients in the independent variables $x_1, x_2, ..., x_n$ i.e., the members of $R$ are finite sums of the form $ax_1^{r_1}x_2^{r_2}...x_n^{r_n}$, where $a$ is any integer and $r_1, ..., r_n$ are nonnegative integers.

For each $\sigma \in S_n$ define a map
\begin{equation*}
  \sigma: R \rightarrow R \text{ by } \sigma \cdot p(x_1, x_2, ..., x_n) = p(x_{\sigma(1)}, x_{\sigma(2)}, ..., x_{\sigma(n)}).
\end{equation*}
Prove that this defines a (left) group action of $S_n$ on $R$.

\begin{proof}
  To show that this is a group action, we must show that $(1) \cdot p = p$ for all $p \in R$, and that $\sigma_1 \cdot (\sigma_2 \cdot p) = (\sigma_1 \circ \sigma_2) \cdot p$ for all $p \in R, \sigma_1, \sigma_2 \in S_n$.

  First, $(1) \cdot p(x_1, x_2, ..., x_n) = p(x_1, x_2, ..., x_n)$ satisfies the identity condition.

  Next, let $\sigma_1, \sigma_2 \in S_n$. Then:
  \begin{flalign*}
      \sigma_1 \cdot (\sigma_2 \cdot p(x_1, x_2, ..., x_n)) = \sigma_1 \cdot p(x_{\sigma_2(1)}, x_{\sigma_2(2)}, ..., x_{\sigma_2(n)}) = \\ p(x_{\sigma_1(\sigma_2(1))}, x_{\sigma_1(\sigma_2(2))}, ..., x_{\sigma_1(\sigma_2(n))}) = \\ (\sigma_1 \circ \sigma_2) \cdot p(x_1, x_2, ..., x_n),
  \end{flalign*}
  as desired.
\end{proof}

\section*{14. (6/18/23)}

Let $H(F)$ be the Heisenberg group over the field $F$ introduced in Exercise 11 of Section 1.4. Determine which matrices lie in the center of $H(F)$ and prove that $Z(H(F))$ is isomorphic to the additive group $F$.

\begin{proof}
  Let $H(F)$ be the Heisenberg group over the field $F$, that is, the group of $3 \times 3$ matrices of the form $\begin{pmatrix}1 & a & b \\ 0 & 1 & c \\ 0 & 0 & 1\end{pmatrix}$, with $a, b, c \in F$, under the operation of matrix multiplication. From 1.4., it can be shown through matrix multiplication that the two matrices $X = \begin{pmatrix}1 & a & b \\ 0 & 1 & c \\ 0 & 0 & 1\end{pmatrix}$ and $Y = \begin{pmatrix}1 & d & e \\ 0 & 1 & f \\ 0 & 0 & 1\end{pmatrix}$ commute if and only if the upper-right entries of the products $XY$ and $YX$ are equal, namely $e + af + b = b + cd + e \Rightarrow af = cd$. In this case, if $a \neq 0$ and $c \neq 0$, then one can always choose $d = 0, f \neq 0$ so that $cd = 0$ but $af \neq 0$, which implies that the two matrices do not commute. Therefore, we must have $a = c = 0$ for the given matrix $X$ to be guaranteed to commute with $Y$ (regardless of the values of the entries $d$ and $f$ in $Y$). Thus the center of $H(F)$ is comprised of matrices of the form $\begin{pmatrix}1 & 0 & x \\ 0 & 1 & 0 \\ 0 & 0 & 1\end{pmatrix}, x \in F$.

  Next, let $\varphi$ be a map from $Z(H(F))$ to $F^+$, the additive group $F$, defined by $\varphi(\begin{pmatrix}1 & 0 & x \\ 0 & 1 & 0 \\ 0 & 0 & 1\end{pmatrix}) = x$. For $A, B \in Z(H(F))$, let $a$ and $b$ be the upper-right entries, respectively. Then $\varphi(A)\varphi(B) = a + b$ and $\varphi(AB) = \varphi(\begin{pmatrix}1 & 0 & a + b \\ 0 & 1 & 0 \\ 0 & 0 & 1\end{pmatrix}) = a + b$, so $\varphi$ is a homomorphism.

  In fact, $\varphi$ is an isomorphism. It is one-to-one: Let $\varphi(A) = \varphi(B)$. Then $a = b$, so $A = B$. It is also onto: Let $x \in F^+$. Then $\varphi(X) = x$. Since $\varphi$ is a bijective homomorphism, it is an isomorphism, and so $Z(H(F)) \cong F^+$.
\end{proof}

\end{document}