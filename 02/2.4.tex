\documentclass{article}

\title{Dummit \& Foote Ch. 2.4: Subgroups Generated by Subsets of a Group}
\author{Scott Donaldson}
\date{Jul. - Aug. 2023}
\usepackage{amsmath, amsthm, amsfonts, enumitem, tabu}

\begin{document}

\maketitle

\section*{1. (7/13/23)}

Prove that if $H$ is a subgroup of $G$ then $\langle H \rangle = H$.

\begin{proof}
    Let $H \leq G$. To show that $\langle H \rangle = H$, we must show that each is contained in the other. By definition, $H \subseteq \langle H \rangle$, so it remains to be proven that $\langle H \rangle \subseteq H$.

    Let $h \in \langle H \rangle$. Recall that:
    \begin{equation*}
        \langle H \rangle = \bigcap_{\substack{H \subseteq K \\ K \leq G}} K,
    \end{equation*}
    that is, for all subset $K \leq G$ with $H \subseteq K$, we have $h \in K$. In particular, since $H$ is a subgroup of $G$, we have $h \in H$, since $H \leq G$ and $H \subseteq H$. Therefore $\langle H \rangle \subseteq H$, and it follows that $\langle H \rangle = H$.
\end{proof}

\section*{2. (7/17/23)}

Prove that if $A$ is a subset of $B$ then $\langle A \rangle \leq \langle B \rangle$. Give an example where $A \subseteq B$ with $A \neq B$ but $\langle A \rangle = \langle B \rangle$.

\begin{proof}
    Let $G$ be a group and let $A \subseteq B \subseteq G$. Recall that one definition of $\langle A \rangle$ is the set of all finite words of elements and inverses of elements of $A$, that is, every element of $\langle A \rangle$ can be written $a_1^{\varepsilon_1} a_2^{\varepsilon_2} ... a_n^{\varepsilon_n}$, where $n \in \mathbb{Z}, n \geq 0$ and $a_i \in A, \varepsilon_i = \pm 1$ for each $i$. Since $A$ is a subset of $B$, $a_i \in A \Rightarrow a_i \in B$, and so each element $a_1^{\varepsilon_1} a_2^{\varepsilon_2} ... a_n^{\varepsilon_n} \in \langle A \rangle$ is also in $\langle B \rangle$. Therefore $\langle A \rangle \leq \langle B \rangle$.

    Now let $G = \mathbb{Z}/3\mathbb{Z}$, $A = \{ 1 \}$, and $B = \{ 0, 1 \}$. Then we have $A \subseteq B$ with $A \neq B$ but $\langle A \rangle = \langle B \rangle = G$.
\end{proof}

\section*{3. (7/17/23)}

Prove that if $H$ is an abelian subgroup of $G$ then $\langle H, Z(G) \rangle$ is abelian. Give an explicit example of an abelian subgroup $H$ of a group $G$ such that $\langle H, C_G(H) \rangle$ is not abelian.

\begin{proof}
    Let $G$ be a group and let $H$ be an abelian subgroup of $G$. Recall that $Z(G) = \{ g \in G \mid xg = gx \text{ for all } x \in G \}$, that is, the set of elements of $G$ that commute with every element of $G$. We will show that $\langle H, Z(G) \rangle$ is an abelian subgroup of $G$.

    First, we will show that the product of any two elements commutes with both elements. Let $a, b \in G$ be commuting elements. Then:
    \begin{equation*}
        (ab)a = aba = aab = a(ab), \text{ and } (ab)b = abb = bab = b(ab),
    \end{equation*}
    as desired.

    Now the generated subgroup $\langle H, Z(G) \rangle$ is constructed from finite words of elements and inverses of elements from $H$ and $Z(G)$. Since $H$ is an abelian subgroup and elements of $Z(G)$ (and therefore their inverses) commute with every element of $G$ (and therefore $H$), it follows that every element in $\langle H, Z(G) \rangle$ is a product of commuting elements. Every such element therefore commutes with every other element in $H$ and $Z(G)$, as well as any other product of elements of $H$ and $Z(G)$. Thus $\langle H, Z(G) \rangle$ is an abelian subgroup of $G$.

    However, it does not follow that $\langle H, C_G(H) \rangle$ is an abelian subgroup of $G$. Let $G = D_8$ and $H = \{ 1, r^2 \}$. The centralizer of $H$ in $G$ is all of $G$, since every element of $H$ commutes with every other element of $G$ (that is, $H = Z(G)$). Then the generated subgroup $\langle H, C_G(H) \rangle = \langle H, G \rangle = G$, which is non-abelian.
\end{proof}

\section*{4. (7/17/23)}

Prove that if $H$ is a subgroup of $G$ then $H$ is generated by the set $H - \{ 1 \}$.

\begin{proof}
    Let $H \leq G$ and consider $\langle H - \{ 1 \} \rangle$. If $H = \{ 1 \}$, then $H - \{ 1 \} = \emptyset$, and so by definition $\langle H - \{ 1 \} \rangle = \{ 1 \} = H$.

    Suppose $H \neq \{ 1 \}$. Then there exists some $h \in H$ with $h \neq 1$. Since $H$ is a subgroup, it is closed under inverses, so $h^{-1} \in H$. We generate $\langle H - \{ 1 \} \rangle$ by taking finite products of elements of $H$, and so $hh^{-1} = 1 \in \langle H - \{ 1 \} \rangle$. Further, we cannot construct any element outside of $H$ by taking products of elements of $H$, so we must therefore have $\langle H - \{ 1 \} \rangle = (H - \{ 1 \}) \cup \{ 1 \} = H$.
\end{proof}

\section*{5. (7/20/23)}

Prove that the subgroup generated by any two distinct elements of order 2 in $S_3$ is all of $S_3$.

\begin{proof}
    The elements of order 2 in $S_3$ are $(1, 2), (1, 3)$, and $(2, 3)$. Since any two of these elements permute one of $\{ 1, 2, 3 \}$ to the other two, without loss of generality we can consider the subgroup generated by a single pair of them. We will consider the subgroup generated by $(1, 2)$ and $(1, 3)$.

    The subgroup contains the identity element, since $(1, 2)(1, 2) = (1)$. It also contains both elements of order 3, since $(1, 2)(1, 3) = (1, 3, 2)$ and $(1, 3)(1, 2) = (1, 2, 3)$. Finally, the subgroup contains the third element of order 2, since $(1, 2)(1, 2, 3) = (2, 3)$. Together these are all the elements of $S_3$.

    Therefore the subgroup generated by any two elements of $S_3$ is all of $S_3$.
\end{proof}

\section*{6. (7/20/23)}

Prove that the subgroup of $S_4$ generated by $(1, 2)$ and $(1, 2)(3, 4)$ is a noncyclic group of order 4.

\begin{proof}
    Let us construct the subgroup of $S_4$ generated by $(1, 2)$ and $(1, 2)(3, 4)$. Both elements have order 2, so we will not consider any higher powers of each. Their product is $(3, 4)$, which also has order 2. At this point the subgroup consists of $\{ (1), (1, 2), (1, 2)(3, 4), (3, 4) \}$. Taking the product of $(3, 4)$ with either of $(1, 2)$ or $(1, 2)(3, 4)$ results in the other element, respectively. Therefore there is no way to obtain new elements not already in this subgroup.

    Thus the subgroup of $S_4$ generated by $(1, 2)$ and $(1, 2)(3, 4)$ has order 4. Further, it is noncyclic, since it contains no elements of order 4 (in fact, it is isomorphic to the Klein 4-group $V_4$).
\end{proof}

\section*{7. (7/22/23)}

Prove that the subgroup of $S_4$ generated by $(1, 2)$ and $(1, 3)(2, 4)$ is isomorphic to the dihedral group of order 8.

\begin{proof}
    Let $A \leq S_4 = \langle (1, 2), (1, 3)(2, 4) \rangle$. Now $A$ naturally contains the product $(1, 2) \cdot (1, 3)(2, 4) = (1, 3, 2, 4)$. So let us consider a map $\varphi: D_8 \rightarrow A$ defined by $\varphi(s) = (1, 2)$ and $\varphi(r) = (1, 3, 2, 4)$. In order to show that $\varphi$ is an isomorphism, we must show that the generators and relations in $D_8$ hold under $\varphi$ in $A$.

    In $S_4$, $(1, 2)$ has order 2 and $(1, 3, 2, 4)$ has order 4 (like $s$ and $r$ respectively in $D_8$). It remains to be shown that the relation $sr = r^{-1}s$ holds under $\varphi$. Now $\varphi(s)\varphi(r) = (1, 2) \cdot (1, 3, 2, 4) = (1, 3)(2, 4)$. Also, $\varphi(r)^{-1}\varphi(s) = (1, 3, 2, 4)^{-1} \cdot (1, 2) = (1, 4, 2, 3) \cdot (1, 2) = (1, 3)(2, 4)$, and so the relation holds as well under $\varphi$.

    So far, this shows that $\varphi$ is a homomorphism into $A$; it remains to be shown that is is both one-to-one and onto. The below table demonstrates exhaustively that $\varphi$ is injectiv, because no two elements in $D_8$ have the same image under $\varphi$ in $A$:

    \begin{center}
        \begin{tabu}{ |c|c| } 
            \hline
            $x \in D_8$ & $\varphi(x) \in A$ \\
            \hline
            1 & $(1)$ \\
            \hline
            $r$ & $(1, 3, 2, 4)$ \\
            \hline
            $r^2$ & $(1, 2)(3, 4)$ \\
            \hline
            $r^3$ & $(1, 4, 2, 3)$ \\
            \hline
            $s$ & $(1, 2)$ \\
            \hline
            $sr$ & $(1, 3)(2, 4)$ \\
            \hline
            $sr^2$ & $(3, 4)$ \\
            \hline
            $sr^3$ & $(1, 4)(2, 3)$ \\
            \hline
        \end{tabu}
    \end{center}

    This shows that $A$ contains at least 8 elements, but not that it contains exactly 8 elements. The multiplication table below shows that $A$ is closed among the 8 elements we know to be included. It is not possible to generate any other element of $S_4$ outside of $A$, and thus $\varphi$ is an isomorphism, and so $A = \langle (1, 2), (1, 3)(2, 4) \rangle$ is isomorphic to $D_8$.

    \begin{table}[h]
        \scriptsize
        \setlength{\tabcolsep}{0.4em} % for the horizontal padding
        \begin{tabu}{ |c|c|c|c|c|c|c|c| } 
            \hline
            $(1)$ & $(1, 3, 2, 4)$ & $(1, 2)(3, 4)$ & $(1, 4, 2, 3)$ & $(1, 2)$ & $(1, 3)(2, 4)$ & $(3, 4)$ & $(1, 4)(2, 3)$ \\
            \hline
            $(1, 3, 2, 4)$ & $(1, 2)(3, 4)$ & $(1, 4, 2, 3)$ & $(1)$ & $(1, 3)(2, 4)$ & $(3, 4)$ & $(1, 4)(2, 3)$ & $(1, 2)(3, 4)$ \\ 
            \hline
            $(1, 2)(3, 4)$ & $(1, 4, 2, 3)$ & $(1)$ & $(1, 3, 2, 4)$ & $(3, 4)$ & $(1, 4)(2, 3)$ & $(1, 2)$ & $(1, 3)(2, 4)$ \\
            \hline
            $(1, 4, 2, 3)$ & $(1)$ & $(1, 3, 2, 4)$ & $(1, 3, 2, 4)$ & $(1, 4)(2, 3)$ & $(1, 2)(3, 4)$ & $(1, 3)(2, 4)$ & $(3, 4)$ \\
            \hline
            $(1, 2)$ & $(1, 4)(2, 3)$ & $(3, 4)$ & $(1, 3)(2, 4)$ & $(1)$ & $(1, 4, 2, 3)$ & $(1, 2)(3, 4)$ & $(1, 3, 2, 4)$ \\
            \hline
            $(1, 3)(2, 4)$ & $(1, 2)(3, 4)$ & $(1, 4)(2, 3)$ & $(3, 4)$ & $(1, 3, 2, 4)$ & $(1)$ & $(1, 4, 2, 3)$ & $(1, 2)(3, 4)$ \\
            \hline
            $(3, 4)$ & $(1, 3)(2, 4)$ & $(1, 2)$ & $(1, 4)(2, 3)$ & $(1, 2)(3, 4)$ & $(1, 3, 2, 4)$ & $(1)$ & $(1, 4, 2, 3)$ \\
            \hline
            $(1, 4)(2, 3)$ & $(3, 4)$ & $(1, 3)(2, 4)$ & $(1, 2)(3, 4)$ & $(1, 4, 2, 3)$ & $(1, 2)(3, 4)$ & $(1, 3, 2, 4)$ & $(1)$ \\
            \hline
        \end{tabu}
    \end{table}
\end{proof}

\section*{8. (7/24/23)}

Prove that $S_4$ = $\langle (1, 2, 3, 4), (1, 2, 4, 3) \rangle$.

\begin{proof}
    By Lagrange's theorem, the order of a subgroup must divide the order of its parent group. Therefore it suffices to show that, if the subgroup of $S_4$ generated by $(1, 2, 3, 4)$ and $(1, 2, 4, 3)$ contains more than half of the elements of $S_4$, then it must be all of $S_4$.

    Beginning with the two elements $(1, 2, 3, 4)$ and $(1, 2, 4, 3)$, we obtain the cyclic subgroups generated by each. For $(1, 2, 3, 4)$, this is $(1), (1, 3)(2, 4)$, and $(1, 4, 3, 2)$. For $(1, 2, 4, 3)$, we also obtain $(1, 4)(2, 3)$ and $(1, 3, 4, 2)$. We now have 7 elements.

    We can obtain as products of the generating elements the 3-cycles $(1, 3, 2) = (1, 2, 3, 4)(1, 2, 4, 3)$ and $(1, 4, 2) = (1, 2, 4, 3)(1, 2, 3, 4)$ [9 elements]. Then, we can obtain a pair of 2-cycles, namely $(2, 4) = (1, 3, 2)(1, 2, 4, 3)$ and $(2, 3) = (1, 4, 2)(1, 2, 3, 4)$ [11 elements]. Next, we can obtain two more 3-cycles $(2, 3, 4) = (2, 4)(2, 3)$ and $(2, 4, 3) = (2, 3)(2, 4)$ [13 elements].

    At this point, noting that the order of $S_4$ is $4! = 24$, we can stop, because we have generated more than half of its elements. Since the subgroup generated by $(1, 2, 3, 4)$ and $(1, 2, 4, 3)$ contains more than half of the elements of $S_4$, it must in fact be $S_4$.
\end{proof}

\section*{9. (7/24/23)}

Prove that $SL_2(\mathbb{F}_3)$ is the subgroup of $GL_2(\mathbb{F}_3)$ generated by $\begin{pmatrix}1 & 1 \\ 0 & 1\end{pmatrix}$ and $\begin{pmatrix}1 & 0 \\ 1 & 1\end{pmatrix}$.

\begin{proof}
    We first construct $SL_2(\mathbb{F}_3)$ by considering all those 2x2 matrices with entries from $\{ 0, 1, 2 \}$ whose determinant is 1 (under modular arithmetic). These, and only these, are:

    \begin{center}
        {\tabulinesep=0.2em
        \begin{tabu}{cccccc}
            $\begin{pmatrix}0 & 1 \\ 2 & 0\end{pmatrix}$ & $\begin{pmatrix}0 & 1 \\ 2 & 1\end{pmatrix}$ & $\begin{pmatrix}0 & 1 \\ 2 & 2\end{pmatrix}$ &
            $\begin{pmatrix}0 & 2 \\ 1 & 0\end{pmatrix}$ & $\begin{pmatrix}0 & 2 \\ 1 & 1\end{pmatrix}$ & $\begin{pmatrix}0 & 2 \\ 1 & 2\end{pmatrix}$ \\
            $\begin{pmatrix}1 & 0 \\ 0 & 1\end{pmatrix}$ & $\begin{pmatrix}1 & 0 \\ 1 & 1\end{pmatrix}$ & $\begin{pmatrix}1 & 0 \\ 2 & 1\end{pmatrix}$ &
            $\begin{pmatrix}1 & 1 \\ 0 & 1\end{pmatrix}$ & $\begin{pmatrix}1 & 1 \\ 1 & 2\end{pmatrix}$ & $\begin{pmatrix}1 & 1 \\ 2 & 0\end{pmatrix}$ \\
            $\begin{pmatrix}1 & 2 \\ 0 & 1\end{pmatrix}$ & $\begin{pmatrix}1 & 2 \\ 1 & 0\end{pmatrix}$ & $\begin{pmatrix}1 & 2 \\ 2 & 2\end{pmatrix}$ \\
            $\begin{pmatrix}2 & 0 \\ 0 & 2\end{pmatrix}$ & $\begin{pmatrix}2 & 0 \\ 1 & 2\end{pmatrix}$ & $\begin{pmatrix}2 & 0 \\ 2 & 2\end{pmatrix}$ &
            $\begin{pmatrix}2 & 1 \\ 0 & 2\end{pmatrix}$ & $\begin{pmatrix}2 & 1 \\ 1 & 1\end{pmatrix}$ & $\begin{pmatrix}2 & 1 \\ 2 & 0\end{pmatrix}$ \\
            $\begin{pmatrix}2 & 2 \\ 0 & 2\end{pmatrix}$ & $\begin{pmatrix}2 & 2 \\ 1 & 0\end{pmatrix}$ & $\begin{pmatrix}2 & 2 \\ 2 & 1\end{pmatrix}$
        \end{tabu} }
    \end{center}

    Next, we will consider the subgroup of $GL_2(\mathbb{F}_3)$ generated by $\begin{pmatrix}1 & 1 \\ 0 & 1\end{pmatrix}$ ($A$) and $\begin{pmatrix}1 & 0 \\ 1 & 1\end{pmatrix}$ ($B$). However, we will find it expedient to generate elements not with these two matrices, but with the products: 
    \begin{flalign*}
        ABA &= \begin{pmatrix}2 & 0 \\ 1 & 2\end{pmatrix} \text{ (C), and} \\
        BAB &= \begin{pmatrix}2 & 1 \\ 0 & 2\end{pmatrix} \text{ (D).}
    \end{flalign*}
    
    The reason that the matrices $C$ and $D$ are more helpful to work with is that they each have order 6, and so we can generate more elements by considering their powers than we do with the given matrices. In fact, we have $C^2 = B$ and $D^2 = A$, respectively.

    Through exploration, we can rewrite the above table of the 24 matrices of $SL_2(\mathbb{F}_3)$ using only products of $C$ and $D$:

    \begin{center}
        {\tabulinesep=0.2em
        \begin{tabu}{cccccc}
            $C^2 D C^2$ & $(C^5 D)^5$ & $DC^2$ & $D^2 C D^2$ & $DC^5$ & $(C^2 D)^2$ \\
            $I = C^6 D^6$ & $C^2$ & $C^4$ & $D^2$ & $(DC)^3$ & $(DC^5)^5$ \\
            $D^4$ & $C^5 D$ & $CD$ \\
            $C^3$ & $C$ & $C^5$ & $D$ & $(CD)^3$ & $C^2 D$ \\
            $D^5$ & $(DC^2)^2$ & $DC$
        \end{tabu} }
    \end{center}

    The process toward completing this table encompassed a rather brute force approach. We took arbitrary powers and products of $C$ and $D$, which quickly yielded many new elements but soon tapered off and resulted in already existing elements. For the final matrix left outstanding, we determined computationally which entries would have been possible in a pair of matrices to be multiplied together to obtain it, and then found those in the table of existing elements (the matrix in question was $\begin{pmatrix}0 & 1 \\ 2 & 0\end{pmatrix} = (C^2) (D C^2)$).

    While this process is a bit unsatisfying methodologically and aesthetically, it does prove that $\begin{pmatrix}1 & 1 \\ 0 & 1\end{pmatrix}$ and $\begin{pmatrix}1 & 0 \\ 1 & 1\end{pmatrix}$ generate the subgroup $SL_2(\mathbb{F}_3)$ of $GL_2(\mathbb{F}_3)$.
\end{proof}

\section*{10. (7/30/23)}

Prove that the subgroup of $SL_2(\mathbb{F}_3)$ generated by $\begin{pmatrix}0 & -1 \\ 1 & 0\end{pmatrix}$ and $\begin{pmatrix}1 & 1 \\ 1 & -1\end{pmatrix}$ is isomorphic to the quaternion group of order 8.

\begin{proof}
    In our notation, we will write the generating matrices as $\begin{pmatrix}0 & 2 \\ 1 & 0\end{pmatrix}$ and $\begin{pmatrix}1 & 1 \\ 1 & 2\end{pmatrix}$, and let $A$ be the subgroup generated by them. Recall that $Q_8$ is generated by $\langle i, j \mid i^2 = j^2, i^4 = j^4 = 1, ji = i^{-1}j \rangle$.

    Now let $\varphi: Q_8 \rightarrow A$ be defined by $\varphi(i) = \begin{pmatrix}0 & 2 \\ 1 & 0\end{pmatrix}$ and $\varphi(j) = \begin{pmatrix}1 & 1 \\ 1 & 2\end{pmatrix}$. For $\varphi$ to be a homomorphism, the generators and relations of $Q_8$ must hold under $\varphi$ in $A$.

    We have
    \begin{equation*}
        \varphi(i)^2 = \begin{pmatrix}0 & 2 \\ 1 & 0\end{pmatrix}\begin{pmatrix}0 & 2 \\ 1 & 0\end{pmatrix} = \begin{pmatrix}2 & 0 \\ 0 & 2\end{pmatrix} \text{ and } \varphi(j)^2 = \begin{pmatrix}1 & 1 \\ 1 & 2\end{pmatrix}\begin{pmatrix}1 & 1 \\ 1 & 2\end{pmatrix} = \begin{pmatrix}2 & 0 \\ 0 & 2\end{pmatrix},
    \end{equation*}
    so $\varphi(i)^2 = \varphi(j)^2$. And $\begin{pmatrix}2 & 0 \\ 0 & 2\end{pmatrix}^2 = \begin{pmatrix}1 & 0 \\ 0 & 1\end{pmatrix}$, so $\varphi(i)^4 = \varphi(j)^4 = I$.

    Finally, note that $\varphi(i)^{-1} = \begin{pmatrix}0 & 1 \\ 2 & 0\end{pmatrix}$. It follows that
    \begin{align*}
        \varphi(i)^{-1} \varphi(j) = \begin{pmatrix}0 & 1 \\ 2 & 0\end{pmatrix}\begin{pmatrix}1 & 1 \\ 1 & 2\end{pmatrix} &= \begin{pmatrix}1 & 2 \\ 2 & 2\end{pmatrix} \text{ and } \\ 
        \varphi(j)\varphi(i) = \begin{pmatrix}1 & 1 \\ 1 & 2\end{pmatrix}\begin{pmatrix}0 & 2 \\ 1 & 0\end{pmatrix} &= \begin{pmatrix}1 & 1 \\ 1 & 2\end{pmatrix},
    \end{align*}
    so all of the generators and relations of $Q_8$ hold under $\varphi$ in $A$. Therefore $\varphi$ is a homomorphism. Further, $\varphi$ is surjective, since every element of $A$ can be written as the image of some element of $Q_8$ under $\varphi$. Then $|A| \leq |Q_8| = 8$. However, since we have shown that there exist at least 5 unique elements of $A$ (the generating matrices, their product, the inverse of $\varphi(i)$, and the identity), by Lagrange's theorem $A$ must contain 8 elements, and $\varphi$ must then also be one-to-one, that is, an isomorphism. Thus $A$ is isomorphic to $Q_8$.
\end{proof}

\section*{11. (7/30/23)}

Show that $SL_2(\mathbb{F}_3)$ and $S_4$ are two nonisomorphic groups of order 24.

\begin{proof}
    From Exercise 9., $SL_2(\mathbb{F}_3)$ contains 24 elements, as does $S_4$. However, in $SL_2(\mathbb{F}_3)$, the element $\begin{pmatrix}2 & 0 \\ 1 & 2\end{pmatrix}$ has order 6, while no element in $S_4$ has order greater than 4. Therefore for any map $\varphi: SL_2(\mathbb{F}_3) \rightarrow S_4$, there is no element in $S_4$ such that $\varphi(\begin{pmatrix}2 & 0 \\ 1 & 2\end{pmatrix}^6) = \varphi(\begin{pmatrix}2 & 0 \\ 1 & 2\end{pmatrix})^6$. Then no such $\varphi$ is an isomorphism, and so the groups are not isomorphic to each other.
\end{proof}

\section*{12. (8/3/23)}

Prove that the subgroup of upper triangular matrices in $GL_3(\mathbb{F}_2)$ is isomorphic to the dihedral group of order 8.

\begin{proof}
    Let $A$ be the subgroup of upper triangular matrices in $GL_3(\mathbb{F}_2)$ and let $\varphi: D_8 \rightarrow A$ be defined on generators by
    \begin{equation*}
        \varphi(s) = \begin{pmatrix}1 & 1 & 0 \\ 0 & 1 & 0 \\ 0 & 0 & 1\end{pmatrix} \text{ and } \varphi(r) = \begin{pmatrix}1 & 1 & 0 \\ 0 & 1 & 1 \\ 0 & 0 & 1\end{pmatrix}.
    \end{equation*}
    We will show that $\varphi$ is an isomorphism.

    The orders of the generators in $D_8$ hold under $\varphi$ in $A$:
    \begin{equation*}
        \varphi(s)^2 = \begin{pmatrix}1 & 1 & 0 \\ 0 & 1 & 0 \\ 0 & 0 & 1\end{pmatrix}\begin{pmatrix}1 & 1 & 0 \\ 0 & 1 & 0 \\ 0 & 0 & 1\end{pmatrix} = \begin{pmatrix}1 & 0 & 0 \\ 0 & 1 & 0 \\ 0 & 0 & 1\end{pmatrix}
    \end{equation*}
    and
    \begin{equation*}
        \varphi(r)^4 = \begin{pmatrix}1 & 1 & 0 \\ 0 & 1 & 1 \\ 0 & 0 & 1\end{pmatrix}^4 = \Biggl(\begin{pmatrix}1 & 0 & 1 \\ 0 & 1 & 0 \\ 0 & 0 & 1\end{pmatrix}^2\Biggr)^2 = \begin{pmatrix}1 & 0 & 0 \\ 0 & 1 & 0 \\ 0 & 0 & 1\end{pmatrix}.
    \end{equation*}

    We also note that $\varphi^{-1}(r) = \begin{pmatrix}1 & 1 & 1 \\ 0 & 1 & 1 \\ 0 & 0 & 1\end{pmatrix}$, and that
    \begin{align*}
        \varphi(s)\varphi(r) = \begin{pmatrix}1 & 1 & 0 \\ 0 & 1 & 0 \\ 0 & 0 & 1\end{pmatrix}\begin{pmatrix}1 & 1 & 0 \\ 0 & 1 & 1 \\ 0 & 0 & 1\end{pmatrix} &= \begin{pmatrix}1 & 0 & 1 \\ 0 & 1 & 1 \\ 0 & 0 & 1\end{pmatrix} \\ &= \begin{pmatrix}1 & 1 & 1 \\ 0 & 1 & 1 \\ 0 & 0 & 1\end{pmatrix}\begin{pmatrix}1 & 1 & 0 \\ 0 & 1 & 0 \\ 0 & 0 & 1\end{pmatrix} = \varphi^{-1}(r)\varphi(s).
    \end{align*}
    Then $\varphi$ is a surjective homomorphism. Since the order of $A$ is 8 (the upper-middle, upper-right, and middle-right entries may each be 0 or 1, so there are $2^3 = 8$ possible upper triangular matrices), $\varphi$ is also injective, and is thus an isomorphism. Thus the subgroup of upper triangular matrices in $GL_3(\mathbb{F}_2)$ is isomorphic to $D_8$.
\end{proof}

\section*{13. (8/3/23)}

Prove that the multiplicative group of positive rational numbers is generated by the set $\{ \frac{1}{p} \mid p \text{ is a prime} \}$.

\begin{proof}
    Consider the subgroup $\langle \{ \frac{1}{p} \mid p \text{ is a prime } \} \rangle$ of the positive rational numbers under multiplication. Since this generated group is closed under inverses, it contains every prime $p$. It is also closed under multiplication, and so contains 1, the identity, and every integer greater than 1, which can be expressed as a product of primes.
    
    It follows that for any positive rational number $\frac{n}{m} = nm^{-1}$, but $n$ and $m$ are contained in the generated subgroup, and by the subgroup criterion, so is $nm^{-1}$. Therefore the group of positive rational numbers under multiplication is equal to $\langle \{ \frac{1}{p} \mid p \text{ is a prime } \} \rangle$.
\end{proof}

\section*{14. (8/3/23)}

A group $H$ is called \emph{finitely generated} if there is a finite set $A$ such that $H = \langle A \rangle$.

\begin{enumerate}[label=(\alph*), itemsep=0em]
    \item Prove that every finite group is finitely generated.
          \begin{proof}
            Let $A$ be a finite group. From Exercise 4., the finite subset $A - \{ 1 \}$ generates $A$.
          \end{proof}
    \item Prove that $\mathbb{Z}$ is finitely generated.
          \begin{proof}
            From Chapter 1.2, Exercise 14., the subset $\langle 1, -1 \rangle$ generates $\mathbb{Z}$ (in fact, $\langle 1 \rangle$ is sufficient).
          \end{proof}
    \item Prove that every finitely generated subgroup of the additive group $\mathbb{Q}$ is cyclic.
          \begin{proof}
            Let $H = \langle \dfrac{a_1}{b_1}, \dfrac{a_2}{b_2}, ..., \dfrac{a_n}{b_n} \rangle$. Let $k = b_1 \cdot b_2 \cdot ... \cdot b_n$ and consider the subgroup generated by $\dfrac{1}{k}$. Now $b_1$ divides $k$, so if $b_1 c = k$, then $\underbrace{\frac{1}{k} + ... + \frac{1}{k}}_{\text{$c$ times}} = \dfrac{1}{b_1}$. Further, $\underbrace{\frac{1}{b_1} + ... + \frac{1}{b_1}}_{\text{$a_1$ times}} = \dfrac{a_1}{b_1}$, so the first generator of $H$ is contained in $\langle \dfrac{1}{k} \rangle$. Similarly, every generator of $H$ is contained in $\langle \dfrac{1}{k} \rangle$.

            Therefore every finitely generated group of $\mathbb{Q}$ is a cyclic group generated by the inverse of the product of the denominators of the generators.
          \end{proof}
    \item Prove that $\mathbb{Q}$ is not finitely generated.
          \begin{proof}
            If $\mathbb{Q}$ were finitely generated, then we would have
            \begin{equation*}
                \mathbb{Q} = \langle \frac{a_1}{b_1}, \frac{a_2}{b_2}, ..., \frac{a_n}{b_n} \rangle = \langle \frac{1}{k} \mid k = b_1 \cdot b_2 \cdot ... \cdot b_n \rangle.
            \end{equation*} 
            However, $\dfrac{1}{2k} \in \mathbb{Q}$ but $\dfrac{1}{2k} \notin \langle \dfrac{1}{k} \rangle$ (since we can only obtain multiples of $\dfrac{1}{k}$). Therefore $\mathbb{Q}$ is not finitely generated.
          \end{proof}
\end{enumerate}

\section*{15. (8/3/23)}

Exhibit a proper subgroup of $\mathbb{Q}$ which is not cyclic.

\begin{proof}
    Let $A$ be the subgroup of $\mathbb{Q}$ generated by the set $\{ \dfrac{1}{2^n} \mid n \in \mathbb{Z}^+ \}$. If we assume that $A$ is cyclic, then there exists a $k > 0$ such that $A = \langle \dfrac{1}{k} \rangle$. However, we can always choose an $n$ such that $\dfrac{1}{2^n} < \dfrac{1}{k}$, and so $A$ contains an element that is not generated by $\dfrac{1}{k}$, a contradiction. Therefore $A$ is not cyclic.

    To show that $A$ is not equal to all of $\mathbb{Q}$, we note that only elements whose denominator is a power of 2 may be generated from $\{ \dfrac{1}{2^n} \mid n \in \mathbb{Z}^+ \}$. That is, $A$ contains $\frac{5}{16} = \frac{1}{4} + \frac{1}{16}$ but not $\frac{5}{13}$. Thus $A$ is a proper subgroup of $\mathbb{Q}$ which is not cyclic.
\end{proof}

\end{document}