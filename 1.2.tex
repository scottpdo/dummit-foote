\documentclass{article}

\title{Dummit \& Foote Ch. 1.2: Dihedral Groups}
\author{Scott Donaldson}
\date{Jan. 2023}
\usepackage{amsmath, amsthm, amsfonts, enumitem}

\begin{document}

\maketitle

\section*{1. (1/23/23)}

Compute the order of each of the elements in the following groups:

\begin{enumerate}[label=(\alph*)]
    \item $D_6$
        \begin{itemize}
            \item $r, r^2$: 3
            \item $s, sr, sr^2$: 2
        \end{itemize}
    
    \item $D_8$
        \begin{itemize}
            \item $r$: 4
            \item $r^2$: 2
            \item $r^3$: 4
            \item $s, sr, sr^2, sr^3$: 2
        \end{itemize}

    \item $D_{10}$
        \begin{itemize}
            \item $r, r^2, r^3, r^4$: 5
            \item $s, sr, sr^2, sr^3, sr^4$: 2
        \end{itemize}
    
\end{enumerate}

\section*{2. (1/23/23)}

Use the generators and relations of $D_{2n} = \langle r, s\hspace{0.5ex}|\hspace{0.5ex} r^n = s^2 = 1, rs = sr^{-1} \rangle$ to show that if $x$ is any element of $D_{2n}$ which is not a power of $r$, then $rx = xr^{-1}$. 

\begin{proof}
    Let $x \in D_{2n}$ such that $x \neq r^k$ for all $k \in \mathbb{Z}$. Then, since all elements of $D_{2n}$ can be written as a product of generators $s$ and $r$, we must have $x = sr^k$ for some $k \in \{ 1, 2, ..., n - 1 \}$. Therefore:
    \begin{equation*}
        rx = rsr^k = sr^{-1} r^k = sr^{k - 1} = sr^k r^{-1} = x r^{-1},
    \end{equation*}
    as desired.
\end{proof}

\section*{3. (1/25/23)}

Use the generators and relations above to show that every element of $D_{2n}$ which is not a power of $r$ has order 2. Deduce that $D_{2n}$ is generated by the two elements $s$ and $sr$, both of which have order 2.

\begin{proof}
    Let $sr^k \in D_{2n}$. $(sr^k)(sr^k) = s(r^k s) r^k = s (s r^{-k}) r^{k} = ss r^{-k} r^k = 1 \cdot 1 = 1$. Thus the order of elements of the form $sr^k$, that is, every element which is not a power of $r$, has order 2.

    To show that $D_{2n}$ is generated by $s$ and $sr$, let $r^k, sr^k \in D_{2n}$. Now $s \cdot sr = r$, so $(s \cdot sr)^k = r^k$. To obtain $sr^k$, we simply left-multiply the previous by $s$: $s(s \cdot sr)^k = sr^k$. Thus every element of $D_{2n}$ can be written as a product of $s$ and $sr$, and so $\langle s, sr \rangle$ is a generator for $D_{2n}$.
\end{proof}

\section*{4. (1/25/23)}

If $n = 2k$ is even and $n \geq 4$, show that $z = r^k$ is an element of order 2 which commutes with all elements of $D_{2n}$. Show also that $z$ is the only nonidentity element of $D_{2n}$ which commutes with all elements of $D_{2n}$.

\begin{proof}
    Let $n = 2k, n \geq 4$, and let $z = r^k \in D_{2n}$. $z \cdot z = r^k r^k = r^{2k} = r^n = 1$, so $z$ has order 2.

    Since $r^k r^k = 1$, it follows that $r^k = r^{-k}$ (equivalently, $z = z^{-1}$). Elements of the form $r^m$ obviously commute with each other, so we only need to show that $z = r^k$ commutes with elements of the form $sr^m$. Now:

    \begin{multline*}
        r^k sr^m = r^k r^{-m} s = r^{-k} r^{-m} s = r^{-k - m} s = (r^{k + m})^{-1} s = \\
        s r^{k + m} = s r^{m + k} = s r^m r^k,
    \end{multline*}

    which shows that $z = r^k$ commutes with elements of the form $sr^m$.

    Finally, to show that $z$ is the only nonidentity element which commutes with all elements, we will consider the possible separate cases of the forms of arbitrary elements of $D_{2n}$. Let $a, b \in D_{2n}$.

    \begin{itemize}
        \item Let $a = r^m$. From above, $a$ commutes with all elements of the form $r^p$. Does $a$ commute with elements of the form $s r^p$? $r^m sr^p = r^m r^{-p} s = r^{m - p} s$. On the other hand, we have $sr^p r^m = sr^{p + m} = r^{-p - m} s$. These two are equal when $m - p = -p - m$, that is, when $m = -m$ (in $\mathbb{Z}/n\mathbb{Z}$). This only occurs when $m = n / 2 = k$, and so $z = r^k$ is the only element of the form $r^m$ which commutes with all elements of $D_{2n}$.
        \item Let $a = sr^m$. As a counterexample, it suffices to show that there is at least one element of $D_{2n}$ which $a$ does not commute with: $r$. $sr^m r = sr^{m + 1}$, while $r sr^m = r r^{-m} s = r^{1 - m} s = sr^{m - 1}$. Because $n \geq 4$, there are no values of $m \in \mathbb{Z}/n\mathbb{Z}$ for which $m + 1 = m - 1$. Thus elements of the form $sr^m$ do not commute in $D_{2n}$.
    \end{itemize}
    This completes the proof that $z = r^k$ is the only nonidentity element of $D_{2n}$ which commutes with all other elements.

\end{proof}

\section*{5. (1/26/23)}

If $n$ is odd and $n \geq 3$, show that the identity is the only element of $D_{2n}$ which commutes with all elements of $D_{2n}$.

\begin{proof}
    This proof is nearly identical to that of Exercise 4. above, only with $n$ odd instead of even. The proof that elements of the form $sr^m$ is the same as above. To show that elements of the form $r^m$ do not commute, we again consider $r^m sr^p$ and $sr^p r^m$ and see that we must have $m = -m$ (in $\mathbb{Z}/n\mathbb{Z}$). Adding $m$ to both sides, we must have $2m = 0 \Rightarrow 2m = n$. However, because $n$ is odd, this does not occur, and so there are no nonidentity elements of $D_{2n}$ which commute with all elements of $D_{2n}$.
\end{proof}

\section*{6. (1/26/23)}

Let $x, y$ be elements of order 2 in any group $G$. Prove that if $t = xy$ then $tx = xt^{-1}$ (so that if $n = |xy| < \infty$ then $x, t$ satisfy the same relations in $G$ as $s, r$ do in $D_{2n}$).

\begin{proof}
    Let $x, y \in G, |x| = |y| = 2$ and let $t = xy$. From $x^2 = y^2 = 1$, we have $x = x^{-1}$ and $y = y^{-1}$. Then:

    \begin{equation*}
        t = xy \Rightarrow tx = xyx = x(y^{-1} x^{-1}) = x(xy)^{-1} = xt^{-1},
    \end{equation*}
    as desired.

    If $|xy| = |t| = n < \infty$, then we have $t^n = x^2 = 1, tx = xt^{-1}$. These are the same relations in $G$ for $x, t$ as $s, r$ do in $D_{2n}$.
\end{proof}

\end{document}