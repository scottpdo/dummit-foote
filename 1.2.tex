\documentclass{article}

\title{Dummit \& Foote Ch. 1.2: Dihedral Groups}
\author{Scott Donaldson}
\date{Jan. 2023}
\usepackage{amsmath, amsthm, amsfonts, enumitem}

\begin{document}

\maketitle

\section*{1. (1/23/23)}

Compute the order of each of the elements in the following groups:

\begin{enumerate}[label=(\alph*)]
    \item $D_6$
        \begin{itemize}
            \item $r, r^2$: 3
            \item $s, sr, sr^2$: 2
        \end{itemize}
    
    \item $D_8$
        \begin{itemize}
            \item $r$: 4
            \item $r^2$: 2
            \item $r^3$: 4
            \item $s, sr, sr^2, sr^3$: 2
        \end{itemize}

    \item $D_{10}$
        \begin{itemize}
            \item $r, r^2, r^3, r^4$: 5
            \item $s, sr, sr^2, sr^3, sr^4$: 2
        \end{itemize}
    
\end{enumerate}

\section*{2. (1/23/23)}

Use the generators and relations of $D_{2n} = \langle r, s\hspace{0.5ex}|\hspace{0.5ex} r^n = s^2 = 1, rs = sr^{-1} \rangle$ to show that if $x$ is any element of $D_{2n}$ which is not a power of $r$, then $rx = xr^{-1}$. 

\begin{proof}
    Let $x \in D_{2n}$ such that $x \neq r^k$ for all $k \in \mathbb{Z}$. Then, since all elements of $D_{2n}$ can be written as a product of generators $s$ and $r$, we must have $x = sr^k$ for some $k \in \{ 1, 2, ..., n - 1 \}$. Therefore:
    \begin{equation*}
        rx = rsr^k = sr^{-1} r^k = sr^{k - 1} = sr^k r^{-1} = x r^{-1},
    \end{equation*}
    as desired.
\end{proof}

\section*{3. (1/25/23)}

Use the generators and relations above to show that every element of $D_{2n}$ which is not a power of $r$ has order 2. Deduce that $D_{2n}$ is generated by the two elements $s$ and $sr$, both of which have order 2.

\begin{proof}
    Let $sr^k \in D_{2n}$. $(sr^k)(sr^k) = s(r^k s) r^k = s (s r^{-k}) r^{k} = ss r^{-k} r^k = 1 \cdot 1 = 1$. Thus the order of elements of the form $sr^k$, that is, every element which is not a power of $r$, has order 2.

    To show that $D_{2n}$ is generated by $s$ and $sr$, let $r^k, sr^k \in D_{2n}$. Now $s \cdot sr = r$, so $(s \cdot sr)^k = r^k$. To obtain $sr^k$, we simply left-multiply the previous by $s$: $s(s \cdot sr)^k = sr^k$. Thus every element of $D_{2n}$ can be written as a product of $s$ and $sr$, and so $\langle s, sr \rangle$ is a generator for $D_{2n}$.
\end{proof}

\end{document}