\documentclass{article}

\title{Dummit \& Foote Ch. 1: Groups}
\author{Scott Donaldson}
\date{2022}
\usepackage{amsmath, amsthm, amsfonts, enumitem}

\begin{document}

\maketitle

\section*{1. (11/14/22)}

Let $G$ be a group. Determine which of the following binary operations are associative:

\begin{enumerate}[label=\alph*)]
    \item The operation $\star$ on $\mathbb{Z}$ defined by $a \star b = a - b:$
    
          Not associative. $3 \star (2 \star 1) = 3 - 1 = 2$ but $(3 \star 2) \star 1 = 3 - 2 = 1$.
    \item The operation $\star$ on $\mathbb{R}$ defined by $a \star b = a + b + ab:$
    
          Associative. 
          \begin{equation*}
            a \star (b \star c) = a \star (b + c + bc) = a + b + c + bc + ab + ac + abc =
            (a + b + ab) \star c = (a \star b) \star c
          \end{equation*}
    \item The operation $\star$ on $\mathbb{Q}$ defined by $a \star b = \frac{a + b}{5}$:
    
          Not associative. $0 \star (1 \star 1) = 0 + 2/5 = 2/5$ but $(0 \star 1) \star 1 = 1/5 \star 1 = 6/5 * 1/5 = 6/25$.
    \item The operation $\star$ on $\mathbb{Z} \times \mathbb{Z}$ defined by $(a, b) \star (c, d) = (ad + bc, bd):$
    
          Associative.
          \begin{multline*}
            ((a,b)\star(c,d))\star(e,f) = (ad + bc, bd)\star(e,f) = \\
            (adf + bcf + bde, bdf) = (a,b)\star(cf + de, df) = (a,b)\star((c,d)\star(e,f)).
          \end{multline*}
    \item The operation $\star$ on $\mathbb{Q} - \{0\}$ defined by $a \star b = a / b:$
    
          Not associative. $(1 \star 2) \star 3 = 1/6$ but $1 \star (2 \star 3) = 3/2$.
\end{enumerate}

\section*{2. (11/14/22)}

Decide which of the binary operations in the preceding exercise are commutative.

\begin{enumerate}[label=\alph*)]
    \item Not commutative. $1 - 2 = -1$ but $2 - 1 = 1$.
    \item Commutative. $a \star b = a + b + ab = b + a + ba = b \star a$.
    \item Commutative. $a \star b = \frac{a + b}{5} = \frac{b + a}{5} = b \star a$.
    \item Commutative. $(a,b)\star(c,d) = (ad + bc, bd) = (cb + da, db) = (c,d)\star(a,b)$.
    \item Not commutative. $1 / 2 = 1/2$ but $2 / 1 = 2$.
\end{enumerate}

\section*{3. (11/16/22)}

Prove that addition of residue classes in $\mathbb{Z}/n\mathbb{Z}$ is associative.

\begin{proof}

      First, we will show that subtraction in $\mathbb{Z}/n\mathbb{Z}$ is well-defined. Given a representative element $\bar{a}$, $1 \leq \bar{a} \leq n - 1$, the element $n - \bar{a}$ is $\bar{a}$'s inverse. $1 \leq n - \bar{a} \leq n - 1$, so $n - \bar{a}$ is also a representative element. Also, $\bar{a} + (n - \bar{a}) = n \sim 0$. Thus, subtracting an element $\bar{a}$ from $\bar{b}$ is the same as adding $n - \bar{a}$ to $\bar{b}$, and so subtraction is well-defined.

      Now, to show that addition is associative, let $\bar{a}, \bar{b}, \bar{c} \in \mathbb{Z}/n\mathbb{Z}$. Suppose that $(\bar{a} + \bar{b}) + \bar{c} = \bar{d}$ and $\bar{a} + (\bar{b} + \bar{c}) = \bar{e}$. Then:

      \begin{equation*}
            \bar{d} - \bar{c} = \bar{a} + \bar{b} \Rightarrow \bar{a} = (\bar{d} - \bar{c}) - \bar{b}
      \end{equation*}

      And:

      \begin{equation*}
            \bar{e} - \bar{a} = \bar{b} + \bar{c} \Rightarrow \bar{e} = ((\bar{d} - \bar{c}) - \bar{b}) + \bar{b} + \bar{c} = \bar{d} - \bar{c} + \bar{c} = \bar{d} 
      \end{equation*}

      Therefore $\bar{d} = \bar{e}$, so $(\bar{a} + \bar{b}) + \bar{c} = \bar{a} + (\bar{b} + \bar{c})$.
\end{proof}

\section*{4. (11/16/22)}

Prove that multiplication of residue classes in $\mathbb{Z}/n\mathbb{Z}$ is associative.

\begin{proof}
      Let $\bar{a}, \bar{b}, \bar{c} \in \mathbb{Z}/n\mathbb{Z}$. Then:

      \begin{equation*}
            \overline{a}(\overline{b}\overline{c}) = \overline{a}(\overline{bc}) = \overline{a(bc)}
      \end{equation*}

      Since the latter expression involves arbitrary integers $a, b, c$ whose representative elements in $\mathbb{Z}/n\mathbb{Z}$ are $\overline{a}, \overline{b}, \overline{c}$, we can use the associative property of standard multiplication:

      \begin{equation*}
            \overline{a(bc)} = \overline{(ab)c} = (\overline{ab})\overline{c} = (\overline{a}\overline{b})\overline{c} 
      \end{equation*}

      Therefore multiplication of residue classes is associative.
\end{proof}

\section*{5. (11/16/22)}

Prove for all $n > 1$ that $\mathbb{Z}/n\mathbb{Z}$ is not a group under multiplication of residue classes.

\begin{proof}
      Let $\mathbb{Z}/n\mathbb{Z}$ with $n > 1$. The element $1$ is the identity element, since (by multiplication of standard integers), $1 \cdot \bar{a} = \bar{a}$ for all $\bar{a} \in \mathbb{Z}/n\mathbb{Z}$. However, the element $0$ has no inverse, since (again by standard multiplication), there is no element $\bar{a}$ such that $0 \cdot \bar{a} = 1$. Thus, $\mathbb{Z}/n\mathbb{Z}$ is not a group under multiplication.
\end{proof}

\section*{6. (11/18/22)}

Determine which of the following are sets are groups under addition:

\begin{enumerate}[label=\alph*)]
      \item the set of rational numbers (including $0 = 0/1$) in lowest terms whose denominators are odd:
      
            This is a group. The identity element is $0$ and addition is associative by definition, so we only need to show that it is closed. Let $\frac{a}{b}$ and $\frac{c}{d}$ be two elements of the set. Then $\frac{a}{b} + \frac{c}{d} = \frac{ad + bc}{bd}$. The product of two odd numbers is odd, so $bd$ is odd. Further, if $\frac{ad + bc}{bd}$ is not in lowest terms, then the denominator must remain negative, since an odd number has no even divisors. Thus the set is closed under addition.
      \item the set of rational numbers (including $0 = 0/1$) in lowest terms whose denominators are even:

            Not a group. $1/2 + 1/2 = 1/1$, a rational number whose denominator is odd.
      \item the set of rational numbers of absolute value $< 1$.

            Not a group. $3/4 + 3/4 = 3/2$, a rational number whose absolute value is $\geq 1$.
      \item the set of rational numbers of absolute value $\geq 1$ together with $0$.

            Not a group. $3/2 + (-3/4) = 1/4$, a rational number whose absolute value is $< 1$. 
      \item the set of rational numbers with denominators equal to $1$ or $2$.

            This is a group. Let $a, b$ be members of the set. If both have denominator $1$ or $2$, then their sum has denominator $1$. Otherwise, if one has denominator $1$ and the other denominator $2$, their sum has denominator $2$. Therefore the set is closed under addition.
      \item the set of rational numbers with denominators equal to $1$, $2$, or $3$.

            Not a group. $1/2 + 1/3 = 5/6$.
\end{enumerate}

\section*{7. (11/18/22)}

Let $G = \{x \in \mathbb{R} \mid 0 \leq x < 1\}$ and for $x, y \in G$ let $x \star y$ be the fractional part of $x + y$. Prove that $\star$ is a well-defined binary operation on $G$ and that $G$ is an abelian group under $\star$.

\begin{proof}
      $\star$ is a well-defined binary operation on $G$. Let $x, y \in G$. Then $x, y \in [0, 1)$. Suppose that $x + y = z \in \mathbb{R}$. By definition, $x \star y$ is the fractional part of $z$, which is unique. Therefore $\star$ is well-defined, and commutative, since $+$ is commutative.

      The identity element of $G$ is $0$, since for all $x \in [0, 1)$, $0 + x = x$.

      $G$ is closed under $\star$. For any $z = x + y$, the fractional part of $z$ is (by definition) greater than or equal to $0$ and strictly less than $1$. Therefore $x \star y$ is in $G$.

      Finally, $\star$ is associative. Let $a, b, c \in G$. $(a \star b) \star c$ is equal to the fractional part of $(a \star b) + c$. And, $a \star b$ is equal to the fractional part of $a + b$. Now, taking the fractional part of a number is an idempotent operation; that is, performing it more than once yields the same value. So the fractional part of $(a \star b) + c$, that is, the fractional part of the fractional part of $(a + b) + c$ is just the fractional part of $(a + b) + c = a + b + c$. Similarly, $a \star (b \star c)$ is equal to the fractional part of $a + b + c$, and so $\star$ is associative.

      Thus $G$ is an abelian group under $\star$.
\end{proof}

\end{document}