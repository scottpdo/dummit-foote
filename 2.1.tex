\documentclass{article}

\title{Dummit \& Foote Ch. 2.1: Subgroups, Definition and Examples}
\author{Scott Donaldson}
\date{May 2023}
\usepackage{amsmath, amsthm, amsfonts, enumitem}

\begin{document}

\maketitle

Let $G$ be a group.

\section*{1. (5/22/23)}

In each of (a) - (e) prove that the specified subset $H$ is a subgroup of the given group $G$:

\begin{enumerate}[label=(\alph*)]
    \item $H = $ the set of complex numbers of the form $a + ai, a \in \mathbb{R}$, $G = \mathbb{C}$ (under addition)
          \begin{proof}
            Let $a + ai, b + bi \in H$. $(b + bi) + (-b - bi) = 0$, so the inverse of $b + bi$ is $-b - bi$.

            Then $a + ai - b + bi = (a - b) + (a - b)i \in H$. By the subgroup criterion, $H$ is a subgroup of $G$.
          \end{proof}
    \item $H = $ the set of complex numbers of absolute value 1, i.e., the unit circle in the complex plane, $G = \mathbb{C}$ (under multiplication)
          \begin{proof}
            Let $a + bi, c + di \in H$. Since $|a + bi| = 1, \sqrt{a^2 + b^2} = 1$. The multiplicative inverse of $a$ is $\frac{a - bi}{\sqrt{a^2 + b^2}} = a - bi$. And the absolute value of $a - bi$ is $\sqrt{a^2 + (-b)^2} = \sqrt{a^2 + b^2} = 1$. Thus $H$ is closed under inverses.

            Further, the product $(a + bi)(c + di) = ac - bd + (ad + bc)i$ has absolute value $\sqrt{(ac - bd)^2 + (ad + bc)^2}$. This simplifies to:
            \begin{multline*}
                \sqrt{a^2c^2 - 2abcd + b^2d^2 + a^2d^2 + 2abcd + b^2c^2} = \\
                \sqrt{a^2c^2 + a^2d^2 + b^2c^2 + b^2d^2} = \sqrt{a^2(c^2 + d^2) + b^2(c^2 + d^2)} = \\
                \sqrt{(a^2 + b^2)(c^2 + d^2)} = \sqrt{a^2 + b^2}\sqrt{c^2 + d^2} = 1,
            \end{multline*}
            and so $H$ is closed under multiplication. Thus it is a subgroup of $G$.
          \end{proof}
    \item $H = $ for fixed $n \in \mathbb{Z}^+$ the set of rational numbers whose denominators divide $n$, $G = \mathbb{Q}$ (under addition)
          \begin{proof}
            Formally, $H = \{ p/q \in \mathbb{Q} \mid q \text{ divides } n \}$. Let $p_1/q_1, p_2/q_2 \in H$. Since $q_1, q_2$ divide $n$, let $aq_1 = bq_2 = n$. Then $p_1/q_1 = ap_1/aq_1 = ap_1/n$ and $p_2/q_2 = bp_2/bq_2 = bp_2/n$. The additive inverse of $p_2/q_2 = bp_2/n$ is $-bp_2/n$. The sum $ap_1/n + (-bp_2/n) = (ap_1 - bp_2)/n$ has a denominator that divides $n$ (or else simplifies to a denominator that divides $n$), and so it is an element of $H$. By the subgroup criterion, $H$ is a subgroup of $G$.
          \end{proof}
    \item $H = $ for fixed $n \in \mathbb{Z}^+$ the set of rational numbers whose denominators are relatively prime to $n$, $G = \mathbb{Q}$ (under addition)
          \begin{proof}
            As immediately above, let $p_1/q_1, p_2/q_2 \in H$. Let $a$ be the greatest common divisor of $q_1$ and $q_2$, and let $q_1 = ar_1, q_2 = ar_2$. Since $q_1, q_2$ are relatively prime to $n$, so too are the corresponding divisors $a, r_1, \text{ and } r_2$. Now the sum of the first element with the inverse of the second element is:
            \begin{equation*}
                p_1/q_1 - p_2/q_2 = p_1/ar_1 - p_2/ar_2 = \frac{p_1r_2 - p_2r_1}{ar_1r_2},
            \end{equation*}
            and since the factors in the divisor are all relatively prime to $n$, so is their product, and so the result is an element of $H$. Thus by the subgroup criterion, $H$ is a subgroup of $G$.
          \end{proof}
    \item $H = $ the set of nonzero real numbers whose square is a rational number, $G = \mathbb{R}$ (under multiplication)
          \begin{proof}
            Let $x_1, x_2 \in H$, with $x_1^2 = p_1/q_1 \in \mathbb{Q}, x_2^2 = p_1/q_1 \in \mathbb{Q}$.
            
            The multiplicative inverse of $x_2$ is $1/x_2$. Consider $x_1/x_2$. Now $(x_1/x_2)^2 = \frac{p_1/q_1}{p_2/q_2} = \frac{p_1}{q_1} \cdot \frac{q_2}{p_1} = \frac{p_1q_2}{p_2q_1} \in \mathbb{Q}$. Thus by the subgroup criterion, $H$ is a subgroup of $G$.
          \end{proof}
\end{enumerate}

\end{document}