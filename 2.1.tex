\documentclass{article}

\title{Dummit \& Foote Ch. 2.1: Subgroups, Definition and Examples}
\author{Scott Donaldson}
\date{May 2023}
\usepackage{amsmath, amsthm, amsfonts, enumitem}

\begin{document}

\maketitle

Let $G$ be a group.

\section*{1. (5/22/23)}

In each of (a) - (e) prove that the specified subset $H$ is a subgroup of the given group $G$:

\begin{enumerate}[label=(\alph*)]
    \item $H = $ the set of complex numbers of the form $a + ai, a \in \mathbb{R}$, $G = \mathbb{C}$ (under addition)
          \begin{proof}
            Let $a + ai, b + bi \in H$. $(b + bi) + (-b - bi) = 0$, so the inverse of $b + bi$ is $-b - bi$.

            Then $a + ai - b + bi = (a - b) + (a - b)i \in H$. By the subgroup criterion, $H$ is a subgroup of $G$.
          \end{proof}
    \item $H = $ the set of complex numbers of absolute value 1, i.e., the unit circle in the complex plane, $G = \mathbb{C}$ (under multiplication)
          \begin{proof}
            Let $a + bi, c + di \in H$. Since $|a + bi| = 1, \sqrt{a^2 + b^2} = 1$. The multiplicative inverse of $a$ is $\frac{a - bi}{\sqrt{a^2 + b^2}} = a - bi$. And the absolute value of $a - bi$ is $\sqrt{a^2 + (-b)^2} = \sqrt{a^2 + b^2} = 1$. Thus $H$ is closed under inverses.

            Further, the product $(a + bi)(c + di) = ac - bd + (ad + bc)i$ has absolute value $\sqrt{(ac - bd)^2 + (ad + bc)^2}$. This simplifies to:
            \begin{multline*}
                \sqrt{a^2c^2 - 2abcd + b^2d^2 + a^2d^2 + 2abcd + b^2c^2} = \\
                \sqrt{a^2c^2 + a^2d^2 + b^2c^2 + b^2d^2} = \sqrt{a^2(c^2 + d^2) + b^2(c^2 + d^2)} = \\
                \sqrt{(a^2 + b^2)(c^2 + d^2)} = \sqrt{a^2 + b^2}\sqrt{c^2 + d^2} = 1,
            \end{multline*}
            and so $H$ is closed under multiplication. Thus it is a subgroup of $G$.
          \end{proof}
    \item $H = $ for fixed $n \in \mathbb{Z}^+$ the set of rational numbers whose denominators divide $n$, $G = \mathbb{Q}$ (under addition)
          \begin{proof}
            Formally, $H = \{ p/q \in \mathbb{Q} \mid q \text{ divides } n \}$. Let $p_1/q_1, p_2/q_2 \in H$. Since $q_1, q_2$ divide $n$, let $aq_1 = bq_2 = n$. Then $p_1/q_1 = ap_1/aq_1 = ap_1/n$ and $p_2/q_2 = bp_2/bq_2 = bp_2/n$. The additive inverse of $p_2/q_2 = bp_2/n$ is $-bp_2/n$. The sum $ap_1/n + (-bp_2/n) = (ap_1 - bp_2)/n$ has a denominator that divides $n$ (or else simplifies to a denominator that divides $n$), and so it is an element of $H$. By the subgroup criterion, $H$ is a subgroup of $G$.
          \end{proof}
    \item $H = $ for fixed $n \in \mathbb{Z}^+$ the set of rational numbers whose denominators are relatively prime to $n$, $G = \mathbb{Q}$ (under addition)
          \begin{proof}
            As immediately above, let $p_1/q_1, p_2/q_2 \in H$. Let $a$ be the greatest common divisor of $q_1$ and $q_2$, and let $q_1 = ar_1, q_2 = ar_2$. Since $q_1, q_2$ are relatively prime to $n$, so too are the corresponding divisors $a, r_1, \text{ and } r_2$. Now the sum of the first element with the inverse of the second element is:
            \begin{equation*}
                p_1/q_1 - p_2/q_2 = p_1/ar_1 - p_2/ar_2 = \frac{p_1r_2 - p_2r_1}{ar_1r_2},
            \end{equation*}
            and since the factors in the divisor are all relatively prime to $n$, so is their product, and so the result is an element of $H$. Thus by the subgroup criterion, $H$ is a subgroup of $G$.
          \end{proof}
    \item $H = $ the set of nonzero real numbers whose square is a rational number, $G = \mathbb{R}$ (under multiplication)
          \begin{proof}
            Let $x_1, x_2 \in H$, with $x_1^2 = p_1/q_1 \in \mathbb{Q}, x_2^2 = p_1/q_1 \in \mathbb{Q}$.
            
            The multiplicative inverse of $x_2$ is $1/x_2$. Consider $x_1/x_2$. Now $(x_1/x_2)^2 = \frac{p_1/q_1}{p_2/q_2} = \frac{p_1}{q_1} \cdot \frac{q_2}{p_1} = \frac{p_1q_2}{p_2q_1} \in \mathbb{Q}$. Thus by the subgroup criterion, $H$ is a subgroup of $G$.
          \end{proof}
\end{enumerate}

\section*{2. (5/22/23)}

In each of (a) - (e) prove that the specified subset $H$ is \emph{not} a subgroup of the given group $G$:

\begin{enumerate}[label=(\alph*)]
    \item $H = $ the set of 2-cycles, $G = D_{2n}$ for $n \geq 3$
          \begin{proof}
            $H$ is not closed. Let $\sigma_1 = (1, 2), \sigma_2 = (2, 3)$, then $\sigma_1 \sigma_2 = (1, 3, 2)$, a 3-cycle and therefore not in $H$.
          \end{proof}
    \item $H = $ the set of reflections, $G = D_{2n}$ for $n \geq 3$
          \begin{proof}
            Formally, $H = \{ sr^k \in D_{2n} \mid 0 \leq k < n \}$. $H$ is not closed. For example, $sr, sr^2 \in H$ but $sr^2sr = sr^2r^{-1}s = srs = ssr^{-1} = r^{-1} \notin H$.
          \end{proof}
    \item $H = \{ x \in G \mid |x| = n \} \cup \{ 1 \}$, $G$ a group containing an element of order $n$ where $n$ is a composite integer greater than 1
          \begin{proof}
            By counterexample, let $G = \mathbb{Z}/8\mathbb{Z}$ under modular addition. Let $n = 8$. The elements 1 and 3 have order 8, so both are in $H$. However, their sum, 4, has order 2, and so is not an element of $H$.
          \end{proof}
    \item $H = $ the set of (positive and negative) odd integers together with 0, $G = \mathbb{Z}$
          \begin{proof}
            Let $k_1, k_2 \in H$. Since both are odd, there exist $n_1, n_2 \in \mathbb{Z}$ such that $k_1 = 2n_1 + 1 \text{ and } k_2 = 2n_2 + 1$. Their sum, then, is $2n_1 + 1 + 2n_2 + 1 = 2n_1 + 2n_2 + 2 = 2(n_1 + n_2 + 1)$, which is an even integer, and so is not an element of $H$.
          \end{proof}
    \item $H = $ the set of real numbers whose square is a rational number, $G = \mathbb{R}$ (under addition)
          \begin{proof}
            By counterexample, consider $\sqrt{2}, \sqrt{3} \in H$. Their sum, $\sqrt{2} + \sqrt{3}$, when squared, is equal to $(\sqrt{2} + \sqrt{3})^2 = 2 + 2\sqrt{6} + 3 = 5 + 2\sqrt{6} \notin \mathbb{Q}$. Therefore $H$ is not closed, and is not a subset of $G$.
          \end{proof}
\end{enumerate}

\section*{3. (5/22/23)}

Show that the following subsets of the dihedral group $D_8$ are actually subgroups:

\begin{enumerate}[label=(\alph*)]
    \item $\{ 1, r^2, s, sr^2 \}$
          \begin{proof}
            For these 4 elements, we will exhaustively show that the subset fulfills the criteria for a subgroup of $D_8$.

            Each element is its own inverse in $D_8$, so the set is closed under inverses.

            It is also closed under the product of two elements. Considering only the non-trivial products, starting with $r^2$: $r^2s = sr^{-2} = sr^2$ and $r^2sr^2 = sr^{-2}r^2 = s$. For $s$: $ssr^2 = r^2$. Finally for $sr^2$: $sr^2 r^2 = s$; $sr^2 s = ssr^{-2} = r^2$. Since the subset of closed under inverses and the binary operation, it is a subgroup.
          \end{proof}
    \item $\{ 1, r^2, sr, sr^3 \}$
          \begin{proof}
            Similar to above, each element is its own inverse. To show it is closed, then, starting with $r^2$: $r^2 sr = sr^{-2} r = sr^{-1} = sr^3$; $r^2 sr^3 = sr^{-2} r^3 = sr$. For $sr$: $sr r^2 = sr^3$; $sr sr^3 = ssr^{-1} r^3 = r^2$. Finally for $sr^3$: $sr^3 r^2 = sr^{-1} r^2 = sr$; $sr^3 sr = ss r^{-3} r = r^{-2} = r^2$. Thus it is a subgroup of $D_8$.
          \end{proof}
\end{enumerate}

\section*{4. (5/22/23)}

Give an explicit example of a group $G$ and an infinite subset $H$ of $G$ that is closed under the group operation but is not a subgroup of $G$.

\begin{proof}
    Let $G = \mathbb{Z}, H = \mathbb{Z}^+$. For any two $n, m \in H$, we have $n > 0$ and $m > 0$. Their sum, $n + m$, is also greater than zero, and so is an element of $H$. However, $H$ does not contain the identity element 0 (as well as containing no additive inverses of any elememts), and so is not a subgroup of $G$.
\end{proof}

\section*{5. (5/22/23)}

Prove that $G$ cannot have a subgroup $H$ with $|H| = n - 1$, where $n = |G| > 2$.

\begin{proof}
    Let $G$ be a finite group of order $n > 2$ and suppose (toward contradiction) that $H$ is a subgroup of $G$ with order $n - 1$. Since $H$ is a subgroup, $1 \in H$. There is exactly one element of $G$ that is not an element of $H$, and it is not the identity. Call that element $g$. Then $g^{-1}$ must be an element of $H$. However, $g^{-1}$ has no inverse in $H$, since by definition $g$ is not in $H$. Therefore $H$ cannot be a subgroup, contradicting the initial assumption that $H$ is a subgroup of $G$ with order $n - 1$.
\end{proof}

\section*{6. (5/23/23)}

Let $G$ be an abelian group. Prove that $\{g \in G \mid |g| < \infty \}$ is a subgroup of $G$ (called the \emph{torsion subgroup} of $G$). Give an explicit example where this set is not a subgroup when $G$ is non-abelian.

\begin{proof}
    We will show that the given set is closed and closed under inverses, and is thus a subgroup of $G$.

    First, let $g_1, g_2 \in G$ with $|g_1| = n$ and $|g_2| = m$. Let $k$ be the least common multiple of $n$ and $m$. Then $g_1^k = g_2^k = 1$. And, given that $G$ is abelian, we have $g_1^k g_2^k = (g_1 g_2)^k = 1$. Thus the order of $g_1 g_2$ is finite, so the set is closed.

    Next, it suffices to demonstrate that, for all $g \in G$ with $|g| = n$, we have $|g^{-1}| = n$, so the set is also closed under inverses and is thus a subgroup of $G$.

    In the non-abelian group $G = GL_2(\mathbb{R})$, however, consider the two elements $\begin{pmatrix}1 & 0 \\ 0 & -1\end{pmatrix}$ and $\begin{pmatrix}1 & 0 \\ 1 & -1\end{pmatrix}$. Each has order 2. However, their product is $\begin{pmatrix}1 & 0 \\ -1 & 1\end{pmatrix}$. Multiplied by itself, this results in $\begin{pmatrix}1 & 0 \\ -2 & 1\end{pmatrix}$, and in general, it can be proven through induction that $\begin{pmatrix}1 & 0 \\ -1 & 1\end{pmatrix}^n =\begin{pmatrix}1 & 0 \\ -n & 1\end{pmatrix}$; that is, it has infinite order. Thus the set of elements with finite order is not closed in $GL_2(\mathbb{R})$ and so it is not a subgroup.
\end{proof}

\section*{7. (5/23/23)}

Fix some $n \in \mathbb{Z}$ with $n > 1$. Find the torsion subgroup of $\mathbb{Z} \times (\mathbb{Z}/n\mathbb{Z})$. Show that the set of elements infinite order together with the identity is \emph{not} a subgroup of this direct product.

\begin{proof}
    The torsion subgroup of $\mathbb{Z} \times (\mathbb{Z}/n\mathbb{Z})$ is $\{ (k, m) \mid |(k, m)| < \infty, k \in \mathbb{Z}, 0 \leq m < n \}$. Considering $\mathbb{Z} \times (\mathbb{Z}/n\mathbb{Z})$ under addition, suppose that $(k, m)$ has order $p$, so $p(k, m) = (0, 0)$. Then $pk = 0$ (in $\mathbb{Z}$) and $mk = 0$ (in $\mathbb{Z}/n\mathbb{Z}$). The only value of $k$ that satisfies this in 0. Because $\mathbb{Z}/n\mathbb{Z}$ is finite, all values of $m$ have finite order (that is, there exists a $p$ for all $m \in \mathbb{Z}/n\mathbb{Z}$ such that $pm = 0$).

    It follows that the torsion subgroup of $\mathbb{Z} \times (\mathbb{Z}/n\mathbb{Z})$ is $\{ (0, m) \mid 0 \leq m < n \}$.

    Now let $A = $ the set of elements of infinite order together with identity, that is, $A = \{ (k, m) \mid |(k, m)| = \infty \} \cup \{ (0, 0) \}$. $(1, 1)$ and $(-1, 1)$ are both in $A$. However, their sum, $(0, 1)$, has finite order (it is in the torsion subgroup, above), and so $A$ is not closed, and is therefore not a subgroup of $\mathbb{Z} \times (\mathbb{Z}/n\mathbb{Z})$.
\end{proof}

\section*{8. (5/27/23)}

Let $H$ and $K$ be subgroups of $G$. Prove that $H \cup K$ is a subgroup if and only if either $H \subseteq K$ or $K \subseteq H$.

\begin{proof}
    First, to show that $H \cup K$ a subgroup implies that $H \subset K$, let $h \in H$ and $k \in K$. Because $H \cup K$ is a subgroup, it is closed, and since $h, k \in H \cup K$, it follows that $hk \in H \cup K$. Then either $hk \in H$ or $hk \in K$.
    
    If $hk \in H$, then, since $H$ is closed under inverses, $h^{-1} \in H$, we have $h^{-1}hk \in H \Rightarrow k \in H$. This implies that $K \subseteq H$. Or, if $hk \in K$, then similarly $hkk^{-1} = h \in K \Rightarrow H \subseteq K$.

    Next, to show that one subgroup being contained in the other implies that their union is a subgroup, without loss of generality let $H \subseteq K$. In this case $H \subset K \Rightarrow H \cup K = K$, which by definition is a subgroup of $G$.

    Thus $H \cup K$ is a subgroup if and only if $H \subseteq K$ or $K \subseteq H$.
\end{proof}\

\section*{9. (5/27/23)}

Let $G = GL_n(F)$, where $F$ is any field. Define
\begin{equation*}
    SL_n(F) = \{ A \in GL_n(F) \mid \det{A} = 1 \}
\end{equation*}
(called the \emph{special linear group}). Prove that $SL_n(F) \leq GL_n(F)$.

\begin{proof}
    Clearly $SL_n(F)$ is a subset of $GL_n(F)$, since by definition $A \in SL_n(F)$ implies $A \in GL_n(F)$. It remains to be proven that $SL_n(F)$ is a subgroup, which we will show by the subgroup criterion.

    Let $B \in SL_n(F)$. From elementary linear algebra, the determinant of the product of two matrices is equal to the product of the two determinants of each matrix. So $1 = \det{I_n} = \det{B^{-1} B} = \det{B^{-1}} \det{B} = \det{B^{-1}}$. Then the determinant of $B$'s inverse is also 1, so $SL_n(F)$ is closed under inverses.

    Next, let $A \in SL_n(F)$ and consider the product $AB^{-1}$. From above, the determinant of this matrix is equal to $\det{A} \det{B^{-1}} = 1 \cdot 1 = 1$. Thus $SL_n(F)$ is also closed under matrix multiplication, and so is a subgroup of $GL_n(F)$.
\end{proof}

\section*{10. (5/27/23)}

\begin{enumerate}[label=(\alph*)]
    \item Prove that if $H$ and $K$ are subgroups of $G$ then so is their intersection $H \cap K$.
          \begin{proof}
            Let $g_1, g_2 \in H \cap K$. Then $g_1 \in H, g_2 \in H, g_1 \in K, \text{ and } g_2 \in K$. It follows that the product $g_1 g_2$ is an element of both $H$ and $K$, since both subgroups are closed. Thus $g_1 g_2 \in H \cap K$, and so $H \cap K$ is closed.

            Similarly, 
            \begin{equation*}
                g \in H \cap K \Rightarrow g \in H, g \in K \Rightarrow g^{-1} \in H, g^{-1} \in K \Rightarrow g^{-1} \in H \cap K,
            \end{equation*}
            which shows that $H \cap K$ is also closed under inverses and is therefore a subgroup of $G$.
          \end{proof}
    \item Prove that the intersection of an arbitrary nonempty collection of subgroups of $G$ is again a subgroup of $G$ (do not assume the collection is countable).
          \begin{proof}
            Let $\mathcal{H}$ be a collection of nonempty subgroups of $G$. Consider $\bigcap_{H \in \mathcal{H}} = \{ h \in G \mid h \in H \text{ for all } H \in \mathcal{H} \}$. Let $h_1, h_2$ be in this subset. Then for all $H \in \mathcal{H}$, $h_1 \in H$ and $h_2 \in H$, so $h_1 h_2 \in H$. So this intersection is closed under the binary operation of $G$. Similarly, for all $H \in \mathcal{H}$, $h \in H \Rightarrow h^{-1} \in H$, and so it is also closed under inverses.

            Thus an arbitrary nonempty collection of subgroups is a subgroup.
          \end{proof}
\end{enumerate}

\end{document}