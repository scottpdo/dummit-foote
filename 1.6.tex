\documentclass{article}

\title{Dummit \& Foote Ch. 1.6: Homomorphisms and Isomorphisms}
\author{Scott Donaldson}
\date{Mar. 2023}
\usepackage{amsmath, amsthm, amsfonts, enumitem}

\begin{document}

\maketitle

\section*{1. (3/25/23)}

Let $\varphi: G \rightarrow H$ be a homomorphism.

\begin{enumerate}[label=(\alph*)]
    \item Prove that $\varphi(x^n) = \varphi(x)^n$ for all $n \in \mathbb{Z}^+$.
          \begin{proof}
            By induction. When $n = 1, \varphi(x^1) = \varphi(x) = \varphi(x)^1$.

            Suppose for some $n$, $\varphi(x^n) = \varphi(x)^n$. Then $\varphi(x^{n + 1}) = \varphi(x^n x)$. By definition, because $\varphi$ is a homomorphism from $G$ to $H$, $\varphi(ab) = \varphi(a)\varphi(b)$ for all $a, b \in G$. So $\varphi(x^n x) = \varphi(x^n) \varphi(x)$. By the induction hypothesis, $\varphi(x^n) = \varphi(x)^n$, so this equals $\varphi(x)^{n + 1}$.

            Therefore $\varphi(x^n) = \varphi(x)^n$ for all $n \in \mathbb{Z}^+$.
          \end{proof}

    \item Do part (a) for $n = -1$ and deduce that $\varphi(x^n) = \varphi(x)^n$ for all $n \in \mathbb{Z}$.
    
          This proof diverges slightly from the directions but arrives at the same result.
          
          Note that, for all $x \in G$, $\varphi(x) = \varphi(1 \cdot x) = \varphi(1) \varphi(x)$. Therefore $\varphi(1) = 1$ (in $H$). Now $1 = \varphi(1) = \varphi(x^n \cdot x^{-n}) = \varphi(x^n) \varphi(x^{-n})$. From part a), this equals $\varphi(x)^n \varphi(x^{-n})$. Left-multiplying both sides by $\varphi(x)^{-n}$, we obtain $\varphi(x^{-n}) = \varphi(x)^{-n}$, as desired.
          
\end{enumerate}

\section*{2. (3/26/23)}

If $\varphi: G \rightarrow H$ is an isomorphism, prove that $|\varphi(x)| = |x|$ for all $x \in G$. Deduce that any two isomorphic groups have the same number of elements of order $n$ for each $n \in \mathbb{Z}^+$.

\begin{proof}
    Let $\varphi: G \rightarrow H$ be an isomorphism and let $x \in G$. If $|x|$ is finite, then (from 1.a) $\varphi(x^n) = \varphi(x)^n$ and (from 1.b) $\varphi(1) = \varphi(x^n) = \varphi(x)^n = 1 \in H$. The order of the element $\varphi(x)^n \in H$ is therefore at most $n$. Because $\varphi$ is an isomorphism, there is only one element whose image is 1, and that is $\varphi(1) = 1$. Therefore for no $m < n$ do we have $\varphi(x)^m = 1$, and so the $|\varphi(x) = n$.

    Next, suppose that $x$ has infinite order in $G$. Then $x^n \neq 1$ for all $n > 0$. Because $\varphi$ is an isomorphism, we know that only $\varphi(1) = 1 \in H$. Therefore $\varphi(x^n) = \varphi(x)^n \neq 1$ for all $n > 0$. Therefore $|\varphi(x)| = \infty$.

    This result is not necessarily true if $\varphi$ is a homomorphism. For example, $\varphi$ could send every element of $G$ to the identity in $H$. (This is a homomorphism: $\varphi(x)\varphi(y) = 1 \cdot 1 = 1$ and $\varphi(x)\varphi(y) = \varphi(xy) = 1$.) Then for all $x \in G$, $|\varphi(x)| = 1$, regardless of the order of $x$.
\end{proof}

\section*{3. (3/27/23)}

If $\varphi: G \rightarrow H$ is an isomorphism, prove that $G$ is abelian if and only if $H$ is abelian. If $\varphi$ is a homomorphism, what additional conditions on $\varphi$ (if any) are sufficient to ensure that if $G$ is abelian, then so is $H$?

\begin{proof}
    First, let $G$ be an abelian group and $\varphi: G \rightarrow H$ be an isomorphism. Given arbitrary distinct elements of $H$, because $\varphi$ is surjective, there are two distinct elements in $G$ whose images are these elements in $H$. Let $\varphi(x), \varphi(y) \in H$ be distinct elements and $x, y \in G$. Then $\varphi(xy) = \varphi(x)\varphi(y)$. Also, because $x$ and $y$ commute, $\varphi(xy) = \varphi(yx) = \varphi(y)\varphi(x)$. Therefore $\varphi(x)\varphi(y) = \varphi(y)\varphi(x)$, so $H$ is an abelian group.

    Next, let $H$ be an abelian group. Again let $\varphi(x), \varphi(y) \in H$ and $x, y \in G$. Then $\varphi(x)\varphi(y) = \varphi(xy)$. Also, $\varphi(x)\varphi(y) = \varphi(y)\varphi(x) = \varphi(yx)$. So $\varphi(xy) = \varphi(yx)$. Because $\varphi$ is one-to-one, this implies that $xy = yx$, and so $G$ is an abelian group.

    If $\varphi$ is a homomorphism, then $G$ being an abelian group does not imply that $H$ is abelian. For example, $H$ could be a non-abelian group and $\varphi$ could send every element of $G$ to the identity in $H$.

    A sufficient condition for a homomorphism $\varphi: G \rightarrow H$ to ensure that if $G$ is abelian, then so is $H$, is that $\varphi$ is surjective. Then for all $h \in H$, $h = \varphi(x)$ for some $x \in G$ (possibly more than one $x$). Let $h_1, h_2 \in H$ with $h_1 = \varphi(x_1) = \varphi(x_2) = ...$ and $h_2 = \varphi(y_1) = \varphi(y_2) = ...$ and with $x_i, y_j \in G$. $\varphi$ is a homomorphism, so for any $i, j$, $\varphi(x_i y_j) = \varphi(x_i) \varphi(y_j) = h_1 h_2$. Also, because $G$ is abelian, $\varphi(x_i y_j) = \varphi(y_j x_i) = \varphi(y_j) \varphi(x_i) = h_2 h_1$. Therefore $h_1 h_2 = h_2 h_1$, so $H$ is abelian.
\end{proof}

\section*{4. (3/27/23)}

Prove that the multiplicative groups $\mathbb{R} - \{0\}$ and $\mathbb{C} - \{0\}$ are not isomorphic.

\begin{proof}
    For any $x \in \mathbb{R} - \{0\}$, $x \neq \pm{1}$, $x$ has infinite order. The proof of this is as follows: Let $x \in \mathbb{R} - \{0, \pm{1}\}$. If the absolute value of $x$ is greater than 1, then the absolute value of $x^n$ is greater than 1 for all $n$, and by induction $x$ has infinite order. If the absolute value of $x$ is less than 1, then the absolute value of $x^n$ is less than 1 for all $n$, and by induction $x$ has infinite order. So $1$ and $-1$ are the only elements of $\mathbb{R} - \{0\}$ with finite order.

    In $\mathbb{C} - \{0\}$, $i$ and $-i$ have order 4. From 2., isomorphic groups have the same number of elements of order $n$ for each $n \in \mathbb{Z}^+$. However, $\mathbb{R} - \{0\}$ has no elements of order 4, and $\mathbb{C} - \{0\}$ has at least 2. Therefore they are not isomorphic.
\end{proof}

\end{document}