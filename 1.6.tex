\documentclass{article}

\title{Dummit \& Foote Ch. 1.6: Homomorphisms and Isomorphisms}
\author{Scott Donaldson}
\date{Mar. 2023}
\usepackage{amsmath, amsthm, amsfonts, enumitem}

\begin{document}

\maketitle

\section*{1. (3/25/23)}

Let $\varphi: G \rightarrow H$ be a homomorphism.

\begin{enumerate}[label=(\alph*)]
    \item Prove that $\varphi(x^n) = \varphi(x)^n$ for all $n \in \mathbb{Z}^+$.
          \begin{proof}
            By induction. When $n = 1, \varphi(x^1) = \varphi(x) = \varphi(x)^1$.

            Suppose for some $n$, $\varphi(x^n) = \varphi(x)^n$. Then $\varphi(x^{n + 1}) = \varphi(x^n x)$. By definition, because $\varphi$ is a homomorphism from $G$ to $H$, $\varphi(ab) = \varphi(a)\varphi(b)$ for all $a, b \in G$. So $\varphi(x^n x) = \varphi(x^n) \varphi(x)$. By the induction hypothesis, $\varphi(x^n) = \varphi(x)^n$, so this equals $\varphi(x)^{n + 1}$.

            Therefore $\varphi(x^n) = \varphi(x)^n$ for all $n \in \mathbb{Z}^+$.
          \end{proof}

    \item Do part (a) for $n = -1$ and deduce that $\varphi(x^n) = \varphi(x)^n$ for all $n \in \mathbb{Z}$.
    
          This proof diverges slightly from the directions but arrives at the same result.
          
          Note that, for all $x \in G$, $\varphi(x) = \varphi(1 \cdot x) = \varphi(1) \varphi(x)$. Therefore $\varphi(1) = 1$ (in $H$). Now $1 = \varphi(1) = \varphi(x^n \cdot x^{-n}) = \varphi(x^n) \varphi(x^{-n})$. From part a), this equals $\varphi(x)^n \varphi(x^{-n})$. Left-multiplying both sides by $\varphi(x)^{-n}$, we obtain $\varphi(x^{-n}) = \varphi(x)^{-n}$, as desired.
          
\end{enumerate}

\end{document}