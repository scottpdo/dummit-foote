\documentclass{article}

\title{Dummit \& Foote Ch. 1.3: Symmetric Groups}
\author{Scott Donaldson}
\date{Feb. - Mar. 2023}
\usepackage{amsmath, amsthm, amsfonts, enumitem}

\begin{document}

\maketitle

\section*{1. (2/16/23)}

\begin{itemize}
    \item $\sigma: (1, 3, 5)(2, 4)$
    \item $\tau: (1, 5)(2, 3)$
    \item $\sigma^2: (1, 5, 3)$
    \item $\sigma \tau: (2, 5, 3, 4)$
    \item $\tau \sigma: (1, 2, 4, 3)$
    \item $\tau^2 \sigma: (1, 3, 5)(2, 4)$ (because $\tau^2 = 1$, so $\tau^2 \sigma = \sigma$)
\end{itemize}

\section*{2. (2/16/23)}

\begin{itemize}
    \item $\sigma: (1,13,5,10)(3,15,8)(4,14,11,7,12,9)$
    \item $\tau: (1, 14)(2, 9, 15, 13, 4)(3, 10)(5, 12, 7)(8, 11)$
    \item $\sigma^2: (1, 5)(3, 8, 15)(4, 11, 12)(7, 9, 4)(10, 13)$
    \item $\sigma \tau: (1, 11, 3)(2, 4)(5, 9, 8, 7, 10, 15)(13, 14)$
    \item $\tau \sigma: (1, 4)(2, 9)(3, 13, 12, 15, 11, 5)(8, 10, 14)$
    \item $\tau^2 \sigma: (1, 2, 15, 8, 3, 4, 14, 11, 12, 13, 7, 5, 10)$
\end{itemize}

\section*{3. (2/16/23)}

Compute the order of each of the permutations whose cycle decompositions were computed above.

\begin{enumerate}
    \item $|\sigma| = 6$; $|\tau| = 2$; $|\sigma^2| = 3$; $|\sigma \tau| = 4$; $|\tau \sigma| = 4$; $|\tau^2 \sigma| = 6$
    \item $|\sigma| = 12$; $|\tau| = 30$; $|\sigma^2| = 6$; $|\sigma \tau| = 6$; $|\tau \sigma| = 6$; $|\tau^2 \sigma| = 13$
\end{enumerate}

\section*{4. (2/16/23)}

Compute the order of each of the elements in the following groups:

\begin{enumerate}[label=(\alph*)]
    \item $S_3$
        \begin{itemize}
            \item $(1)$: 1
            \item $(1, 2); (1, 3); (2, 3)$: 2
            \item $(1, 2, 3); (1, 3, 2)$: 3
        \end{itemize}
    \item $S_4$
        \begin{itemize}
            \item $(1)$: 1
            \item $(1, 2); (1, 3); (1, 4); (2, 3); (2, 4); (3, 4); (1, 2)(3, 4);\\(1, 3)(2, 4); (1, 4)(2, 3)$: 2
            \item $(1, 2, 3); (1, 3, 2); (1, 2, 4); (1, 4, 2); (1, 3, 4); (1, 4, 3);\\(2, 3, 4); (2, 4, 3)$: 3
            \item $(1, 2, 3, 4); (1, 4, 2, 3); (1, 3, 2, 4); (1, 3, 4, 2); (1, 4, 2, 3); (1, 4, 3, 2)$: 4
        \end{itemize}
\end{enumerate}

\section*{5. (2/16/23)}

Find the order of $(1, 12, 8, 10, 4)(2, 13)(5, 11, 7)(6, 9)$.

\begin{proof}
    The order of a permutation in a symmetric group is the least common multiple of its cycles. However, since we have not yet proven this, we will calculate the first few multiples of the permutation manually, and extrapolate from there. Let $\sigma = (1, 12, 8, 10, 4)(2, 13)(5, 11, 7)(6, 9)$.

    $\sigma^2 = (1, 8, 4, 12, 10)(5, 7, 11)$.

    $\sigma^3 = (1, 10, 12, 4, 8)(2, 13)(6, 9)$.

    $\sigma^4 = (1, 4, 10, 8, 12)(5, 11, 7)$.
    
    $\sigma^5 = (2, 13)(5, 7, 11)(6, 9)$.

    From this pattern, we see that each constituent cycle vanishes from the cycle decomposition when the exponent is a multiple of the cycle's length. Thus, the order of $\sigma$ is the least common multiple of the lengths of its cycles, which is $2 \cdot 3 \cdot 5 = 30$.
\end{proof}

\section*{6. (2/17/23)}

Write out the cycle decomposition of each element of order 4 in $S_4$.

\begin{itemize}
    \item $(1, 2, 3, 4)$
    \item $(1, 2, 4, 3)$
    \item $(1, 3, 2, 4)$
    \item $(1, 3, 4, 2)$
    \item $(1, 4, 2, 3)$
    \item $(1, 4, 3, 2)$
\end{itemize}

\section*{7. (2/20/23)}

Write out the cycle decomposition of each element of order 2 in $S_4$.

\begin{itemize}
    \item $(1, 2)$
    \item $(1, 3)$
    \item $(1, 4)$
    \item $(2, 3)$
    \item $(2, 4)$
    \item $(3, 4)$
    \item $(1, 2)(3, 4)$
    \item $(1, 3)(2, 4)$
    \item $(1, 4)(2, 3)$
\end{itemize}

\section*{8. (2/22/23)}

Prove that if $\Omega = \{1, 2, 3... \}$ then $S_\Omega$ is an infinite group.

\begin{proof}

    Let $\Omega = \{1, 2, 3... \}$. Consider the subset $T$ consisting of all elements whose cycle decomposition is a single 2-cycle permuting $1 \in \Omega$, for example $(1, 2), (1, 10)$ but not $(2, 3)$.

    There is a bijection $f: \mathbb{Z}^+ \rightarrow T$ defined by $f(n) = (1, n + 1)$. Because there is a bijection between these two sets, they have the same cardinality; that is, like $\mathbb{Z}^+$, $T$ is infinite.

    Because $\Omega$ contains a proper subset of infinite size, $\Omega$ has infinite elements and is therefore an infinite group.

\end{proof}

\section*{9. (2/22/23)}

\begin{enumerate}[label=(\alph*)]
    \item Let $\sigma$ be the 12-cycle $(1, 2, 3, 4, 5, 6, 7, 8, 9, 10, 11, 12)$. For which positive integers $i$ is $\sigma^i$ also a 12-cycle?
        \begin{itemize}
            \item $\sigma^5 = (1, 6, 11, 4, 9, 2, 7, 12, 5, 10, 3, 8)$
            \item $\sigma^7 = (1, 8, 3, 10, 5, 12, 7, 2, 9, 4, 11, 6)$
            \item $\sigma^11 = (1, 12, 11, 10, ..., 2)$
        \end{itemize}
    \item Let $\tau$ be the 8-cycle $(1, 2, 3, 4, 5, 6, 7, 8)$. For which positive integers $i$ is $\tau^i$ also a 12-cycle?
        \begin{itemize}
            \item $\tau^3 = (1, 4, 7, 2, 5, 8, 3, 6)$
            \item $\tau^5 = (1, 6, 3, 8, 5, 2, 7, 4)$
            \item $\tau^7 = (1, 8, 7, 6, 5, 4, 3, 2)$
        \end{itemize}
    \item Let $\omega$ be the 14-cycle $(1, 2, 3, 4, 5, 6, 7, 8, 9, 10, 11, 12, 13, 14)$. For which positive integers $i$ is $\omega^i$ also a 12-cycle?
        \begin{itemize}
            \item $\omega^3 = (1, 4, 7, 10, 13, 2, 5, 8, 11, 14, 3, 6, 9, 12)$
            \item $\omega^5 = (1, 6, 11, 2, 7, 12, 3, 8, 13, 4, 9, 14, 5, 10)$
            \item $\omega^9 = (1, 10, 5, 14, 9, 4, 13, 8, 3, 12, 7, 2, 11, 6)$
            \item $\omega^{11} = (1, 12, 9, 6, 3, 14, 11, 8, 5, 2, 13, 10, 7, 4)$
            \item $\omega^{13} = (1, 14, 13, 12, ..., 2)$
        \end{itemize}
\end{enumerate}

\section*{10. (2/23/23)}

Prove that if $\sigma$ is the $m$-cycle $(a_1, a_2, ..., a_m)$, then for all $i \in \{1, 2, 3, ..., m \},\\
\sigma^i(a_k) = a_{k + i}$, where $k + i$ is replaced by its least residue mod $m$ when $k + i > m$. Deduce that $|\sigma| = m$.

\begin{proof}
    We will prove this by induction on $i$. For the base case, $i = 1$, we have, by definition, $\sigma^1 = \sigma$ and $\sigma(a_k) = a_{k + 1}$ for $k < m$. For $m$, $\sigma(a_m) = a_1$, and since $1 = (m + 1)$ mod $m$, this holds for all $i \in \{1, 2, 3, ..., m \}$.

    For the induction case, suppose that for some $n \in \{1, 2, 3, ..., m \}$, $\sigma^n(a_k) = a_{k + n}$ (where $k + n$ is assumed mod $m$) for all valid $k$. Consider $\sigma^{n + 1} = \sigma^n \sigma^1 = \sigma^n \sigma$. For an arbitrary element $a_k$, then, $\sigma^{n + 1}(a_k) = \sigma^n(\sigma(a_k)) = \sigma^n(a_{k + 1})$ (by the base case), which equals $a_{k + n + 1}$ (by the induction hypothesis). Therefore, $\sigma^{n + 1}(a_k) = a_{k + (n + 1)}$.

    Thus, by induction, $\sigma^i(a_k) = a_{k + i}$ for all $i \in \{1, 2, 3, ..., m \}$. It follows that $\sigma^m(a_k) = a_{k + m} = a_{k}$ for all $k$, so $\sigma^m = 1$. Therefore, $|\sigma| = m$.
\end{proof}

\section*{11. (3/12/23)}

Let $\sigma$ be the $m$-cycle $(1, 2, ..., m)$. Show that $\sigma^i$ is also an $m$-cycle if and only if $i$ is relatively prime to $m$.

\begin{proof}
    First, we will show that if $i$ and $m$ are relatively prime, then $\sigma^i$ is also an $m$-cycle. Let $\sigma$ be the $m$-cycle $(1, 2, ..., m)$ and let $i$ be relatively prime to $m$.

    From 10., we know that $\sigma^i(k) = k + i$ (mod $m$). So $\sigma^i$ is the permutation which sends 1 to $1 + i$, $1 + i$ to $1 + 2i$, and so on. We can represent it with the cycle decomposition $(1, 1 + i, 1 + 2i, ...)$ (all mod $m$). It may be the case that this includes multiple disjoint cycles -- we have not yet proven that $\sigma^i$ can be represented by a single cycle. However, every element of the first cycle in its cycle decomposition can be represented as $1 + ki$, with $0 \leq k < m$. The final value in the first cycle is $1 + ki$ such that $1 + (k + 1)i = 1$. For this to occur, because $i$ and $m$ are relatively prime, $m$ must divide $k + 1$, and so $m \leq k + 1$. In fact, we cannot have $m < k + 1$, because when the sequence arrives at $1 + mi$, this is equal to 1. So $m = k + 1$; that is, $\sigma^i$ is also an $m$-cycle.

    Next, we will prove the contrapositive: Namely, that if $i$ and $m$ are not relatively prime, then $\sigma^i$ is not an $m$-cycle. Without loss of generality, let $\sigma$ be the $m$-cycle $(1, 2, ..., m)$ and let $j = $ gcd$\{i, m\} > 1$. Again from 10., $\sigma^i(k) = k + i$ (mod $m$). The first (and possibly only) cycle of the cycle decomposition of $\sigma^i$ is $(1, 1 + i, 1 + 2i, ...)$ (all mod $m$). The last element of this cycle decomposition is $1 + (\frac{m}{j} - 1)n$. Its successor, $1 + \frac{mn}{j}$, is equal to 1 mod $m$, because $m$ divides $\frac{mn}{j}$. In fact, $\frac{mn}{j}$ is the least common multiple of $m$ and $n$ (because $j$ is the greatest common divisor of $m$ and $n$). It follows that $\frac{m}{j} - 1$ is the smallest candidate integer for the coefficient of $n$ in the final element of the first cycle. Therefore, $\sigma^i$ contains a cycle of length $\frac{m}{j} < m$, and so it is not an $m$-cycle. Further, $\sigma^i$ contains exactly $j$ disjoint cycles of length $\frac{m}{j}$.
\end{proof}

\section*{12. (3/12/23)}

\begin{enumerate}[label=(\alph*)]
    \item If $\tau = (1, 2)(3, 4)(5, 6)(7, 8)(9, 10)$ determine whether there is an $n$-cycle $\sigma$ ($n \geq 10$) with $\tau = \sigma^k$ for some integer $k$.

          Let $\sigma$ be the 10-cycle $(1, 3, 5, 7, 9, 2, 4, 6, 8, 10)$.

          $\sigma^2 = (1, 5, 9, 4, 8)(2, 6, 10, 3, 7)$.

          $\sigma^3 = (1, 7, 4, 10, 5, 2, 8, 3, 9, 6)$.

          $\sigma^4 = (1, 9, 8, 5, 4)(2, 10, 7, 6, 3)$, and

          $\sigma^5 = (1, 2)(3, 4)(5, 6)(7, 8)(9, 10)$.
    \item If $\tau = (1, 2)(3, 4, 5)$ determine whether there is an $n$-cycle $\sigma$ ($n \geq 5$) with $\tau = \sigma^k$ for some integer $k$.

          From 11., we know that if $\sigma$ is a 5-cycle, then $\sigma^k$ must be another 5-cycle (because 5 is prime). And, if $\sigma$ is an $n$-cycle with $n > 5$, then its cycle decomposition contains disjoint cycles of equal length. Because $\tau = (1, 2)(3, 4, 5)$ is the product of a 2-cycle and a 3-cycle, there are no $n$-cycles $\sigma$ with $n > 5$ and $\tau = \sigma^k$.
\end{enumerate}

\end{document}