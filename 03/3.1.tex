\documentclass{article}

\title{Dummit \& Foote Ch. 3.1: Quotient Groups and Homomorphisms}
\author{Scott Donaldson}
\date{Aug. - Sep. 2023}
\usepackage{amsmath, amsthm, amsfonts, enumitem, tabu}

\begin{document}

\maketitle

Let $G$ and $H$ be groups.

\section*{1. (8/21/23)}

Let $\varphi: G \rightarrow H$ be a homomorphism and let $E \leq H$. Prove that $\varphi^{-1}(E) \leq G$ (i.e., the preimage or pullback of a subgroup under a homomorphism is a subgroup). If $E \unlhd H$ prove that $\varphi^{-1}(E) \unlhd G$. Deduce that ker $\varphi \unlhd G$. 

\begin{proof}
    Let $x, y \in \varphi^{-1}(E) \subseteq G$. Suppose that $\varphi(x) = a, \varphi(y) = b, a, b \in E \leq H$. Since $\varphi$ is a homomorphism, we have $\varphi(y^{-1}) = \varphi(y)^{-1} = b^{-1}$. Then:
    \begin{equation*}
        \varphi(xy^{-1}) = \varphi(x)\varphi(y^{-1}) = \varphi(x)\varphi(y)^{-1} = ab^{-1} \in E,
    \end{equation*}
    which implies that $xy^{-1} \in \varphi^{-1}(E)$. It follows that, by the subgroup criterion, $\varphi^{-1}(E) \leq G$.
\end{proof}

\section*{2. (8/23/23)}

Let $\varphi: G \rightarrow H$ be a homomorphism of groups with kernel $K$ and let $a, b \in \varphi(G)$. Let $X \in G/K$ be the fiber above $a$ and $Y$ be the fiber above $b$, i.e., $X = \varphi^{-1}(a), Y = \varphi^{-1}(b)$. Fix an element $x \in X$ (so $\varphi(x) = a$). Prove that if $XY = Z$ in the quotient group $G/K$ and $z$ is any member of $Z$, then there is some $y \in Y$ such that $xy = z$.

\begin{proof}
    We know that, for any $x \in X, y \in Y$, $\varphi(x) = a$ and $\varphi(y) = b$. Since $\varphi$ is a homomorphism, it follows that $\varphi(xy) = \varphi(x)\varphi(y) = ab$, and so the image of any element of $XY = Z$ under $\varphi$ is $ab \in H$.
    
    Next, consider the element $x^{-1}z \in G$, as well as its image under $\varphi$. Since $\varphi$ is a homomorphism, we have $\varphi(x^{-1}) = \varphi(x)^{-1}$. So $\varphi(x^{-1}z) = \varphi(x^{-1})\varphi(z) = \varphi(x)^{-1}\varphi(z) = a^{-1}ab = b$. The set $Y$ consists of all elements of $G$ whose image under $\varphi$ is $b$, and so we must have $x^{-1}z \in Y$.

    Now if we fix some element $x \in X$, then for any $z \in Z$, we have $x^{-1}z \in Y$ such that its product with $x$ is $z$: $x x^{-1}z = z$.
\end{proof}

\section*{3. (8/23/23)}

Let $A$ be an abelian group and let $B$ be a subgroup of $A$. Prove that $A/B$ is abelian. Give an example of a non-abelian group $G$ containing a proper normal subgroup $N$ such that $G/N$ is abelian.

\begin{proof}
    Because $A$ is abelian, all subgroups of $A$ are normal, so $A/B$ is well-defined for every $B \leq A$.

    Let $C, D \in A/B$ with $C = cB$ and $D = dB$ for some $c, d \in A$. Then:
    \begin{equation*}
        CD = (cB)(dB) = (cd)B = (dc)B = (dB)(cB) = DC,
    \end{equation*}
    which implies that $A/B$ is abelian.

    Now if we let $G$ be the dihedral group $D_8$, then $G$ is non-abelian. Let $N$ be the cyclic subgroup generated by $r: \{ 1, r, r^2, r^3 \}$. The only coset of $N$ is $sN$; together these two sets cover $G$. Then $G/N = \{ N, sN \}$. There is only one group of order 2 up to isomorphism, and it is abelian. Thus $G/N$ is abelian.
\end{proof}

\section*{4. (8/23/23)}

Prove that in the quotient group $G/N$, $(gN)^\alpha = (g^\alpha)N$ for all $\alpha \in \mathbb{Z}$.

\begin{proof}
    We start by induction: In the base case, $\alpha = 1$, we have $(gN)^1 = gN = (g^1)N$. Next, suppose that for some $\alpha > 1$, we have $(gN)^\alpha = (g^\alpha)N$. Then:
    \begin{equation*}
        (gN)^{\alpha + 1} = (gN)^\alpha gN = g^\alpha N \cdot gN = (g^{\alpha + 1})N,
    \end{equation*}
    as desired. We have now proven that $(gN)^\alpha = (g^\alpha)N$ for $\alpha \geq 1$.

    Next, consider $(gN)^\alpha (gN)^{-\alpha}$, where $\alpha \geq 1$. In the quotient group $G/N$, for any subset $X \in G/N$, we must have $X^\alpha X^{-\alpha} = N$ (the identity of $G/N$), so $(gN)^\alpha (gN)^{-\alpha} = N$. From above, $(gN)^\alpha = (g^\alpha)N$, so $(g^\alpha)N \cdot (gN)^{-\alpha} = N$. Also, from the operation on left cosets, we know that $N = (g^\alpha)N \cdot (g^{-\alpha})N$. Since both $(g^\alpha)N \cdot (gN)^{-\alpha} = N$ and $(g^\alpha)N \cdot (g^{-\alpha})N = N$, we must have $(gN)^{-\alpha} = (g^{-\alpha})N$. We have now proven for all nonzero integers.

    Finally, we note that $(gN)^0 = N$ (the identity of $G/N$) and that $(g^0)N = eN = N$, so $(gN)^0 = (g^0)N$. This concludes the proof that $(gN)^\alpha = (g^\alpha)N$ for all $\alpha \in \mathbb{Z}$.
\end{proof}

\section*{5. (8/23/23)}

Use the preceding exercise to prove that the order of the element $gN$ in $G/N$ is $n$, where $n$ is the smallest positive integer such that $g^n \in N$ (and $gN$ has infinite order if no such positive integer exists). Give an example to show that the order of $gN$ in $G/N$ may be strictly smaller than the order of $g$ in $G$.

\begin{proof}
    Let $gN \in G/N$, and let $n$ be the smallest positive integer such that $g^n \in N$. Suppose that $g^n = h \in N$.

    From Exercise 4., $(gN)^n = (g^n)N = hN = N$ (because $h \in N$), so the order of $gN$ must divide $n$.

    Suppose (toward contradiction) that the order of $gN$ is $k$, where $k < n$. Then $(gN)^k = (g^k)N = N$, which implies that $g^k$ lies in $N$, contradicting our assumption that $n$ is the smallest such positive integer. Therefore the order of $gN$ is $n$.

    If there is no positive integer $n$ such that $g^n \in N$, then for all $k \in \mathbb{Z}^+$, we have $(gN)^k = (g^k)N \neq N$, so $gN$ has infinite order.

    As an example where $|gN| < |g|$, let $G = Z_9 = \langle x \rangle$ and let $N = \langle x^3 \rangle$. Because all cyclic groups are abelian, $N$ is normal in $G$, and so $G/N$ is well-defined. The quotient group $G/N$ contains three elements: $N, xN$, and $(x^2)N$. The element $xN \in G/N$ has order 3: $(xN)^3 = (x^3)N = N$ (because $x^3 \in N$). However, the generating element $x \in G$ has order 9.
\end{proof}

\section*{6. (8/24/23)}

Define $\varphi: \mathbb{R}^\times \rightarrow \{ \pm 1 \}$ by letting $\varphi(x)$ be $x$ divided by the absolute value of $x$. Describe the fibers of $\varphi$ and prove that $\varphi$ is a homomorphism.

\begin{proof}
    We consider the two cases where $x < 0$ and $x > 0$ (0 is not an element of $\mathbb{R}^\times$). If $x > 0$, then $\varphi(x) = x/|x| = x/x = 1$. If $x < 0$, then $\varphi(x) = x/|x| = x/-x = -1$. Therefore the fiber above $-1$ is every negative real number and the fiber above 1 is every positive real number.

    To show that $\varphi$ is a homomorphism, we let $x, y \in \mathbb{R}^\times$ and again consider the different cases: Where $x$ and $y$ are both positive, where they are both negative, and where one is positive and the other negative.

    If both $x$ and $y$ are positive, then $\varphi(x)\varphi(y) = 1 \cdot 1 = 1$ and $\varphi(xy) = \frac{xy}{|xy|} = \frac{xy}{xy} = 1$, so $\varphi(x)\varphi(y) = \varphi(xy)$.

    If both $x$ and $y$ are negative, then $\varphi(x)\varphi(y) = -1 \cdot -1 = 1$ and $\varphi(xy) = \frac{xy}{|xy|} = \frac{xy}{xy} = 1$, so $\varphi(x)\varphi(y) = \varphi(xy)$.

    Suppose $x$ is positive and $y$ is negative. Then $\varphi(x)\varphi(y) = 1 \cdot -1 = -1$ and $\varphi(xy) = \frac{xy}{|xy|} = \frac{xy}{-xy} = -1$, so $\varphi(x)\varphi(y) = \varphi(xy)$.
    
    Thus, in every case of $x, y \in \mathbb{R}^\times$, we have $\varphi(x)\varphi(y) = \varphi(xy)$, and $\varphi$ is thus a homomorphism.
\end{proof}

\section*{7. (8/24/23)}

Define $\pi: \mathbb{R}^2 \rightarrow \mathbb{R}$ by $\pi((x, y)) = x + y$. Prove that $\pi$ is a surjective homomorphism and the describe the kernel and fibers of $\pi$ geometrically.

\begin{proof}
    First, to show that $\pi$ is surjective, let $z \in \mathbb{R}$. Now $z = z + 0$, so $(z, 0)$ is an element of $\mathbb{R}^2$ such that $\pi((z, 0)) = z + 0 = z$.

    Next, to show that $\pi$ is a homomorphism, let $(x_1, y_1), (x_2, y_2) \in \mathbb{R}^2$. We have $\pi((x_1, y_1) + (x_2, y_2)) = \pi((x_1 + x_2, y_1 + y_2)) = x_1 + x_2 + y_1 + y_2$, and $\pi((x_1, y_1)) + \pi((x_2, y_2)) = x_1 + y_1 + x_2 + y_2$. By the commutativity of addition in $\mathbb{R}$, these are equal to each other, and so $\pi$ is a surjective homomorphism.

    The kernel of $\pi$ consists of all points $(x, y) \in \mathbb{R}^2$ such that $x + y = 0$, that is, the diagonal line running from the upper-left to the bottom-right of the Cartesian plane. Geometrically, the fibers of $\pi$ are translations of this line, such that for any $z \in \mathbb{R}$, the fiber of $\pi$ above $z$ is the diagonal line intersecting both $(z, 0)$ and $(0, z)$.
\end{proof}

\section*{8. (8/24/23)}

Let $\varphi: \mathbb{R}^\times \rightarrow \mathbb{R}^\times$ be the map sending $x$ to the absolute value of $x$. Prove that $\varphi$ is a homomorphism and find the image of $\varphi$. Describe the kernel and the fibers of $\varphi$.

\begin{proof}
    Let $x, y \in \mathbb{R}^\times$ (so $x \neq 0, y \neq 0$). If both $x$ and $y$ are positive or both are negative, then:
    \begin{equation*}
        \varphi(xy) = |xy| = |x| |y| = \varphi(x) \varphi(y),
    \end{equation*}
    and if $x$ is positive and $y$ is negative, then:
    \begin{equation*}
        \varphi(xy) = |xy| = x(-y) = |x||y| = \varphi(x)\varphi(y),
    \end{equation*}
    so $\varphi$ is a homomorphism.

    The image of $\varphi$ consists of every positive real number. The kernel of $\varphi$ is the set $\{ x \in \mathbb{R}^\times \mid |x| = 1 \}$, that is, $\{ \pm 1 \}$. For a given element $z > 0$, the fiber of $\varphi$ above $z$ is the set $\{ \pm z \}$.
\end{proof}

\section*{9. (8/25/23)}

Define $\varphi: \mathbb{C}^\times \rightarrow \mathbb{R}^\times$ by $\varphi(a + bi) = a^2 + b^2$. Prove that $\varphi$ is a homomorphism and find the image of $\varphi$. Describe the kernel and the fibers of $\varphi$ geometrically (as subsets of the plane).

\begin{proof}
    To show that $\varphi$ is a homomorphism, let $z_1 = a_1 + b_1 i, z_2 = a_2 + b_2 i \in \mathbb{C}^\times$. We calculate:
    \begin{flalign*}
        \varphi(z_1 z_2) &= \varphi((a_1 + b_1 i)(a_2 + b_2 i)) \\ &= \varphi((a_1 a_2 - b_1 b_2) + (a_1 b_2 + a_2 b_1)i) \\ &= (a_1 a_2 - b_1 b_2)^2 + (a_1 b_2 + a_2 b_1)^2 \\ &= a_1^2 a_2^2 - 2a_1 a_2 b_1 b_2 + b_1^2 b_2^2 + a_1^2 b_2^2 + 2a_1 a_2 b_1 b_2 + a_2^2 b_1^2 \\ &= a_1^2 a_2^2 + b_1^2 b_2^2 + a_1^2 b_2^2 + a_2^2 b_1^2, \text{ and } \\
        \varphi(z_1)\varphi(z_2) &= \varphi(a_1 + b_1 i)\varphi(a_2 + b_2 i) = (a_1^2 + b_1^2)(a_2^2 + b_2^2) \\ &= a_1^2 a_2^2 + b_1^2 b_2^2 + a_1^2 b_2^2 + a_2^2 b_1^2,
    \end{flalign*}
    which proves that $\varphi$ is a homomorphism.

    The image of a complex number $a + bi$ under $\varphi$ is $a^2 + b^2$, which is always non-negative because it is the sum of two non-negative numbers. Since both $\mathbb{C}^\times$ and $\mathbb{R}^\times$ exclude 0, the image of $\varphi$ is therefore all positive real numbers.

    The kernel of $\varphi$ are those complex numbers whose image under $\varphi$ is 1. Geometrically, $\varphi$ is a map from a point in the complex plane to its length, or distance from zero. Therefore the kernel of $\varphi$ is the unit circle in the complex plane. The fibers of a given positive real number $x$ is the circle of radius $x$ centered at the origin in the complex plane.
\end{proof}

\end{document}