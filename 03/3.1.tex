\documentclass{article}

\title{Dummit \& Foote Ch. 3.1: Quotient Groups and Homomorphisms}
\author{Scott Donaldson}
\date{Aug. - Sep. 2023}
\usepackage{amsmath, amsthm, amsfonts, enumitem, tabu}

% 3.4.11 has a matrix multiplication that overflows but looks OK, so override the warning here
\hfuzz=1pt

\begin{document}

\maketitle

Let $G$ and $H$ be groups.

\section*{1. (9/1/23)}

Let $\varphi: G \rightarrow H$ be a homomorphism and let $E \leq H$. Prove that $\varphi^{-1}(E) \leq G$ (i.e., the preimage or pullback of a subgroup under a homomorphism is a subgroup). If $E \unlhd H$ prove that $\varphi^{-1}(E) \unlhd G$. Deduce that ker $\varphi \unlhd G$. 

\begin{proof}
    Let $x, y \in \varphi^{-1}(E) \subseteq G$. Suppose that $\varphi(x) = a, \varphi(y) = b, a, b \in E \leq H$. Since $\varphi$ is a homomorphism, we have $\varphi(y^{-1}) = \varphi(y)^{-1} = b^{-1}$. Then:
    \begin{equation*}
        \varphi(xy^{-1}) = \varphi(x)\varphi(y^{-1}) = \varphi(x)\varphi(y)^{-1} = ab^{-1} \in E,
    \end{equation*}
    which implies that $xy^{-1} \in \varphi^{-1}(E)$. It follows that, by the subgroup criterion, $\varphi^{-1}(E) \leq G$.

    Next, let $E \unlhd H$ (to show that $\varphi^{-1}(E) \unlhd G$). Again let $x \in \varphi^{-1}(E) \leq G$ and suppose $\varphi(x) = a$. Now for some $g \in G$ (not necessarily in $\varphi^{-1}(E)$), consider $\varphi(gxg^{-1})$. Suppose also that $\varphi(g) = h \in H$. Because $E$ is normal in $H$ and $a \in E$, we have $hah^{-1} \in E$. Then:
    \begin{equation*}
        \varphi(gxg^{-1}) = \varphi(g)\varphi(x)\varphi(g^{-1}) = \varphi(g)\varphi(x)\varphi(g)^{-1} = hah^{-1} \in E,
    \end{equation*}
    which implies that $gxg^{-1} \in \varphi^{-1}(E)$. Since the conjugate of any element of $\varphi^{-1}(E)$ by any other element of $G$ lies in $\varphi^{-1}(E)$, we therefore conclude that $\varphi^{-1}(E) \unlhd G$.

    Finally, we note that ker $\varphi$ = $\{ g \in G \mid \varphi(g) = 1_H \}$. Since the trivial subgroup consisting of the identity of $H$ is normal (the conjugate of $1_H$ by any element of $H$ is $1_H$), we therefore have $\varphi^{-1}(\{ 1_H \}) = \text{ker } \varphi \unlhd G$.
\end{proof}

\section*{2. (8/23/23)}

Let $\varphi: G \rightarrow H$ be a homomorphism of groups with kernel $K$ and let $a, b \in \varphi(G)$. Let $X \in G/K$ be the fiber above $a$ and $Y$ be the fiber above $b$, i.e., $X = \varphi^{-1}(a), Y = \varphi^{-1}(b)$. Fix an element $x \in X$ (so $\varphi(x) = a$). Prove that if $XY = Z$ in the quotient group $G/K$ and $z$ is any member of $Z$, then there is some $y \in Y$ such that $xy = z$.

\begin{proof}
    We know that, for any $x \in X, y \in Y$, $\varphi(x) = a$ and $\varphi(y) = b$. Since $\varphi$ is a homomorphism, it follows that $\varphi(xy) = \varphi(x)\varphi(y) = ab$, and so the image of any element of $XY = Z$ under $\varphi$ is $ab \in H$.
    
    Next, consider the element $x^{-1}z \in G$, as well as its image under $\varphi$. Since $\varphi$ is a homomorphism, we have $\varphi(x^{-1}) = \varphi(x)^{-1}$. So $\varphi(x^{-1}z) = \varphi(x^{-1})\varphi(z) = \varphi(x)^{-1}\varphi(z) = a^{-1}ab = b$. The set $Y$ consists of all elements of $G$ whose image under $\varphi$ is $b$, and so we must have $x^{-1}z \in Y$.

    Now if we fix some element $x \in X$, then for any $z \in Z$, we have $x^{-1}z \in Y$ such that its product with $x$ is $z$: $x x^{-1}z = z$.
\end{proof}

\section*{3. (8/23/23)}

Let $A$ be an abelian group and let $B$ be a subgroup of $A$. Prove that $A/B$ is abelian. Give an example of a non-abelian group $G$ containing a proper normal subgroup $N$ such that $G/N$ is abelian.

\begin{proof}
    Because $A$ is abelian, all subgroups of $A$ are normal, so $A/B$ is well-defined for every $B \leq A$.

    Let $C, D \in A/B$ with $C = cB$ and $D = dB$ for some $c, d \in A$. Then:
    \begin{equation*}
        CD = (cB)(dB) = (cd)B = (dc)B = (dB)(cB) = DC,
    \end{equation*}
    which implies that $A/B$ is abelian.

    Now if we let $G$ be the dihedral group $D_8$, then $G$ is non-abelian. Let $N$ be the cyclic subgroup generated by $r: \{ 1, r, r^2, r^3 \}$. The only coset of $N$ is $sN$; together these two sets cover $G$. Then $G/N = \{ N, sN \}$. There is only one group of order 2 up to isomorphism, and it is abelian. Thus $G/N$ is abelian.
\end{proof}

\section*{4. (8/23/23)}

Prove that in the quotient group $G/N$, $(gN)^\alpha = (g^\alpha)N$ for all $\alpha \in \mathbb{Z}$.

\begin{proof}
    We start by induction: In the base case, $\alpha = 1$, we have $(gN)^1 = gN = (g^1)N$. Next, suppose that for some $\alpha > 1$, we have $(gN)^\alpha = (g^\alpha)N$. Then:
    \begin{equation*}
        (gN)^{\alpha + 1} = (gN)^\alpha gN = g^\alpha N \cdot gN = (g^{\alpha + 1})N,
    \end{equation*}
    as desired. We have now proven that $(gN)^\alpha = (g^\alpha)N$ for $\alpha \geq 1$.

    Next, consider $(gN)^\alpha (gN)^{-\alpha}$, where $\alpha \geq 1$. In the quotient group $G/N$, for any subset $X \in G/N$, we must have $X^\alpha X^{-\alpha} = N$ (the identity of $G/N$), so $(gN)^\alpha (gN)^{-\alpha} = N$. From above, $(gN)^\alpha = (g^\alpha)N$, so $(g^\alpha)N \cdot (gN)^{-\alpha} = N$. Also, from the operation on left cosets, we know that $N = (g^\alpha)N \cdot (g^{-\alpha})N$. Since both $(g^\alpha)N \cdot (gN)^{-\alpha} = N$ and $(g^\alpha)N \cdot (g^{-\alpha})N = N$, we must have $(gN)^{-\alpha} = (g^{-\alpha})N$. We have now proven for all nonzero integers.

    Finally, we note that $(gN)^0 = N$ (the identity of $G/N$) and that $(g^0)N = eN = N$, so $(gN)^0 = (g^0)N$. This concludes the proof that $(gN)^\alpha = (g^\alpha)N$ for all $\alpha \in \mathbb{Z}$.
\end{proof}

\section*{5. (8/23/23)}

Use the preceding exercise to prove that the order of the element $gN$ in $G/N$ is $n$, where $n$ is the smallest positive integer such that $g^n \in N$ (and $gN$ has infinite order if no such positive integer exists). Give an example to show that the order of $gN$ in $G/N$ may be strictly smaller than the order of $g$ in $G$.

\begin{proof}
    Let $gN \in G/N$, and let $n$ be the smallest positive integer such that $g^n \in N$. Suppose that $g^n = h \in N$.

    From Exercise 4., $(gN)^n = (g^n)N = hN = N$ (because $h \in N$), so the order of $gN$ must divide $n$.

    Suppose (toward contradiction) that the order of $gN$ is $k$, where $k < n$. Then $(gN)^k = (g^k)N = N$, which implies that $g^k$ lies in $N$, contradicting our assumption that $n$ is the smallest such positive integer. Therefore the order of $gN$ is $n$.

    If there is no positive integer $n$ such that $g^n \in N$, then for all $k \in \mathbb{Z}^+$, we have $(gN)^k = (g^k)N \neq N$, so $gN$ has infinite order.

    As an example where $|gN| < |g|$, let $G = Z_9 = \langle x \rangle$ and let $N = \langle x^3 \rangle$. Because all cyclic groups are abelian, $N$ is normal in $G$, and so $G/N$ is well-defined. The quotient group $G/N$ contains three elements: $N, xN$, and $(x^2)N$. The element $xN \in G/N$ has order 3: $(xN)^3 = (x^3)N = N$ (because $x^3 \in N$). However, the generating element $x \in G$ has order 9.
\end{proof}

\section*{6. (8/24/23)}

Define $\varphi: \mathbb{R}^\times \rightarrow \{ \pm 1 \}$ by letting $\varphi(x)$ be $x$ divided by the absolute value of $x$. Describe the fibers of $\varphi$ and prove that $\varphi$ is a homomorphism.

\begin{proof}
    We consider the two cases where $x < 0$ and $x > 0$ (0 is not an element of $\mathbb{R}^\times$). If $x > 0$, then $\varphi(x) = x/|x| = x/x = 1$. If $x < 0$, then $\varphi(x) = x/|x| = x/-x = -1$. Therefore the fiber above $-1$ is every negative real number and the fiber above 1 is every positive real number.

    To show that $\varphi$ is a homomorphism, we let $x, y \in \mathbb{R}^\times$ and again consider the different cases: Where $x$ and $y$ are both positive, where they are both negative, and where one is positive and the other negative.

    If both $x$ and $y$ are positive, then $\varphi(x)\varphi(y) = 1 \cdot 1 = 1$ and $\varphi(xy) = \frac{xy}{|xy|} = \frac{xy}{xy} = 1$, so $\varphi(x)\varphi(y) = \varphi(xy)$.

    If both $x$ and $y$ are negative, then $\varphi(x)\varphi(y) = -1 \cdot -1 = 1$ and $\varphi(xy) = \frac{xy}{|xy|} = \frac{xy}{xy} = 1$, so $\varphi(x)\varphi(y) = \varphi(xy)$.

    Suppose $x$ is positive and $y$ is negative. Then $\varphi(x)\varphi(y) = 1 \cdot -1 = -1$ and $\varphi(xy) = \frac{xy}{|xy|} = \frac{xy}{-xy} = -1$, so $\varphi(x)\varphi(y) = \varphi(xy)$.
    
    Thus, in every case of $x, y \in \mathbb{R}^\times$, we have $\varphi(x)\varphi(y) = \varphi(xy)$, and $\varphi$ is thus a homomorphism.
\end{proof}

\section*{7. (8/24/23)}

Define $\pi: \mathbb{R}^2 \rightarrow \mathbb{R}$ by $\pi((x, y)) = x + y$. Prove that $\pi$ is a surjective homomorphism and the describe the kernel and fibers of $\pi$ geometrically.

\begin{proof}
    First, to show that $\pi$ is surjective, let $z \in \mathbb{R}$. Now $z = z + 0$, so $(z, 0)$ is an element of $\mathbb{R}^2$ such that $\pi((z, 0)) = z + 0 = z$.

    Next, to show that $\pi$ is a homomorphism, let $(x_1, y_1), (x_2, y_2) \in \mathbb{R}^2$. We have $\pi((x_1, y_1) + (x_2, y_2)) = \pi((x_1 + x_2, y_1 + y_2)) = x_1 + x_2 + y_1 + y_2$, and $\pi((x_1, y_1)) + \pi((x_2, y_2)) = x_1 + y_1 + x_2 + y_2$. By the commutativity of addition in $\mathbb{R}$, these are equal to each other, and so $\pi$ is a surjective homomorphism.

    The kernel of $\pi$ consists of all points $(x, y) \in \mathbb{R}^2$ such that $x + y = 0$, that is, the diagonal line running from the upper-left to the bottom-right of the Cartesian plane. Geometrically, the fibers of $\pi$ are translations of this line, such that for any $z \in \mathbb{R}$, the fiber of $\pi$ above $z$ is the diagonal line intersecting both $(z, 0)$ and $(0, z)$.
\end{proof}

\section*{8. (8/24/23)}

Let $\varphi: \mathbb{R}^\times \rightarrow \mathbb{R}^\times$ be the map sending $x$ to the absolute value of $x$. Prove that $\varphi$ is a homomorphism and find the image of $\varphi$. Describe the kernel and the fibers of $\varphi$.

\begin{proof}
    Let $x, y \in \mathbb{R}^\times$ (so $x \neq 0, y \neq 0$). If both $x$ and $y$ are positive or both are negative, then:
    \begin{equation*}
        \varphi(xy) = |xy| = |x| |y| = \varphi(x) \varphi(y),
    \end{equation*}
    and if $x$ is positive and $y$ is negative, then:
    \begin{equation*}
        \varphi(xy) = |xy| = x(-y) = |x||y| = \varphi(x)\varphi(y),
    \end{equation*}
    so $\varphi$ is a homomorphism.

    The image of $\varphi$ consists of every positive real number. The kernel of $\varphi$ is the set $\{ x \in \mathbb{R}^\times \mid |x| = 1 \}$, that is, $\{ \pm 1 \}$. For a given element $z > 0$, the fiber of $\varphi$ above $z$ is the set $\{ \pm z \}$.
\end{proof}

\section*{9. (8/25/23)}

Define $\varphi: \mathbb{C}^\times \rightarrow \mathbb{R}^\times$ by $\varphi(a + bi) = a^2 + b^2$. Prove that $\varphi$ is a homomorphism and find the image of $\varphi$. Describe the kernel and the fibers of $\varphi$ geometrically (as subsets of the plane).

\begin{proof}
    To show that $\varphi$ is a homomorphism, let $z_1 = a_1 + b_1 i, z_2 = a_2 + b_2 i \in \mathbb{C}^\times$. We calculate:
    \begin{flalign*}
        \varphi(z_1 z_2) &= \varphi((a_1 + b_1 i)(a_2 + b_2 i)) \\ &= \varphi((a_1 a_2 - b_1 b_2) + (a_1 b_2 + a_2 b_1)i) \\ &= (a_1 a_2 - b_1 b_2)^2 + (a_1 b_2 + a_2 b_1)^2 \\ &= a_1^2 a_2^2 - 2a_1 a_2 b_1 b_2 + b_1^2 b_2^2 + a_1^2 b_2^2 + 2a_1 a_2 b_1 b_2 + a_2^2 b_1^2 \\ &= a_1^2 a_2^2 + b_1^2 b_2^2 + a_1^2 b_2^2 + a_2^2 b_1^2, \text{ and } \\
        \varphi(z_1)\varphi(z_2) &= \varphi(a_1 + b_1 i)\varphi(a_2 + b_2 i) = (a_1^2 + b_1^2)(a_2^2 + b_2^2) \\ &= a_1^2 a_2^2 + b_1^2 b_2^2 + a_1^2 b_2^2 + a_2^2 b_1^2,
    \end{flalign*}
    which proves that $\varphi$ is a homomorphism.

    The image of a complex number $a + bi$ under $\varphi$ is $a^2 + b^2$, which is always non-negative because it is the sum of two non-negative numbers. Since both $\mathbb{C}^\times$ and $\mathbb{R}^\times$ exclude 0, the image of $\varphi$ is therefore all positive real numbers.

    The kernel of $\varphi$ are those complex numbers whose image under $\varphi$ is 1. Geometrically, $\varphi$ is a map from a point in the complex plane to its length, or distance from zero. Therefore the kernel of $\varphi$ is the unit circle in the complex plane. The fibers of a given positive real number $x$ is the circle of radius $x$ centered at the origin in the complex plane.
\end{proof}

\section*{10. (8/28/23)}

Let $\varphi: \mathbb{Z}/8\mathbb{Z} \rightarrow \mathbb{Z}/4\mathbb{Z}$ by $\varphi(\overline{a}) = \overline{a}$. Show that this is a well-defined, surjective homomorphism and describe its fibers and kernel explicitly (showing that $\varphi$ is well-defined involves the fact that $\overline{a}$ has a different meaning in the domain and range of $\varphi$).

\begin{proof}
    The map $\varphi$ is well-defined because it assigns to each member of $\mathbb{Z}/8\mathbb{Z}$ a single, unique element of $\mathbb{Z}/4\mathbb{Z}$. Let $a \in \{ 0, ... 7 \}$ be equal to $\overline{a}$ mod 8. Then we have $\varphi(\overline{a}) = \varphi(a)$. Further, $\varphi$ assigns each $a \in \{ 0, ... 7 \}$ to $a$ mod 4; that is, it assigns 0 and 4 to 0, 1 and 5 to 1, 2 and 6 to 2, and 3 and 7 to 3. This also shows that $\varphi$ is surjective, since each $\overline{a} \cong \mathbb{Z}/4\mathbb{Z}$ (represented by $a = \overline{a}$ mod 4) has a preimage in $\mathbb{Z}/8\mathbb{Z}$.

    The kernel of $\varphi$ is $\{ 0, 4 \} \leq \mathbb{Z}/8\mathbb{Z}$, and the fiber of any $a \in \mathbb{Z}/4\mathbb{Z}$ is the tuple $\{ a, a + 4 \}$.
\end{proof}

\section*{11. (8/28/23)}

Let $F$ be a field and let $G = \{ \begin{pmatrix}a & b \\ 0 & c\end{pmatrix} \mid a, b, c \in F, ac \neq 0 \} \leq GL_2(F)$.

\begin{enumerate}[label=(\alph*), itemsep=0em]
    \item Prove that the map $\varphi: \begin{pmatrix}a & b \\ 0 & c\end{pmatrix} \mapsto a$ is a surjective homomorphism from $G$ onto $F^\times$ (recall that $F^\times$ is the multiplicative group of nonzero elements in $F$). Describe the fibers and kernel of $\varphi$.
          \begin{proof}
            To show that $\varphi$ is surjective, let $a \in F^\times$ (so $a \neq 0$). Then we have $\varphi(\begin{pmatrix}a & 0 \\ 0 & 1\end{pmatrix}) = a$, so $\varphi$ is onto.

            Next, to show that it is a homomorphism, we note that:
            \begin{equation*}
                \varphi(\begin{pmatrix}a & b \\ 0 & c\end{pmatrix}\begin{pmatrix}d & e \\ 0 & f\end{pmatrix}) = \varphi(\begin{pmatrix}ad & ae + bf \\ 0 & cf\end{pmatrix}) = ad = \varphi(\begin{pmatrix}a & b \\ 0 & c\end{pmatrix})\varphi(\begin{pmatrix}d & e \\ 0 & f\end{pmatrix}),
            \end{equation*}
            so $\varphi$ is also a homomorphism.

            The kernel of $\varphi$ is $\{ \begin{pmatrix}1 & b \\ 0 & c\end{pmatrix} \mid b, c \in F, c \neq 0 \}$, and the fiber of $\varphi$ over a given element $a \in F^\times$ is $\{ \begin{pmatrix} a & b \\ 0 & c\end{pmatrix} \mid b, c \in F, c \neq 0 \}$.
          \end{proof}
    \item Prove that the map $\psi: \begin{pmatrix}a & b \\ 0 & c\end{pmatrix} \mapsto (a, c)$ is a surjective homomorphism from $G$ onto $F^\times \times F^\times$. Describe the fibers and kernel of $\psi$.
          \begin{proof}
            To show that $\psi$ is surjective, let $(a, c) \in F^\times \times F^\times$ (so $a, c \neq 0$). Then we have $\psi(\begin{pmatrix}a & 0 \\ 0 & c\end{pmatrix}) = (a, c)$, so $\psi$ is onto.

            Next, to show that it is a homomorphism, we note that:
            \begin{align*}
                \psi(\begin{pmatrix}a & b \\ 0 & c\end{pmatrix}\begin{pmatrix}d & e \\ 0 & f\end{pmatrix}) &= \psi(\begin{pmatrix}ad & ae + bf \\ 0 & cf\end{pmatrix}) = (ad, cf) \\ &= (a, c)(d, f) = \psi(\begin{pmatrix}a & b \\ 0 & c\end{pmatrix})\psi(\begin{pmatrix}d & e \\ 0 & f\end{pmatrix}),
            \end{align*}
            so $\psi$ is also a homomorphism.

            The kernel of $\psi$ is the preimage of $(1, 1)$, that is, $\{ \begin{pmatrix}1 & b \\ 0 & 1\end{pmatrix} \mid b \in F \}$, and the fiber of $\psi$ over a given element $(a, c) \in F^\times \times F^\times$ is $\{ \begin{pmatrix}a & b \\ 0 & c\end{pmatrix} \mid b \in F \}$.
          \end{proof}
    \item Let $H = \{ \begin{pmatrix}1 & b \\ 0 & 1\end{pmatrix} \mid b \in F \}$. Prove that $H$ is isomorphic to the additive group $F$.
          \begin{proof}
            As usual, to show that $H$ is isomorphic to the additive group $F$, we must show that there exists a bijective homomorphism $\varphi: H \rightarrow F$. Define $\varphi$ by $\varphi(\begin{pmatrix}1 & b \\ 0 & 1\end{pmatrix}) = b$. We will show that it is an isomorphism.

            First, $\varphi$ is injective: Suppose that $\varphi(\begin{pmatrix}1 & a \\ 0 & 1\end{pmatrix}) = \varphi(\begin{pmatrix}1 & b \\ 0 & 1\end{pmatrix}) = c$. Then we have $a = c$ and $b = c$, so the two matrices are the same, and $\varphi$ is injective.

            Next, $\varphi$ is surjective: Let $b \in F$. Then we have $\varphi(\begin{pmatrix}1 & b \\ 0 & 1\end{pmatrix}) = b$.

            Finally, $\varphi$ is a homomorphism:
            \begin{equation*}
                \varphi(\begin{pmatrix}1 & a \\ 0 & 1\end{pmatrix}\begin{pmatrix}1 & b \\ 0 & 1\end{pmatrix}) = \varphi(\begin{pmatrix}1 & a + b \\ 0 & 1\end{pmatrix}) = a + b = \varphi(\begin{pmatrix}1 & a \\ 0 & 1\end{pmatrix}) + \varphi(\begin{pmatrix}1 & b \\ 0 & 1\end{pmatrix}).
            \end{equation*}
          \end{proof}
\end{enumerate}

\section*{12. (8/30/23)}

Let $G$ be the additive group of real numbers, let $H$ be the multiplicative group of complex numbers of absolute value 1 (the unit circle $S^1$ in the complex plane) and let $\varphi: G \rightarrow H$ be the homomorphism $\varphi: r \mapsto e^{2 \pi i r}$. Draw the points on the real line which lie in the kernel of $\varphi$. Describe similarly the elements in the fibers of $\varphi$ above the points $-1$, $i$, and $e^{4 \pi i / 3}$ of $H$.

\begin{proof}
    The kernel of $\varphi$ is the set $\{ r \in \mathbb{R} \mid e^{2 \pi i r} = 1 \}$. Recall that $e^{2 \pi i r} = \cos{2 \pi r} + i \sin{2 \pi r}$, so the values of $r$ for which $e^{2 \pi i r} = 1$ are those where $\cos{2 \pi r} = 1$, that is, all of the integers.

    We similarly obtain the fiber of $\varphi$ above $-1$ by considering when $\cos{2 \pi r} = -1$, which occurs when $r = 1/2, 3/2, 5/2, ...$, that is, $r \in \{ n + \frac{1}{2} \mid n \in \mathbb{Z} \}$. For the fiber above $i$, we must have $\sin{2 \pi r} = 1$, which occurs when $r = 1/4, 5/4, 9/4, ...$, that is, $r \in \{ n + \frac{1}{4} \mid n \in \mathbb{Z} \}$. Finally, we have $4 \pi / 3 = \frac{2}{3} \cdot 2 \pi$, so the fiber above $e^{4 \pi i / 3}$ is $\{ n + \frac{2}{3} \mid n \in \mathbb{Z} \}$.

    We can also write these as cosets of $\mathbb{Z}$, so the fibers are $\frac{1}{2} + \mathbb{Z}$, $\frac{1}{4} + \mathbb{Z}$, and $\frac{2}{3} + \mathbb{Z}$, respectively.
\end{proof}

\section*{13. (8/31/23)}

Repeat the preceding exercise with the map $\varphi$ replaced by the map $\varphi: r \mapsto e^{4 \pi i r}$.

\begin{proof}
    In this case, the kernel of $\varphi$ consists of values of $r$ for which $e^{4 \pi i r} = 1 \Rightarrow \cos{4 \pi r} = 1$. The period is now halved, so this occurs when $r \in \{ 1/2, 1, 3/2, ... \}$; the kernel is $\{ \frac{n}{2} \mid n \in \mathbb{Z} \}$.

    The fiber of $\varphi$ above $-1$ has $\cos{4 \pi r} = -1$, when $r = 1/4, 3/4, 5/4, ...$, that is, $r \in \{ \frac{1}{4} + \frac{n}{2} \mid n \in \mathbb{Z} \}$. Above $i$, we have $\sin{4 \pi r} = 1$, so $r \in \{ \frac{1}{8}, \frac{5}{8}, ... \}$, and the fiber is $\{ \frac{1}{8} + \frac{n}{2} \mid n \in \mathbb{Z} \}$. Finally, above $4 \pi / 3$, the fiber is $\{ \frac{1}{3} + \frac{n}{2} \mid n \in \mathbb{Z} \}$.

    If we denote the kernel in this exercise as $\frac{1}{2}\mathbb{Z}$, then as cosets, the fibers are $\frac{1}{4} + \frac{1}{2}\mathbb{Z}$, $\frac{1}{8} + \frac{1}{2}\mathbb{Z}$, and $\frac{1}{3} + \frac{1}{2}\mathbb{Z}$, respectively.
\end{proof}

\section*{14. (8/31/23)}

Consider the additive quotient group $\mathbb{Q}/\mathbb{Z}$.

\begin{enumerate}[label=(\alph*), itemsep=0em]
    \item Show that every coset of $\mathbb{Z}$ in $\mathbb{Q}$ contains exactly one representative $q \in \mathbb{Q}$ in the range $0 \leq q < 1$.
          \begin{proof}
            The rational numbers under addition constitutes an abelian group, so $\mathbb{Z}$ is a normal subgroup of $\mathbb{Q}$, and $\mathbb{Q}/\mathbb{Z}$ is therefore well-defined. The elements of the quotient group $\mathbb{Q}/\mathbb{Z}$ are cosets of $\mathbb{Z}$ in $\mathbb{Q}$, for example, $\mathbb{Z}$ itself (the identity), as well as $\frac{1}{2} + \mathbb{Z}$, $\frac{7}{4} + \mathbb{Z}$, and so on.

            Let $q + \mathbb{Z}$ be a coset of $\mathbb{Z}$ (for arbitrary $q \in \mathbb{Q}$). If $q > 1$, then let $n \in \mathbb{Z}$ be the largest integer such that $q - n \geq 0$ (such an integer exists by the well-ordering property). Then $q - n$ is the unique representative for $q + \mathbb{Z}$ in the range $[0, 1)$, since $q - n - 1 < 0$ and $q - n + 1 > 1$. Similarly, if $q < 0$, there exists a unique $n$ such that $0 \leq q + n < 1$. Finally, if $0 \leq q < 1$, then $q$ itself is the unique representative for $q + \mathbb{Z}$ lying between 0 (inclusive) and 1 (exclusive).
          \end{proof}
    \item Show that every element of $\mathbb{Q}/\mathbb{Z}$ has finite order but that there are elements of arbitrarily large order.
          \begin{proof}
            Let $\frac{a}{b} + \mathbb{Z} \in \mathbb{Q}/\mathbb{Z}$ (with $0 \leq \frac{a}{b} < 1$, as above, and suppose that $\frac{a}{b}$ is in lowest terms). Then we have:
            \begin{equation*}
                \underbrace{(\frac{a}{b} + \mathbb{Z}) + ... + (\frac{a}{b} + \mathbb{Z})}_{b \text{ times}} = \underbrace{(\frac{a}{b} + ... + \frac{a}{b})}_{b \text{ times}} + \mathbb{Z} = a + \mathbb{Z} = \mathbb{Z},
            \end{equation*}
            so the order of $\frac{a}{b} + \mathbb{Z} \in \mathbb{Q}/\mathbb{Z}$ is at most $b$, and it therefore has finite order.

            However, given a coset $\frac{1}{b} + \mathbb{Z}$ of order $b$, there always exists an element of higher order, for example $\frac{1}{b + 1} + \mathbb{Z}$ and $\frac{1}{2b} + \mathbb{Z}$, which have order $b + 1$ and $2b$, respectively.
          \end{proof}
    \item Show that $\mathbb{Q}/\mathbb{Z}$ is the torsion subgroup of $\mathbb{R}/\mathbb{Z}$.
          \begin{proof}
            Recall that the torsion subgroup of $\mathbb{R}/\mathbb{Z}$ is the set of elements of $\mathbb{R}/\mathbb{Z}$ of finite order (by Chapter 2.1, Exercise 6., this set is a subgroup when the parent group is abelian).

            First, let $q + \mathbb{Z} \in \mathbb{Q}/\mathbb{Z}$. Since rational numbers are also real numbers, $q + \mathbb{Z}$ also lies in $\mathbb{R}/\mathbb{Z}$. From 14.b), it has finite order. Therefore it is an element of the torsion subgroup of $\mathbb{R}/\mathbb{Z}$.

            Next, let $x + \mathbb{Z}$ be an element of the torsion subgroup of $\mathbb{R}/\mathbb{Z}$. Suppose that $|x + \mathbb{Z}| = n < \infty$. Then we have:
            \begin{equation*}
                \underbrace{(x + \mathbb{Z}) + ... + (x + \mathbb{Z})}_{n \text{ times}} = \underbrace{(x + ... + x)}_{n \text{ times}} + \mathbb{Z} = nx + \mathbb{Z} = \mathbb{Z},
            \end{equation*}
            which implies that $nx$ is an integer. Suppose that $nx = m \in \mathbb{Z}$. Then $x = m/n$, and so we have $x \in \mathbb{Q}$, which implies that $x + \mathbb{Z} \in \mathbb{Q}/\mathbb{Z}$.

            Therefore, because inclusion in one implies inclusion in the other and vice-versa, these groups are equal.
          \end{proof}
    \item Prove that $\mathbb{Q}/\mathbb{Z}$ is isomorphic to the multiplicative group of roots of unity in $\mathbb{C}^\times$.
          \begin{proof}
            Let $\varphi: \mathbb{Q}/\mathbb{Z} \rightarrow \mathbb{C}^\times$ be defined by $\varphi(r + \mathbb{Z}) = e^{2 \pi i r}$, where $0 \leq r < 1$. We will show that $\varphi$ is a bijective homomorphism, and that the groups are thus isomorphic to each other.

            First, to show that $\varphi$ is a homomorphism, note that:
            \begin{align*}
                \varphi((q + \mathbb{Z}) + (r + \mathbb{Z})) = \varphi((q + r) + \mathbb{Z}) &= e^{2 \pi i (q + r)}, \text{ and } \\
                \varphi(q + \mathbb{Z}) \varphi(r + \mathbb{Z}) = e^{2 \pi i q} e^{2 \pi i r} = e^{2 \pi i q + 2 \pi i r} &= e^{2 \pi i (q + r)},
            \end{align*}
            as desired.

            Next, $\varphi$ is one-to-one: Suppose $e^{2 \pi i r} = \varphi(r + \mathbb{Z}) = \varphi(q + \mathbb{Z})$ for some $r, q \in [0,1)$. In fact, there are many possible rational numbers fulfilling this if we open the range to all of $\mathbb{Q}$; however, because the period of $e^{2 \pi i r}$ is 1, there is only one unique value in the range $[0, 1)$, so we must have $r = q$. Therefore $\varphi$ is injective.

            Finally, $\varphi$ is surjective: Let $z$ be a root of unity with order $n$. Then $z$ can be expressed as $e^{2 \pi i t / n}$ for some $t \in \{ 0, 1, ..., n - 1 \}$. By definition of $\varphi$, the rational number $t/n \in [0, 1)$ has $\varphi(t/n) = e^{2 \pi i t / n} = z$. Thus $\varphi$ is a bijective homomorphism, and so $\mathbb{Q}/\mathbb{Z}$ is isomorphic to the roots of unity in $\mathbb{C}^\times$.
          \end{proof}
\end{enumerate}

\section*{15. (9/1/23)}

Prove that the quotient of a divisible abelian group by any proper subgroup is also divisible. Deduce that $\mathbb{Q}/\mathbb{Z}$ is divisible.

\begin{proof}
    Let $A$ be a divisible abelian group and let $B$ be a proper subgroup of $A$. Since $A$ is abelian, all of its subgroups are normal, so the quotient group $A/B$ is well-defined.

    Let $aB \in A/B$ and let $k > 0$. Since $A$ is divisible, there exists an $x \in A$ such that $x^k = a$. Then we have $aB = (x^k)B = (xB)^k$ for $xB \in A/B$, so $aB$ has a $k$-th root in $A/B$. Therefore $A/B$ is divisible.

    Note that the rational numbers under addition form a divisible abelian group (from Ch. 2.4, Exercise 19.) and the integers are a proper subgroup of the rational numbers. It follows that the quotient group $\mathbb{Q}/\mathbb{Z}$ is divisible.
\end{proof}

\section*{16. (9/5/23)}

Let $G$ be a group, let $N$ be a normal subgroup of $G$, and let $\overline{G} = G/N$. Prove that if $G = \langle x, y \rangle$ then $\overline{G} = \langle \overline{x}, \overline{y} \rangle$. Prove more generally that if $G = \langle S \rangle$ for any subset $S$ of $G$ then $\overline{G} = \langle \overline{S} \rangle$.

\begin{proof}
    If $G = \langle x, y \rangle$, then we can write any element $g$ as a finite product of $x$ and $y$, say $g = x^{a_1} y^{b_1} ... x^{a_n} y^{b_n}$. It follows that, for $\overline{g} \in \overline{G}$, we have:
    \begin{multline*}
        \overline{g} = gN = (x^{a_1} y^{b_1} ... x^{a_n} y^{b_n})N = (x^{a_1})N (y^{b_1})N ... (x^{a_n})N (y^{b_n})N = \\ (xN)^{a_1} (yN)^{b_1} ... (xN)^{a_n} (yN)^{b_n} = \overline{x}^{a_1} \overline{y}^{b_1} ... \overline{x}^{a_n} \overline{y}^{b_n},
    \end{multline*}
    that is, we can write $\overline{g}$ as a finite product of $\overline{x}, \overline{y} \in \overline{G}$, and so $\overline{G} = \langle \overline{x}, \overline{y} \rangle$.

    More generally, if $G = \langle S \rangle$, then any element $g$ can be written as a finite product of elements of $S$, say $g = (s_1^{a_{11}}...s_n^{a_{n1}})(s_1^{a_{12}}...s_n^{a_{n2}})...(s_1^{a_{1k}}...s_n^{a_{nk}})$. Then we have:
    \begin{equation*}
        \overline{g} = gN = \Bigl( \prod_{j = 1}^{k} \bigl(\prod_{i = 1}^{n} s_i^{a_{ij}}\bigr) \Bigr) N = \prod_{j = 1}^{k} \prod_{i = 1}^{n} (s_i^{a_{ij}} N) = \prod_{j = 1}^{k} \prod_{i = 1}^{n} (s_i N)^{a_{ij}} = \prod_{j = 1}^{k} \prod_{i = 1}^{n} \overline{s_i}^{a_{ij}},
    \end{equation*}
    and so similar to above, this means that any element $\overline{g} = gN \in G/N$ can be written as a finite product of $\overline{s_1}, \overline{s_2}, ..., \overline{s_n}$, and therefore $\overline{G} = \langle \overline{S} \rangle$.
\end{proof}

\section*{17. (9/6/23)}

Let $G$ be the dihedral group of order 16: $G = \langle r, s \mid r^8 = s^2 = 1, rs = sr^{-1} \rangle$ and let $\overline{G} = G/\langle r^4 \rangle$ be the quotient of $G$ by the subgroup generated by $r^4$ (this subgroup is the center of $G$, hence is normal).

\begin{enumerate}[label=(\alph*), itemsep=0em]
    \item Show that the order of $\overline{G}$ is 8.
          
          The quotient group $\overline{G}$ consists of cosets of the cyclic subgroup of $G$ generated by $r^4$, that is, cosets of $\{ 1, r^4 \}$. For example, the coset $s \langle r^4 \rangle$ is $\{ s, sr^4 \}$. Notice that the coset for $sr^4$ is the same as for $s$, and because $\langle r^4 \rangle$ consists of two elements, for each element $x \in G$, there is another element whose coset is the same (namely $xr^4$). Thus the order of $\overline{G}$ is $16/2 = 8$.
    \item Exhibit each element of $\overline{G}$ in the form $\overline{s}^a \overline{r}^b$, for some integers $a$ and $b$.

          The elements of $\overline{G}$ are: 
          \begin{align*}
            \overline{1} &= \{ 1, r^4 \} & \overline{s} &= \{ s, sr^4 \} \\ 
            \overline{r} &= \{ r, r^5 \} & \overline{s}\cdot\overline{r} &= \{ sr, sr^5 \} \\
            \overline{r}^2 &= \{ r^2, r^6 \} & \overline{s}\cdot\overline{r}^2 &= \{ sr^2, sr^6 \} \\
            \overline{r}^3 &= \{ r^3, r^7 \} & \overline{s}\cdot\overline{r}^3 &= \{ sr^3, sr^7 \}
          \end{align*}
    \item Find the order of each of the elements of $\overline{G}$ exhibited in (b).

          The orders of the elements of $\overline{G}$ are: $\overline{1}: 1, \overline{r}: 4, \overline{r}^2: 2, \overline{r}^3: 4, \overline{s}: 2, \\ \overline{s}\cdot\overline{r}: 2, \overline{s}\cdot\overline{r}^2: 2, \overline{s}\cdot\overline{r}^3: 2$.
    \item Write each of the following elements of $\overline{G}$ in the form $\overline{s}^a \overline{r}^b$, for some integers $a$ and $b$ as in (b):
        \begin{itemize}[itemsep=0em]
            \item $\overline{rs} = \overline{sr^7} = \overline{s}\cdot\overline{r}^3$
            \item $\overline{sr^{-2}s} = \overline{sr^6 s} = \overline{ssr^2} = \overline{r}^2$
            \item $\overline{s^{-1}r^{-1}sr} = \overline{sr^7sr} = \overline{ssrr} = \overline{r}^2$
        \end{itemize}
    \item Prove that $\overline{H} = \langle \overline{s}, \overline{r}^2 \rangle$ is a normal subgroup of $\overline{G}$ and $\overline{H}$ is isomorphic to the Klein 4-group. Describe the isomorphism type of the complete preimage of $\overline{H}$ in $G$.
          \begin{proof}
            There is a clear isomorphism between $\overline{G}$ and $D_8$ given by $\overline{x} \in \overline{G} \mapsto x \in D_8$. Because of this, we know that the elements $\overline{s}$ and $\overline{r}$ generate $\overline{G}$. Since we know the generators of both $\overline{G}$ and $\overline{H}$, in order to test for normality, we only have to check that the conjugates of the generators of $\overline{H}$ by the generators of $\overline{G}$ are in $\overline{H}$.

            Now powers of $\overline{s}$ and $\overline{r}$ commute with other powers of $\overline{s}$ and $\overline{r}$, respectively, so we can proceed to:
            \begin{align*}
                &\overline{r}\cdot\overline{s}\cdot\overline{r}^{-1} = \overline{rsr^{-1}} = \overline{rsr^7} = \overline{sr^7r^7} = \overline{sr^{14}} = \overline{sr^{6}} = \overline{s}\cdot\overline{r}^2 \in \overline{H}, \text{ and } \\
                &\overline{s}\cdot\overline{r}^2\cdot\overline{s} = \overline{sr^2 s} = \overline{ssr^6} = \overline{r^6} = \overline{r}^2 \in \overline{H}.
            \end{align*}
            This demonstrates that the conjugates of the generators of $\overline{H}$ by the generators of $\overline{G}$ lie in $\overline{H}$, and so $\overline{H} \unlhd \overline{G}$.

            The elements of $\overline{H}$ are $\overline{1}, \overline{s}, \overline{r}^2, \text{ and } \overline{s}\cdot\overline{r}^2$. Any other product of elements gives an element of $\overline{H}$. All of these elements have order 2, and so from Ch. 1.1, Exercise 36, $\overline{H} \cong V_4$.

            The complete preimage of $\overline{H}$ under the natural projection homomorphism $\pi(g) \mapsto \overline{g} = g \langle r^4 \rangle$ is the set $\{ g \in G \mid \pi(g) \in \overline{H} \}$. The elements of $G$ in the complete preimage of $\overline{H}$ are $1, r^2, r^4, r^6, s, sr^2, sr^4, \text{ and } sr^6$. This set of elements is isomorphic to $D_4$ (given by $s, r^2 \in \pi^{-1}(\overline{H}) \mapsto s, r \in D_4$).
          \end{proof}
    \item Find the center of $\overline{G}$ and describe the isomorphism type of $\overline{H}/Z(\overline{G})$.
          
          The center of $\overline{G}$ consists of the elements of $\overline{G}$ that commute with all other elements of $\overline{G}$. This is the subgroup $\langle \overline{r}^2 \rangle$. Now the quotient group $\overline{H}/Z(\overline{G}) = \langle \overline{s}, \overline{r}^2 \rangle / \langle \overline{r}^2 \rangle$ consists of the cosets of $\langle \overline{r}^2 \rangle$ in $\overline{H}$, that is, the elements $\langle \overline{r}^2 \rangle, \overline{s}\langle \overline{r}^2 \rangle$. We do not have $\overline{r}^2$ as a unique element in $\overline{H}/Z(\overline{G})$, because
          \begin{equation*}
            \overline{r}^2 \langle \overline{r}^2 \rangle = \overline{r}^2 \{ \overline{1}, \overline{r}^2 \} = \{ \overline{r}^2, \overline{r}^4 \} = \{ \overline{1}, \overline{r}^2 \} = \langle \overline{r}^2 \rangle.
          \end{equation*}
          Similarly, $\overline{s}\cdot\overline{r}^2 \notin \overline{H}/Z(\overline{G})$. Therefore it is isomorphic to the cyclic roup $Z_2$.
\end{enumerate}

\section*{18. (9/10/23)}

Let $G$ be the quasidihedral group of order 16: $G = \langle \sigma, \tau \mid \sigma^8 = \tau^2 = 1, \sigma \tau = \tau \sigma^3 \rangle$ and let $\overline{G} = G/\langle \sigma^4 \rangle$ be the quotient of $G$ by the subgroup generated by $\langle \sigma^4 \rangle$ (this subgroup is the center of $G$, hence is normal).

\begin{enumerate}[label=(\alph*), itemsep=0em]
    \item Show that the order of $\overline{G}$ is 8.

          The elements of $\overline{G}$ are the cosets of the subgroup generated by $\sigma^4$. For example, for $\tau \in G$, the element $\overline{\tau} \in \overline{G} = \{ \tau, \tau \sigma^4 \}$. As with 17.a), there are two elements in this set, and the cosets of $\langle \sigma^4 \rangle$ partition $G$. Thus $\overline{G}$ has $16/2 = 8$ elements.
    \item Exhibit each element of $\overline{G}$ in the form $\overline{\tau}^a \overline{\sigma}^b$, for some integers $a$ and $b$.

          The elements of $\overline{G}$ are:
          \begin{align*}
            \overline{1} &= \{ 1, \sigma^4 \} & \overline{\tau} &= \{ \tau, \tau\sigma^4 \} \\ 
            \overline{\sigma} &= \{ \sigma, \sigma^5 \} & \overline{\tau}\cdot\overline{\sigma} &= \{ \tau\sigma, \tau\sigma^5 \} \\
            \overline{\sigma}^2 &= \{ \sigma^2, \sigma^6 \} & \overline{\tau}\cdot\overline{\sigma}^2 &= \{ \tau\sigma^2, \tau\sigma^6 \} \\
            \overline{\sigma}^3 &= \{ \sigma^3, \sigma^7 \} & \overline{\tau}\cdot\overline{\sigma}^3 &= \{ \tau\sigma^3, \tau\sigma^7 \}
          \end{align*}
    \item Find the order of each of the elements of $\overline{G}$ exhibited in (b).

          The orders of the elements of $\overline{G}$ are: $\overline{1}: 1, \overline{\sigma}: 4, \overline{\sigma}^2: 2, \overline{\sigma}^3: 4, \overline{\tau}: 2, \\ \overline{\tau}\cdot\overline{\sigma}: 2, \overline{\tau}\cdot\overline{\sigma}^2: 2, \overline{\tau}\cdot\overline{\sigma}^3: 2$.
    \item Write the following elements of $\overline{G}$ in the form $\overline{\tau}^a \overline{\sigma}^b$, for some integers $a$ and $b$ as in (b):
          \begin{itemize}[itemsep=0em]
            \item $\overline{\sigma \tau} = \overline{\tau \sigma^3} = \overline{\tau} \cdot \overline{\sigma}^3$
            \item $\overline{\tau \sigma^{-2} \tau} = \overline{\tau \sigma^6 \tau} = \overline{\tau \tau \sigma^{18}} = \overline{\sigma^2} = \overline{\sigma}^2$
            \item $\overline{\tau^{-1} \sigma^{-1} \tau \sigma} = \overline{\tau \sigma^7 \tau \sigma} = \overline{\tau \tau \sigma^{21} \sigma} = \overline{\sigma^{22}} = \overline{\sigma^6} = \overline{\sigma}^2$
          \end{itemize}
    \item Prove that $\overline{G} \cong D_8$.
          \begin{proof}
            Let $\varphi: \overline{G} \rightarrow D_8$ be defined by $\varphi(\overline{\sigma}) = r$ and $\varphi(\overline{\tau}) = s$. Now $\overline{\sigma}$ and $\overline{\tau}$ are generators for $\overline{G}$, since (as shown above) every element can be written in the form $\overline{\tau}^a \overline{\sigma}^b$, for some integers $a$ and $b$. Then $\varphi$ is a map from $\overline{G}$ to $D_8$ defined on the generators of $\overline{G}$ to the generators of $D_8$. Since both groups have the same cardinality, in order to show that $\varphi$ is an isomorphism, it only remains to check that the relations of $\overline{G}$ are the same as those in $D_8$.

            In $D_8$, we have $s^2 = r^4 = 1$ and $rs = sr^{-1}$. In part (c) above, we computed the orders of $\overline{\tau}$ and $\overline{\sigma}$, which are 2 and 4, respectively, matching their counterparts in $D_8$. Finally, we have $\overline{\sigma}\cdot\overline{\tau} = \overline{\sigma\tau} = \overline{\tau\sigma^3} = \overline{\tau}\cdot\overline{\sigma^3} = \overline{\tau}\cdot\overline{\sigma}^{-1}$, and so the relations hold. Thus $\overline{G} \cong D_8$.
          \end{proof}
\end{enumerate}

\section*{19. (9/13/23)}

Let $G$ be the modular group of order 16: $G = \langle u, v \mid u^2 = v^8 = 1, vu = uv^5 \rangle$ and let $\overline{G} = G/\langle v^4 \rangle$ be the quotient of $G$ by the subgroup generated by $v^4$ (this subgroup is contained in the center of $G$, hence is normal). 

\begin{enumerate}[label=(\alph*), itemsep=0em]
    \item Show that the order of $\overline{G}$ is 8.
    
          The elements of $\overline{G}$ are the cosets of the subgroup generated by $v^4$. For example, for $u \in G$, the element $\overline{u} \in \overline{G} = \{ u, u v^4 \}$. As with 17.a), there are two elements in this set, and the cosets of $\langle v^4 \rangle$ partition $G$. Thus $\overline{G}$ has $16/2 = 8$ elements.
    \item Exhibit each element of $\overline{G}$ in the form $\overline{u}^a \overline{v}^b$, for some integers $a$ and $b$.

          The elements of $\overline{G}$ are:
          \begin{align*}
            \overline{1} &= \{ 1, v^4 \} & \overline{u} &= \{ u, uv^4 \} \\ 
            \overline{v} &= \{ v, v^5 \} & \overline{u}\cdot\overline{v} &= \{ uv, uv^5 \} \\
            \overline{v}^2 &= \{ v^2, v^6 \} & \overline{u}\cdot\overline{v}^2 &= \{ uv^2, uv^6 \} \\
            \overline{v}^3 &= \{ v^3, v^7 \} & \overline{u}\cdot\overline{v}^3 &= \{ uv^3, uv^7 \}
          \end{align*}
    \item Find the order of each of the elements of $\overline{G}$ exhibited in (b).

          The orders of the elements of $\overline{G}$ are: $\overline{1}: 1, \overline{v}: 4, \overline{v}^2: 2, \overline{v}^3: 4, \overline{u}: 2, \\ \overline{u}\cdot\overline{v}: 4, \overline{u}\cdot\overline{v}^2: 2, \overline{u}\cdot\overline{v}^3: 4$.
    \item Write each of the following elements of $\overline{G}$ in the form $\overline{u}^a \overline{v}^b$, for some integers $a$ and $b$ as in (b):
        \begin{itemize}[itemsep=0em]
            \item $\overline{v u} = \overline{u v^5} = \overline{u} \cdot \overline{v}$
            \item $\overline{u v^{-2} u} = \overline{u v^6 u} = \overline{u u v^{30}} = \overline{v^{30}} = \overline{v^{6}} = \overline{v}^2$
            \item $\overline{u^{-1} v^{-1} u v} = \overline{u v^{7} u v} = \overline{u u v^{35} v} = \overline{v^{36}} = \overline{v^4} = \overline{1}$
        \end{itemize}
    \item Prove that $\overline{G}$ is abelian and is isomorphic to $Z_2 \times Z_4$.
        \begin{proof}
            From part (d) above, we deduced that $\overline{v u} = \overline{u v^5} = \overline{u v}$. Since the generators of $\overline{G}$ commute, $\overline{G}$ is an abelian group.

            For clarity, let us write the elements of $Z_2 \times Z_4$ as $(u^k, v^j)$, with $k \in \{ 0, 1 \}$ and $j \in \{ 0, 1, 2, 3 \}$. Then $(u, 1)$ and $(1, v)$ are generators of $Z_2 \times Z_4$.

            Now let $\varphi: \overline{G} \rightarrow Z_2 \times Z_4$ be defined on generators $\overline{u}$ and $\overline{v}$ by $\varphi(\overline{u}) = (u, 1)$ and $\varphi(\overline{v}) = (1, v)$. As above, since $\varphi$ is a map from $\overline{G}$ to $Z_2 \times Z_4$, two groups of equal order, and $\varphi$ is defined on and to the generators of each, respectively, we only have to check that the relations hold.

            In $\overline{G}$, we have $\overline{u}^2 = 1$, and in $Z_2 \times Z_4$, we have $\varphi(\overline{u})^2 = (u, 1)^2 = (u^2, 1) = (1, 1)$, the identity of $Z_2 \times Z_4$. Also, we have $\overline{v}^4 = 1$ and $\varphi(\overline{v})^4 = (1, v)^4 = (1, v^4) = (1, 1)$. Since $\overline{G}$ and $Z_2 \times Z_4$ are both abelian, there are no other relations we need to check. We conclude that $\varphi$ is an isomorphism, and that the two groups are isomorphic.
        \end{proof}
\end{enumerate}

\section*{20. (9/14/23)}

Let $G = \mathbb{Z}/24\mathbb{Z}$ and let $\widetilde{G} = G/\langle \overline{12} \rangle$, where for each integer $a$ we simplify notation by writing $\widetilde{\overline{a}}$ as $\widetilde{a}$.

\begin{enumerate}[label=(\alph*), itemsep=0em]
    \item Show that $\widetilde{G} = \{ \widetilde{0}, \widetilde{1}, ..., \widetilde{11} \}$.

          Now $\widetilde{G}$ consists of the cosets of $\langle \overline{12} \rangle = \{ 0, 12 \}$ in $\mathbb{Z}/24\mathbb{Z}$, for example, $\widetilde{4} = 4 + \{ 0, 12 \} = \{ 4, 16 \}$ and $\widetilde{21} = 21 + \{ 0, 12 \} = \{ 21, 33 \} = \{ 9, 21 \} = \widetilde{9}$. For each $n \in \{ 0, ..., 11 \}$, the element $n + 12 \in \mathbb{Z}/24\mathbb{Z}$ has the same coset as $n$, since $n + 12 \cong n$ (mod 12). Thus the elements of $\widetilde{G}$ are:
          \begin{align*}
            \widetilde{0} &= \{ 0, 12 \} & \widetilde{4} &= \{ 4, 16 \} & \widetilde{8} &= \{ 8, 20 \} \\
            \widetilde{1} &= \{ 1, 13 \} & \widetilde{5} &= \{ 5, 17 \} & \widetilde{9} &= \{ 9, 21 \} \\
            \widetilde{2} &= \{ 2, 14 \} & \widetilde{6} &= \{ 6, 18 \} & \widetilde{10} &= \{ 10, 22 \} \\
            \widetilde{3} &= \{ 3, 15 \} & \widetilde{7} &= \{ 7, 19 \} & \widetilde{11} &= \{ 11, 23 \}
          \end{align*}
    \item Find the order of each element of $\widetilde{G}$.
          \begin{align*}
            \widetilde{0} &: 1 & \widetilde{4} &: 3 & \widetilde{8} &: 3 \\
            \widetilde{1} &: 12 & \widetilde{5} &: 12 & \widetilde{9} &: 4 \\
            \widetilde{2} &: 6 & \widetilde{6} &: 2 & \widetilde{10} &: 6 \\
            \widetilde{3} &: 4 & \widetilde{7} &: 12 & \widetilde{11} &: 12
          \end{align*}
    \item Prove that $\widetilde{G} \cong \mathbb{Z}/12\mathbb{Z}$. (Thus $(\mathbb{Z}/24\mathbb{Z}) / (12\mathbb{Z}/24\mathbb{Z}) \cong \mathbb{Z}/12\mathbb{Z}$, just as if we inverted and cancelled the $24\mathbb{Z}$'s.)
          \begin{proof}
            From Ch. 2.3, Theorem 4, $\mathbb{Z}/n\mathbb{Z}$ is another presentation of the unique cyclic group of order $n$. It suffices, then, to prove that $\widetilde{G}$ is cyclic in order to show that it is isomorphic to $\mathbb{Z}/12\mathbb{Z}$.

            We claim that $\widetilde{1}$ is a generator for $\overline{G}$. For any element $\widetilde{a} \in \widetilde{G}$ ($0 \leq a < 12$), we can write:
            \begin{align*}
                \widetilde{a} &= \{ a, a + 12 \} = a + \{ 0, 12 \} = (\underbrace{1 + ... + 1}_{a \text{ times}}) + \{ 0, 12 \} \\ 
                &= \underbrace{(1 + \{ 0, 12 \}) + ... + (1 + \{ 0, 12 \})}_{a \text{ times}} = \underbrace{\widetilde{1} + ... + \widetilde{1}}_{a \text{ times}} \\
                &= a \cdot \widetilde{1},
            \end{align*}
            and so any element of $\widetilde{G}$ is generated from $\widetilde{1}$. Thus $\widetilde{G}$ is isomorphic to the cyclic group of order 12, which is isomorphic to $\mathbb{Z}/12\mathbb{Z}$.
          \end{proof}
\end{enumerate}

\section*{22. (9/14/23)}

\begin{enumerate}[label=(\alph*), itemsep=0em]
    \item Prove that if $H$ and $K$ are normal subgroups of $G$ then their intersection $H \cap K$ is also a normal subgroup of $G$.
          \begin{proof}
            Let $H$ and $K$ be normal subgroups of $G$. Let $h \in H \cap K$, so $h \in H$ and $h \in K$. Since both $H$ and $K$ are normal, we have $ghg^{-1} \in H$ and $ghg^{-1} \in K$ for all $g \in G$. It follows that $ghg^{-1} \in H \cap K$ for all $g \in G$. Therefore $H \cap K$ is a normal subgroup of $G$.
          \end{proof}
    \item Prove that the intersection of an arbitrary nonempty collection of normal subgroups of a group is a normal subgroup (do not assume the collection is countable).
          \begin{proof}
            Let $\mathcal{H}$ be a nonempty collection of normal subgroups of $G$. Consider $\bigcap_{H \in \mathcal{H}} = \{ h \in G \mid h \in H \text{ for all } H \in \mathcal{H} \}$. From Ch. 2.1, Exercise 10., we know that $\mathcal{H}$ is itself a subgroup of $G$. We will show that in this case it is normal in $G$.

            Let $h \in \bigcap_{H \in \mathcal{H}}$. Then for all $H \in \mathcal{H}$, we have $h \in H$. Since each $H$ is normal in $G$, we have $ghg^{-1} \in H$ for all $g \in G, H \in \mathcal{H}$. It follows that $ghg^{-1} \in \bigcap_{H \in \mathcal{H}}$, and therefore $\bigcap_{H \in \mathcal{H}}$ is normal in $G$.
          \end{proof}
\end{enumerate}

\section*{23. (9/16/23)}

Prove that the join of any nonempty collection of normal subgroups of a group is a normal subgroup.

\begin{proof}
    Let $\mathcal{H}$ be a nonempty collection of subgroups of $G$ and let $\langle \mathcal{H} \rangle$ be their join.

    Let $h \in \langle \mathcal{H} \rangle$. Then $h$ can be written as a finite product of elements, say $h_1, h_2, ..., h_n$, where each $h_i$ is an element of a corresponding normal subgroup $H_i \in \mathcal{H}$. We write this product:
    \begin{equation*}
        h = (h_1^{a_{11}}...h_n^{a_{n1}})(h_1^{a_{12}}...h_n^{a_{n2}})...(h_1^{a_{1k}}...h_n^{a_{nk}}) = \prod_{j = 1}^{k} \prod_{i = 1}^{n} h_i^{a_{ij}}.
    \end{equation*}
    Since each $h_i$ belongs to a normal subgroup $H_i$ of $G$, we have $gh_ig^{-1} \in H_i$ for all $g \in G$. It follows that, for any $m > 0$, we have $gh_i^kg^{-1} \in H_i$ (because $(gh_ig^{-1})^k = gh_ig^{-1}$). Now note that, since $(ga_1g^{-1})(ga_2g^{-1})...(ga_ng^{-1}) = g(a_1 a_2 ... a_n)g^{-1}$, the product of conjugates of the constituent elements of $h$ is equal to the conjugate of the product of those elements:
    \begin{equation*}
        \prod_{j = 1}^{k} \prod_{i = 1}^{n} gh_i^{a_{ij}}g^{-1} = g \Bigl(\prod_{j = 1}^{k} \prod_{i = 1}^{n} h_i^{a_{ij}}\Bigr) g^{-1} = ghg^{-1}.
    \end{equation*}
    The left-hand side of the equation is the product of conjugates of elements $h_i$ that each belong to the corresponding normal subgroup $H_i$. Therefore the product is an element of the join $\langle \mathcal{H} \rangle$. Since it is equal to the right-hand side, the conjugate of $h$ by any element $g \in G$, we must have $ghg^{-1} \in \langle \mathcal{H} \rangle$ for all $g \in G$. Thus the join of any nonempty collection of normal subgroups of a group is a normal subgroup.
\end{proof}

\section*{24. (9/16/23)}

Prove that if $N \unlhd G$ and $H$ is any subgroup of $G$ then $N \cap H \unlhd H$.

\begin{proof}
    Let $N \unlhd G$, $H \leq G$, and let $n \in N \cap H$, $h \in H$. Consider the conjugate element $hnh^{-1}$.

    Since $N$ is normal in $G$ and $h \in H \Rightarrow h \in G$, we have $hnh^{-1} \in N$.

    Since $H$ is a subgroup of $G$, it is closed and closed under inverses. Also, $n \in N \cap H \Rightarrow n \in H$, so the product $hnh^{-1}$ lies in $H$. We have both $hnh^{-1} \in N$ and $hnh^{-1} \in H$, so $hnh^{-1} \in N \cap H$.

    So the conjugate of any element of $N \cap H$ by any element of $H$ is again an element of $N \cap H$. Therefore $N \cap H$ is normal in $H$.
\end{proof}

\section*{25. (9/17/23)}

\begin{enumerate}[label=(\alph*), itemsep=0em]
    \item Prove that a subgroup $N$ of $G$ is normal if and only if $gNg^{-1} \subseteq G$ for all $g \in G$.
          \begin{proof}
            Recall that $N$ is defined to be normal in $G$ if $gNg^{-1} = N$ for all $g \in G$. Now if $N \unlhd G$, then clearly $gNg^{-1} \subseteq N$, since $gNg^{-1} = N$.

            Suppose that $gNg^{-1} \subseteq N$ for all $g \in G$. Let $x \in N, g \in G$. The conjugate of $x$ by $g^{-1}$, $g^{-1}x(g^{-1})^{-1}$, must lie in $N$. Let us write $g^{-1}x(g^{-1})^{-1} = n \in N$. Then we have:
            \begin{equation*}
                x = gg^{-1}xgg^{-1} = g\bigl(g^{-1}x(g^{-1})^{-1}\bigr)g^{-1} = gng^{-1},
            \end{equation*}
            and so $x \in gNg^{-1}$. This implies that $N \subseteq gNg^{-1}$. Therefore $gNg^{-1} = N$ for all $g \in G$, and so $N \unlhd G$.
          \end{proof}
    \item Let $G = GL_2(\mathbb{Q})$, let $N$ be the subgroup of upper triangular matrices with integer entries and 1's on the diagonal, and let $g$ be the diagonal matrix with entries 2, 1. Show that $gNg^{-1} \subseteq N$ but $g$ does \emph{not} normalize $N$.
          \begin{proof}
            Let $N = \begin{pmatrix}1 & n \\ 0 & 1\end{pmatrix}$, where $n \in \mathbb{Z}$ and let $g = \begin{pmatrix}2 & 0 \\ 0 & 1\end{pmatrix}$. The inverse element $g^{-1} = \begin{pmatrix}1/2 & 0 \\ 0 & 1\end{pmatrix}$.

            Then we have:
            \begin{equation*}
                gNg^{-1} = \left\{ \begin{pmatrix}2 & 0 \\ 0 & 1 \end{pmatrix}\begin{pmatrix}1 & n \\ 0 & 1 \end{pmatrix}\begin{pmatrix}1/2 & 0 \\ 0 & 1 \end{pmatrix} \mid n \in \mathbb{Z} \right\} = \left\{ \begin{pmatrix}1 & 2n \\ 0 & 1\end{pmatrix} \mid n \in \mathbb{Z} \right\}.
            \end{equation*}
            Since $2n \in \mathbb{Z}$ for all $n \in \mathbb{Z}$, we have $gNg^{-1} \subseteq N$. However, there is no $n \in \mathbb{Z}$ such that $g \begin{pmatrix}1 & n \\ 0 & 1 \end{pmatrix} g^{-1} = \begin{pmatrix}1 & 1 \\ 0 & 1\end{pmatrix}$. In order for $g$ to normalize $N$, we must have $gNg^{-1} = N$. Therefore $g$ does not normalize $N$.
          \end{proof}
\end{enumerate}

\section*{26. (9/18/23)}

Let $a, b \in G$.

\begin{enumerate}[label=(\alph*), itemsep=0em]
    \item Prove that the conjugate of the product of $a$ and $b$ is the product of the conjugate of $a$ and the conjugate of $b$. Prove that the order of $a$ and the order of any conjugate of $a$ are the same.
          \begin{proof}
            Let $g \in G$. Then:
            \begin{equation*}g(ab)g^{-1} = gabg^{-1} = gag^{-1}gbg^{-1} = (gag^{-1})(gbg^{-1}),\end{equation*}
            as desired.

            Next, we show that $a^n = 1$ if and only if $(gag^{-1})^n = 1$. If $a^n = 1$, then we have $(gag^{-1})^n = ga^ng^{-1} = gg^{-1} = 1$. And, if $(gag^{-1})^n = 1$, then we have $ga^ng^{-1} = 1$. Left multiplying by $g^{-1}$ and right-multiplying by $g$, we obtain $a^n = 1$. Therefore the order of $a$ is equal to the order of any conjugate of $a$.
          \end{proof}
    \item Prove that the conjugate of $a^{-1}$ is the inverse of the conjugate of $a$.
          \begin{proof}
            We can see that:
            \begin{equation*}(gag^{-1})(ga^{-1}g^{-1}) = gag^{-1}ga^{-1}g^{-1} = gaa^{-1}g^{-1} = gg^{-1} = 1,\end{equation*}
            and so the conjugate of $a^{-1}$ is the inverse of the conjugate of $a$.
          \end{proof}
    \item Let $N = \langle S \rangle$ for some subset $S$ of $G$. Prove that $N \unlhd G$ if $gSg^{-1} \subseteq N$ for all $g \in G$.
          \begin{proof}
            Let $x \in N$. Since $N = \langle S \rangle$, we can write $x$ as a finite product of elements of $S$: $x = (s_1^{a_{11}}...s_n^{a_{n1}})(s_1^{a_{12}}...s_n^{a_{n2}})...(s_1^{a_{1k}}...s_n^{a_{nk}})$. Now for each $s_i^{ij}$, we have $gs_i^{ij} \in N$ (since $gSg^{-1} \subseteq N$). Therefore $gxg^{-1} = g \Bigl( \prod_{j = 1}^{k} \prod_{i = 1}^{n} s_i^{a_{ij}} \Bigr) g^{-1} = \prod_{j = 1}^{k} \prod_{i = 1}^{n} (g s_i^{a_{ij}} g^{-1})$ lies in $N$ (for all $g \in G$), since it is a finite product of elements of $N$. Thus $N \unlhd G$.
          \end{proof}
    \item Deduce that if $N$ is the cyclic group $\langle x \rangle$, then $N$ is normal in $G$ if and only if for each $g \in G$, $gxg^{-1} = x^k$ for some $k \in \mathbb{Z}$.

          If $N = \langle x \rangle$ is normal in $G$, then for all $g \in G$, we have $gNg^{-1} = N$, which implies that $gxg^{-1} \in N$. Since all elements of $N$ can be written as $x^k$ for some $k \in \mathbb{Z}$, we have $gxg^{-1} = x^k$.

          Conversely, if for all $g \in G$, we have $gxg^{-1} = x^k$ for some $k \in \mathbb{Z}$, then we clearly have $gxg^{-1} \in N$, which implies that $gNg^{-1} \subseteq N$. From Exercise 25. above, this implies that $N \unlhd G$.

          Therefore $N \unlhd G$ if and only for each $g \in G, gxg^{-1} = x^k$ for some $k \in \mathbb{Z}$.
    \item Let $n$ be a positive integer. Prove that the subgroup $N$ of $G$ generated by all the elements of $G$ of order $n$ is a normal subgroup of $G$.
          \begin{proof}
            Let $S \subseteq G$ be the subset of elements of order $n$ in $G$ and let $N = \langle S \rangle$. For each $x \in N$, $x$ can be written as a finite product of elements of $S$: $x = (s_1^{a_{11}}...s_n^{a_{n1}})(s_1^{a_{12}}...s_n^{a_{n2}})...(s_1^{a_{1k}}...s_n^{a_{nk}})$, where $|s_i| = n$ for each $s_i \in S$. From part (a) above, the conjugate of any element has the same order as the element itself, so $|gs_ig^{-1}| = n$ for each $s_i \in S$, $g \in G$. Then $gs_ig^{-1} \in S \Rightarrow gs_ig^{-1} \in N$, and it follows that:
            \begin{equation*}
                gxg^{-1} = g \Bigl( \prod_{j = 1}^{k} \prod_{i = 1}^{n} s_i^{a_{ij}} \Bigr) g^{-1} = \prod_{j = 1}^{k} \prod_{i = 1}^{n} (g s_i^{a_{ij}} g^{-1})
            \end{equation*}
            is the product of a elements of $N$, and so belongs to $N$ itself. Then $gxg^{-1} \in N$ for all $g \in G$, which implies that $gNg^{-1} \subseteq N$, and thus $N$ is normal in $G$.
          \end{proof}
\end{enumerate}

\end{document}