\documentclass{article}

\title{Dummit \& Foote Ch. 3.5: Transpositions and the Alternating Group}
\author{Scott Donaldson}
\date{Dec. 2023}
\usepackage{amsmath, amsthm, amsfonts, amssymb, enumitem, tabu, tikz}

\begin{document}

\maketitle

\section*{1. (12/6/23)}

In Exercises 1 and 2 of Section 1.3 you were asked to find the cycle decompositions of some permutations. Write each of these permutations as a product of transpositions. Determine which of these is an even permutation and which is an odd permutation.

\vspace{0.5em}
In Exercise 1,
\begin{align*}
    \sigma &= (1, 3, 5)(2, 4) = (1, 3)(1, 5)(2, 4), \text{ odd.} \\
    \tau &= (1, 5)(2, 3), \text{ even.} \\
    \sigma^2 &= (1, 5, 3) = (1, 3)(1, 5), \text{ even.} \\
    \sigma \tau &= (2, 5, 3, 4) = (2, 4)(2, 3)(2, 5), \text{ odd.} \\
    \tau^2 \sigma &= (1, 3, 5)(2, 4) = (1, 5)(1, 3)(2, 4), \text{ odd.}
\end{align*}

In Exercise 2,
\begin{align*}
    \sigma &= (1, 13, 5, 10)(3, 15, 8)(4, 14, 11, 7, 12, 9) \\ &= (1, 10)(1, 5)(1, 13)(3, 8)(3, 15)(4, 9)(4, 12)(4, 7)(4, 11)(4, 14), \text{ even.} \\
    \tau &= (1, 14)(2, 9, 15, 13, 4)(3, 10)(5, 12, 7)(8, 11) \\ &= (1, 14)(2, 4)(2, 13)(2, 15)(2, 9)(3, 10)(5, 7)(5, 12)(8, 11), \text{ odd.} \\
    \sigma^2 &= (1, 5)(3, 8, 15)(4, 11, 12)(7, 9, 4)(10, 13) \\ &= (1, 15)(3, 15)(3, 8)(4, 12)(4, 11)(7, 4)(7, 9)(10, 13), \text{ even.} \\
    \sigma \tau &= (1, 11, 3)(2, 4)(5, 9, 8, 7, 10, 15)(13, 14) \\ &= (1, 3)(1, 11)(2, 4)(5, 15)(5, 10)(5, 7)(5, 8)(5, 9)(13, 14), \text{ odd.} \\
    \tau \sigma &= (1, 4)(2, 9)(3, 13, 12, 15, 11, 5)(8, 10, 14) \\ &= (1, 4)(2, 9)(3, 5)(3, 11)(3, 15)(3, 12)(3, 13)(8, 14)(8, 10), \text{ odd.} \\
    \tau^2 \sigma &= (1, 2, 15, 8, 3, 4, 14, 11, 12, 13, 7, 5, 10) \\ &= (1, 10)(1, 5)(1, 7)(1, 13)(1, 12)(1, 11)(1, 14)(1, 4)(1, 3)(1, 8)(1, 15)(1, 2), \\ & \hspace{1.1em} \text{ even.}
\end{align*}

\section*{2. (12/6/23)}

Prove that $\sigma^2$ is an even permutation for every permutation $\sigma$.

\begin{proof}
    We take as given the homomorphism $\epsilon: S_n \rightarrow \{ \pm 1 \}$ defined in this chapter, which determines the sign of every permutation $\sigma \in S_n$.

    If $\sigma$ is an even permutation, then $\epsilon(\sigma) = 1$. It follows that:
    \begin{equation*}
        \epsilon(\sigma^2) = \epsilon(\sigma)\epsilon(\sigma) = 1 \cdot 1 = 1,
    \end{equation*}
    and so $\sigma^2$ is an even permutation.

    If $\sigma$ is an odd permutation, then $\epsilon(\sigma) = -1$. It follows that:
    \begin{equation*}
        \epsilon(\sigma^2) = \epsilon(\sigma)\epsilon(\sigma) = -1 \cdot -1 = 1,
    \end{equation*}
    and so $\sigma^2$ is an even permutation.

    Since for every $\sigma \in S_n$, $\sigma$ is either an even or an odd permutation, this proves that $\sigma^2$ is an even permutation for every permutation $\sigma$.
\end{proof}

\section*{3. (12/6/23)}

Prove that $S_n$ is generated by $\{ (i, i + 1) \mid 1 \leq i \leq n - 1 \}$.

\begin{proof}
    Since any element of $S_n$ may be written as a product of transpositions, it suffices to show that the set $\{ (i, i + 1) \mid 1 \leq i \leq n - 1 \}$ can generate any transposition. Writing an arbitrary transposition in $S_n$ as $(i, i + a)$, we will prove this by strong induction on $a$ (where $1 \leq a \leq n - i$).

    The base case $a = 1$ is given, since $(i, i + 1)$ is a member of the generating set for all $i \in \{ 1, ..., n - 1 \}$.

    Next, suppose that for all $i \in \{ 1, ..., n - 1\}$ and $a \in \{ 1, ..., n - i \}$, the transposition $(i, i + a - 1)$ can be obtained from the generating set. So we have the transpositions $(i + a - 1, i + a)$ (in the generating set) and $(i, i + a - 1)$ (from the inductive hypothesis). Then:
    \begin{equation*}
        (i + a - 1, i + a)(i, i + a - 1)(i + a - 1, i + a) = (i, i + a),
    \end{equation*}
    so we can obtain the transposition $(i, i + a)$. This concludes the proof that the set $\{ (i, i + 1) \mid 1 \leq i \leq n - 1 \}$ can generate any transposition, and therefore generates all of $S_n$.
\end{proof}

\end{document}