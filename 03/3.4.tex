\documentclass{article}

\title{Dummit \& Foote Ch. 3.4: Composition Series and the Hölder Program}
\author{Scott Donaldson}
\date{Oct. 2023}
\usepackage{amsmath, amsthm, amsfonts, amssymb, enumitem, tabu, tikz}

\begin{document}

\maketitle

\section*{1. (11/2/23)}

Prove that if $G$ is an abelian simple group then $G \cong Z_p$ for some prime $p$ (do not assume $G$ is a finite group).

\begin{proof}
    Since $G$ is simple, the only normal subgroups of $G$ are 1 and $G$ itself. However, since $G$ is abelian, any subgroup of $G$ must be normal, so it follows that $G$ contains \emph{no} subgroups other than 1 and itself.

    If $x_1, x_2 \in G$ are distinct generators for $G$, then $\langle x_1 \rangle$ and $\langle x_2 \rangle$ would be distinct subgroups of $G$; therefore $G$ is generated by a single element and is a cyclic group. Let us write $G = \langle x \rangle$. If $G$ were infinite, then for any $n > 1$, $\langle x^n \rangle$ would be a distinct subgroup of $G$, so $G$ must be finite.

    Finally, if $n$ divides $|G|$, then from Chapter 2, Theorem 7.(3), $G$ contains a proper subgroup of order $n$. Therefore $|G|$ has no divisors other than 1 and itself, so we have $|G| = p$ for some prime $p$. We conclude that $G \cong Z_p$ for some prime $p$.
\end{proof}

\end{document}