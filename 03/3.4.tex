\documentclass{article}

\title{Dummit \& Foote Ch. 3.4: Composition Series and the Hölder Program}
\author{Scott Donaldson}
\date{Oct. 2023}
\usepackage{amsmath, amsthm, amsfonts, amssymb, enumitem, tabu, tikz}

\begin{document}

\maketitle

\section*{1. (11/2/23)}

Prove that if $G$ is an abelian simple group then $G \cong Z_p$ for some prime $p$ (do not assume $G$ is a finite group).

\begin{proof}
    Since $G$ is simple, the only normal subgroups of $G$ are 1 and $G$ itself. However, since $G$ is abelian, any subgroup of $G$ must be normal, so it follows that $G$ contains \emph{no} subgroups other than 1 and itself.

    If $x_1, x_2 \in G$ are distinct generators for $G$, then $\langle x_1 \rangle$ and $\langle x_2 \rangle$ would be distinct subgroups of $G$; therefore $G$ is generated by a single element and is a cyclic group. Let us write $G = \langle x \rangle$. If $G$ were infinite, then for any $n > 1$, $\langle x^n \rangle$ would be a distinct subgroup of $G$, so $G$ must be finite.

    Finally, if $n$ divides $|G|$, then from Chapter 2, Theorem 7.(3), $G$ contains a proper subgroup of order $n$. Therefore $|G|$ has no divisors other than 1 and itself, so we have $|G| = p$ for some prime $p$. We conclude that $G \cong Z_p$ for some prime $p$.
\end{proof}

\section*{2. (11/3/23)}

Exhibit all 3 composition series for $Q_8$ and all 7 composition series for $D_8$. List the composition factors in each case.

The 3 composition series for $Q_8$ are:
\begin{enumerate}[itemsep=0em]
    \item $1 \leq \langle -1 \rangle \leq \langle i \rangle \leq Q_8$
    \item $1 \leq \langle -1 \rangle \leq \langle j \rangle \leq Q_8$
    \item $1 \leq \langle -1 \rangle \leq \langle k \rangle \leq Q_8$
\end{enumerate}
In each series, each composition factor is isomorphic to $Z_2$ (thus each $N_i$ is normal in $N_{i + 1}$; since there is only one left coset it must equal the only right coset).

The 7 composition series for $D_8$ are:
\begin{enumerate}[itemsep=0em]
    \item $1 \leq \langle s \rangle \leq \langle s, r^2 \rangle \leq D_8$
    \item $1 \leq \langle sr^2 \rangle \leq \langle s, r^2 \rangle \leq D_8$
    \item $1 \leq \langle r^2 \rangle \leq \langle s, r^2 \rangle \leq D_8$
    \item $1 \leq \langle r^2 \rangle \leq \langle r \rangle \leq D_8$
    \item $1 \leq \langle r^2 \rangle \leq \langle sr, r^2 \rangle \leq D_8$
    \item $1 \leq \langle sr \rangle \leq \langle sr, r^2 \rangle \leq D_8$
    \item $1 \leq \langle sr^3 \rangle \leq \langle sr, r^2 \rangle \leq D_8$
\end{enumerate}
Again each composition factor is isomorphic to $Z_2$.

\section*{3. (11/3/23)}

Find a composition series for the quasidihedral group of order 16 (cf. Exercise 11, Section 2.5). Deduce that $QD_{16}$ is solvable.

\begin{proof}[Solution]
    Recall that $QD_{16} = \langle \sigma, \tau \mid \sigma^8 = \tau^2 = 1, \sigma \tau = \tau \sigma^3 \rangle$.

    A composition series for $QD_{16}$ is:
    \begin{equation*}
        1 \leq \langle \sigma^4 \rangle \leq \langle \sigma^2 \rangle \leq \langle \sigma \rangle \leq QD_{16},
    \end{equation*}
    where each composition factor is isomorphic to $Z_2$. Since $Z_2$ is abelian, each composition factor is solvable, and so $QD_{16}$ is solvable.
\end{proof}

\section*{4. (11/4/23)}

Use Cauchy's Theorem and induction to show that a finite abelian group has a subgroup of order $n$ for each positive divisor $n$ of its order.

\begin{proof}
    Let $G$ be a finite abelian group. Let us suppose that, for all groups $H$, $|H| < |G|$, $H$ has a subgroup of order $n$ for each positive divisor $n$ of its order.

    Let $p$ be a prime dividing $|G|$. From Cauchy's Theorem, there is an $x \in G$ with $|x| = p$. Since $G$ is abelian, $\langle x \rangle$ is normal in $G$. So the quotient group $G/\langle x \rangle$ is well-defined and has order $|G|/p < |G|$, thus it has a subgroup of order $n$ for each $n$ dividing $|G|/p$.
    
    Let $n$ be a positive divisor of $|G|$. Since $|G| = p \cdot \frac{|G|}{p}$, $n$ divides $\frac{|G|}{p}$. From the induction hypothesis, let $\overline{K}$ be a subgroup of $G/\langle x \rangle$ of order $n$. For each $\overline{k} \in \overline{K}, \overline{k} \neq \overline{1}$, we must have $k \notin \langle x \rangle$, or else we would have $\overline{k} = k \cdot \langle x \rangle = \langle x \rangle$. Then there is a bijection from $\overline{K}$ onto $K$ given by $\overline{k} \mapsto k$. Thus $K$ is a subgroup of $G$ of order $n$.
\end{proof}

\end{document}