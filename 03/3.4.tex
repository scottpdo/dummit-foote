\documentclass{article}

\title{Dummit \& Foote Ch. 3.4: Composition Series and the Hölder Program}
\author{Scott Donaldson}
\date{Nov. 2023}
\usepackage{amsmath, amsthm, amsfonts, amssymb, enumitem, tabu, tikz}

\begin{document}

\maketitle

\section*{1. (11/2/23)}

Prove that if $G$ is an abelian simple group then $G \cong Z_p$ for some prime $p$ (do not assume $G$ is a finite group).

\begin{proof}
    Since $G$ is simple, the only normal subgroups of $G$ are 1 and $G$ itself. However, since $G$ is abelian, any subgroup of $G$ must be normal, so it follows that $G$ contains \emph{no} subgroups other than 1 and itself.

    If $x_1, x_2 \in G$ are distinct generators for $G$, then $\langle x_1 \rangle$ and $\langle x_2 \rangle$ would be distinct subgroups of $G$; therefore $G$ is generated by a single element and is a cyclic group. Let us write $G = \langle x \rangle$. If $G$ were infinite, then for any $n > 1$, $\langle x^n \rangle$ would be a distinct subgroup of $G$, so $G$ must be finite.

    Finally, if $n$ divides $|G|$, then from Chapter 2, Theorem 7.(3), $G$ contains a proper subgroup of order $n$. Therefore $|G|$ has no divisors other than 1 and itself, so we have $|G| = p$ for some prime $p$. We conclude that $G \cong Z_p$ for some prime $p$.
\end{proof}

\section*{2. (11/3/23)}

Exhibit all 3 composition series for $Q_8$ and all 7 composition series for $D_8$. List the composition factors in each case.

The 3 composition series for $Q_8$ are:
\begin{enumerate}[itemsep=0em]
    \item $1 \leq \langle -1 \rangle \leq \langle i \rangle \leq Q_8$
    \item $1 \leq \langle -1 \rangle \leq \langle j \rangle \leq Q_8$
    \item $1 \leq \langle -1 \rangle \leq \langle k \rangle \leq Q_8$
\end{enumerate}
In each series, each composition factor is isomorphic to $Z_2$ (thus each $N_i$ is normal in $N_{i + 1}$; since there is only one left coset it must equal the only right coset).

The 7 composition series for $D_8$ are:
\begin{enumerate}[itemsep=0em]
    \item $1 \leq \langle s \rangle \leq \langle s, r^2 \rangle \leq D_8$
    \item $1 \leq \langle sr^2 \rangle \leq \langle s, r^2 \rangle \leq D_8$
    \item $1 \leq \langle r^2 \rangle \leq \langle s, r^2 \rangle \leq D_8$
    \item $1 \leq \langle r^2 \rangle \leq \langle r \rangle \leq D_8$
    \item $1 \leq \langle r^2 \rangle \leq \langle sr, r^2 \rangle \leq D_8$
    \item $1 \leq \langle sr \rangle \leq \langle sr, r^2 \rangle \leq D_8$
    \item $1 \leq \langle sr^3 \rangle \leq \langle sr, r^2 \rangle \leq D_8$
\end{enumerate}
Again each composition factor is isomorphic to $Z_2$.

\section*{3. (11/3/23)}

Find a composition series for the quasidihedral group of order 16 (cf. Exercise 11, Section 2.5). Deduce that $QD_{16}$ is solvable.

\begin{proof}[Solution]
    Recall that $QD_{16} = \langle \sigma, \tau \mid \sigma^8 = \tau^2 = 1, \sigma \tau = \tau \sigma^3 \rangle$.

    A composition series for $QD_{16}$ is:
    \begin{equation*}
        1 \leq \langle \sigma^4 \rangle \leq \langle \sigma^2 \rangle \leq \langle \sigma \rangle \leq QD_{16},
    \end{equation*}
    where each composition factor is isomorphic to $Z_2$. Since $Z_2$ is abelian, each composition factor is solvable, and so $QD_{16}$ is solvable.
\end{proof}

\section*{4. (11/4/23)}

Use Cauchy's Theorem and induction to show that a finite abelian group has a subgroup of order $n$ for each positive divisor $n$ of its order.

\begin{proof}
    Let $G$ be a finite abelian group. Let us suppose that, for all groups $H$, $|H| < |G|$, $H$ has a subgroup of order $n$ for each positive divisor $n$ of its order.

    Let $p$ be a prime dividing $|G|$. From Cauchy's Theorem, there is an $x \in G$ with $|x| = p$. Since $G$ is abelian, $\langle x \rangle$ is normal in $G$. So the quotient group $G/\langle x \rangle$ is well-defined and has order $|G|/p < |G|$, thus it has a subgroup of order $n$ for each $n$ dividing $|G|/p$.
    
    Let $n$ be a positive divisor of $|G|$. Since $|G| = p \cdot \frac{|G|}{p}$, $n$ divides $\frac{|G|}{p}$. From the induction hypothesis, let $\overline{K}$ be a subgroup of $G/\langle x \rangle$ of order $n$. For each $\overline{k} \in \overline{K}, \overline{k} \neq \overline{1}$, we must have $k \notin \langle x \rangle$, or else we would have $\overline{k} = k \cdot \langle x \rangle = \langle x \rangle$. Then there is a bijection from $\overline{K}$ onto $K$ given by $\overline{k} \mapsto k$. Thus $K$ is a subgroup of $G$ of order $n$.
\end{proof}

\section*{5. (11/7/23)}

Prove that subgroups and quotient groups of a solvable group are solvable.

\begin{proof}
    Let $G$ be a solvable group. Then there exists a chain of subgroups
    \begin{equation*}
        1 = G_0 \unlhd G_1 \unlhd G_2 \unlhd ... \unlhd G_s = G
    \end{equation*}
    such that $G_{i + 1}/G_i$ is abelian for each $i \in \{ 0, ..., s - 1 \}$.

    Let $N \leq G$ and let $G_i$ be the smallest subgroup in the above series such that $N \leq G_i$. Since $G_{i - 1} \unlhd G_i$, we have $G_i \leq N_G(G_{i - 1})$ and so $N \leq N_G(G_{i - 1})$. Then by the Diamond Isomorphism Theorem it follows that
    \begin{align*}
        NG_{i - 1} \leq G_i, \hspace{0.4em}
        N \cap G_{i - 1} \unlhd N, \text{ and }
        NG_{i - 1}/G_{i - 1} \cong N/N \cap G_{i - 1}.
    \end{align*}
    Since the quotient group $G_i/G_{i - 1}$ is abelian, its subgroup $NG_{i - 1}/G_{i - 1}$ is as well. Then, since $N/N \cap G_{i - 1} \cong NG_{i - 1}/G_{i - 1}$, it follows that $N/N \cap G_{i - 1}$ is abelian.

    We can repeat the above process with $N \cap G_{i - 1} \leq G_{i - 1}$ to conclude that $N \cap G_{i - 2} \unlhd N \cap G_{i - 1}$, with $N \cap G_{i - 1}/N \cap G_{i - 2}$ abelian. Continuing this way we produce the chain
    \begin{equation*}
        1 = N \cap G_0 \unlhd N \cap G_1 \unlhd ... \unlhd N \cap G_{i - 1} \unlhd N \cap G_i = N
    \end{equation*}
    where $N \cap G_{i + 1} / N \cap G_i$ is abelian for $i \in \{ 0, ..., i - 1 \}$, which shows that $N$ is solvable.
\end{proof}

\section*{6. (11/9/23)}

Prove part (1) of the Jordan-Hölder Theorem by induction on $|G|$.

\begin{proof}
    Part (1) of the Jordan-Hölder Theorem states that if $G$ is a finite group, $G \neq 1$, then $G$ has a composition series. Suppose that for all groups $H$, $|H| < G$, $H$ has a composition series.

    If $G$ is a simple group, then $1 \leq G$ is a composition series, since $G/1 \cong G$ is simple.
    
    Therefore assume that $G$ contains at least one proper normal subgroup $N$. Then we have $|N| < |G|$, so by assumption $N$ has a composition series
    \begin{equation*}
        1 = N_0 \leq N_1 \leq ... \leq N_{k - 1} \leq N_k = N,
    \end{equation*}
    where the quotient group $N_{i + 1}/N_i$ is simple for $i \in \{ 0, ..., k - 1 \}$. And, the quotient group $G/N$ has order $|G/N| = \frac{|G|}{|N|} < |G|$, so it also contains a composition series
    \begin{equation*}
        N/N = G_0/N \leq G_1/N \leq ... \leq G_{m + 1}/N \leq G_m/N = G,
    \end{equation*}
    where each $(G_{i + 1}/N)/(G_i/N)$ is simple for $i \in \{ 0, ..., m - 1 \}$. By the Third Isomorphism Theorem, this implies that each $G_{i + 1}/G_i$ is simple.

    We now have a chain
    \begin{equation*}
        1 = N_0 \leq N_1 \leq ... \leq N_{k - 1} \leq N_k = N = G_0 \leq G_1 \leq ... \leq G_{m + 1} \leq G_m = G
    \end{equation*}
    where the quotient of each successive subgroup by the previous is a simple group. Thus it is a composition series for $G$.
\end{proof}

\section*{7. (11/9/23)}

If $G$ is a finite group and $H \unlhd G$ prove that there is a composition series for $G$, one of whose terms is $H$.

\begin{proof}
    By the Jordan-Hölder Theorem, $H$ and the quotient group $G/H$ both have composition series. Then we can construct a chain (identical to the immediately above proof) such that
    \begin{equation*}
        1 = H_0 \leq H_1 \leq ... \leq H_{k - 1} \leq H_k = H = G_0 \leq G_1 \leq ... \leq G_{m + 1} \leq G_m = G
    \end{equation*}
    is a composition series for $G$, one of whose terms is $H$.
\end{proof}

\section*{8. (11/12/23)}

Let $G$ be a \emph{finite} group. Prove that the following are equivalent:
\begin{enumerate}[label=(\roman*), itemsep=0em]
    \item $G$ is solvable
    \item $G$ has a chain of subgroups: $1 = H_0 \unlhd H_1 \unlhd H_2 \unlhd ... \unlhd H_s = G$ such that $H_{i + 1}/H_i$ is cyclic, $0 \leq i \leq s - 1$
    \item all composition factors of $G$ are of prime order
    \item $G$ has a chain of subgroups: $1 = N_0 \unlhd N_1 \unlhd N_2 \unlhd ... \unlhd N_t = G$ such that each $N_i$ is a normal subgroup of $G$ and $N_{i + 1}/N$ is abelian, $0 \leq i \leq t - 1$.
\end{enumerate}

To show that (i) implies (ii), let $G$ be a finite solvable group. Then there exists a chain
\begin{equation*}
    1 = G_0 \unlhd G_1 \unlhd G_2 \unlhd ... \unlhd G_s = G
\end{equation*}
such that $G_{i + 1}/G_i$ is abelian for $0 \leq i \leq s - 1$. If for some $i$, $G_{i + 1}/G_i$ is not simple, then there exists a proper normal subgroup $H$ of $G_{i + 1}$ that contains but is not equal to $G_i$. Since $G_i \unlhd G_{i + 1}$, we also have $G_i \unlhd H$, and since $G_{i + 1}/G_i$ is abelian, $H/G_i$ is as well. So we can subdivide every link in the original chain to produce another chain:
\begin{equation*}
    1 = H_0 \unlhd H_1 \unlhd H_2 \unlhd ... \unlhd H_t = G,
\end{equation*}
where each $H_{i + 1}/H_i$ is an abelian simple group for $0 \leq i \leq t - 1$. From Exercise 1, an abelian simple group is isomorphic to $Z_p$ for some prime $p$. Therefore each quotient in the chain is cyclic.

Similarly (ii) implies (iii). If $G$ has a chain of subgroups such that each quotient is cyclic, then each quotient is also abelian. If there is a quotient $H_{i + 1}/H_i$ that is composite, then we can find another proper normal subgroup $K$ of $H_{i + 1}$ that is not equal to $H_i$. We continue to do this until the links in the chain are of prime order.

Next we show that (iii) implies (iv). All composition factors of $G$ are of prime order, so they are all isomorphic to $Z_p$ for some prime $p$, thus cyclic and abelian. Let $M$ be a minimal nontrivial normal subgroup of $G$. From Exercise 7, there is a composition series of $G$ that includes $M$. Let $N \unlhd M$ be of prime index, so $M/N$ is abelian. From Chapter 3.1, Exercise 40, it follows that, for all $x, y \in M$, the commutator element $x^{-1}y^{-1}xy$ lies in $N$. 

We claim next that, for all $g \in G$, $gNg^{-1} \unlhd M$. Let $x = gng^{-1} \in gNg^{-1}$. Then for all $h \in G$:
\begin{equation*}
    hxh^{-1} = hgng^{-1}h^{-1} = (hg)n(hg)^{-1} \in M \text{ (since $hg \in G$ and $n \in M$)},
\end{equation*}
which shows that $gNg^{-1} \unlhd M$. Since $|gNg^{-1}| = |N|$, $M/gNg^{-1}$ has the same prime order as $M/N$, and the quotient group is therefore abelian, so for all $g \in G, x, y \in M$, we have $x^{-1}y^{-1}xy \in gNg^{-1}$. It follows that $gNg^{-1} = N$ for all $g \in G$, and so $N \unlhd G$, which contradicts $M$ being a minimal normal subgroup. Therefore we must have $N = 1$. In turn, we conclude that $x^{-1}y^{-1}xy = 1$ for all $x, y \in M$, and so $xy = yx$, hence $M$ is abelian.

Since $M \unlhd G$, next let $\overline{G} = G/M$ and let $\overline{M_1} \in \overline{G}$ be a minimal nontrivial normal subgroup. Then $\overline{M_1} = M_1/M$ is an abelian quotient group. We continue inductively until we produce the chain
\begin{equation*}
    1 = M \unlhd M_1 \unlhd M_2 \unlhd ... \unlhd M_r = G,
\end{equation*}
where each $M_i$ is normal in $G$ and $M_{i + 1}/M_i$ is abelian, $0 \leq i \leq r - 1$.

Finally, (iv) implies (i), for each $M_{i + 1}/M_i$ is already abelian, and so $G$ is abelian. This concludes the proof that the four statements above are equivalent.
\end{document}