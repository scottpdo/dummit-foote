\documentclass{article}

\title{Dummit \& Foote Ch. 1.7: Group Actions}
\author{Scott Donaldson}
\date{Apr. 2023}
\usepackage{amsmath, amsthm, amsfonts, enumitem}

\begin{document}

\maketitle

\section*{1. (4/27/23)}

Let $F$ be a field. Show that the multiplicative group of nonzero elements of $F$ (denoted by $F^\times$) acts on the set $F$ by $g \cdot a = ga$, where $g \in F^\times, a \in F$ and $ga$ is the usual product in $F$ of the two field elements.

\begin{proof}
    To show that $F^\times$ acts on $F$, we must show that $g_1 \cdot (g_2 \cdot a) = (g_1 g_2) \cdot a$ for all $g_1, g_2 \in F^\times, a \in F$, and $1 \cdot a = a$ for all $a \in F$.

    First, let $g_1, g_2 \in F^\times$ and $a \in F$. By the definition of the action, $g_1 \cdot (g_2 \cdot a) = g_1 \cdot (g_2 a) = g_1 g_2 a$. By the associativity of multiplication, $g_1 g_2 a = (g_1 g_2) a$. Again by the action definition, this equals $(g_1 g_2) \cdot a$.
    
    It follows directly from the field axiom of multiplicative identity that $1 \cdot a = a$ for all $a \in A$. Thus $F^\times$ acts on $F$ by $g \cdot a = ga$.
\end{proof}

\section*{2. (4/27/23)}

Show that the additive group $\mathbb{Z}$ acts on itself by $z \cdot a = z + a$ for all $z, a \in \mathbb{Z}$.

\begin{proof}
    First, $z_1 \cdot (z_2 \cdot a) = z_1 \cdot (z_2 + a) = z_1 + z_2 + a = (z_1 + z_2) + a = (z_1 + z_2) \cdot a$.

    Also, $0 \cdot a = 0 + a = a$ for all $a \in \mathbb{Z}$. Thus $\mathbb{Z}$ acts on itself by $z \cdot a = z + a$.
\end{proof}

\section*{3. (4/27/23)}

Show that the additive group $\mathbb{R}$ acts on the $x, y$ plane $\mathbb{R} \times \mathbb{R}$ by $r \cdot (x, y) = (x + ry, y)$.

\begin{proof}
    First, $r_1 \cdot (r_2 \cdot (x, y)) = r_1 \cdot (x + r_2 y, y) = (x + r_2 y + r_1 y, y) = \\ (x + (r_1 + r_2)y, y) = (r_1 + r_2) \cdot (x, y)$.

    Also, $0 \cdot (x, y) = (x + 0y, y) = (x, y)$ for all $(x, y) \in \mathbb{R} \times \mathbb{R}$. Thus $\mathbb{R}$ acts on $\mathbb{R} \times \mathbb{R}$ by $r \cdot (x, y) = (x + ry, y)$.
\end{proof}

\section*{4. (4/27/23)}

Let $G$ be a group acting on a set $A$ and fix some $a \in A$. Show that the following sets are subgroups of $G$:

\begin{enumerate}[label=(\alph*)]
    \item the kernel of the action,
          \begin{proof}
            The kernel of $G$ is the set $\{ g \in G \mid g \cdot a = a \text{ for all } a \in A \}$. It is closed under the binary operation of $G$: If $g_1, g_2$ are in the kernel, then $g_1 \cdot (g_2 \cdot a) = g_1 \cdot a = a$ for all $a \in A$. And, by definition of a group action, $g_1 \cdot (g_2 \cdot a) = (g_1 g_2) \cdot a$, which implies that $(g_1 g_2) \cdot a = a$, so $g_1 g_2$ is in the kernel of $G$.

            The kernel is also closed under inverses: Let $g$ be in the kernel of $G$. Then $1 \cdot a = (g^{-1} g) \cdot a = g^{-1} \cdot (g \cdot a) = g^{-1} \cdot a$. By definition, $1 \cdot a = a$, so $g^{-1} \cdot a = a$ for all $a$, so $g^{-1}$ is in the kernel. Thus the kernel of the action is a subgroup of $G$.
          \end{proof}
    \item $\{ g \in G \mid ga = a \}$ — this subgroup is called the \emph{stabilizer} of $G$.
          \begin{proof}
            The proof that this set of elements if a subgroup is identical to the one immediately above, but for a fixed $a$ as opposed to all $a \in A$.
          \end{proof}
\end{enumerate}

\section*{5. (4/28/23)}

Prove that the kernel of an action of the group $G$ on the set $A$ is the same as the kernel of the corresponding permutation representation $G \rightarrow S_A$.

\begin{proof}
    Let $\varphi$ be the permutation representation $G \rightarrow S_A$ corresponding to $G$ acting on $A$. Let $g$ be in the kernel of the action of $G$ (to show that $\varphi(g)$ is in the kernel of $\varphi$). Then $g \cdot a = a$ for all $a \in A$. If $\sigma_g$ is the permutation of $S_A$ corresponding to $g$, then $\sigma_g$ is the identity permutation, because $\sigma_g(a) = a$ for all $a \in A$. Thus $\sigma_g = \varphi(g)$ is in the kernel of $\varphi$.

    Next, let $\varphi(g)$ be in the kernel of $\varphi$ (to show that $g$ is in the kernel of $G$). Then $\varphi(g)$ is the identity permutation, so $\varphi(g) \cdot a = \sigma_g(a) = a$ for all $a \in A$. Also, by definition, $\sigma_g(a) = g \cdot a$, so $g \cdot a = a$ for all $a \in A$. Thus $g$ is in the kernel of the action of $G$.

    Having shown that membership in one implies membership in the other, this proves that the kernel of $G$ acting on $A$ is thus equal to the kernel of the permutation representation $\varphi: G \rightarrow S_A$.
\end{proof}

\section*{6. (4/28/23)}

Prove that a group $G$ acts faithfully on a set $A$ if and only if the kernel of the action is the set consisting of only the identity.

\begin{proof}
    First, let $G$ act on $A$. Suppose that $G$ acts on $A$ faithfully (to show that the kernel of the action of $G$ is the set consisting of only the identity). Consider the permutation representation $\varphi: G \rightarrow S_A$. Since $G$ acts on $A$ faithfully, $\varphi$ is injective (that is, $g_1, g_2 \in G$ induce different permutations $\varphi(g_1), \varphi(g_2)$). Thus the identity permutation $\varphi(1)$ is the only permutation that assigns $a$ to $a$ for all $a \in A$. From 5., the kernel of the action of $G$ is the same as the kernel of $\varphi$, so the identity of $G$ is the only element in the kernel of the action of $G$.

    Next, suppose that the kernel of the action of $G = \{ 1 \}$ (to show that $G$ acts on $A$ faithfully). Suppose for some $g_1, g_2 \in G$, we have $\varphi(g_1) = \varphi(g_2)$, that is, $\sigma_{g_1}(a) = \sigma_{g_2}(a)$ for all $a \in A$. Consider the permutation obtained by composing $\varphi(g_1)^{-1} \circ \varphi(g_2)$. Applying the resulting permutation to some $a \in A$ (and saying that $\sigma_{g_1}(a) = \sigma_{g_2}(a) = b$), we obtain $(\varphi(g_1)^{-1} \circ \varphi(g_2))(a) = \sigma_{g_1}^{-1}(\sigma_{g_2}(a)) = \sigma_{g_1}^{-1}(b) = a$. This implies that $\varphi(g_1)^{-1} \circ \varphi(g_2)$ is the identity permutation. Since $\varphi$ is a homomorphism, $\varphi(g_1)^{-1} \circ \varphi(g_2) = \varphi(g_1^{-1}) \circ \varphi(g_2) = \varphi(g_1^{-1} g_2)$. However, because the kernel of the action of $G$ is $\{ 1 \}$, and from 5., the kernel of $\varphi$ is also $\{ 1 \}$, this implies that $g_1^{-1} g_2 = 1 \Rightarrow g_1 = g_2$.
\end{proof}

\section*{7. (4/29/23)}

Prove that the action of the multiplicative group $\mathbb{R}^\times$ on $\mathbb{R}^n$ defined by \\ $\alpha \cdot (r_1, r_2, ..., r_n) = (\alpha r_1, \alpha r_2, ..., \alpha r_n)$ is faithful.

\begin{proof}
    From 6., a group acts faithfully on a set if and only if the kernel of the action consists only of the group's identity. Therefore, to show that the given action of $\mathbb{R}^\times$ on $\mathbb{R}^n$ is faithful, it suffices to show that the kernel of the action is $\{ 1 \}$.

    By definition, the kernel of the action is the set of all $\alpha \in \mathbb{R}$ such that $\alpha \cdot (r_1, r_2, ..., r_n) = (r_1, r_2, ..., r_n)$ for all such elements of $\mathbb{R}^n$. By definition of the group action, then, for an element $\alpha$ of $\mathbb{R}^\times$ to be in the kernel of the action, we must have $\alpha r_1 = r_1, \alpha r_2 = r_2, ..., \alpha r_n = r_n$. The only element for which this holds is $1$. Thus the kernel of the action is $\{ 1 \}$, and so $\mathbb{R}^\times$ acts faithfully on $\mathbb{R}^n$.
\end{proof}

\section*{8. (4/30/23)}

Let $A$ be a nonempty set and let $k$ be a positive integer with $k \leq |A|$. The symmetric group $S_A$ acts on $B$ consisting of all subsets of $A$ of cardinality $k$ by $\sigma \cdot \{ a_1, ..., a_k \} = \{ \sigma(a_1), ..., \sigma(a_k) \}$.

\begin{enumerate}[label=(\alph*)]
    \item Prove that this is a group action.
          \begin{proof}
            The identity permutation acts on an arbitrary element of $B$ by $(1) \cdot \{ a_1, ..., a_k \} = \{ a_1, ..., a_k \}$, as desired.

            Further, $\sigma_1 \cdot (\sigma_2 \cdot \{ a_1, ..., a_k \}) = \sigma_1 \cdot \{ \sigma_2(a_1), ..., \sigma_2(a_k) \} = \{ \sigma_1(\sigma_2(a_1)), ..., \\ \sigma_1(\sigma_2(a_k)) \} = \{ (\sigma_1 \circ \sigma_2)(a_1), ..., (\sigma_1 \circ \sigma_2)(a_k) \} = (\sigma_1 \circ \sigma_2) \cdot \{ a_1, ..., a_k \}$. Together these two equations prove that this action of $S_A$ on $B$ is a group action.
          \end{proof}
    \item Describe exactly how the permutations $(1, 2)$ and $(1, 2, 3)$ act on the six 2-element subsets of $\{1, 2, 3, 4\}$.
          \begin{itemize}
            \item $(1, 2) \cdot \{1, 2\} = \{2, 1\} = \{1, 2\}$
            \item $(1, 2) \cdot \{1, 3\} = \{2, 3\}$
            \item $(1, 2) \cdot \{1, 4\} = \{2, 4\}$
            \item $(1, 2) \cdot \{2, 3\} = \{1, 3\}$
            \item $(1, 2) \cdot \{2, 4\} = \{1, 4\}$
            \item $(1, 2) \cdot \{3, 4\} = \{3, 4\}$
            \item $(1, 2, 3) \cdot \{1, 2\} = \{2, 3\}$
            \item $(1, 2, 3) \cdot \{1, 3\} = \{2, 1\} = \{1, 2\}$
            \item $(1, 2, 3) \cdot \{1, 4\} = \{2, 4\}$
            \item $(1, 2, 3) \cdot \{2, 3\} = \{3, 1\} = \{1, 3\}$
            \item $(1, 2, 3) \cdot \{2, 4\} = \{3, 4\}$
            \item $(1, 2, 3) \cdot \{3, 4\} = \{1, 4\}$
          \end{itemize}
\end{enumerate}

\section*{9. (4/30/23)}

Do both parts of the preceding exercise with "ordered $k$-tuples" in place of "$k$-element subsets," where the action on $k$-tuples is defined as above but with set braces replaced by parentheses (note that, for example, the 2-tuples $(1, 2)$ and $(2, 1)$ are different even though the sets $\{1, 2\}$ and $\{2, 1\}$ are the same).

\begin{enumerate}[label=(\alph*)]
    \item The proof is identical to that in 8., but with set braces replaced by parentheses. For the identity permutation, $(1) \cdot ( a_1, ..., a_k ) = ( a_1, ..., a_k )$. Similarly for arbitrary $\sigma_1, \sigma_2$ and $( a_1, ..., a_k )$, the logic holds.
    \item Describe exactly how the permutations $(1, 2)$ and $(1, 2, 3)$ act on the twelve 2-element tuples of $(1, 2, 3, 4)$.
          \begin{itemize}
            \item $(1, 2) \cdot (1, 2) = (2, 1); (1, 2) \cdot (2, 1) = (1, 2)$
            \item $(1, 2) \cdot (1, 3) = (2, 3); (1, 2) \cdot (3, 1) = (3, 2)$
            \item $(1, 2) \cdot (1, 4) = (2, 4); (1, 2) \cdot (4, 1) = (4, 2)$
            \item $(1, 2) \cdot (2, 3) = (1, 3); (1, 2) \cdot (3, 2) = (3, 1)$
            \item $(1, 2) \cdot (2, 4) = (1, 4); (1, 2) \cdot (4, 2) = (4, 1)$
            \item $(1, 2) \cdot (3, 4) = (3, 4); (1, 2) \cdot (4, 3) = (4, 3)$
            \item $(1, 2, 3) \cdot (1, 2) = (2, 3); (1, 2, 3) \cdot (2, 1) = (3, 2)$
            \item $(1, 2, 3) \cdot (1, 3) = (2, 1); (1, 2, 3) \cdot (3, 1) = (1, 2)$
            \item $(1, 2, 3) \cdot (1, 4) = (2, 4); (1, 2, 3) \cdot (4, 1) = (4, 2)$
            \item $(1, 2, 3) \cdot (2, 3) = (3, 1); (1, 2, 3) \cdot (3, 2) = (1, 3)$
            \item $(1, 2, 3) \cdot (2, 4) = (3, 4); (1, 2, 3) \cdot (4, 2) = (4, 3)$
            \item $(1, 2, 3) \cdot (3, 4) = (1, 4); (1, 2, 3) \cdot (4, 3) = (4, 1)$
          \end{itemize}
\end{enumerate}

\section*{10. (5/4/23)}

With reference to the two preceding exercises determine:

\begin{enumerate}[label=(\alph*)]
  \item for which values of $k$ the action of $S_n$ on $k$-element subsets is faithful, and
  \item for which values of $k$ the action of $S_n$ on ordered $k$-tuples is faithful.
\end{enumerate}

For the action of $S_n$ on $k$-element subsets, the action is faithful if $n > 1$ and $k < n$.
\begin{proof}
  In the case where $n = 1$, then the action is trivially faithful (because the symmetric group $S_n$ consists only of the identity).
  
  So suppose that $n > 1$ and let $k < n$, with $B$ the set of all $k$-element subsets of $A = \{ 1, 2, ..., n \}$. Let $\sigma \in S_n$ be a non-identity permutation. Then $\sigma$ assigns at least one element of $A$ to a different element of $A$. Suppose that $\sigma(a_1) = a_2$ for some $a_1, a_2 \in A$. Because $k < n$, there exists a subset $b \in B$ such that $a_1 \in b$ and $a_2 \notin b$. Then $\sigma \cdot b = \{ \sigma(a_1), ... \} = \{ a_2, ... \} \neq b$, and so $\sigma$ is not in the kernel of the action. Therefore the kernel of the action consists only of the identity permutation, and so the action is faithful.

  Now, let $n > 1$ and let $k = n$. Then $B$, the set of all $k$-element subsets of $A = \{ 1, 2, ..., n \}$, consists only of $A$ itself. Now let $\sigma \in S_n$ and let $a_1, a_2 \in A$ with $\sigma(a_1) = a_2$. For all $b \in B$ (because $b = A$), $a_1, a_2 \in b \Rightarrow \sigma(a_2) \in b$. Therefore $\sigma \cdot b = b$ for all $b \in B$. Thus every permutation of $S_n$ is in the kernel of the action, and so the action is not faithful.

  This proves that the action of $S_n$ on $k$-element subsets is faithful if and only if $n > 1$ and $k < n$.
\end{proof}

For the action of $S_n$ on ordered $k$-tuples, the action is faithful for all values of $k$ (if $n > 1$).

\begin{proof}
  As above, the action is trivially faithful if $n = 1$, so suppose that $n > 1$, let $\sigma$ be a non-identity permutation in $S_n$, and let $1 \leq k \leq n$, such that $B$ is the set of all $k$-element tuples of $A = \{ 1, 2, ..., n \}$ (ex. $(1,2)$ and $(2,1) \in B$). Let $a_1 \in A$ and let $a_2 = \sigma(a_1)$. Let $b$ be the $k$-tuple consisting only of $a_1$, that is, $\underbrace{(a_1, ..., a_1)}_{\text{$k$ times}}$. Then $\sigma \cdot b = \sigma \cdot (a_1, ..., a_1) = (\sigma(a_1), ..., \sigma(a_1)) = (a_2, ..., a_2)$. Then for all non-identity $\sigma \in S_n$, there exists a $b \in B$ such that $\sigma \cdot b \neq b$. Therefore the only permutation in the kernel of the action is the identity permutation, and so the action is faithful for all values of $k$.
\end{proof}

\end{document}