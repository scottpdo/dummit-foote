\documentclass{article}

\title{Dummit \& Foote Ch. 7.1: Introduction to Rings}
\author{Scott Donaldson}
\date{Jul. 2024}
\usepackage{amsmath, amsthm, amsfonts, amssymb, enumitem, tabu, tikz}

\begin{document}

\maketitle

Let $R$ be a ring with 1.

\section*{1. (7/1/24)}

Show that $(-1)^2 = 1$ in $\mathbb{R}$.

\begin{proof}
    We have:
    \begin{align*}
        (-1) + (-1)^2 = \underbrace{(-1)(1)}_{\text{identity}} + (-1)(-1) = \underbrace{(-1)(1 + (-1))}_{\text{distribution}} = (-1)\underbrace{(0)}_{\text{inverses}} = 0,
    \end{align*}
    and therefore, since $(-1) + (-1)^2 = 0$, $(-1)^2 = 1$.
\end{proof}

\section*{2. (7/1/24)}

Prove that if $u$ is a unit in $R$ then so is $-u$.

\begin{proof}
    Recall that $u$ is a unit in $R$ if there exists some $v \in R$ such that $uv = vu = 1$.

    Now:
    \begin{align*}
        (-u)(v) = -(uv) &= -1, \text{ which implies that} \\
        (-u)(v)(-1) = (-1)^2 &= 1, \text{ so} \\
        (-u)(-v) = 1,
    \end{align*}
    which implies that $-u$ is also a unit in $R$.
\end{proof}

\end{document}