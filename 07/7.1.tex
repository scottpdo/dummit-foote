\documentclass{article}

\title{Dummit \& Foote Ch. 7.1: Introduction to Rings}
\author{Scott Donaldson}
\date{Jul. 2024}
\usepackage{amsmath, amsthm, amsfonts, amssymb, enumitem, tabu, tikz}

\begin{document}

\maketitle

Let $R$ be a ring with 1.

\section*{1. (7/1/24)}

Show that $(-1)^2 = 1$ in $\mathbb{R}$.

\begin{proof}
    We have:
    \begin{align*}
        (-1) + (-1)^2 = \underbrace{(-1)(1)}_{\text{identity}} + (-1)(-1) = \underbrace{(-1)(1 + (-1))}_{\text{distribution}} = (-1)\underbrace{(0)}_{\text{inverses}} = 0,
    \end{align*}
    and therefore, since $(-1) + (-1)^2 = 0$, $(-1)^2 = 1$.
\end{proof}

\section*{2. (7/1/24)}

Prove that if $u$ is a unit in $R$ then so is $-u$.

\begin{proof}
    Recall that $u$ is a unit in $R$ if there exists some $v \in R$ such that $uv = vu = 1$.

    Now:
    \begin{align*}
        (-u)(v) = -(uv) &= -1, \text{ which implies that} \\
        (-u)(v)(-1) = (-1)^2 &= 1, \text{ so} \\
        (-u)(-v) = 1,
    \end{align*}
    which implies that $-u$ is also a unit in $R$.
\end{proof}

\section*{7. (7/5/24)}

The \emph{center} of a ring $R$ is $\{ z \in R \mid zr = rz \text{ for all } r \in R \}$ (i.e., is the set of all elements which commute with every element of $R$). Prove that the center of a ring is a subring that contains the identity. Prove that the center of a division ring is a field.

\begin{proof}
    Let $a, b \in R$ be in the center of $R$ and let $x \in R$. Then:
    \begin{equation*}
        (a - b)x = ax - bx = xa - xb = x(a - b),
    \end{equation*}
    so $a - b$ is in the center of $R$. And, since $a$ and $b$ both commute with $x$, we have $(ab)x = abx = xab = x(ab)$, so $ab$ lies in the center of $R$ as well. Since by definition 1 commutes with every element of $R$, the center of $R$ is a subring of $R$ containing the identity.

    If $R$ is a division ring, then every element in its center (except 0) has a multiplicative inverse (is a unit). Every element in its center also commutes with every other element. A field is a commutative ring where every nonzero element is a unit; therefore the center of a division ring is a field.
\end{proof}

\section*{8. (7/9/24)}

Describe the center of the Hamilton Quaternions $\mathbb{H}$. Prove that $\{ a + bi \mid a, b \in \mathbb{R} \}$ is a subring of $\mathbb{H}$ which is a field but is not contained in the center of $\mathbb{H}$.

\begin{proof}
    Let $a + bi + cj + dk$ ($a, b, c, d \in \mathbb{R}$) lie in the center of $\mathbb{H}$. It must commute with $i$ ($= 0 + 1i + 0j + 0k$). Then:
    \begin{align*}
        (a + bi + cj + dk)i &= ai + bi^2 + cji + dki \\
        &= -b + ai + dj - ck, \text{ and } \\
        i(a + bi + cj + dk) &= ai + bi^2 + cij + dik \\
        &= -b + ai - dj + ck.
    \end{align*}
    If these are equal, then we have:
    \begin{align*}
        -b + ai + dj - ck &= -b + ai - dj + ck \\
        dj - ck &= -dj + ck \\ 
        2dj &= 2ck \\
        dj &= ck,
    \end{align*}
    and since $c, d \in \mathbb{R}$, there are no nonzero values of $c, d$ such that $dj = ck$. Thus we must have $c = d = 0$.

    Repeating the above steps for the product of $a + bi + cj + dk$ and $j$ or $k$, we see that $b$ must also be 0.

    Now because real coefficients of $i, j, k$ commute, $a$ may take any value, and so the center of $\mathbb{H}$ consists of the real numbers (that is, quaternions of the form $a + 0i + 0j + 0k$).

    Consider the subset $\{ a + bi \mid a, b \in \mathbb{R} \}$. Let $a + bi, c + di$ be two elements of this subset. We see that $(a + bi) - (c + di) = (a - c) + (b - d)i$ and $(a + bi)(c + di) = ac + adi + bci + bdi^2 = (ac - bd) + (ad + bc)i$. Since this subset is closed under subtraction and multiplication, it is a subring of $\mathbb{H}$. However, since it includes elements with nonzero $i$ components, it is not contained in the center of $\mathbb{H}$.
\end{proof}

\section*{9. (7/9/24)}

For a fixed element $a \in R$ define $C(a) = \{ r \in R \mid ra = ar \}$. Prove that $C(a)$ is a subring of $R$ containing $a$. Prove that the center of $R$ is the intersection of the subrings $C(a)$ over all $a \in R$.

\begin{proof}
    Let $a \in R$ and let $c, d \in C(a)$. Then:
    \begin{align*}
        (c - d)a = ca - da = ac - ad = a(c - d),& \text{ and } \\
        (cd)a = cda = cad = acd = a(cd),&
    \end{align*}
    so $C(a)$ is a subring of $R$. Since elements commute with themselves, $a \in C(a)$.

    Next, consider the intersection of all subrings $C(a)$ for $a \in R$, $\bigcap_{a \in R} C(a)$. Let $c \in \bigcap_{a \in R} C(a)$. Then $ca = ac$ for all $a \in R$, so $c$ is in the center of $R$. Conversely, if $c$ is in the center of $R$, then for all $a \in R$, $ca = ac$, and so $c \in \bigcap_{a \in R} C(a)$. Thus the center of $R$ is the intersection of the subrings $C(a)$ over all $a \in R$.
\end{proof}

\section*{10. (7/9/24)}

Prove that if $D$ is a division ring then $C(a)$ is a division ring for all $a \in D$.

\begin{proof}
    Let $D$ be a division ring and let $a \in D$. Recall that, in a division ring, every nonzero element has a multiplicative inverse (denote $x$'s inverse by $x^{-1}$).

    Let $c \neq 0 \in C(a)$. We see that:
    \begin{align*}
        a &= a \\
        a &= a c c^{-1} \text{ ($cc^{-1} = 1$)}\\
        a &= c a c^{-1} \text{ ($ca = ac$)}\\
        c^{-1} a &= a c^{-1} \text{ (left-multiply by $c^{-1}$)},
    \end{align*}
    so $c^{-1} \in C(a)$. Since the multiplicative inverse of every element $c \in C(a)$ lies in $C(a)$, it is therefore a division ring.
\end{proof}

\section*{11. (7/9/24)}

Prove that if $R$ is an integral domain and $x^2 = 1$ for some $x \in R$ then $x = \pm 1$.

\begin{proof}
    Recall that an integral domain is a commutative ring with identity $1 \neq 0$ and no zero divisors. Suppose that $x^2 = 1$ for some $x \in R$. Then:
    \begin{align*}
        x^2 &= 1 \\
        x^2 + x &= x + 1 \\
        x(x + 1) &= x + 1.
    \end{align*}
    By the cancellation property of integral domains (Proposition 2), either $x + 1 = 0$, which implies that $x = -1$, or else $x = 1$.
\end{proof}

\end{document}