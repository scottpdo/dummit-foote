\documentclass{article}

\title{Dummit \& Foote Ch. 7.1: Introduction to Rings}
\author{Scott Donaldson}
\date{Jul. 2024}
\usepackage{amsmath, amsthm, amsfonts, amssymb, enumitem, tabu, tikz}

\begin{document}

\maketitle

Let $R$ be a ring with 1.

\section*{1. (7/1/24)}

Show that $(-1)^2 = 1$ in $\mathbb{R}$.

\begin{proof}
    We have:
    \begin{align*}
        (-1) + (-1)^2 = \underbrace{(-1)(1)}_{\text{identity}} + (-1)(-1) = \underbrace{(-1)(1 + (-1))}_{\text{distribution}} = (-1)\underbrace{(0)}_{\text{inverses}} = 0,
    \end{align*}
    and therefore, since $(-1) + (-1)^2 = 0$, $(-1)^2 = 1$.
\end{proof}

\section*{2. (7/1/24)}

Prove that if $u$ is a unit in $R$ then so is $-u$.

\begin{proof}
    Recall that $u$ is a unit in $R$ if there exists some $v \in R$ such that $uv = vu = 1$.

    Now:
    \begin{align*}
        (-u)(v) = -(uv) &= -1, \text{ which implies that} \\
        (-u)(v)(-1) = (-1)^2 &= 1, \text{ so} \\
        (-u)(-v) = 1,
    \end{align*}
    which implies that $-u$ is also a unit in $R$.
\end{proof}

\section*{7. (7/5/24)}

The \emph{center} of a ring $R$ is $\{ z \in R \mid zr = rz \text{ for all } r \in R \}$ (i.e., is the set of all elements which commute with every element of $R$). Prove that the center of a ring is a subring that contains the identity. Prove that the center of a division ring is a field.

\begin{proof}
    Let $a, b \in R$ be in the center of $R$ and let $x \in R$. Then:
    \begin{equation*}
        (a - b)x = ax - bx = xa - xb = x(a - b),
    \end{equation*}
    so $a - b$ is in the center of $R$. And, since $a$ and $b$ both commute with $x$, we have $(ab)x = abx = xab = x(ab)$, so $ab$ lies in the center of $R$ as well. Since by definition 1 commutes with every element of $R$, the center of $R$ is a subring of $R$ containing the identity.

    If $R$ is a division ring, then every element in its center (except 0) has a multiplicative inverse (is a unit). Every element in its center also commutes with every other element. A field is a commutative ring where every nonzero element is a unit; therefore the center of a division ring is a field.
\end{proof}

\section*{8. (7/9/24)}

Describe the center of the Hamilton Quaternions $\mathbb{H}$. Prove that $\{ a + bi \mid a, b \in \mathbb{R} \}$ is a subring of $\mathbb{H}$ which is a field but is not contained in the center of $\mathbb{H}$.

\begin{proof}
    Let $a + bi + cj + dk$ ($a, b, c, d \in \mathbb{R}$) lie in the center of $\mathbb{H}$. It must commute with $i$ ($= 0 + 1i + 0j + 0k$). Then:
    \begin{align*}
        (a + bi + cj + dk)i &= ai + bi^2 + cji + dki \\
        &= -b + ai + dj - ck, \text{ and } \\
        i(a + bi + cj + dk) &= ai + bi^2 + cij + dik \\
        &= -b + ai - dj + ck.
    \end{align*}
    If these are equal, then we have:
    \begin{align*}
        -b + ai + dj - ck &= -b + ai - dj + ck \\
        dj - ck &= -dj + ck \\ 
        2dj &= 2ck \\
        dj &= ck,
    \end{align*}
    and since $c, d \in \mathbb{R}$, there are no non-zero values of $c, d$ such that $dj = ck$. Thus we must have $c = d = 0$.

    Repeating the above steps for the product of $a + bi + cj + dk$ and $j$ or $k$, we see that $b$ must also be 0.

    Now because real coefficients of $i, j, k$ commute, $a$ may take any value, and so the center of $\mathbb{H}$ consists of the real numbers (that is, quaternions of the form $a + 0i + 0j + 0k$).

    Consider the subset $\{ a + bi \mid a, b \in \mathbb{R} \}$. Let $a + bi, c + di$ be two elements of this subset. We see that $(a + bi) - (c + di) = (a - c) + (b - d)i$ and $(a + bi)(c + di) = ac + adi + bci + bdi^2 = (ac - bd) + (ad + bc)i$. Since this subset is closed under subtraction and multiplication, it is a subring of $\mathbb{H}$. However, since it includes elements with non-zero $i$ components, it is not contained in the center of $\mathbb{H}$.
\end{proof}

\end{document}