\documentclass{article}

\title{Dummit \& Foote Ch. 7.2: Polynomial Rings, Matrix Rings, and Group Rings}
\author{Scott Donaldson}
\date{Oct. 2024}
\usepackage{amsmath, amsthm, amsfonts, amssymb, enumitem, tabu, tikz}

\begin{document}

\maketitle

Let $R$ be a commutative ring with 1.

\section*{2. (10/7/24)}

Let $p(x) = a_n x^n + a_{n - a} x^{n - 1} + ... + a_1 x + a_0$ be an element of the polynomial ring $R[x]$. Prove that $p(x)$ is a zero divisor in $R[x]$ if and only if there is a nonzero $b \in R$ such that $bp(x) = 0$.

\begin{proof}
    If there exists $b \in R$ such that $bp(x) = 0$, then the polynomial $b(x) = b \in R[x]$ is an element such that $b(x)p(x) = 0$, so $p(x)$ is a zero divisor.

    Conversely, suppose that $p(x)$ is zero divisor in $R$. Let $g(x) = b_m x^m + b_{m - 1} x^{m - 1} + ... + b_0$ be a nonzero polynomial of minimal degree such that $p(x) g(x) = 0$. Then:
    \begin{align*}
        p(x) g(x) &= (a_n x^n + ... + a_0)(b_m x^m + ... + b_0) \\
        &= a_n b_m x^{n + m} + ... + a_0 b_0 = 0,
    \end{align*}
    which implies that $a_n b_m = 0$.

    Then $a_n g(x) = a_n b_m x^m + ... a_n b_0 = a_n b_{m - 1} x^{m - 1} + ... a_n b_0$ is a polynomial of degree $m - 1$. And, because $p(x) g(x) = 0$, we have $a_n g(x) p(x) = 0$, contradicting $g(x)$ being a polynomial of minimal degree such that multiplying it by $p(x)$ is zero. Therefore we must have $a_n g(x) = 0$.

    Now suppose inductively that $a_{n - i} g(x) = 0$ and $a_{n - k} g(x) = 0$ for some $i \in \{ 0, ..., n \}$ and all $k \leq i$. Given $p(x) = a_n x^n + ... a_0$, let us write $p_{n - i}(x) = a_{n - i} x^{n - i} + ... + a_0$. Then:
    \begin{align*}
        p(x) g(x) &= (a_n x^n + ... + a_0)g(x) \\
        &= (a_n x^n + ... + a_{n - i} x^{n _ i} + p_{n - i - 1}(x))g(x) \\
        &= p_{n - i - 1}(x)g(x) \text{ (since } a_n g(x) = ... = a_{n - i} g(x) = 0 \text{)} \\
        &= a_{n - i - 1}x^{n - i - 1}g(x) + ... + a_0 g(x) = 0.
    \end{align*}
    It follows that the leading coefficient of $x^{n + m - i - 1}$, $a_{n - i - 1} b_m$, must equal zero. By induction, this implies that $a_i b_m = 0$ for all $i \in \{ 0, ..., n \}$; that is, $b_m p(x) = 0$ and so there exists a $b \in R$ such that $bp(x) = 0$.
\end{proof}

\end{document}